\documentclass{sciposter}
\usepackage{lipsum}
\usepackage{epsfig}
\usepackage{amsmath}
\usepackage{amssymb}
\usepackage{multicol}
\usepackage{graphicx,url}
\usepackage[portuges, brazil]{babel}   
\usepackage[utf8]{inputenc}
\usepackage{listings}
\usepackage{color}

\newtheorem{Def}{Definição}

\definecolor{codegray}{rgb}{0.5,0.5,0.5}
\definecolor{codeblue}{rgb}{0.25,0.5,0.75}

\lstdefinestyle{mystyle}{
    backgroundcolor=\color{white},
    commentstyle=\color{codegray},
    keywordstyle=\color{codeblue},
    numberstyle=\tiny\color{codegray},
    stringstyle=\color{codeblue},
    basicstyle=\ttfamily\footnotesize,
    breakatwhitespace=false,
    breaklines=true,
    captionpos=b,
    keepspaces=true,
    numbers=left,
    numbersep=5pt,
    showspaces=false,
    showstringspaces=false,
    showtabs=false,
    tabsize=2
}

\lstset{style=mystyle}

\title{Modelo de Consulta de Python para Ciência de Dados}
% Título do projeto

\institute 
{Bacharelado em Economia\\
Insper - Instituto de Ensino e Pesquisa\\
São Paulo, Brasil}
% Nome e endereço da Instituição

\rightlogo[1]{logo-insper.png}  % Substitua pelo logo do Insper

\begin{document}

\conference{{\bf Python para Ciência de Dados}, Curso de Economia - Insper, 2024, São Paulo, Brasil}

\maketitle

%%% Início do ambiente Multicolunas
\begin{multicols}{3}

%%% Resumo
\begin{abstract}
Este documento fornece um modelo de consulta para a eletiva Python para Ciência de Dados. Ele organiza os principais conceitos e tópicos relevantes para facilitar a revisão do curso.
\end{abstract}

\section{Aula 01}

\subsection{Tipos de Dados Fundamentais}

\subsubsection{Integer}
Os números inteiros são representados pelo tipo \texttt{int}. Útil para contagens e cálculos sem frações.

\subsubsection{Float}
Os números decimais são representados por \texttt{float}. Usado em cálculos que envolvem frações ou números contínuos.

\subsubsection{String}
Strings são sequências de caracteres e podem ser manipuladas de várias formas. Usadas para textos e concatenações.

\subsubsection{Boolean}
Valores lógicos, podendo ser \texttt{True} ou \texttt{False}, muito utilizados em condicionais e comparações.

\subsection{Coerção de Tipos}
Conversão entre tipos de dados, como de string para número ou vice-versa. É útil quando precisamos realizar operações entre diferentes tipos.

\subsection{Manipulação de Strings}
As strings podem ser manipuladas com métodos como:
- \texttt{.upper()}: Converte a string para letras maiúsculas.
- \texttt{.replace()}: Substitui uma substring por outra.
- Formatação com f-strings: Permite a inserção de variáveis dentro de strings.

\subsection{Função para Verificar Palíndromos}
Uma função que verifica se uma frase ou palavra é um palíndromo (ou seja, pode ser lida da mesma forma de trás para frente, ignorando espaços e maiúsculas/minúsculas).

\subsection{Listas e Tuplas}

\subsubsection{Listas}
Listas são coleções mutáveis de itens. Isso significa que podemos alterar, adicionar ou remover elementos de uma lista após sua criação.
Funções comuns para manipular listas:
- \texttt{.append()}: Adiciona um item ao final da lista.
- \texttt{.remove()}: Remove um item específico.
- \texttt{.pop()}: Remove e retorna o último item da lista.

\subsubsection{Tuplas}
Tuplas são coleções imutáveis, ou seja, após criadas, seus elementos não podem ser alterados. Útil para armazenar dados que não precisam ser modificados, como coordenadas.

\subsection{Índices em Listas}
Cada elemento em uma lista possui um índice, que começa em 0. Acessar elementos pelo índice é uma operação comum:
- O primeiro elemento da lista pode ser acessado por \texttt{lista[0]}.
- O último elemento pode ser acessado por \texttt{lista[-1]}.
- Podemos também acessar fatias da lista, como \texttt{lista[1:4]}, que retorna os elementos do índice 1 até o 3 (o último índice não é incluído).

\subsection{Quando Usar e Quando Evitar}

\subsubsection{Integer e Float}
Usar quando for necessário realizar cálculos numéricos. Use \texttt{int} para contagens e \texttt{float} para valores com frações. Evitar se o dado não for numérico.

\subsubsection{String}
Usar para armazenar e manipular textos. Útil para concatenações e exibição de informações textuais. Evitar se precisar realizar cálculos numéricos diretamente.

\subsubsection{Boolean}
Usar para tomadas de decisão, comparações e estruturas de controle (condicionais). Evitar para armazenar dados não binários.

\subsubsection{Listas}
Usar para armazenar coleções de dados que precisam ser manipuladas, como adicionar ou remover elementos. Evitar se os dados forem imutáveis.

\subsubsection{Tuplas}
Usar para armazenar dados que não devem ser alterados durante a execução do programa. Evitar se precisar alterar ou reorganizar os elementos.

\section{Aula 02}

\subsection{Objetos Básicos: Dicionários e Conjuntos}

\subsubsection{Dicionários}
Dicionários são coleções de pares \texttt{chave:valor}. São úteis para armazenar dados que precisam ser acessados por meio de uma chave, em vez de uma posição numérica.
- Chaves podem ser usadas para buscar valores rapidamente.
- Métodos úteis incluem:
  - \texttt{.keys()}: Retorna as chaves do dicionário.
  - \texttt{.values()}: Retorna os valores.
  - \texttt{.items()}: Retorna uma lista de pares \texttt{chave:valor}.
- Usos: Dicionários são ideais para armazenar informações como registros de dados, onde é importante referenciar o valor por uma chave descritiva.

\subsubsection{Conjuntos}
Conjuntos são coleções não ordenadas de elementos únicos. Eles são usados para realizar operações matemáticas como interseção, união e diferença de conjuntos.
- Operações comuns:
  - \texttt{.intersection()}: Retorna a interseção de dois conjuntos.
  - \texttt{.union()}: Retorna a união de dois conjuntos.
  - \texttt{.difference()}: Retorna os elementos que estão em um conjunto, mas não no outro.
- Usos: Conjuntos são ideais para cenários onde a presença única de itens é importante, como em operações de comparação e filtragem de duplicatas.

\subsection{Condicionais e Estruturas Repetitivas}

\subsubsection{Condicionais}
As estruturas condicionais em Python permitem a execução de blocos de código com base em condições específicas.
- A instrução \texttt{if} executa um bloco de código se a condição for verdadeira.
- A instrução \texttt{else} executa um bloco de código se a condição for falsa.
- O Python permite verificações concisas com \texttt{if} aninhado, como \texttt{0 <= x <= 1}.

\subsubsection{Estruturas Repetitivas}
Estruturas como o \texttt{for} e \texttt{while} são usadas para repetir blocos de código.
- O \texttt{for} é frequentemente usado para iterar sobre listas, dicionários e outras coleções.
- O \texttt{while} continua executando enquanto uma condição for verdadeira.
- O \texttt{continue} interrompe a iteração atual e vai para a próxima.
- O \texttt{break} termina o loop completamente.

\subsection{Funções Lambda}
Funções lambda são funções anônimas em Python, definidas usando a palavra-chave \texttt{lambda}. Elas são úteis para operações curtas e rápidas, especialmente quando você não precisa definir uma função completa.
- A sintaxe básica é: \texttt{lambda argumentos: expressão}.
- Funções lambda são frequentemente usadas em conjunto com funções como \texttt{map()} e \texttt{filter()}, que aplicam a função a todos os elementos de uma lista ou conjunto.
- Usos: Lambdas são úteis em funções de ordem superior e operações que exigem uma transformação rápida de dados sem a necessidade de uma função nomeada.

\subsection{List Comprehensions}
São uma forma concisa de criar listas em Python. Elas permitem aplicar uma operação a cada item de uma sequência e criar uma nova lista baseada no resultado.
- Exemplo: \texttt{[x**2 for x in range(10)]} cria uma lista dos quadrados dos números de 0 a 9.
- Usos: List comprehensions são úteis para manipulação de listas e filtragem de dados de forma concisa e eficiente.

\section{Aula 03}

\subsection{Sequência de Fibonacci}
A sequência de Fibonacci é definida recursivamente com \(a_0 = a_1 = 1\) e \(a_n = a_{n-2} + a_{n-1}\) para \(n \geq 2\). À medida que \(n\) aumenta, a razão entre números consecutivos da sequência converge para o número de ouro \(\phi = \frac{1 + \sqrt{5}}{2}\).

\subsection{NumPy e Matrizes}
O **NumPy** é uma biblioteca poderosa para computação numérica em Python. Ele permite a criação e manipulação de arrays e matrizes, além de oferecer uma ampla gama de funções matemáticas e estatísticas.

- **Arrays**: Estruturas de dados semelhantes a listas, porém mais eficientes para operações numéricas. Arrays podem ser multidimensionais.
- **Operações vetorizadas**: No NumPy, operações matemáticas em arrays podem ser feitas de forma vetorizada, o que significa que operações em todos os elementos de um array são realizadas de uma vez, sem necessidade de loops.
- **Matrizes**: Arrays bidimensionais (ou mais), úteis para representar dados tabulares ou realizar cálculos matemáticos.

Exemplo de criação de uma matriz e suas operações:
- **Criação de matriz**: \texttt{np.array([[1, 2], [3, 4]])}
- **Soma de matrizes**: \texttt{A + B}
- **Multiplicação de matrizes**: \texttt{np.dot(A, B)}

\subsection{Classes e Objetos}
Classes são a base da programação orientada a objetos. Uma classe define a estrutura e o comportamento dos objetos que podem ser criados a partir dela.
- **Método de classe:** Funções associadas a uma classe que podem acessar ou modificar atributos de classe (compartilhados por todos os objetos da classe).
- **Método de instância:** Funções que operam sobre objetos específicos da classe, usando atributos e métodos do objeto.
- **Atributos:** Características de uma classe ou de um objeto. Atributos de classe são compartilhados por todos os objetos, enquanto atributos de instância são únicos para cada objeto.

\subsection{Exceções}
Exceções permitem capturar erros que ocorrem durante a execução do programa e tratá-los de forma apropriada, sem interromper o fluxo do código.
- A palavra-chave \texttt{try} tenta executar um bloco de código.
- A palavra-chave \texttt{except} captura exceções e executa um bloco de código alternativo quando um erro é detectado.

\subsection{Lei dos Grandes Números}
Essa lei estatística afirma que, à medida que o tamanho da amostra aumenta, a média das amostras se aproxima da média esperada da população.

\subsection{Teorema Central do Limite}
Esse teorema afirma que, para um número suficientemente grande de amostras independentes e identicamente distribuídas, a distribuição das médias dessas amostras tende a ser normal, independentemente da distribuição original.

\section{Aula 01}

\subsection{Tipos de Dados Fundamentais}

\subsubsection{Integer}
Os números inteiros são representados pelo tipo \texttt{int}. Exemplo de operação com inteiros:
\begin{lstlisting}[language=Python]
x = 10
y = 3
print(x // y)  # Divisão inteira: 3
\end{lstlisting}

\subsubsection{Float}
Os números decimais são representados por \texttt{float}. Exemplo de cálculo com decimais:
\begin{lstlisting}[language=Python]
a = 3.5
b = 2.0
print(a * b)  # Resultado: 7.0
\end{lstlisting}

\subsubsection{String}
Strings são sequências de caracteres. Aqui está um exemplo de manipulação de string com f-string:
\begin{lstlisting}[language=Python]
name = "Alice"
age = 30
print(f"{name} tem {age} anos.")  # Alice tem 30 anos.
\end{lstlisting}

\subsubsection{Boolean}
Valores booleanos são usados em comparações. Exemplo de condicional simples:
\begin{lstlisting}[language=Python]
x = 10
print(x > 5)  # Retorna: True
\end{lstlisting}

\subsection{Coerção de Tipos}
A coerção de tipos converte um tipo de dado em outro. Exemplo de conversão de string para inteiro:
\begin{lstlisting}[language=Python]
num_str = "123"
num = int(num_str)
print(type(num))  # Saída: <class 'int'>
\end{lstlisting}

\subsection{Manipulação de Strings}
As strings podem ser manipuladas de várias formas, como alterar para maiúsculas ou substituir caracteres:
\begin{lstlisting}[language=Python]
text = "insper"
print(text.upper())  # INSper
print(text.replace("i", "I"))  # Insper
\end{lstlisting}

\subsection{Listas e Tuplas}
\subsubsection{Listas}
Listas são coleções mutáveis. Exemplo de adição e remoção de elementos:
\begin{lstlisting}[language=Python]
fruits = ["apple", "banana", "cherry"]
fruits.append("orange")  # Adiciona "orange" à lista
print(fruits)
fruits.remove("banana")  # Remove "banana"
print(fruits)
\end{lstlisting}

\subsubsection{Tuplas}
Tuplas são coleções imutáveis. Exemplo de uso de tuplas:
\begin{lstlisting}[language=Python]
coordinates = (10, 20)
print(coordinates[0])  # Acessa o primeiro item da tupla
\end{lstlisting}

\section{Aula 02}

\subsection{Dicionários e Conjuntos}
\subsubsection{Dicionários}
Dicionários armazenam pares chave-valor. Exemplo de como acessar valores em um dicionário:
\begin{lstlisting}[language=Python]
student = {"name": "Alice", "age": 25}
print(student["name"])  # Saída: Alice
\end{lstlisting}

\subsubsection{Conjuntos}
Conjuntos armazenam elementos únicos. Exemplo de operação de união entre conjuntos:
\begin{lstlisting}[language=Python]
set_a = {1, 2, 3}
set_b = {3, 4, 5}
print(set_a.union(set_b))  # {1, 2, 3, 4, 5}
\end{lstlisting}

\subsection{Condicionais}
Exemplo de estrutura condicional:
\begin{lstlisting}[language=Python]
x = 10
if x > 5:
    print("Maior que 5")
else:
    print("Menor ou igual a 5")
\end{lstlisting}

\subsection{Estruturas Repetitivas}
Exemplo de loop \texttt{for} para iterar em uma lista:
\begin{lstlisting}[language=Python]
numbers = [1, 2, 3, 4, 5]
for num in numbers:
    print(num)
\end{lstlisting}

\subsection{Funções Lambda}
Funções lambda permitem definir funções anônimas rapidamente. Exemplo com \texttt{filter} para filtrar números pares:
\begin{lstlisting}[language=Python]
numbers = [1, 2, 3, 4, 5]
even_numbers = list(filter(lambda x: x % 2 == 0, numbers))
print(even_numbers)  # Saída: [2, 4]
\end{lstlisting}

\subsection{List Comprehensions}
Exemplo de list comprehension para criar uma lista de quadrados:
\begin{lstlisting}[language=Python]
squares = [x**2 for x in range(10)]
print(squares)  # Saída: [0, 1, 4, ..., 81]
\end{lstlisting}

\section{Aula 03}

\subsection{NumPy e Matrizes}
NumPy facilita o trabalho com arrays e matrizes. Exemplo de criação de uma matriz e soma de elementos:
\begin{lstlisting}[language=Python]
import numpy as np
matrix = np.array([[1, 2], [3, 4]])
print(matrix.sum())  # Saída: 10
\end{lstlisting}

\subsection{Classes e Objetos}
Exemplo de criação de uma classe e instanciamento de um objeto:
\begin{lstlisting}[language=Python]
class Dog:
    def __init__(self, name):
        self.name = name

    def bark(self):
        print(f"{self.name} is barking")

my_dog = Dog("Rex")
my_dog.bark()  # Rex is barking
\end{lstlisting}

\subsection{Exceções}
Tratamento de exceções usando \texttt{try} e \texttt{except}:
\begin{lstlisting}[language=Python]
try:
    x = 10 / 0
except ZeroDivisionError:
    print("Divisão por zero!")
\end{lstlisting}

\subsection{Lei dos Grandes Números e Teorema Central do Limite}
Estes conceitos são demonstrados frequentemente usando simulações de Monte Carlo e análise estatística. No contexto de Python, você pode usar bibliotecas como NumPy e matplotlib para realizar experimentos como simulações de médias e convergência para distribuições normais.



\end{multicols}

\end{document}
