\documentclass[a4paper,12pt]{article}[abntex2]
\bibliographystyle{abntex2-alf}
\usepackage{siunitx} % Fornece suporte para a tipografia de unidades do Sistema Internacional e formatação de números
\usepackage{booktabs} % Melhora a qualidade das tabelas
\usepackage{tabularx} % Permite tabelas com larguras de colunas ajustáveis
\usepackage{graphicx} % Suporte para inclusão de imagens
\usepackage{newtxtext} % Substitui a fonte padrão pela Times Roman
\usepackage{ragged2e} % Justificação de texto melhorada
\usepackage{setspace} % Controle do espaçamento entre linhas
\usepackage[a4paper, left=3.0cm, top=3.0cm, bottom=2.0cm, right=2.0cm]{geometry} % Personalização das margens do documento
\usepackage{lipsum} % Geração de texto dummy 'Lorem Ipsum'
\usepackage{fancyhdr} % Customização de cabeçalhos e rodapés
\usepackage{titlesec} % Personalização dos títulos de seções
\usepackage[portuguese]{babel} % Adaptação para o português (nomes e hifenização
\usepackage{hyperref} % Suporte a hiperlinks
\usepackage{indentfirst} % Indentação do primeiro parágrafo das seções
\sisetup{
  output-decimal-marker = {,},
  inter-unit-product = \ensuremath{{}\cdot{}},
  per-mode = symbol
}
\setlength{\headheight}{14.49998pt}

\DeclareSIUnit{\real}{R\$}
\newcommand{\real}[1]{R\$#1}
\usepackage{float} % Melhor controle sobre o posicionamento de figuras e tabelas
\usepackage{footnotehyper} % Notas de rodapé clicáveis em combinação com hyperref
\hypersetup{
    colorlinks=true,
    linkcolor=black,
    filecolor=magenta,      
    urlcolor=cyan,
    citecolor=black,        
    pdfborder={0 0 0},
}
\usepackage[normalem]{ulem} % Permite o uso de diferentes tipos de sublinhados sem alterar o \emph{}
\makeatletter
\def\@pdfborder{0 0 0} % Remove a borda dos links
\def\@pdfborderstyle{/S/U/W 1} % Estilo da borda dos links
\makeatother
\onehalfspacing

\begin{document}

\begin{titlepage}
    \centering
    \vspace*{1cm}
    \Large\textbf{INSPER – INSTITUTO DE ENSINO E PESQUISA}\\
    \Large ECONOMIA\\
    \vspace{1.5cm}
    \Large\textbf{Estudos de H.E.B I}\\
    \vspace{1.5cm}
    Prof. Heleno Piazenini Vieira\\
    Prof. Auxiliar \\
    \vfill
    \normalsize
    Hicham Munir Tayfour, \href{mailto:hichamt@al.insper.edu.br}{hichamt@al.insper.edu.br}\\
    5º Período - Economia A\\
    \vfill
    São Paulo\\
    Agosto/2024
\end{titlepage}

\newpage
\tableofcontents
\thispagestyle{empty} % This command removes the page number from the table of contents page
\newpage
\setcounter{page}{1} % This command sets the page number to start from this page
\justify
\onehalfspacing

\pagestyle{fancy}
\fancyhf{}
\rhead{\thepage}

\section{\textbf{Resumos dos Textos}}
\subsection{\textbf{Formação Econômica do Brasil- Celso Furtado}}
\subsubsection{\textbf{Capítulo 1: Da Expansão Comercial à Empresa Agrícola}}

O capítulo 1 do livro *Formação Econômica do Brasil* de Celso Furtado oferece uma análise profunda sobre como a ocupação econômica das terras americanas, especificamente o Brasil, foi um desdobramento da expansão comercial europeia dos séculos XV e XVI. A ocupação não foi motivada por um excesso populacional, como em casos históricos de migrações, mas sim pela busca de novas rotas e mercados, especialmente após as invasões turcas que dificultaram o comércio com o Oriente.

Inicialmente, a descoberta das Américas parecia secundária, com os portugueses focados no lucrativo comércio oriental. Porém, o ouro extraído das civilizações mexicanas e andinas pelos espanhóis rapidamente transformou a América em um objetivo central para as potências europeias, suscitando grande interesse e rivalidade. Espanha e Portugal, detentores do direito sobre essas terras através do Tratado de Tordesilhas, se viram pressionados por outras nações europeias, especialmente a França, que tentaram estabelecer colônias nas novas terras.

A necessidade de consolidar a ocupação das terras brasileiras tornou-se evidente para Portugal após incursões francesas, levando os portugueses a desviar recursos do Oriente para o Brasil. Esse movimento foi parcialmente motivado pela esperança de encontrar ouro no interior do território, uma expectativa que, embora não imediata, foi crucial para justificar o investimento na colonização.

Com o objetivo de viabilizar economicamente a defesa dessas terras, Portugal passou de uma economia extrativa, centrada na exploração de pau-brasil e outros recursos naturais, para uma economia agrícola. A experiência prévia dos portugueses na produção de açúcar nas ilhas atlânticas foi vital para o sucesso desse empreendimento no Brasil. Eles não apenas resolveram os desafios técnicos relacionados à produção de açúcar, mas também criaram uma base industrial para a fabricação dos engenhos necessários.

No campo comercial, o açúcar produzido em Portugal inicialmente enfrentou um mercado restrito, mas a crise de superprodução e a subsequente queda de preços no final do século XV indicaram que os canais tradicionais, controlados por comerciantes italianos, não eram suficientes. A expansão do comércio para a Flandres e o envolvimento dos holandeses, que se tornaram parceiros chave na distribuição e refinação do açúcar, foram essenciais para a consolidação desse mercado. Os holandeses não só ajudaram na comercialização, mas também forneceram o capital necessário para a expansão da produção no Brasil, incluindo o financiamento da importação de mão de obra escrava da África.

O problema da mão de obra foi particularmente desafiador. A escassez de trabalhadores na Europa, combinada com os altos custos de transporte e as duras condições de trabalho no Brasil, tornou inviável a importação de trabalhadores europeus em grande escala. No entanto, os portugueses já dominavam o mercado africano de escravos, o que possibilitou a criação de um fluxo constante de mão de obra barata para sustentar a produção açucareira no Brasil.

O capítulo conclui que o sucesso da empresa agrícola no Brasil não foi fruto de um planejamento rigoroso, mas sim de uma série de circunstâncias favoráveis que foram habilmente aproveitadas. O desejo do governo português de manter suas possessões na América, associado à lucratividade da produção açucareira, garantiu a continuidade da ocupação portuguesa no Brasil. Esse sucesso não apenas assegurou a presença portuguesa em grande parte do território americano, mas também estabeleceu as bases para a expansão territorial e a transformação do Brasil em uma colônia economicamente viável.

A análise de Furtado revela como a economia brasileira começou a se formar a partir desses primeiros empreendimentos agrícolas, que, ao transformar o Brasil em um importante produtor de açúcar, integraram o país na economia mundial e moldaram suas estruturas econômicas e sociais nas décadas seguintes.

\subsubsection{\textbf{Capítulo 2: Fatores do Êxito da Empresa Agrícola}}

No capítulo 2, Celso Furtado explora os fatores que permitiram o sucesso da empresa agrícola portuguesa no Brasil, destacando a importância de uma combinação de elementos técnicos, comerciais e financeiros.

Primeiramente, Furtado ressalta que os portugueses já tinham experiência na produção de açúcar nas ilhas do Atlântico, especialmente na Madeira e em São Tomé, antes de se aventurarem no Brasil. Essa experiência foi crucial para resolver os desafios técnicos da produção açucareira, desde a construção dos engenhos até o desenvolvimento de uma indústria de equipamentos em Portugal. A capacidade técnica acumulada possibilitou aos portugueses superar as dificuldades de exportação de equipamentos e conhecimento, fatores que teriam tornado o sucesso da empreitada brasileira mais difícil sem esse avanço prévio.

No campo comercial, o açúcar produzido pelos portugueses inicialmente entrou nos mercados europeus através de canais controlados por comerciantes italianos, principalmente venezianos. No entanto, a crise de superprodução no final do século XV e a consequente queda de preços sugeriram que esses canais não eram suficientemente amplos para absorver o aumento da produção. Como resultado, o comércio se expandiu para novas áreas, particularmente para Flandres, rompendo o monopólio veneziano. Essa expansão foi facilitada pelos flamengos, que começaram a refinar e distribuir o açúcar português, ampliando significativamente o mercado na Europa.

A partir de meados do século XVI, a parceria entre portugueses e flamengos, especialmente os holandeses, intensificou-se. Os flamengos não só refinaram e distribuíram o açúcar por toda a Europa, mas também forneceram capitais essenciais para a expansão da produção no Brasil. Os investimentos dos holandeses não se limitaram à refinação e comercialização; eles também financiaram a construção de engenhos no Brasil e a importação de mão de obra escrava africana. A viabilidade e rentabilidade da empresa açucareira brasileira foram amplamente demonstradas, o que atraiu ainda mais investimentos de poderosos grupos financeiros europeus.

Um dos maiores desafios enfrentados foi a questão da mão de obra. A importação de trabalhadores europeus era inviável devido aos altos custos e às condições adversas no Brasil. No entanto, os portugueses, já envolvidos no tráfico de escravos africanos, resolveram esse problema ao importar maciçamente escravos para trabalharem nas plantações de açúcar. Essa solução foi essencial para manter a competitividade e a lucratividade da empresa açucareira.

Finalmente, Furtado conclui que o sucesso da empresa agrícola no Brasil não foi fruto de um plano meticulosamente preestabelecido, mas sim de uma série de circunstâncias favoráveis que foram aproveitadas de maneira eficiente. A determinação do governo português em conservar suas terras na América, motivada pela esperança de encontrar ouro, foi fundamental para o apoio contínuo à colonização e à produção açucareira. Esse êxito garantiu a presença portuguesa nas terras americanas e permitiu que, no século seguinte, Portugal avançasse significativamente na ocupação e exploração dessas terras, mesmo diante das mudanças no equilíbrio de poder na Europa.

O capítulo mostra, assim, como cada um dos problemas enfrentados — técnica de produção, criação de mercado, financiamento e mão de obra — foi superado oportunamente, estabelecendo as bases para o sucesso econômico do Brasil colonial.

\subsubsection{\textbf{Capítulo 3: Razões do Monopólio}}

No capítulo 3, Celso Furtado discute as razões por trás do monopólio comercial português sobre a economia colonial brasileira, particularmente no contexto da produção açucareira. O sucesso financeiro extraordinário da colonização agrícola no Brasil tornou as novas terras altamente atraentes para a exploração econômica. No entanto, ao contrário dos portugueses, os espanhóis concentraram-se na extração de metais preciosos em suas colônias americanas, isolando suas áreas ricas em minerais das pressões concorrenciais.

As colônias espanholas, densamente povoadas, dependiam principalmente da exploração da mão de obra indígena e estavam estruturadas para produzir um excedente líquido na forma de metais preciosos, que era transferido periodicamente para a metrópole. Esse fluxo constante de riqueza gerou profundas transformações na economia espanhola, incluindo uma inflação crônica e um déficit comercial persistente, que limitavam a capacidade do país de expandir outras atividades econômicas além da mineração.

Por outro lado, a política econômica espanhola, voltada para a autossuficiência de suas colônias, resultou em uma escassez de transporte e fretes elevados, dificultando o comércio regular entre as colônias e a metrópole. Enquanto isso, a estrutura econômica da colonização portuguesa, menos focada na extração mineral, permitiu a criação de um sistema agrícola, centrado na produção de açúcar, que aproveitava o mercado europeu em expansão.

O capítulo também analisa como a decadência econômica espanhola beneficiou indiretamente a empresa colonial portuguesa. A falta de interesse da Espanha em desenvolver uma agricultura de exportação competitiva deixou um espaço que Portugal, aliado a parceiros comerciais como os holandeses, preencheu eficientemente. Essa parceria permitiu a Portugal não só consolidar sua posição no mercado de açúcar, mas também sobreviver aos desafios impostos pela guerra com a Espanha e a subsequente ocupação holandesa de partes do Brasil.

Além disso, Furtado examina o impacto das rivalidades europeias nas Américas, enfatizando que a absorção de Portugal pela Espanha e a guerra com a Holanda determinaram o fim de uma fase de cooperação econômica benéfica entre portugueses e holandeses. A invasão holandesa no nordeste brasileiro, por um quarto de século, resultou na transferência de conhecimento técnico e organizacional da indústria açucareira para os holandeses, que depois implantaram essa indústria em larga escala no Caribe, competindo diretamente com o Brasil.

O monopólio que os portugueses mantinham sobre o comércio de açúcar começou a se desintegrar à medida que o Caribe emergia como uma região produtora significativa. No final do século XVII, os preços do açúcar caíram para metade dos valores anteriores, e as exportações brasileiras diminuíram drasticamente. Isso marcou o início do fim da era de ouro da produção açucareira brasileira, afetando gravemente a economia de Portugal e do Brasil.

Furtado conclui que o sucesso inicial da empresa agrícola portuguesa foi possível devido a uma série de circunstâncias favoráveis, incluindo a aliança com os holandeses e a inércia econômica da Espanha. Entretanto, esses fatores também tornaram a economia colonial brasileira altamente vulnerável às mudanças no cenário econômico global, levando à perda do monopólio português no comércio de produtos tropicais e à necessidade de reavaliar a estratégia econômica colonial.

\subsubsection{\textbf{Capítulo 14: Fluxo da Renda}}

No capítulo 14, Celso Furtado analisa de forma minuciosa o fluxo de renda gerado pela economia mineradora no Brasil durante o ciclo do ouro no século XVIII, destacando seus efeitos econômicos e sociais tanto na colônia quanto na metrópole portuguesa. O ciclo do ouro foi um dos períodos mais prósperos da história colonial brasileira, transformando regiões como Minas Gerais, Goiás e Mato Grosso em centros dinâmicos de extração de riquezas. No entanto, essa prosperidade trouxe consigo uma série de desafios e distorções econômicas que Furtado examina com profundidade.

A economia mineradora era altamente especializada na extração de ouro e diamantes, atividades que geravam enormes fluxos de renda, mas que também criavam uma dependência perigosa do mercado internacional. A exportação de ouro atingiu seu auge por volta de 1760, mas o declínio subsequente foi inevitável à medida que as minas se esgotavam e os custos de extração aumentavam. Essa dependência da exportação de recursos naturais expunha a economia colonial às flutuações nos preços internacionais do ouro, o que tornava o crescimento econômico extremamente volátil e insustentável a longo prazo.

Um ponto central da análise de Furtado é a falta de diversificação econômica nas regiões mineradoras. A riqueza gerada pela mineração foi em grande parte consumida ou transferida para a metrópole, em vez de ser reinvestida em outros setores produtivos, como agricultura, manufatura ou infraestrutura. Essa ausência de reinvestimento impediu o desenvolvimento de uma base econômica diversificada e autossustentável, perpetuando um ciclo de exploração intensiva dos recursos naturais sem perspectivas de desenvolvimento futuro. As regiões mineradoras, após o esgotamento das jazidas, caíram em um estado de estagnação econômica e social, evidenciando a fragilidade desse modelo de desenvolvimento.

Furtado também destaca as profundas desigualdades na distribuição da renda gerada pela mineração. A maior parte da riqueza ficava concentrada nas mãos da elite colonial e da Coroa portuguesa, enquanto a população local, incluindo os escravos que constituíam a força de trabalho principal nas minas, pouco se beneficiava dessa abundância. A concentração de renda e poder nas mãos de poucos reforçou as estruturas sociais e econômicas desiguais, criando um abismo entre ricos e pobres que perpetuou a marginalização da maior parte da população.

Além disso, Furtado explora as implicações políticas desse fluxo de renda. A riqueza gerada pelo ouro fortaleceu o poder da Coroa portuguesa, permitindo-lhe financiar guerras e projetos de expansão na Europa e em outras partes do mundo. No entanto, essa riqueza também gerou tensões internas na colônia, com conflitos entre as autoridades coloniais e os mineradores, e fomentou uma economia dependente de ciclos de exploração, que não incentivava o desenvolvimento de atividades econômicas autônomas e diversificadas.

O capítulo conclui que, embora o ciclo do ouro tenha proporcionado um impulso econômico significativo no curto prazo, ele também criou uma estrutura econômica e social frágil, marcada pela concentração de renda, dependência externa e falta de diversificação produtiva. A herança da economia mineradora é, segundo Furtado, um legado de desigualdade e subdesenvolvimento que moldou de forma duradoura o futuro econômico do Brasil. O esgotamento das minas não apenas sinalizou o fim de um período de prosperidade, mas também expôs as fraquezas estruturais de uma economia colonial excessivamente dependente da exportação de recursos naturais, deixando profundas marcas que seriam sentidas ao longo dos séculos seguintes.

\subsubsection{\textbf{Capítulo 15: Regressão Econômica e Expansão da Área de Subsistência}}

No capítulo 15, Celso Furtado examina as consequências econômicas e sociais do declínio da economia mineradora no Brasil colonial e como isso levou a uma regressão econômica e à expansão da economia de subsistência. Esse processo de regressão ocorreu quando a produção de ouro entrou em colapso, afetando drasticamente as regiões dependentes da mineração, como Minas Gerais, Goiás e Mato Grosso.

Furtado explica que, com a queda na produção de ouro, as grandes empresas mineradoras começaram a se desintegrar, perdendo capital e mão de obra, que já não podia ser reposta devido à diminuição dos lucros. Muitos empresários de lavras, outrora prósperos, se viram reduzidos a simples faiscadores, sobrevivendo apenas com a extração manual de pequenas quantidades de ouro. Esse processo de decadência não foi rápido; pelo contrário, foi marcado por uma lenta e contínua diminuição do capital investido no setor minerador, o que eventualmente levou à desagregação completa da economia mineira.

Em contraste com o colapso da mineração, houve uma expansão da agricultura de subsistência, que se tornou a principal atividade econômica nas regiões afetadas. Essa expansão não representou um avanço econômico, mas sim uma adaptação às novas circunstâncias, em que a população local, sem outra fonte de renda, se voltou para a produção de alimentos para o consumo próprio. A economia de subsistência, que se espalhou por vastas áreas do território brasileiro, estava marcada por técnicas agrícolas rudimentares e por uma baixa densidade econômica, caracterizada pela ausência de especialização e pelo uso extensivo e ineficiente da terra.

Furtado também analisa as consequências regionais dessa regressão econômica. Algumas áreas conseguiram se manter por meio da diversificação econômica, enquanto outras regiões experimentaram um empobrecimento acentuado. A falta de um setor produtivo diversificado agravou as desigualdades regionais, criando uma disparidade significativa entre as regiões que conseguiram manter alguma forma de dinamismo econômico e aquelas que entraram em estagnação.

Outro ponto importante destacado por Furtado é a concentração da propriedade da terra. Através do sistema de sesmarias, a terra, antes monopólio real, passou para as mãos de poucos proprietários, que detinham vastas extensões de terras, mas que não as utilizavam de maneira eficiente. Essa concentração de terras contribuiu para a perpetuação de uma economia de subsistência e para a manutenção de uma estrutura social hierárquica, em que a maior parte da população dependia de uma pequena elite de proprietários de terras.

O capítulo conclui que a regressão econômica e a expansão da economia de subsistência tiveram impactos duradouros no desenvolvimento do Brasil. A falta de diversificação econômica e a dependência excessiva de ciclos de exploração de recursos naturais criaram uma base frágil para o crescimento econômico sustentável. Esse período de estagnação contribuiu para a perpetuação das desigualdades regionais e para a formação de uma economia nacional que, durante muito tempo, se manteve vulnerável às flutuações externas e dependente de um modelo econômico baseado na exploração de recursos primários.

Furtado argumenta que o legado dessa fase histórica foi a criação de uma estrutura econômica que, ao invés de promover o desenvolvimento, reforçou as desigualdades e as disparidades regionais, perpetuando um ciclo de pobreza e subdesenvolvimento que seria difícil de romper nas décadas seguintes.

\subsection{\textbf{O Desenvolvimento Econômico no Brasil Pré-1945 - Villela e Suzigan}}

\subsubsection{\textbf{Capítulo 4: Regressão Econômica e Expansão da Área de Subsistência}}

Neste capítulo, Villela e Suzigan exploram em profundidade o processo de regressão econômica que ocorreu no Brasil durante o século XVIII, especialmente nas regiões que haviam prosperado durante o ciclo do ouro. O esgotamento gradual das minas de ouro, que durante décadas foram o motor da economia colonial, resultou em um colapso econômico que transformou radicalmente a estrutura produtiva dessas áreas.

Com a progressiva exaustão das minas, a produção de ouro diminuiu significativamente, levando ao abandono das atividades mineradoras por parte de muitos trabalhadores. Aqueles que permaneciam nas regiões mineradoras, na ausência de alternativas, começaram a se dedicar à agricultura de subsistência. Esse movimento marcou uma transição forçada da economia exportadora, baseada na mineração, para uma economia voltada ao autoconsumo, com baixa produtividade e escasso excedente.

Os autores destacam que essa expansão da agricultura de subsistência foi uma resposta às novas condições impostas pelo declínio da mineração. No entanto, essa expansão não foi acompanhada de melhorias tecnológicas ou de uma reorganização produtiva capaz de aumentar a eficiência agrícola. Pelo contrário, a agricultura que emergiu nas regiões mineradoras após o declínio do ouro foi caracterizada por técnicas rudimentares, uso extensivo e ineficiente da terra, e uma produção orientada principalmente para a sobrevivência das famílias locais, sem gerar excedentes significativos para o comércio.

Villela e Suzigan também abordam as consequências sociais dessa transformação. A concentração fundiária tornou-se ainda mais pronunciada à medida que as terras mineradoras foram sendo gradualmente incorporadas por grandes proprietários rurais, que transformaram as antigas áreas de mineração em grandes propriedades agrícolas. Essa concentração de terras reforçou a desigualdade social, uma vez que a maioria da população permaneceu sem acesso a recursos suficientes para melhorar suas condições de vida. A concentração fundiária, combinada com a falta de diversificação econômica, resultou em uma estrutura social altamente hierarquizada, onde uma pequena elite agrária dominava vastas áreas de terra e os recursos econômicos, enquanto a maior parte da população vivia em condições de extrema pobreza e marginalização.

Os autores argumentam que a regressão econômica do Brasil no século XVIII não foi um fenômeno isolado, mas sim o resultado de uma combinação de fatores estruturais, incluindo a falta de diversificação econômica, a dependência de um único recurso exportador e a concentração de poder econômico e político nas mãos de uma elite agrária. Essa regressão teve consequências duradouras para o desenvolvimento econômico do Brasil, perpetuando um modelo econômico baseado na exploração extensiva de recursos naturais, com pouca ou nenhuma atenção ao desenvolvimento de setores produtivos diversificados que pudessem sustentar o crescimento a longo prazo.

O capítulo conclui que a expansão da área de subsistência, longe de ser um sinal de desenvolvimento econômico, refletia uma adaptação ao colapso das atividades mineradoras, marcando um retrocesso significativo na estrutura econômica do Brasil. Essa fase histórica deixou um legado de desigualdade e subdesenvolvimento que moldou a trajetória econômica do país nas décadas seguintes. A falta de investimentos em setores produtivos diversificados e a concentração fundiária contribuíram para a criação de uma economia dependente, vulnerável às flutuações externas e incapaz de promover o desenvolvimento sustentável a longo prazo.

\subsection{\textbf{Adeus, Senhor Portugal - Rafael Cariello e Thales Zamberlam Pereira}}

\subsubsection{\textbf{Capítulo 1: A Crise Inaugural}}

O primeiro capítulo de *Adeus, Senhor Portugal* aborda a complexa crise fiscal que não só marcou o nascimento do Brasil como também precipitou a independência do país e a queda do absolutismo português. Rafael Cariello e Thales Zamberlam Pereira iniciam o capítulo descrevendo o Brasil como fruto de uma crise fiscal, onde o déficit e a inflação foram os "pais" e "mães" do novo país. Este cenário de crise econômica não apenas moldou a política da época, mas também teve implicações profundas e duradouras, desencadeando uma série de eventos que culminariam na emancipação do Brasil.

O capítulo contextualiza a crise dentro de um período em que Portugal já enfrentava dificuldades financeiras há décadas. Desde o final do século XVIII, e especialmente após o início das Guerras Napoleônicas, as finanças do reino estavam em frangalhos. A necessidade de financiar um exército para defender o território e de sustentar uma marinha que pudesse proteger suas rotas comerciais fez com que as despesas militares consumissem uma parcela enorme do orçamento português, frequentemente excedendo 50\% das receitas. Apesar desses gastos, Portugal não conseguiu evitar a invasão francesa em 1807, o que forçou a corte a fugir para o Brasil, transferindo o centro do poder para a América do Sul.

A chegada da família real ao Brasil trouxe um novo conjunto de desafios. A corte portuguesa, agora instalada no Rio de Janeiro, continuou a acumular dívidas enquanto tentava sustentar o luxo e o estilo de vida palaciano em um novo continente. Além disso, D. João VI decidiu abrir uma nova frente de batalha na América do Sul, na tentativa de anexar a região da Cisplatina (atual Uruguai), o que agravou ainda mais a situação financeira. Para financiar essas empreitadas, a coroa recorreu a uma série de medidas desesperadas, incluindo a elevação de impostos, a criação de novos tributos e a emissão descontrolada de papel-moeda. Essas políticas, no entanto, só conseguiram empurrar a economia brasileira para uma espiral inflacionária, com os preços de produtos essenciais, como farinha de mandioca e carne-seca, subindo de forma vertiginosa.

A crise não foi apenas econômica, mas também social e política. A inflação e a escassez de alimentos afetaram tanto os grandes proprietários de terras, que dependiam desses produtos para alimentar seus escravos, quanto a população urbana, incluindo soldados e milícias que garantiam a ordem. A insatisfação era generalizada e se espalhou por várias camadas da sociedade, desde os cortesãos e burocratas até os cidadãos comuns e soldados. Em 1819, a cidade do Rio de Janeiro experimentou uma das maiores crises de abastecimento de sua história, provocando protestos e petições ao rei por medidas de alívio.

A crise fiscal e a insatisfação popular culminaram em uma série de levantes. O primeiro movimento significativo ocorreu no Porto, em Portugal, em 24 de agosto de 1820, onde a população e os militares se insurgiram contra o governo de D. João VI, exigindo a convocação de uma Constituição que limitasse os poderes do rei. Esse movimento rapidamente se espalhou para Lisboa e, eventualmente, atravessou o Atlântico, chegando ao Brasil. As províncias do Pará, Bahia e Rio de Janeiro se uniram ao clamor por uma Constituição, marcando o início do fim do absolutismo e a transição para uma monarquia constitucional.

Os autores destacam que, além da crise econômica, as ideias iluministas e liberais desempenharam um papel crucial na queda do absolutismo. A liberdade de imprensa, conquistada após as revoluções liberais, permitiu a disseminação rápida de ideias revolucionárias. Obras como *O Contrato Social* de Rousseau começaram a circular amplamente, influenciando a opinião pública e fomentando debates sobre os direitos dos cidadãos e os limites do poder real. No entanto, os autores argumentam que, embora as ideias liberais tenham sido fundamentais para moldar a mentalidade da época, elas não teriam sido suficientes para derrubar o absolutismo por si só. Foi a crise fiscal — com suas consequências tangíveis e imediatas, como a falta de pagamento das tropas e a inflação galopante — que forneceu o impulso necessário para que as revoltas ganhassem força e se traduzissem em uma mudança política concreta.

O capítulo conclui que a crise fiscal não apenas precipitou o fim do absolutismo em Portugal, mas também criou as condições para a independência do Brasil em 1822. A crise econômica desestabilizou o governo, gerou insatisfação generalizada e abriu espaço para a emergência de novas forças políticas. Essas forças, impulsionadas tanto por motivações econômicas quanto por ideais iluministas, foram essenciais para a consolidação de um novo arranjo político que culminaria na separação entre Brasil e Portugal e na formação de uma nova ordem constitucional.

Em síntese, o capítulo 1 de *Adeus, Senhor Portugal* oferece uma análise rica e detalhada das interconexões entre crise fiscal, mudança política e o papel das ideias iluministas no processo que levou à independência do Brasil. A abordagem dos autores revela como fatores econômicos, sociais e intelectuais se entrelaçaram para moldar um dos períodos mais críticos da história brasileira.

\subsection{\textbf{Referêcncias Bibliográficas}}
\begin{thebibliography}{99}

\bibitem{Furtado2005_cap1}
FURTADO, Celso. \textbf{Formação Econômica do Brasil}. 34ª ed. São Paulo: Companhia das Letras, 2005.
\textbf{Capítulo 1: Da Expansão Comercial à Empresa Agrícola}.

\bibitem{Furtado2005_cap2}
FURTADO, Celso. \textbf{Formação Econômica do Brasil}. 34ª ed. São Paulo: Companhia das Letras, 2005.
\textbf{Capítulo 2: Fatores do Êxito da Empresa Agrícola}.

\bibitem{Furtado2005_cap3}
FURTADO, Celso. \textbf{Formação Econômica do Brasil}. 34ª ed. São Paulo: Companhia das Letras, 2005.
\textbf{Capítulo 3: Razões do Monopólio}.

\bibitem{Furtado2005_cap14}
FURTADO, Celso. \textbf{Formação Econômica do Brasil}. 34ª ed. São Paulo: Companhia das Letras, 2005.
\textbf{Capítulo 14: Fluxo da Renda}.

\bibitem{Furtado2005_cap15}
FURTADO, Celso. \textbf{Formação Econômica do Brasil}. 34ª ed. São Paulo: Companhia das Letras, 2005.
\textbf{Capítulo 15: Regressão Econômica e Expansão da Área de Subsistência}.

\bibitem{Villela2001}
VILLELA, André A.; SUZIGAN, Wilson. \textbf{O Desenvolvimento Econômico no Brasil: Pré-1945}. 3ª ed. São Paulo: Editora da Unicamp, 2001.
\textbf{Capítulo: Regressão Econômica e Expansão da Área de Subsistência}.

\bibitem{Cariello2020}
CARIELLO, Rafael; PEREIRA, Thales Zamberlam. \textbf{Adeus, Senhor Portugal: Crise do Absolutismo e a Independência do Brasil}. 1ª ed. São Paulo: Companhia das Letras, 2020.
\textbf{Capítulo 1: A Crise Inaugural}.

\end{thebibliography}


\newpage
\section{\textbf{Guia de Estudos}}

\subsection{\textbf{Questão 1: Como podemos justificar o emprego da mão de obra escrava africana ter se tornado a mais importante na etapa colonial?}}

O emprego da mão de obra escrava africana como principal força de trabalho durante a etapa colonial brasileira pode ser justificado por uma combinação de fatores econômicos, sociais e geopolíticos.

Inicialmente, o desenvolvimento da economia açucareira no Brasil, a partir do século XVI, demandava uma força de trabalho intensiva e resistente, capaz de suportar as duras condições das plantações de cana-de-açúcar e engenhos. A população indígena, embora numerosa, não atendia a essa demanda por diversos motivos: muitos indígenas não se adaptaram ao trabalho agrícola intensivo, havia resistência cultural e física, e as doenças trazidas pelos europeus dizimaram grande parte dessa população. Assim, a solução encontrada foi a importação de escravos africanos, que se mostraram mais resistentes às condições adversas e possuíam habilidades agrícolas já desenvolvidas em suas terras de origem.

Outro ponto crucial para a adoção da mão de obra africana foi o estabelecimento do comércio triangular entre Europa, África e América, que facilitou a logística e a viabilidade econômica do tráfico negreiro. Esse comércio envolvia a exportação de produtos manufaturados da Europa para a África, a troca desses produtos por escravos africanos, que eram então transportados para as colônias americanas, e, finalmente, a exportação de produtos coloniais, como açúcar e tabaco, de volta para a Europa. Este sistema consolidou-se como uma das bases econômicas do mercantilismo colonial, beneficiando tanto os traficantes de escravos quanto os proprietários de terras nas colônias.

Adicionalmente, o uso da mão de obra escrava africana está ligado às características institucionais e econômicas das colônias ibéricas. Ao contrário das colônias do norte da América, onde o trabalho livre foi mais predominante, as colônias portuguesas e espanholas eram caracterizadas por uma forte concentração de terras e pela necessidade de uma mão de obra abundante e barata para a produção de commodities agrícolas voltadas para a exportação. A escravidão fornecia uma forma eficiente de controle e exploração dessa mão de obra, garantindo baixos custos de produção e altos lucros para os colonos.

O estudo de Engerman e Sokoloff também sugere que os endowments de fatores, como o clima e a geografia das regiões coloniais, favoreceram o desenvolvimento de economias baseadas em grandes plantações que dependiam da mão de obra escrava. Regiões como o Brasil, com clima adequado para a produção de cana-de-açúcar e outras commodities agrícolas, tornaram-se propícias à formação de sociedades altamente desiguais, onde uma elite de proprietários de terras controlava vastos recursos e uma grande população de escravos.

Portanto, a predominância da mão de obra escrava africana durante a etapa colonial pode ser explicada pela combinação de necessidade econômica, viabilidade logística e padrões institucionais que favoreciam a exploração intensiva de trabalho em economias voltadas para a exportação de produtos agrícolas. Esses fatores contribuíram para a consolidação de um sistema econômico profundamente desigual, cujas consequências sociais e econômicas perduraram por séculos.

\subsection{\textbf{Questão 2: O aumento da concorrência no mercado de açúcar na 2ª metade do século XVII provocou uma situação de decadência no Nordeste. Reflita sobre essa frase (considere levantar evidências para montar sua reflexão)}}

A afirmação de que o aumento da concorrência no mercado de açúcar na 2ª metade do século XVII provocou uma situação de decadência no Nordeste brasileiro é amplamente fundamentada em uma série de fatores econômicos, geopolíticos e sociais que ocorreram naquele período.

Primeiramente, é importante contextualizar que o Brasil, particularmente a região Nordeste, foi durante boa parte do século XVI e início do XVII o maior produtor mundial de açúcar. A economia açucareira nordestina era baseada em grandes engenhos que utilizavam mão de obra escrava africana para produzir açúcar em larga escala, destinado principalmente ao mercado europeu. Esse produto era altamente lucrativo, e o Brasil detinha uma posição quase monopolística no fornecimento de açúcar para a Europa.

Contudo, a partir da 2ª metade do século XVII, a posição dominante do Brasil no mercado mundial de açúcar começou a ser ameaçada pela entrada de novos competidores. As colônias inglesas e francesas no Caribe, como Barbados, Jamaica e São Domingos (atual Haiti), começaram a produzir açúcar de maneira mais eficiente, utilizando técnicas agrícolas avançadas e beneficiando-se de investimentos significativos por parte das metrópoles europeias. Essas novas regiões produtoras tinham condições naturais igualmente favoráveis e começaram a produzir açúcar com custos mais baixos, o que resultou na redução dos preços internacionais do açúcar.

A presença holandesa nas colônias do Caribe, após a sua expulsão do Nordeste brasileiro durante as invasões holandesas (1630-1654), também contribuiu para a intensificação da concorrência. Os holandeses, que haviam adquirido vasto conhecimento e experiência na produção e comércio de açúcar durante o período em que controlaram partes do Nordeste, aplicaram esses conhecimentos nas Antilhas, onde estabeleceram plantações eficientes e altamente produtivas.

Esse aumento na oferta global de açúcar, sem um crescimento correspondente na demanda, levou a uma queda significativa dos preços internacionais. Como resultado, os engenhos brasileiros, que já enfrentavam altos custos de produção devido à infraestrutura menos desenvolvida e à necessidade constante de importar escravos, começaram a sofrer economicamente. A lucratividade das plantações de açúcar no Nordeste diminuiu drasticamente, e muitos engenhos foram forçados a fechar ou reduzir significativamente suas operações.

Além disso, a guerra luso-holandesa e a subsequente reconquista do Nordeste pelo governo português agravaram ainda mais a situação. Durante e após as guerras, a infraestrutura de produção e distribuição foi severamente danificada, o que também contribuiu para a decadência econômica da região.

Em termos sociais, a decadência econômica do setor açucareiro no Nordeste resultou em um processo de empobrecimento das elites locais, que até então dependiam fortemente da produção e exportação de açúcar. A crise econômica também teve impactos profundos na estrutura social da região, agravando as desigualdades e exacerbando a dependência de outras atividades econômicas menos lucrativas, como a pecuária e a agricultura de subsistência.

Dessa forma, o aumento da concorrência no mercado de açúcar na 2ª metade do século XVII, principalmente por parte das colônias caribenhas, combinado com os efeitos das guerras e a falta de modernização dos engenhos brasileiros, provocou uma situação de decadência econômica significativa no Nordeste do Brasil. Essa decadência marcou o início de uma mudança gradual na economia brasileira, com o deslocamento do eixo econômico para outras regiões e outras atividades, como a mineração no século seguinte.

\subsection{\textbf{Questão 3 : Segundo os autores E\&S, a estrutura econômica colonial impactou fortemente a estrutura formada no século XIX. Isto é, os factor endowments foram fundamentais para compreendermos nossa trajetória de desenvolvimento econômico. Explique essa conexão, deixando claro o papel das instituições.}}

De acordo com os autores Stanley L. Engerman e Kenneth L. Sokoloff (E\&S), a estrutura econômica das colônias no Novo Mundo, incluindo o Brasil, foi profundamente influenciada pelos \textit{factor endowments}—isto é, pelos recursos naturais, características geográficas e sociais disponíveis no momento da colonização. Esses \textit{factor endowments} não apenas moldaram a economia colonial, mas também estabeleceram as bases para a trajetória de desenvolvimento econômico ao longo dos séculos seguintes, incluindo o século XIX.

Os \textit{factor endowments} referem-se a elementos como a abundância de terras férteis, o clima favorável para culturas específicas (como o açúcar e o café), e a presença ou ausência de grandes populações indígenas que poderiam ser exploradas. No caso do Brasil, a aptidão para o cultivo de açúcar em larga escala, que era altamente lucrativo no mercado europeu, e a disponibilidade de mão de obra escrava africana foram determinantes para o estabelecimento de uma economia baseada em grandes propriedades agrárias, conhecidas como latifúndios. Essas grandes propriedades geraram uma estrutura social e econômica caracterizada por uma profunda desigualdade na distribuição de riqueza e poder, com uma pequena elite controlando vastos recursos e uma grande população composta por escravos e trabalhadores subalternos.

Os autores argumentam que essa desigualdade inicial, estabelecida durante o período colonial, teve um impacto duradouro na formação das instituições e na trajetória de desenvolvimento econômico das ex-colônias. As instituições políticas e econômicas que surgiram foram, em grande parte, moldadas para manter o status quo, protegendo os interesses da elite proprietária de terras e limitando a participação econômica e política das camadas mais baixas da sociedade. Dessa forma, as instituições serviram para perpetuar a desigualdade, restringindo o acesso a oportunidades econômicas e mantendo uma estrutura de poder concentrada.

Essa continuidade institucional é crucial para entender como os \textit{factor endowments} iniciais influenciaram o desenvolvimento econômico no século XIX. No Brasil, por exemplo, a transição da economia açucareira para a economia cafeeira no século XIX não alterou substancialmente a estrutura social ou a distribuição de poder, pois as novas elites cafeeiras adotaram e reforçaram as mesmas práticas institucionais estabelecidas durante o período colonial. Assim, o desenvolvimento econômico no Brasil continuou a ser caracterizado por um alto grau de concentração de riqueza e poder, e por um crescimento econômico que beneficiava principalmente as elites, enquanto a maior parte da população permanecia marginalizada.

Além disso, as instituições coloniais também afetaram a forma como o Brasil se integrou à economia global durante o século XIX. A dependência de exportações de produtos agrícolas, como açúcar e café, que eram produzidos em grandes propriedades com mão de obra barata, perpetuou uma economia voltada para a exportação e altamente vulnerável às flutuações dos mercados internacionais. Isso limitou a capacidade do Brasil de desenvolver uma base industrial diversificada e autossuficiente, exacerbando as desigualdades econômicas e sociais.

Portanto, segundo E\&S, a conexão entre os \textit{factor endowments} coloniais e a trajetória de desenvolvimento econômico do Brasil no século XIX é clara: os recursos e condições iniciais moldaram uma estrutura econômica e social desigual, que foi perpetuada e reforçada pelas instituições criadas para proteger os interesses da elite. Essas instituições, por sua vez, influenciaram o caminho do desenvolvimento econômico, limitando o potencial de crescimento inclusivo e sustentável.



\end{document}