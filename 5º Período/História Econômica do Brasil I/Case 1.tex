\documentclass[a4paper,12pt]{article}[abntex2]
\bibliographystyle{abntex2-alf}
\usepackage{siunitx} % Fornece suporte para a tipografia de unidades do Sistema Internacional e formatação de números
\usepackage{booktabs} % Melhora a qualidade das tabelas
\usepackage{tabularx} % Permite tabelas com larguras de colunas ajustáveis
\usepackage{graphicx} % Suporte para inclusão de imagens
\usepackage{newtxtext} % Substitui a fonte padrão pela Times Roman
\usepackage{ragged2e} % Justificação de texto melhorada
\usepackage{setspace} % Controle do espaçamento entre linhas
\usepackage[a4paper, left=3.0cm, top=3.0cm, bottom=2.0cm, right=2.0cm]{geometry} % Personalização das margens do documento
\usepackage{lipsum} % Geração de texto dummy 'Lorem Ipsum'
\usepackage{fancyhdr} % Customização de cabeçalhos e rodapés
\usepackage{titlesec} % Personalização dos títulos de seções
\usepackage[portuguese]{babel} % Adaptação para o português (nomes e hifenização
\usepackage{hyperref} % Suporte a hiperlinks
\usepackage{indentfirst} % Indentação do primeiro parágrafo das seções
\sisetup{
  output-decimal-marker = {,},
  inter-unit-product = \ensuremath{{}\cdot{}},
  per-mode = symbol
}
\setlength{\headheight}{14.49998pt}

\DeclareSIUnit{\real}{R\$}
\newcommand{\real}[1]{R\$#1}
\usepackage{float} % Melhor controle sobre o posicionamento de figuras e tabelas
\usepackage{footnotehyper} % Notas de rodapé clicáveis em combinação com hyperref
\hypersetup{
    colorlinks=true,
    linkcolor=black,
    filecolor=magenta,      
    urlcolor=cyan,
    citecolor=black,        
    pdfborder={0 0 0},
}
\usepackage[normalem]{ulem} % Permite o uso de diferentes tipos de sublinhados sem alterar o \emph{}
\makeatletter
\def\@pdfborder{0 0 0} % Remove a borda dos links
\def\@pdfborderstyle{/S/U/W 1} % Estilo da borda dos links
\makeatother
\onehalfspacing

\begin{document}

\begin{titlepage}
    \centering
    \vspace*{1cm}
    \Large\textbf{INSPER – INSTITUTO DE ENSINO E PESQUISA}\\
    \Large ECONOMIA\\
    \vspace{1.5cm}
    \Large\textbf{Estudos do Case 1 - H.E.B}\\
    \vspace{1.5cm}
    Prof. Heleno Piazenini Vieira\\
    Prof. Auxiliar \\
    \vfill
    \normalsize
    Hicham Munir Tayfour, \href{mailto:hichamt@al.insper.edu.br}{hichamt@al.insper.edu.br}\\
    5º Período - Economia A\\
    \vfill
    São Paulo\\
    Agosto/2024
\end{titlepage}

\newpage
\tableofcontents
\thispagestyle{empty} % This command removes the page number from the table of contents page
\newpage
\setcounter{page}{1} % This command sets the page number to start from this page
\justify
\onehalfspacing

\pagestyle{fancy}
\fancyhf{}
\rhead{\thepage}

\section{\textbf{Resumo das partes do texto}}


\subsection*{\textbf{Abstract}}

\addcontentsline{toc}{subsection}{Abstract}
O artigo aborda a questão de por que os Estados Unidos e o Canadá foram mais bem-sucedidos ao longo do tempo em comparação com outras economias do Novo Mundo. Inicialmente, todas as sociedades do Novo Mundo apresentavam altos níveis de produto per capita, mas as trajetórias divergiram significativamente com o tempo. A divergência pode ser atribuída ao crescimento econômico sustentado alcançado pelos EUA e Canadá no século XVIII e início do século XIX, enquanto outros países só conseguiram esse crescimento no final do século XIX ou no século XX.

Os autores destacam que as diferenças substanciais no grau de desigualdade de riqueza, capital humano e poder político são cruciais para entender a variação nas trajetórias de crescimento. Eles sugerem que as raízes dessas disparidades estão nas diferenças dos endowments de fatores iniciais das respectivas colônias. A aptidão para o cultivo de açúcar e outros cultivos que utilizavam intensivamente trabalho escravo, bem como a presença de grandes concentrações de nativos americanos, foram fatores significativos que geraram desigualdade extrema.

Sociedades onde pequenos grupos de descendentes de europeus detinham grandes parcelas de riqueza, capital humano e poder político, dominando economicamente e politicamente a maioria da população, eram comuns nas colônias com essas características. Notavelmente, as colônias britânicas no norte do continente norte-americano não apresentavam essas condições.

Após demonstrar a importância dos endowments de fatores iniciais na geração de grandes diferenças de desigualdade e na estrutura das economias, o artigo chama a atenção para a tendência das políticas governamentais em manter as condições iniciais ou o mesmo grau de desigualdade ao longo do desenvolvimento econômico das respectivas economias. Além disso, explora-se o impacto do grau de desigualdade na evolução das instituições que promovem a participação ampla na economia comercial, mercados e mudanças tecnológicas durante esse período específico. Os autores sugerem que a maior igualdade de riqueza, capital humano e poder político nos EUA e Canadá pode ter predisposto esses países a alcançar o crescimento econômico sustentado mais cedo.

Em suma, os autores argumentam que o papel dos endowments de fatores foi subestimado, e a independência das instituições de desenvolvimento em relação aos endowments de fatores foi exagerada nas teorias sobre as trajetórias de crescimento diferencial entre as economias do Novo Mundo

\subsection*{\textbf{Introduction}}
\addcontentsline{toc}{subsection}{Introduction}

Os historiadores econômicos dos Estados Unidos tradicionalmente atribuem o crescimento econômico do país aos recursos naturais abundantes e à distribuição relativamente igualitária de renda. Este quadro, que remonta a Adam Smith, sugere que o conhecimento generalizado das tecnologias europeias, aliado à abundância de terras e recursos per capita, resultou em alta produtividade marginal do trabalho e, portanto, uma sociedade igualitária com alto padrão de vida e boas perspectivas de crescimento sustentado.

No entanto, surgem questões quando se compara a experiência dos Estados Unidos com as economias da América Latina. Essas sociedades do Novo Mundo também começaram com vastas reservas de terras e recursos naturais per capita e eram bastante prósperas nos séculos XVII e XVIII. Mesmo assim, os EUA e o Canadá se destacaram em termos de crescimento econômico sustentado ao longo do tempo, em contraste com outras colônias do Novo Mundo. Este contraste sugere que os endowments de fatores, por si só, não explicam a diversidade de resultados econômicos.

Para explicar os caminhos divergentes dos EUA e da América Latina, muitos estudiosos mencionam diferenças institucionais, incluindo a democracia, segurança dos direitos de propriedade, propensão ao trabalho árduo ou ao empreendedorismo, cultura e religião. Essas diferenças institucionais são frequentemente relacionadas a variações entre heranças britânicas, espanholas, portuguesas e indígenas. Apesar do reconhecimento de que os endowments de fatores podem influenciar o desenvolvimento econômico e institucional, poucos estudos exploram padrões sistemáticos dessa influência.

Neste artigo, os autores exploram a possibilidade de que o papel dos endowments de fatores tenha sido subestimado e a independência do desenvolvimento institucional em relação a esses endowments exagerada. A análise é inspirada pela observação de que, apesar de começarem com contextos legais e culturais semelhantes, as colônias britânicas no Novo Mundo evoluíram para sociedades e instituições econômicas bastante distintas. Apenas algumas dessas colônias conseguiram crescimento econômico sustentado, enquanto a maioria, que compartilhou certas características dos endowments de fatores com as sociedades latino-americanas, não teve o mesmo sucesso.

Os autores argumentam que uma perspectiva hemisférica sobre as colônias europeias no Novo Mundo indica que os endowments de fatores e as atitudes políticas em relação a eles tiveram impactos profundos e duradouros na estrutura econômica das colônias e em seus caminhos de desenvolvimento a longo prazo. Todas as colônias começaram com uma abundância de terra e outros recursos em relação à mão de obra, mas outros aspectos dos endowments de fatores variaram, contribuindo para diferenças substanciais na distribuição de terras, riqueza e poder político.

As colônias no Caribe, Brasil e sul dos EUA tinham condições climáticas e de solo adequadas para o cultivo de culturas como açúcar, café, arroz, tabaco e algodão, que eram mais eficientes em grandes plantações com trabalho escravo, gerando uma distribuição de riqueza e poder altamente desigual. As colônias espanholas no México e Peru também foram caracterizadas por desigualdade extrema devido à presença de grandes populações indígenas e práticas de concessão de terras e recursos minerais aos membros da elite.

Em contraste, as colônias do norte dos EUA desenvolveram-se com pequenas fazendas familiares, onde as condições climáticas favoreceram uma agricultura diversificada e sem economias de escala na produção, promovendo uma distribuição mais igualitária de riqueza, instituições políticas mais democráticas, mercados internos mais extensos e políticas mais orientadas para o crescimento.

Os autores concluem que as evidências das experiências das colônias do Novo Mundo apoiam a visão de que regiões com circunstâncias e direitos mais igualitários têm maior probabilidade de alcançar crescimento econômico sustentado. Eles também rejeitam o determinismo simples implícito no conceito de "dependência de trajetória", mas apoiam a ideia de que os padrões de crescimento podem ser influenciados pelo caminho traçado.

Esta introdução estabelece a base para a argumentação de que os endowments de fatores iniciais tiveram um papel crucial na moldagem das trajetórias de desenvolvimento econômico das colônias do Novo Mundo.

\subsection*{\textbf{A Brief Sketch of the Growth of the New World Economies}}
\addcontentsline{toc}{subsection}{A Brief Sketch of the Growth of the New World Economies}

A "descoberta" e a exploração das Américas pelos europeus fizeram parte de um esforço grandioso e de longo prazo para explorar oportunidades econômicas em territórios subpovoados ou mal defendidos ao redor do mundo. As nações europeias competiram por reivindicações e começaram a extrair vantagens materiais e outras através de empreendimentos transitórios como expedições e o estabelecimento de assentamentos. Tanto em nível governamental quanto entre agentes privados, surgiram problemas formidáveis de organização devido aos ambientes radicalmente novos e às dificuldades de realizar os massivos fluxos intercontinentais de trabalho e capital.

Comum a todas as colônias do Novo Mundo era a alta produtividade marginal do trabalho, especialmente do trabalho europeu. Uma indicação desse retorno ao trabalho é o fluxo extensivo e sem precedentes de migrantes que cruzaram o Atlântico da Europa e da África para praticamente todas as colônias, apesar do alto custo de transporte. Mais de 60\% dos migrantes eram africanos trazidos involuntariamente como escravos, o que demonstra a predominância do motivo econômico de capturar os ganhos associados à alta produtividade do trabalho. Escravos fluíam para os locais onde sua produtividade atendia aos padrões internacionais, sendo bem-vindos em colônias de todas as potências europeias.

À medida que a migração para o Novo Mundo acelerou, ocorreram várias mudanças salientes na composição e direção do fluxo. A proporção de migrantes escravizados cresceu continuamente, de aproximadamente 20\% antes de 1580 para quase 75\% entre 1700 e 1760. Além disso, houve uma mudança marcante nos números relativos das colônias espanholas, cuja participação no fluxo de migrantes diminuiu continuamente, enquanto o número de novos colonos em colônias portuguesas e francesas cresceu significativamente.

Houve também um aumento na proporção de imigrantes que se estabeleceram em colônias especializadas na produção de açúcar, tabaco, café e alguns outros cultivos básicos para os mercados mundiais. Isso é evidente pelo crescente número de migrantes que foram para as colônias de Portugal, França e Países Baixos, bem como pela predominância quantitativa das colônias britânicas na América do Norte, onde mais de 90\% dos migrantes se dirigiram para as colônias das Índias Ocidentais e do sul do continente. Virtualmente todas essas colônias eram fortemente orientadas para a produção de tais cultivos, atraindo grandes influxos de mão-de-obra, especialmente escravos, devido às condições favoráveis para a produção desses produtos valiosos.

Os dados demográficos mostram que a maioria das populações das economias do Novo Mundo era composta por pessoas de ascendência africana ou indígena até o século XIX. Nas colônias adequadas para o cultivo de açúcar, como Barbados e Brasil, as populações rapidamente se tornaram dominadas por descendentes de africanos trazidos como escravos para trabalhar nas grandes plantações. As colônias espanholas eram predominantemente habitadas por indígenas ou mestiços, refletindo as políticas restritivas de imigração da Espanha.

Em contraste, os territórios que se tornariam os Estados Unidos e o Canadá tinham poucas populações indígenas antes da chegada dos europeus, e a composição de suas populações foi determinada pelos grupos de imigrantes e suas taxas de crescimento natural. Essas colônias absorveram relativamente mais europeus do que escravos africanos, resultando em populações predominantemente brancas. Por volta de 1825, aproximadamente 80\% da população dos Estados Unidos e do Canadá era branca, enquanto no Brasil e nas demais economias do Novo Mundo, a proporção de brancos era inferior a 25\% e 20\%, respectivamente. A prevalência maior de brancos nos Estados Unidos e no Canadá pode ajudar a explicar por que havia menos desigualdade e mais potencial para crescimento econômico nessas economias.

Os padrões de migração, taxas de salários nos mercados de trabalho livre, medidas antropométricas e dados sobre riqueza sugerem que a produtividade do trabalho e a renda para os europeus no Novo Mundo eram altas em comparação aos padrões do Velho Mundo. As colônias especializadas na produção de açúcar, como a Jamaica, geraram tanta riqueza não humana per capita quanto qualquer grupo de colônias no continente norte-americano, refletindo a alta produtividade das plantações de açúcar.

Estudos sugerem que a vantagem em renda per capita dos Estados Unidos (e Canadá) sobre as economias latino-americanas surgiu durante o final do século XVIII e XIX, quando os Estados Unidos começaram a realizar crescimento econômico sustentado bem antes de seus vizinhos no hemisfério. Durante essa era, aqueles de ascendência europeia no México e Barbados estavam em melhor situação do que seus outros homólogos na América do Norte, pois constituíam uma parcela menor da população e suas rendas eram muito mais altas do que as dos nativos americanos ou escravos.

Os economistas tradicionalmente destacam a influência dos endowments de fatores, e os autores argumentam que os Estados Unidos e o Canadá eram relativamente incomuns entre as colônias do Novo Mundo devido a seus endowments de fatores que favoreciam uma distribuição mais igualitária de riqueza e instituições que promoviam a participação ampla da população na atividade comercial. Em contraste, os endowments de fatores das outras colônias do Novo Mundo levaram a distribuições extremamente desiguais de riqueza, capital humano e poder político desde o início de suas histórias, junto com instituições que protegiam as elites.

O artigo distingue três tipos de colônias do Novo Mundo: as colônias açucareiras, as colônias espanholas com grandes populações indígenas e as colônias do norte do continente norte-americano. As colônias açucareiras, como Barbados e Brasil, eram dominadas por grandes plantações escravistas, gerando uma distribuição extremamente desigual de riqueza. As colônias espanholas, como México e Peru, eram caracterizadas por grandes propriedades concedidas a uma elite privilegiada, resultando em desigualdade extrema. As colônias do norte do continente norte-americano, por outro lado, desenvolveram-se com pequenas fazendas familiares, promovendo uma distribuição mais igualitária de riqueza e instituições mais democráticas.

Em resumo, os autores argumentam que os endowments de fatores e as políticas governamentais desempenharam papéis cruciais na formação das trajetórias de desenvolvimento econômico das colônias do Novo Mundo, resultando em diferentes graus de desigualdade e potencial para crescimento econômico sustentado.

\subsection*{\textbf{The Role of Institutions in Shaping Factor Endowment}}
\addcontentsline{toc}{subsection}{The Role of Institutions in Shaping Factor Endowment}

Os autores argumentam que, embora os endowments de fatores iniciais possam ser tratados como exógenos no início da colonização europeia, a influência desses endowments na trajetória de desenvolvimento econômico e institucional é complexa e duradoura. Instituições, uma vez estabelecidas, podem perpetuar e até reforçar as condições iniciais dos endowments de fatores, criando um ciclo que afeta o desenvolvimento econômico a longo prazo.

Políticas governamentais desempenharam um papel crucial na evolução dos endowments de fatores, especialmente aquelas relacionadas à terra, imigração e regulamentação do comércio. Durante a era colonial, as potências europeias seguiram variantes do mercantilismo, com o objetivo de beneficiar a metrópole. Mudanças significativas ocorreram no final do século XVIII e início do século XIX, com as independências das colônias americanas, alterando as políticas para beneficiar os novos estados independentes.

Políticas de Terra
As políticas de distribuição de terra variavam amplamente entre as colônias e tiveram impactos profundos na distribuição de riqueza e poder político. Colônias onde grandes concessões de terra foram feitas a militares, missionários e outros colonos tenderam a desenvolver distribuições mais desiguais de riqueza e poder, em contraste com aquelas onde pequenas propriedades foram disponibilizadas. Essas políticas iniciais influenciaram diretamente a estrutura econômica e social das colônias, afetando seu desenvolvimento futuro.

Políticas de Imigração
As políticas de imigração também variaram significativamente. As colônias britânicas incentivaram a imigração de trabalhadores europeus, inclusive por meio de servidão contratada, resultando em populações brancas mais diversas e amplamente participativas na economia comercial. Em contraste, a imigração espanhola foi estritamente controlada, com um fluxo decrescente de migrantes ao longo do tempo. Essa restrição foi facilitada pela abundância de mão-de-obra indígena nas colônias espanholas, que substituía a necessidade de trabalhadores europeus.

Após as independências, muitas nações latino-americanas adotaram políticas de imigração relativamente livres para atrair novos trabalhadores, principalmente da Europa. No entanto, a maior parte do fluxo migratório transatlântico do século XIX se dirigiu aos Estados Unidos, refletindo tanto o tamanho maior de sua economia quanto as oportunidades percebidas de maior igualdade e disponibilidade de pequenas propriedades.

Comércio e Infraestrutura
O desenvolvimento de infraestrutura de transporte e instituições financeiras foi crucial para o crescimento econômico. Nas colônias britânicas, especialmente nos Estados Unidos, o desenvolvimento de infraestrutura como bancos e sistemas de transporte foi amplamente apoiado pela participação pública e privada, facilitando o crescimento econômico. A igualdade relativa na distribuição de riqueza e poder político incentivou investimentos em infraestrutura que beneficiaram amplamente a população.

Impacto das Instituições
As instituições desenvolvidas nas colônias britânicas, particularmente nos Estados Unidos e Canadá, eram mais favoráveis à participação ampla na economia comercial. O direito de formar corporações, por exemplo, foi amplamente exercido para promover investimentos em transporte, instituições financeiras e manufaturas. A igualdade econômica entre a população branca incentivou um ambiente legal que favorecia a iniciativa privada e a inovação.

A extensão dos mercados e a associação entre igualdade e crescimento econômico são evidentes nos padrões de produtividade e inovação observados no início do século XIX nos Estados Unidos. O acesso a mercados amplos incentivou a especialização, a adoção de novas tecnologias e métodos, e o aumento da produtividade. As altas taxas de invenção e inovação foram fortemente associadas à expansão dos mercados e à participação de uma ampla gama de indivíduos na economia.

Em resumo, os autores argumentam que as instituições, influenciadas pelos endowments de fatores iniciais, desempenharam um papel crucial na determinação das trajetórias de desenvolvimento econômico das colônias do Novo Mundo. As políticas governamentais e a estrutura institucional resultante dos endowments de fatores iniciais perpetuaram desigualdades ou promoveram a igualdade, afetando o crescimento econômico e o desenvolvimento institucional a longo prazo.

\subsection*{\textbf{The Extent of Inequality and the Timing of Industrialization}}
\addcontentsline{toc}{subsection}{The Extent of Inequality and the Timing of Industrialization}

Os autores argumentam que, embora todas as colônias do Novo Mundo oferecessem altos padrões de vida para os europeus, diferenças fundamentais em seus endowments de fatores e nas políticas governamentais resultaram em diferentes graus de desigualdade e trajetórias de crescimento econômico. A maioria dessas economias desenvolveu distribuições extremamente desiguais de riqueza, capital humano e poder político desde o início, mantendo essas desigualdades após a independência. Estados Unidos e Canadá se destacam como excepcionais, com altos padrões de vida e relativa igualdade desde o começo, o que, segundo os autores, não é coincidência em relação ao seu crescimento econômico mais cedo.

A Relação entre Igualdade e Crescimento
A ideia de que a igualdade ou democracia em uma sociedade pode estar associada ao potencial de crescimento econômico não é nova. Aqueles que argumentam que a desigualdade promove o crescimento destacam as taxas mais altas de poupança e investimento pelos ricos, sugerindo que grandes quantidades de capital eram necessárias para o crescimento sustentado. Em contraste, os que defendem a igualdade acreditam que ela estimula o crescimento ao incentivar a evolução de redes extensas de mercados, incluindo o mercado de trabalho, e a comercialização em geral. Isso, por sua vez, leva a processos autossustentáveis de expansão de mercados, uso mais eficaz dos recursos, economias de escala, maior atividade inventiva e acumulação de capital humano, bem como maior especialização dos fatores de produção.

Evidências dos Estados Unidos
Estudos recentes sobre a industrialização precoce nos Estados Unidos apoiam a hipótese de que economias mais igualitárias estavam melhor posicionadas para o crescimento durante os séculos XVIII e XIX. A evidência vem principalmente de investigações sobre as fontes e a natureza do crescimento da produtividade. Pesquisas indicam que a produtividade aumentou substancialmente nos primeiros estágios da industrialização, principalmente devido a mudanças organizacionais, métodos e designs que não exigiam grandes investimentos de capital.

A expansão dos mercados desempenhou um papel crucial na estimulação do crescimento técnico. Em agricultura, por exemplo, fazendas com acesso fácil aos mercados tendiam a se especializar mais, usar recursos de maneira mais intensiva e adotar novos produtos. Na manufatura, empresas próximas a mercados amplos mantinham níveis mais altos de produtividade devido a novos métodos de organização do trabalho, maior inventividade e inovação, e economias de escala.

Patentes e Inovação
Trabalhos com registros de patentes nos Estados Unidos demonstram que o crescimento da atividade inventiva estava fortemente associado à expansão dos mercados. A construção de infraestrutura de transporte, como canais, resultou em grandes aumentos na atividade de patentes. Além disso, a diversidade de ocupações representadas entre os inventores em áreas com maior atividade de patentes sugere que uma ampla gama de indivíduos estava engajada na busca por melhores métodos de produção.

Igualdade e Instituições
A igualdade na distribuição de riqueza, capital humano e poder político nos Estados Unidos não apenas incentivou a formação de mercados amplos, mas também moldou um ambiente legal favorável à iniciativa privada e inovação. O direito de formar corporações foi amplamente exercido para promover investimentos em infraestrutura de transporte, instituições financeiras e manufatura. Isso foi facilitado por leis de incorporação gerais implementadas por muitos governos estaduais, refletindo a oposição ao privilégio e o apoio à iniciativa privada.

Conclusão
Os autores concluem que os endowments de fatores e as instituições desempenharam papéis fundamentais na determinação das trajetórias de desenvolvimento econômico das colônias do Novo Mundo. As colônias com maior igualdade, como os Estados Unidos e o Canadá, estavam mais bem posicionadas para alcançar o crescimento econômico sustentado devido à formação de mercados amplos e à participação de uma ampla gama de indivíduos na economia comercial. Em contraste, colônias com alta desigualdade enfrentaram maiores desafios para o desenvolvimento econômico, resultando em trajetórias de crescimento diferentes e frequentemente mais lentas.

\subsection*{\textbf{Conclusions}}
\addcontentsline{toc}{subsection}{Conclusions}

No capítulo de conclusões, os autores reafirmam que as diferenças significativas nos endowments de fatores iniciais das colônias do Novo Mundo desempenharam um papel crucial na determinação de suas trajetórias de desenvolvimento econômico e institucional. Eles argumentam que essas diferenças levaram a variações substanciais na distribuição de riqueza, capital humano e poder político, resultando em diferentes padrões de crescimento econômico a longo prazo.

Papel dos Endowments de Fatores
Os autores enfatizam que os endowments de fatores, como a aptidão para o cultivo de culturas específicas (por exemplo, açúcar, tabaco) e a presença de grandes populações indígenas, influenciaram profundamente as estruturas econômicas e sociais das colônias. Colônias com condições favoráveis ao cultivo de culturas que utilizavam intensivamente mão-de-obra escrava tendiam a desenvolver economias com grandes plantações e uma distribuição extremamente desigual de riqueza e poder. Em contraste, colônias onde a agricultura era baseada em pequenas propriedades familiares, como no norte dos Estados Unidos, promoveram uma distribuição mais igualitária de riqueza e instituições mais democráticas.

Instituições e Desigualdade
Os autores argumentam que as instituições desenvolvidas em resposta aos endowments de fatores iniciais perpetuaram as condições de desigualdade ou igualdade. Em colônias com grande desigualdade, as instituições frequentemente protegiam os privilégios das elites, restringindo a participação econômica e política da maioria da população. Isso, por sua vez, limitava o potencial de crescimento econômico sustentado. Em contraste, as colônias com maior igualdade desenvolveram instituições que promoviam a participação ampla na economia, incentivando o crescimento econômico.

Crescimento Econômico e Industrialização
As colônias que conseguiram alcançar o crescimento econômico sustentado e a industrialização precoce, como os Estados Unidos e o Canadá, eram caracterizadas por maior igualdade de riqueza e capital humano. Essa igualdade incentivou o desenvolvimento de mercados amplos, a inovação tecnológica e a participação de uma ampla gama de indivíduos na economia comercial. As evidências sugerem que a expansão dos mercados e o acesso a oportunidades econômicas foram fundamentais para o crescimento econômico nesses países.

Implicações para o Desenvolvimento Econômico
Os autores concluem que as lições aprendidas com o estudo das colônias do Novo Mundo têm implicações importantes para o desenvolvimento econômico moderno. Eles sugerem que a promoção de maior igualdade na distribuição de recursos e oportunidades pode ser crucial para alcançar o crescimento econômico sustentado. As políticas governamentais devem considerar os impactos de longo prazo dos endowments de fatores e das instituições na promoção do desenvolvimento econômico.

Reflexões Finais
Os autores ressaltam a importância de um entendimento abrangente dos endowments de fatores e das instituições no estudo das trajetórias de desenvolvimento econômico. Eles argumentam que a análise histórica das colônias do Novo Mundo oferece insights valiosos sobre os fatores que influenciam o crescimento econômico e o desenvolvimento institucional. Em última análise, o estudo destaca a complexidade das interações entre recursos naturais, políticas governamentais e instituições na formação dos destinos econômicos das nações.

\newpage
\section{\textbf{Relações com os primeiros três capítulos do livro Formação Econômica do Brasil (Celso Furtado)}}

\subsection*{\textbf{Capítulo 1: Da expansão comercial à empresa agrícola}}
\addcontentsline{toc}{subsection}{Capítulo 1: Da expansão comercial à empresa agrícola}

\subsubsection*{Contexto Econômico da Colonização}

No Capítulo 1, Furtado argumenta que a ocupação econômica das terras americanas foi uma extensão da expansão comercial europeia, impulsionada pela busca por novas rotas e mercados devido às dificuldades impostas pelo controle otomano das rotas tradicionais para o Oriente. Ele destaca que a descoberta das Américas foi inicialmente vista como um episódio secundário, especialmente por Portugal, que estava mais focado em suas atividades no Oriente.

Por outro lado, Engerman e Sokoloff discutem como os diferentes \textit{endowments} (disponibilidade de recursos naturais, demografia, etc.) e as instituições moldaram os caminhos de desenvolvimento das economias coloniais nas Américas. Eles argumentam que as condições naturais influenciaram fortemente a estrutura econômica das colônias, levando a diferentes trajetórias de desenvolvimento. No caso do Brasil, a abundância de terras férteis e o clima favorável ao cultivo de açúcar foram determinantes para a adoção de uma economia baseada em plantações e trabalho escravo.

Há uma concordância entre os textos de Furtado e de Engerman e Sokoloff sobre o fato de que a colonização e a exploração econômica das Américas foram profundamente influenciadas por fatores externos, sejam eles a necessidade de novas rotas comerciais (como menciona Furtado) ou as condições naturais e institucionais locais (como discutido por Engerman e Sokoloff). Furtado, no entanto, foca mais no contexto europeu e na decisão de Portugal de ocupar o Brasil, enquanto Engerman e Sokoloff exploram como as condições internas das colônias moldaram seus desenvolvimentos econômicos.

\subsubsection*{A Introdução do Cultivo de Açúcar}

Furtado detalha como a introdução da produção de açúcar no Brasil foi precedida por experimentos bem-sucedidos em ilhas atlânticas como Madeira e Açores. Ele argumenta que a empresa açucareira só se tornou viável devido à experiência técnica e comercial acumulada pelos portugueses, bem como pela criação de um mercado europeu para o açúcar.

Engerman e Sokoloff discutem como as plantações de açúcar, especialmente em áreas como o Brasil, estavam intimamente ligadas a fatores como a disponibilidade de terras e a instituição da escravidão. Eles sugerem que as características econômicas das plantações de açúcar, como a alta rentabilidade e a dependência do trabalho escravo, foram determinantes para o desenvolvimento dessas economias coloniais.

Há uma concordância entre Furtado e Engerman e Sokoloff sobre a importância do cultivo de açúcar para a economia colonial brasileira. Furtado fornece uma visão histórica detalhada de como o Brasil se tornou um grande produtor de açúcar, enfatizando a importância da experiência anterior e dos mercados europeus. Engerman e Sokoloff, por sua vez, colocam essa atividade dentro de um contexto mais amplo de como as plantações de açúcar moldaram as instituições e o desenvolvimento econômico das colônias.

\subsubsection*{Impacto da Escravidão}

Furtado descreve a escravidão como uma condição necessária para a sobrevivência da economia açucareira no Brasil. Ele aponta que, sem a disponibilidade de mão de obra escrava, a empresa açucareira provavelmente não teria prosperado, e Portugal teria dificuldades em justificar a defesa e a ocupação de suas colônias na América.

O artigo de Engerman e Sokoloff argumenta que a escolha de utilizar trabalho escravo nas plantações foi uma decisão influenciada pelas condições econômicas e sociais. Eles mostram que em regiões onde o cultivo de produtos como açúcar era viável, a escravidão tornou-se a forma predominante de trabalho, contribuindo para uma grande desigualdade de renda e concentração de poder nas mãos de uma elite proprietária de terras.

Ambos os textos concordam que a escravidão foi central para o sucesso da economia açucareira no Brasil. Furtado oferece uma explicação histórica e pragmática sobre a necessidade de mão de obra escrava, enquanto Engerman e Sokoloff fornecem uma análise mais teórica sobre como a escravidão se encaixava no contexto econômico das plantações.

\subsubsection*{Expansão e Consolidação do Comércio}

No capítulo, Furtado destaca que a expansão do comércio de açúcar envolveu parcerias estratégicas com mercadores flamengos, que refinavam e distribuíam o açúcar em toda a Europa. Ele argumenta que essas parcerias foram essenciais para o sucesso comercial da empresa açucareira brasileira.

Engerman e Sokoloff exploram como as condições comerciais e as instituições moldaram as diferentes trajetórias econômicas das colônias. Eles argumentam que as colônias que conseguiram integrar suas economias aos mercados internacionais, como o Brasil, prosperaram mais do que aquelas que ficaram isoladas ou dependentes de um único produto ou mercado.

Ambos os textos concordam que a integração ao mercado internacional foi crucial para o sucesso econômico das colônias. Furtado foca nas parcerias comerciais específicas que ajudaram a consolidar a economia açucareira brasileira, enquanto Engerman e Sokoloff discutem o papel mais amplo das instituições e do comércio na determinação do sucesso econômico das colônias.

\subsubsection*{Conclusão}

As relações entre o Capítulo 1 de Celso Furtado e o artigo de Engerman e Sokoloff mostram uma convergência significativa em vários pontos, especialmente na importância dos fatores externos (como mercados e tecnologia) e internos (como a disponibilidade de recursos e a escravidão) na formação da economia colonial brasileira. No entanto, enquanto Furtado fornece uma narrativa histórica rica em detalhes específicos, Engerman e Sokoloff adotam uma abordagem mais analítica, explorando as variáveis que influenciaram o desenvolvimento econômico das colônias americanas em um sentido mais amplo.

Essas análises podem ajudar a entender como diferentes perspectivas históricas e econômicas se complementam na explicação do desenvolvimento das economias coloniais no Novo Mundo.

\subsection*{\textbf{Capítulo 2: Fatores do êxito da empresa agrícola}}
\addcontentsline{toc}{subsection}{Capítulo 2: Fatores do êxito da empresa agrícola}

\subsubsection*{Importância da Experiência Prévia em Produção de Açúcar}

No Capítulo 2, Furtado argumenta que a experiência prévia dos portugueses nas ilhas atlânticas, como Madeira e Açores, foi essencial para o êxito da produção açucareira no Brasil. Ele destaca que essa experiência foi crucial não apenas para resolver problemas técnicos de produção, mas também para fomentar o desenvolvimento da indústria de equipamentos para engenhos açucareiros em Portugal. Sem esses avanços técnicos, o sucesso da empresa açucareira no Brasil teria sido muito mais difícil.

Engerman e Sokoloff, por sua vez, discutem no artigo como as condições naturais e os "endowments" iniciais, como a aptidão para o cultivo de açúcar, moldaram o desenvolvimento econômico das colônias americanas. Eles argumentam que a viabilidade econômica do cultivo de açúcar, associado ao uso intensivo de trabalho escravo, foi determinante para o estabelecimento de economias coloniais baseadas em grandes plantações.

Ambos os textos concordam sobre a importância das condições técnicas e naturais no sucesso da empresa agrícola no Brasil. Enquanto Furtado enfatiza o papel da experiência acumulada pelos portugueses, Engerman e Sokoloff focam mais na influência dos "endowments" e das condições locais na formação de economias baseadas na produção açucareira.

\subsubsection*{Integração Comercial e Parcerias Internacionais}

Furtado descreve como a integração do comércio de açúcar português nos mercados europeus, inicialmente por meio de canais controlados por comerciantes italianos e posteriormente em parceria com mercadores flamengos, foi vital para o sucesso econômico da empresa açucareira. Ele aponta que essas parcerias, especialmente com os flamengos, foram fundamentais para expandir a distribuição de açúcar por toda a Europa.

No artigo de Engerman e Sokoloff, é destacado como as condições institucionais e as parcerias comerciais desempenharam um papel crucial na prosperidade de certas colônias, como o Brasil. Eles sugerem que colônias que conseguiram integrar suas economias aos mercados internacionais, muitas vezes através de tais parcerias, foram mais bem-sucedidas economicamente.

A relação entre os textos mostra uma concordância na importância da integração comercial para o êxito das economias coloniais. Furtado fornece uma narrativa detalhada das parcerias comerciais específicas que impulsionaram o sucesso da produção açucareira no Brasil, enquanto Engerman e Sokoloff abordam o impacto mais amplo das instituições e das relações comerciais no desenvolvimento econômico colonial.

\subsubsection*{Solução do Problema de Mão de Obra}

Furtado também discute a importância da solução do problema de mão de obra para o sucesso da empresa agrícola brasileira. Ele afirma que a importação de mão de obra escrava africana foi a solução encontrada pelos portugueses para viabilizar economicamente a produção em larga escala de açúcar, dado que a importação de mão de obra europeia seria inviável.

Engerman e Sokoloff enfatizam em seu artigo como a disponibilidade e o uso de trabalho escravo em plantações de açúcar foram cruciais para o desenvolvimento econômico das colônias que adotaram este sistema. Eles argumentam que o uso de mão de obra escrava levou a uma concentração significativa de riqueza e poder, moldando profundamente as instituições e o desenvolvimento dessas economias.

Há uma concordância clara entre os dois textos sobre a importância da mão de obra escrava para a economia açucareira no Brasil. Furtado oferece uma visão histórica e pragmática sobre a necessidade dessa mão de obra, enquanto Engerman e Sokoloff fornecem uma análise mais teórica sobre como a escravidão influenciou a estrutura econômica e social das colônias.

\subsubsection*{Financiamento e Apoio Comercial}

Furtado menciona que o financiamento da produção açucareira foi viabilizado por grupos financeiros poderosos, especialmente os holandeses, que não apenas financiaram a refinação e comercialização do açúcar, mas também participaram do financiamento das instalações produtivas no Brasil e da importação de mão de obra escrava. Ele argumenta que o sucesso econômico inicial garantiu a continuidade desse financiamento, o que foi crucial para a expansão da produção.

Engerman e Sokoloff discutem como o acesso a capitais e a instituições comerciais bem desenvolvidas foi um fator determinante para o sucesso econômico de certas colônias. Eles sugerem que o apoio financeiro e comercial que algumas colônias receberam permitiu que essas economias prosperassem e se expandissem.

A relação entre os textos novamente mostra uma convergência na importância do financiamento e do apoio comercial. Furtado detalha como esse apoio foi crucial para o desenvolvimento da economia açucareira brasileira, enquanto Engerman e Sokoloff analisam a importância mais ampla do acesso a capitais e instituições comerciais para o sucesso das colônias.

\subsubsection*{Conclusão}

As relações entre o Capítulo 2 de Celso Furtado e o artigo de Engerman e Sokoloff revelam uma concordância significativa em vários aspectos cruciais para o sucesso da empresa agrícola no Brasil, particularmente no que diz respeito à importância da experiência técnica, integração comercial, solução de problemas de mão de obra e financiamento. Enquanto Furtado fornece uma narrativa detalhada e histórica dessas questões, Engerman e Sokoloff oferecem uma análise mais ampla e teórica que explora como esses fatores contribuíram para o desenvolvimento econômico das colônias do Novo Mundo.

\subsection*{\textbf{Capítulo 3: Razões do monopólio}}
\addcontentsline{toc}{subsection}{Capítulo 3: Razões do monopólio}

\subsubsection*{Estrutura Econômica e Política das Colônias Ibéricas}

No Capítulo 3, Furtado explora como a estrutura econômica e política das colônias ibéricas, especialmente a política de exploração da Espanha, foi centralizada na extração de metais preciosos, enquanto Portugal desenvolveu um modelo baseado na produção agrícola, particularmente de açúcar. Ele destaca que a Espanha, ao concentrar-se na extração mineral, criou uma economia autossuficiente nas colônias, mas que foi fortemente afetada pela crise econômica e pela inflação na metrópole.

Engerman e Sokoloff, por sua vez, discutem em seu artigo como as condições naturais e os "endowments" moldaram as economias das colônias. Eles argumentam que a abundância de recursos minerais na América Espanhola levou à formação de uma economia baseada na exploração intensiva desses recursos, enquanto colônias como o Brasil, com aptidão para a agricultura, seguiram caminhos econômicos diferentes, focando em produtos como o açúcar e utilizando mão de obra escrava.

A análise comparativa entre os textos de Furtado e de Engerman e Sokoloff mostra uma concordância na descrição das diferentes estruturas econômicas das colônias ibéricas. Furtado fornece uma perspectiva histórica detalhada sobre as consequências dessas escolhas econômicas, enquanto Engerman e Sokoloff oferecem uma análise teórica sobre como essas escolhas foram influenciadas pelas condições locais.

\subsubsection*{O Papel das Instituições e do Monopólio Comercial}

Furtado discute como as políticas monopolistas de Portugal, sustentadas pela cooperação com comerciantes holandeses, foram fundamentais para o sucesso da economia açucareira no Brasil. Ele argumenta que, sem esse monopólio, a posição de Portugal no mercado de açúcar teria sido muito mais vulnerável, especialmente diante da concorrência de outras potências coloniais.

No artigo de Engerman e Sokoloff, é discutido como as instituições moldaram as trajetórias de desenvolvimento das colônias americanas. Eles sugerem que colônias com instituições que favoreceram monopólios e grandes plantações, como o Brasil, desenvolveram economias baseadas em alta desigualdade e concentração de riqueza, mas também em grande eficiência econômica na produção de commodities como o açúcar.

Há uma concordância entre os textos sobre a importância das instituições e do monopólio comercial na formação da economia colonial. Furtado fornece uma narrativa histórica do papel central que o monopólio desempenhou no sucesso econômico de Portugal, enquanto Engerman e Sokoloff exploram o impacto dessas instituições na estrutura econômica e social das colônias.

\subsubsection*{Impacto da Decadência Econômica Espanhola}

Furtado analisa como a decadência econômica da Espanha, devido à má gestão dos recursos minerais e à inflação crônica, abriu espaço para que Portugal consolidasse sua posição no mercado de açúcar. Ele argumenta que, se a Espanha tivesse desenvolvido uma economia agrícola mais diversificada, o monopólio português do açúcar poderia ter sido desafiado.

Engerman e Sokoloff também tocam na questão de como as diferentes estratégias econômicas e as instituições moldaram o desenvolvimento das colônias. Eles observam que colônias que se concentraram exclusivamente na extração mineral, como as espanholas, não conseguiram desenvolver economias diversificadas, o que levou a crises econômicas e à estagnação.

A relação entre os textos revela um entendimento comum de que a falta de diversificação econômica e a má gestão dos recursos podem levar à decadência econômica, conforme discutido por Furtado e apoiado pela análise de Engerman e Sokoloff sobre as colônias espanholas.

\subsubsection*{Consequências do Monopólio e da Cooperação Internacional}

Furtado enfatiza que a manutenção do monopólio açucareiro português foi possível graças à cooperação estratégica com os holandeses, que controlavam a distribuição de açúcar na Europa. Ele alerta que essa dependência da cooperação internacional também criou vulnerabilidades, que se manifestaram quando os holandeses começaram a desenvolver a produção açucareira nas colônias do Caribe.

Engerman e Sokoloff discutem como a cooperação e a integração ao comércio internacional foram cruciais para o desenvolvimento das economias coloniais. Eles sugerem que, embora essa integração tenha trazido grandes benefícios econômicos, também criou dependências que, em momentos de crise, poderiam desestabilizar essas economias.

A análise comparativa mostra uma convergência nos textos sobre o papel crucial da cooperação internacional e do monopólio no sucesso inicial das economias coloniais, mas também sobre as vulnerabilidades que esses fatores podiam introduzir a longo prazo.

\subsubsection*{Conclusão}

As relações entre o Capítulo 3 de Celso Furtado e o artigo de Engerman e Sokoloff revelam uma concordância significativa sobre a importância das instituições, do monopólio comercial e da cooperação internacional no desenvolvimento econômico das colônias ibéricas. Enquanto Furtado oferece uma narrativa detalhada dos impactos históricos dessas estratégias, Engerman e Sokoloff fornecem uma análise teórica que contextualiza essas decisões dentro de um quadro mais amplo de desenvolvimento econômico nas Américas.

\newpage
\section{\textbf{Respondendo as perguntas do Guia de Discussão}}

\subsection{\textbf{Questão 1 : A partir da leitura do texto sugerido: o que são "factor endowments"? Por que, segundo os autores desse texto, eles são importantes para entendermos nossa formação econômica?
}}

No artigo de Engerman e Sokoloff, "factor endowments" referem-se aos recursos naturais, características geográficas e demográficas iniciais que uma região ou colônia possui ao longo do tempo. Estes incluem fatores como o tipo de solo, clima, recursos minerais, a abundância ou escassez de mão de obra, e a presença de populações indígenas.

Segundo os autores, os "factor endowments" são fundamentais para entender a formação econômica de diferentes regiões do Novo Mundo porque eles influenciam diretamente a estrutura econômica que as colônias adotaram. Por exemplo, regiões com solos férteis e clima adequado para o cultivo de produtos como açúcar e café, aliadas à disponibilidade de mão de obra (muitas vezes escrava), desenvolveram economias baseadas em grandes plantações com alta concentração de riqueza e poder nas mãos de poucos.

Esses fatores iniciais não apenas determinaram a orientação econômica das colônias, mas também moldaram suas instituições políticas e sociais, criando diferentes trajetórias de desenvolvimento ao longo do tempo. As regiões que começaram com "factor endowments" que favoreciam a produção em larga escala e o uso de mão de obra escrava tendiam a desenvolver economias com maior desigualdade de riqueza e poder político, o que impactou o crescimento econômico sustentado dessas regiões.

Portanto, entender os "factor endowments" permite compreender como as condições iniciais e as escolhas institucionais associadas a elas influenciaram a formação econômica de uma região, incluindo as desigualdades que persistem até os dias atuais.

\subsection{\textbf{Questão 2 : Qual foi a herança econômica colonial brasileira mais importante? Por que essa herança prejudicou o nosso desenvolvimento econômico, segundo os autores do texto?
}}

Segundo Engerman e Sokoloff, a herança econômica colonial mais importante do Brasil foi o sistema de grandes plantações, que dependia intensamente do uso de mão de obra escrava. Esse sistema foi uma consequência direta dos "factor endowments" do Brasil, que incluíam terras férteis e um clima propício para o cultivo de produtos agrícolas como o açúcar.

Essa herança prejudicou o desenvolvimento econômico do Brasil de várias maneiras. Primeiro, o sistema de grandes plantações gerou uma concentração extrema de riqueza e poder nas mãos de uma pequena elite. Isso resultou em uma sociedade profundamente desigual, onde a maioria da população tinha acesso limitado a recursos econômicos e oportunidades. Segundo, a dependência de mão de obra escrava impediu o desenvolvimento de um mercado interno robusto, uma vez que grande parte da população não tinha poder de compra significativo.

Além disso, as instituições políticas e econômicas que emergiram desse sistema eram voltadas para manter essa estrutura desigual, com pouca atenção para a inclusão social ou o desenvolvimento de setores econômicos diversificados. Como resultado, o Brasil experimentou um crescimento econômico mais lento e uma menor inovação tecnológica em comparação com países que desenvolveram economias mais inclusivas e igualitárias.

Assim, os autores argumentam que essa herança colonial criou barreiras duradouras para o crescimento econômico sustentável e equitativo, influenciando negativamente o desenvolvimento econômico do Brasil ao longo dos séculos.

\subsection{\textbf{Questão 3 : A atividade econômica desenvolvida no Brasil ao longo do século XVIII estimulou um grande movimento imigratório de portugueses. De acordo com o texto de Engerman e Sokoloff, seria esperada uma mudança na trajetória de desenvolvimento brasileiro a partir desse movimento migratório? Por que esse movimento não provocou tal mudança? 
}}

De acordo com Engerman e Sokoloff, seria razoável esperar que um grande movimento imigratório pudesse alterar a trajetória de desenvolvimento de uma colônia, particularmente se esses imigrantes trouxessem novas habilidades, conhecimentos e dinâmicas sociais. No entanto, no caso do Brasil, o movimento migratório de portugueses ao longo do século XVIII não resultou em uma mudança significativa na trajetória de desenvolvimento do país.

Isso ocorreu porque, segundo os autores, as instituições e a estrutura econômica já estavam firmemente estabelecidas com base na economia de plantation, que dependia fortemente do trabalho escravo e da produção de commodities agrícolas para exportação. Mesmo com a chegada de novos imigrantes, esses padrões econômicos e sociais profundamente enraizados mantiveram-se intactos, restringindo o potencial de transformação econômica.

Os imigrantes portugueses que chegaram ao Brasil acabaram sendo assimilados na estrutura social e econômica existente, que era caracterizada por uma grande desigualdade de riqueza e poder. A elite colonial, interessada em preservar seu controle sobre os recursos e a mão de obra, não incentivou mudanças nas instituições ou nas práticas econômicas que pudessem democratizar a economia ou promover um desenvolvimento mais inclusivo.

Assim, embora o movimento migratório tenha trazido um aumento na população e potencialmente mais capital humano, a falta de mudanças nas instituições fundamentais e na estrutura econômica significou que o Brasil continuou em um caminho de desenvolvimento caracterizado pela desigualdade e pela dependência de exportações de commodities, limitando as possibilidades de crescimento sustentável e diversificado.

\end{document}