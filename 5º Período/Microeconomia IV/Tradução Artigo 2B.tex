\documentclass[a4paper,12pt]{article}[abntex2]
\bibliographystyle{abntex2-alf}
\usepackage{siunitx} % Fornece suporte para a tipografia de unidades do Sistema Internacional e formatação de números
\usepackage{booktabs} % Melhora a qualidade das tabelas
\usepackage{tabularx} % Permite tabelas com larguras de colunas ajustáveis
\usepackage{graphicx} % Suporte para inclusão de imagens
\usepackage{newtxtext} % Substitui a fonte padrão pela Times Roman
\usepackage{ragged2e} % Justificação de texto melhorada
\usepackage{setspace} % Controle do espaçamento entre linhas
\usepackage[a4paper, left=3.0cm, top=3.0cm, bottom=2.0cm, right=2.0cm]{geometry} % Personalização das margens do documento
\usepackage{lipsum} % Geração de texto dummy 'Lorem Ipsum'
\usepackage{fancyhdr} % Customização de cabeçalhos e rodapés
\usepackage{titlesec} % Personalização dos títulos de seções
\usepackage[portuguese]{babel} % Adaptação para o português (nomes e hifenização
\usepackage{hyperref} % Suporte a hiperlinks
\usepackage{indentfirst} % Indentação do primeiro parágrafo das seções
\sisetup{
  output-decimal-marker = {,},
  inter-unit-product = \ensuremath{{}\cdot{}},
  per-mode = symbol
}
\setlength{\headheight}{14.49998pt}

\DeclareSIUnit{\real}{R\$}
\newcommand{\real}[1]{R\$#1}
\usepackage{float} % Melhor controle sobre o posicionamento de figuras e tabelas
\usepackage{footnotehyper} % Notas de rodapé clicáveis em combinação com hyperref
\hypersetup{
    colorlinks=true,
    linkcolor=black,
    filecolor=magenta,      
    urlcolor=cyan,
    citecolor=black,        
    pdfborder={0 0 0},
}
\usepackage[normalem]{ulem} % Permite o uso de diferentes tipos de sublinhados sem alterar o \emph{}
\makeatletter
\def\@pdfborder{0 0 0} % Remove a borda dos links
\def\@pdfborderstyle{/S/U/W 1} % Estilo da borda dos links
\makeatother
\onehalfspacing

\begin{document}

\begin{titlepage}
    \centering
    \vspace*{1cm}
    \Large\textbf{INSPER – INSTITUTO DE ENSINO E PESQUISA}\\
    \Large ECONOMIA\\
    \vspace{1.5cm}
    \Large\textbf{Tradução do Artigo : Gender wage gaps and risky vs. secure employment: An experimental analysis}\\
    \vspace{1.5cm}
    Prof. Adriano Dutra Teixeira\\
    Prof. Luiz Felipe Campos Fontes\\
    Prof. Auxiliar Pedro Picchetti\\
    \vfill
    \normalsize
    Arthur Di Croce Wohnrath, \href{mailto:arthurcw1@al.insper.edu.br}{arthurcw1@al.insper.edu.br}\\
Erik Hund Bettamio Guimarães, \href{mailto:erikhbg@al.insper.edu.br}{erikhbg@al.insper.edu.br}\\
Érika Kaori Fuzisaka, \href{mailto:erikakf1@al.insper.edu.br}{erikakf1@al.insper.edu.br}\\
Hicham Munir Tayfour, \href{mailto:hichamt@al.insper.edu.br}{hichamt@al.insper.edu.br}\\
Lucas Batista Ferreira, \href{mailto:lucasbf1@al.insper.edu.br}{lucasbf1@al.insper.edu.br}\\
Sarah de Araújo Nascimento Silva, \href{mailto:sarahans@al.insper.edu.br}{sarahans@al.insper.edu.br}

    5º Período - Economia A\\
    \vfill
    São Paulo\\
    Agosto/2024
\end{titlepage}

\newpage
\tableofcontents
\thispagestyle{empty} % This command removes the page number from the table of contents page
\newpage
\setcounter{page}{1} % This command sets the page number to start from this page
\justify
\onehalfspacing

\pagestyle{fancy}
\fancyhf{}
\rhead{\thepage}

\section*{\textbf{Abstract}}
\addcontentsline{toc}{section}{Abstract}

Além da discriminação, do poder de mercado e do capital humano, as diferenças de gênero nas preferências por risco também podem contribuir para as disparidades salariais observadas entre os gêneros. Conduzimos experimentos de laboratório nos quais os participantes escolhem entre um emprego arriscado (em termos de exposição ao desemprego) e um emprego seguro, após serem designados, em rodadas iniciais, para ambos os tipos de empregos. Ambos os empregos envolvem a mesma tarefa de digitação. O emprego arriscado adiciona o elemento de uma probabilidade conhecida de que a oportunidade de digitação não estará disponível em um determinado período. Os participantes foram informados sobre o prêmio de risco exógeno oferecido para o emprego arriscado. As mulheres foram mais propensas do que os homens a escolher o emprego seguro, e essas escolhas explicaram entre 40\% e 77\% da diferença salarial de gênero nos experimentos. Um método para classificar os participantes de acordo com as preferências por risco é derivado do modelo teórico e demonstra ainda a maior incidência de aversão ao risco entre as mulheres.

\section{\textbf{Introduction}}

As disparidades salariais de gênero têm sido o foco de um grande número de estudos empíricos, principalmente baseados em dados de campo gerados por mercados de trabalho naturais. A divisão conceitual fundamental da diferença salarial de gênero é entre discriminação e capital humano. A discriminação pode surgir de três fontes distintas: preferências e gostos de Becker, poder de mercado e discriminação estatística. Consistente com todas as três teorias da discriminação está uma explicação de capital humano para as diferenças de gênero nas dotações de produtividade. As diferenças de gênero nos resultados ocupacionais podem claramente contribuir para a diferença salarial de gênero. Parte desse componente da diferença salarial pode surgir como resultado da segregação ocupacional induzida por gostos por discriminação por agentes econômicos (Baldwin et al., 2001; Shatnawi et al., 2014). O restante da diferença nos resultados ocupacionais pode resultar das diferenças de gênero nas preferências por vários atributos do trabalho que estão associados a diferenciais compensatórios.

Um atributo de trabalho potencialmente saliente é o risco de desemprego involuntário. Pesquisas anteriores revelaram que as taxas de perda involuntária de emprego são significativamente maiores entre os empregados do sexo masculino do que entre as suas contrapartes femininas (Blau e Kahn, 1981). Wilkins e Wooden (2013) argumentam que esse fenômeno decorre das diferenças sistemáticas nos tipos de ocupações que homens e mulheres escolhem. Dependendo da distribuição das atitudes de risco em relação aos períodos de desemprego, haverá algum diferencial compensatório que surgirá no mercado de trabalho. O grau em que homens e mulheres se diferenciam ao se posicionarem em empregos arriscados versus seguros tem implicações para a diferença salarial de gênero. Um estudo recente examinou essa questão no contexto do emprego no setor público versus privado e a diferença salarial de gênero (Jung, 2017). Infelizmente, no mercado de trabalho natural, há uma série de fatores que podem se confundir com a aversão ao risco, dada a natureza multidimensional do pacote de trabalho, como políticas de conciliação entre trabalho e vida familiar, distância de deslocamento, etc.

Vários estudos anteriores utilizando dados de campo com o método de decomposição sugerem que existem diferenças salariais não explicadas, ou seja, Bayard et al. (2003); Hotchkiss e Pitts (2007); Oaxaca (1973), etc. Por outro lado, uma grande quantidade de pesquisas experimentais fornece evidências de que as mulheres são mais avessas ao risco do que os homens em uma variedade de contextos, como a valoração de apostas e/ou escolhas entre apostas (Hartog et al. (2002); Levin et al. (1988)) e decisões arriscadas em ambientes contextuais, como investimento e seguros (Gysler et al. (2002); Schubert (1999)). No entanto, não há uma abundância de evidências que mostrem o vínculo direto entre as diferenças de gênero nas atitudes em relação ao risco e as disparidades salariais.

Até onde sabemos, este estudo é o primeiro a usar o laboratório para identificar o papel potencial da aversão ao risco na explicação das disparidades salariais de gênero em um cenário em que podemos abstrair de uma miríade de fatores normalmente presentes nos mercados de trabalho de campo, incluindo discriminação no mercado de trabalho e aversão à competição. São realizados experimentos nos quais os participantes têm a oportunidade de escolher entre duas tarefas de digitação diferenciadas apenas pela perspectiva de períodos exógenos de desemprego. A tarefa arriscada é acompanhada por um prêmio salarial. As disparidades de gênero que surgem em nosso desenho experimental podem resultar apenas de diferenças de gênero no desempenho de digitação e nas escolhas de emprego. As mulheres eram mais propensas do que os homens a escolher o emprego seguro, e a análise de decomposição salarial revela que essas escolhas de emprego explicaram entre 40\% e 77\% da diferença salarial de gênero nos experimentos.

É claro que os experimentos de laboratório descritos neste artigo não pretendem refletir parâmetros populacionais no mercado de trabalho natural. Em vez disso, a ideia por trás da tarefa de digitação e da decomposição é motivar melhor a relevância do experimento para o mercado de trabalho natural. Com esse objetivo, implementamos um desenho que permite possíveis diferenças de gênero nas características determinantes do salário, bem como diferenças de gênero nas escolhas de emprego, onde há um elemento de risco de ganho e um prêmio salarial por assumir esse risco.


\section{\textbf{Literature}}

O reconhecimento por parte dos economistas da associação entre salários e características dos empregos tem uma longa história. Adam Smith argumentou em "A Riqueza das Nações" que os salários poderiam ser determinados por diferentes características dos empregos, como o risco (Smith, 1776). Desde o tempo de Adam Smith, a teoria dos diferenciais compensatórios de salários tem sido amplamente estudada. Murphy et al. (1987) e Moore (1995) mostram que setores de emprego com maior desemprego e maior risco tendem a ter salários mais altos. Assim, as decisões de alocação de empregos podem variar conforme as atitudes dos indivíduos em relação ao risco. Trabalhos mais recentes, como Hartog et al. (2003), também mostram que empregos com maior risco exigem salários mais altos, contribuindo para a teoria dos diferenciais compensatórios de salários. Trabalhadores que estão mais dispostos a aceitar uma certa quantia de dinheiro por um dado aumento no risco são mais propensos a escolher trabalhar em empregos mais arriscados do que aqueles que são menos inclinados a fazer uma troca entre salários e risco. Embora a escolha do setor de trabalho seja sensível às diferenças nas atitudes em relação ao risco, a priori, também está fortemente correlacionada com as decisões educacionais.

Dependendo do grau de aversão ao risco de um indivíduo, trabalhadores avessos ao risco valorizam mais a estabilidade no emprego, enquanto outros, que são menos avessos ao risco, podem preferir trocar a estabilidade por um salário mais alto (prêmio de risco) no setor privado. Esse argumento tem sido amplamente estudado por décadas. Por exemplo, Bellante e Link (1981) usaram o índice de medida de aversão ao risco inata (proxies como investimento em seguros, uso de cinto de segurança, etc.) e mostraram que a probabilidade de escolher trabalhar no setor público aumenta à medida que o grau de aversão ao risco aumenta. Um estudo recente utilizando o grande painel socioeconômico alemão encontrou que trabalhadores avessos ao risco tendem a se alocar no emprego do setor público, enquanto o risco é recompensado com salários mais altos no setor privado (Pfeifer, 2011). Com o uso de dados de preferências reveladas em relação ao risco, Buurman et al. (2012) validam o argumento de que trabalhadores do setor público são significativamente menos propensos a escolher a opção arriscada (loterias).

Ekelund et al. (2005) usam uma variável psicométrica que mede a evitação de dano como um indicador de atitudes em relação ao risco. Eles descobriram que agentes com uma pontuação mais alta de evitação de dano (ou seja, menos avessos ao risco) são menos propensos a se tornarem autônomos, o que é considerado mais arriscado do que ser empregado assalariado. Em um estudo experimental, Dohmen et al. (2005) mostram que medidas de atitudes subjetivas em relação ao risco, como a aversão ao risco auto-relatada e perguntas sobre loterias, fornecem um preditor válido do comportamento de risco real. Dohmen e Falk (2011) ampliam esses resultados e utilizam a aversão ao risco auto-relatada no Painel Socioeconômico Alemão para examinar se as preferências de risco explicam como os indivíduos se posicionam em ocupações com diferentes variações de ganhos. Pissarides (1974) apresenta um modelo teórico explicando que trabalhadores avessos ao risco têm salários de reserva mais baixos. Cox e Oaxaca (1989) sugerem uma relação negativa entre o grau de aversão ao risco e o nível de salários de reserva, e Cox e Oaxaca (1992) e Cox e Oaxaca (1996) revalidam o argumento com evidências experimentais sobre o comportamento de busca individual. Essa relação é demonstrada empiricamente por Pannenberg (2007). Da mesma forma, Goerke e Pannenberg (2012) mostram que existe uma relação negativa entre aversão ao risco e filiação sindical.

Dado que a alocação de empregos é importante em termos da posição efetivamente ocupada no mercado de trabalho, há boas razões para questionar se a decisão de alocação de empregos interage com a disparidade de gênero observada no mercado de trabalho. Embora o viés de gênero na educação tenha sido reduzido e a diferença educacional entre homens e mulheres tenha diminuído nas últimas décadas (Arnot et al., 1999), ainda há preocupação com a considerável diferença salarial e outros tipos de discriminação com base no gênero no mercado de trabalho. Em uma tentativa de explicar essas descobertas, Bertrand (2011), Croson e Gneezy (2009), Eckel e Grossman (2008) e Filippin e Crosetto (2016) argumentam que as mulheres podem ser mais avessas ao risco e menos competitivas que os homens. Mais interessante para nossa questão, Gneezy et al. (2003), Niederle e Vesterlund (2007) e Croson e Gneezy (2009) sugerem que as diferenças nas atitudes em relação ao risco podem, em parte, explicar a diferença de gênero nos resultados do mercado de trabalho. Da mesma forma, Barsky et al. (1997), Dohmen e Falk (2011) e Bonin et al. (2007) mostram que a seleção do setor de trabalho e os salários estão correlacionados com as atitudes em relação ao risco.

Em seu artigo seminal, Niederle e Vesterlund (2007) buscam determinar a importância relativa de vários fatores - preferência pura, excesso de confiança, aversão ao risco e aversão ao feedback - na explicação das diferenças de gênero nas preferências por competição. Os autores utilizam o mecanismo de entrada em torneios para explorar as diferenças de gênero na competitividade. Os participantes são convidados a escolher entre remuneração por tarefa e um torneio para determinar a forma de compensação pelo desempenho anterior em uma tarefa não competitiva. Em suas descobertas, a diferença na seleção para um ambiente competitivo é impulsionada pelos homens sendo mais excessivamente confiantes e pelas diferenças de gênero nas preferências por desempenho em competição, mas o risco desempenha apenas um papel insignificante. Em contraste, Dohmen e Falk (2011) encontram que parte das diferenças de gênero pode ser atribuída a diferenças de produtividade e preferências de risco. Em um experimento de campo com crianças de 9 a 12 anos na Colômbia e na Suécia, foi observado que os meninos são mais propensos a preferir a competição em geral e a assumir mais riscos (Cardenas et al., 2012).

No nosso contexto, os participantes experimentam esquemas de pagamento correspondentes tanto a um emprego arriscado (em termos de exposição ao desemprego) quanto a um emprego seguro, e, na etapa final, os participantes escolhem entre o emprego arriscado e o seguro. Portanto, nosso contexto compartilha algumas semelhanças com o estudo de Niederle e Vesterlund (2007), em que os participantes experimentam ambos os esquemas de pagamento e são convidados a escolher qual esquema de pagamento preferem na rodada seguinte. No entanto, como nosso objetivo de pesquisa é diferente, nosso design experimental difere significativamente em alguns aspectos importantes daquele encontrado em Niederle e Vesterlund (2007). Não há elemento de competição em nosso design, de modo que o foco pode estar inteiramente nas diferenças de gênero em ambientes de emprego financeiramente arriscados versus seguros e nas implicações dessas escolhas para a geração de disparidades salariais de gênero, independentemente de fatores baseados em produtividade.

\section{\textbf{Conceptual framework}}

\subsection{\textbf{Secure Job}}

Vamos considerar que a remuneração para o emprego seguro segue um esquema de taxa por peça simples: \( W_s = \gamma_s P \), onde \( W_s \) é o rendimento pelo desempenho da tarefa, \( P \) é uma medida discreta de desempenho (a ser definida abaixo), e \( \gamma_s \) é o retorno marginal ao desempenho para o emprego seguro. Para qualquer indivíduo, a produtividade/desempenho é uma variável aleatória: \( P_i = \psi_i + \varepsilon_i \), onde \( \psi_i = E(P_i) > 0 \) (produtividade média) e \( \varepsilon_i \sim i.i.d(0, \sigma_i^2) \). Consequentemente, \( Var(P_i) = \sigma_i^2 \).

Para um indivíduo empregado no emprego seguro, os salários são determinados de acordo com \( W_{si} = \gamma_s P_i \). A média e a variância da distribuição salarial do indivíduo são obtidas como

\begin{equation}
\centering{\[E(W_{si}) = \gamma_s \psi_i\]}

\centering{\[Var(W_{si}) = \left( \gamma_s \right)^2 \sigma_i^2.\]}

\end{equation}

\subsection{\textbf{Risky Job}}

Condicional à produtividade, a remuneração no emprego arriscado (\( W_r \)) é determinada de acordo com

\[
\begin{array}{c|c}
W_r & \text{Prob}(W_r = k) \\
\hline
w_u & \phi \\
w_r & 1 - \phi
\end{array}
\]

onde \( \phi \) é a probabilidade de ocorrência de desemprego, \( w_u \) é a compensação por desemprego, \( w_r = \gamma_r P \) é o rendimento do emprego arriscado, e \( \gamma_r \) é o retorno marginal ao desempenho para o emprego arriscado.

Pode-se demonstrar que a média e a variância da distribuição condicional de salários são dadas por \( E(W_r | P) = \phi w_u + (1 - \phi) \gamma_r P \) e \( Var(W_r | P) = \phi (1 - \phi) (\gamma_r P - w_u)^2 \). A lei das expectativas iteradas e a lei da variância total são usadas para obter os momentos incondicionais da distribuição salarial para um dado indivíduo: \( E(W_{ri}) = \phi w_u + (1 - \phi) \gamma_r \psi_i \) e \( Var(W_{ri}) = \phi (1 - \phi) (\gamma_r \psi_i - w_u)^2 + (1 - \phi)(\gamma_r)^2 \sigma_i^2 \).


\subsection{\textbf{Risk Premium}}

Para um dado retorno marginal ao desempenho no emprego seguro (\( \gamma_s \)), pode-se facilmente resolver o prêmio de risco compensatório para um agente neutro ao risco com produtividade \( \psi_i \):

\begin{equation}
E(W_{ri}) = E(W_{si})
\end{equation}

\begin{equation}
\phi w_u + (1 - \phi) \gamma_r \psi_i = \gamma_s \psi_i 
\end{equation}

\begin{equation}
r_{ri} - \gamma_s = \left( \frac{\phi}{1 - \phi} \right) \left( \gamma_s - \frac{w_u}{\psi_i} \right) > 0,
\end{equation}

onde \( r_{ri} - \gamma_s \) é o prêmio de risco compensatório que tornaria o agente neutro ao risco indiferente entre o emprego seguro e o emprego arriscado.

Note que \( \gamma_s - \frac{w_u}{\psi_i} > 0 \Rightarrow w_u < \gamma_s \psi_i \), ou seja, para garantir um prêmio de risco positivo, a compensação por desemprego deve ser menor que o produto marginal esperado de receita (retorno) no emprego seguro. Além disso, observe que o prêmio de risco compensatório \( r_{ri} - \gamma_s \) aumenta com \( \phi \) (risco de desemprego) e \( \psi_i \) (habilidade), e diminui com \( w_u \) (compensação por desemprego).


\section{\textbf{Experimental design}}

Os participantes foram recrutados na Universidade de Paris I. Ao final do experimento, um questionário foi administrado aos participantes. Isso foi feito não apenas para obter informações demográficas básicas, mas também para medir as atitudes em relação ao risco, elicitadas com base em escolhas hipotéticas de loteria de Holt-Laury fornecidas aos participantes.

\subsection{\textbf{Treatments}}

Em nossos experimentos, os rendimentos do emprego seguro dependem apenas do desempenho. Não há risco de desemprego. Em contraste, os rendimentos do emprego arriscado dependem do desempenho e da chance (risco de desemprego). Além disso, o recebimento do benefício de desemprego \( w_u \) ocorre com probabilidade \( \phi \), e o recebimento dos rendimentos \( w_{ri} = \gamma_r P_i \) ocorre com probabilidade \( 1 - \phi \). Os participantes foram totalmente informados sobre os pagamentos pelo desempenho e a probabilidade de desemprego para o emprego arriscado.

Cada participante participa de três tratamentos:

\begin{itemize}
    \item \textbf{Tratamento 1:} metade dos participantes é aleatoriamente atribuída ao emprego arriscado e a outra metade é atribuída ao emprego seguro.
    \item \textbf{Tratamento 2:} os participantes atribuídos ao emprego arriscado (seguro) no Tratamento 1 são atribuídos ao emprego seguro (arriscado) no Tratamento 2.
    \item \textbf{Tratamento 3:} os participantes escolhem entre o emprego arriscado e o emprego seguro.
\end{itemize}

Realizamos dois conjuntos de experimentos nos quais os Tratamentos 1–3 foram realizados com dois prêmios de risco diferentes (\( \gamma_r - \gamma_s \)). Não houve sobreposição de participantes entre os dois experimentos de prêmio de risco.

\subsection{\textbf{Effort Tasks and Compensation}}

Os participantes ganham renda digitando blocos de 5 letras gerados aleatoriamente. A compensação é derivada do desempenho do participante, medido por \( P_i \), que corresponde ao número de blocos digitados corretamente. Em cada período, há 40 blocos de 5 letras aleatórios, de modo que \( 0 \leq P_i \leq 40 \). Existem 10 períodos em cada ensaio experimental. Os participantes têm 90 segundos para digitar em cada período.

Todos os participantes foram confrontados com a mesma sequência de blocos de letras ao longo dos 10 períodos dentro de um determinado tratamento (Tratamentos 1–3), mas as sequências eram diferentes entre os tratamentos. Após cada período, a tela de cada participante exibe o número de palavras que eles digitaram corretamente e os rendimentos daquele período com base no tipo de trabalho.

Ao final de cada sessão experimental, um período foi sorteado aleatoriamente de cada um dos Tratamentos 1–3 separadamente para cada participante, e cada participante foi então compensado com base em seu desempenho nos 3 períodos selecionados aleatoriamente.

\subsection{\textbf{Experimental Design Parameters}}

É importante definir valores de parâmetros de modo que um número suficiente de participantes seja atraído tanto para o emprego arriscado quanto para o emprego seguro. Naturalmente, não conhecemos as atitudes em relação ao risco de cada participante ex ante, nem conhecemos a distribuição de produtividade de cada participante ex ante. Com base em cálculos simples que assumiram uma produtividade média dos participantes variando de 33\% a 75\% de acurácia, juntamente com diferentes valores de teste para os retornos à produtividade, fomos capazes de adotar valores razoáveis para os parâmetros do desenho experimental.

Para o nosso desenho, fixamos os valores de \( w_u \), \( \phi \) e \( \gamma_r \) e variamos \( \gamma_s \). Os valores do desenho experimental foram definidos como \( w_u = \) €1, \( \phi = 0.3 \), \( \gamma_r = 0.2 \), \( \gamma_s = \) €0.13 ou €0.14. Consequentemente, os dois experimentos de prêmio de risco corresponderam a \( \gamma_r - \gamma_s = \) €0.07 e €0.06. A inclusão de um tratamento de prêmio de risco nos proporciona a oportunidade de obter insights adicionais a partir das respostas dos participantes às mudanças nos incentivos para assumir riscos. Em particular, buscamos aprender se o tratamento de prêmio é neutro em termos de gênero em seus efeitos sobre a escolha de emprego arriscado e sobre as disparidades salariais de gênero.

\section{\textbf{Empirical Analysis and Results}}

\subsection{\textbf{Decomposition Analysis}}

Nosso objetivo é ser capaz de medir o efeito das diferenças de gênero na escolha de emprego sobre quaisquer disparidades salariais de gênero que surjam em nosso ambiente experimental. Isso é realizado através dos métodos de decomposição desenvolvidos no Apêndice Técnico. Para um dado prêmio de risco dentro do tratamento de escolha endógena (Tratamento 3), a decomposição da disparidade salarial de gênero pode ser expressa como

\begin{equation}
w_m - w_f = \underbrace{\gamma_r \left(P_r^m - P_r^f\right)\theta_r^f}_{\text{produtividade}} + \underbrace{\gamma_s \left(P_s^m - P_s^f\right)\left(1 - \theta_r^f\right)}_{\text{produtividade}}
+ \underbrace{\left(\gamma_r P_r^m - \gamma_s P_s^m\right)\left(\theta_r^m - \theta_r^f\right)}_{\text{escolha de emprego}},
\end{equation}

ou alternativamente como

\begin{equation}
w_m - w_f = \underbrace{\gamma_r \left(P_r^m - P_r^f\right)\theta_r^m}_{\text{produtividade}} + \underbrace{\gamma_s \left(P_s^m - P_s^f\right)\left(1 - \theta_r^m\right)}_{\text{produtividade}}
+ \underbrace{\left(\gamma_r P_r^f - \gamma_s P_s^f\right)\left(\theta_r^m - \theta_r^f\right)}_{\text{escolha de emprego}},
\end{equation}

onde \( \theta_r^j \) é a proporção da amostra das observações salariais no tratamento de escolha endógena associada às escolhas de emprego arriscado. Essas decomposições alternativas surgem do peso dado às diferenças de gênero na produtividade, \( \left(P_r^m - P_r^f\right) \) e \( \left(P_s^m - P_s^f\right) \), por \( \theta_r^f \) ou \( \theta_r^m \).

\subsection{\textbf{Identification of Individual Risk Attitude}}

Relevante para nossa análise das disparidades salariais de gênero e das diferenças de gênero nas preferências por risco é uma propriedade de nosso quadro conceitual que leva diretamente a uma classificação de participantes de acordo com o que pode ser interpretado como preferências por risco. Essa classificação de atribuição de risco pode ser vista como uma alternativa à medida HL de preferências por risco. Embora não seja possível que nosso quadro identifique precisamente as atitudes em relação ao risco para cada participante, é possível identificar subconjuntos de indivíduos que são ou avessos ao risco ou amantes do risco. A chave para essa identificação é a presença de compensação por desemprego \( w_u \), que permite comparar a escolha de emprego de cada participante com a diferença entre seu prêmio de risco neutro estimado e o prêmio de risco experimental.

Substituindo uma estimativa de \( \psi_i \) na Eq. (4) para cada participante, pode-se estimar o prêmio de risco compensatório se o participante fosse neutro ao risco:

\begin{equation}
(\gamma_r - \gamma_s)_i = \left( \frac{\phi}{1 - \phi} \right) \left( \gamma_s - \frac{w_u}{\hat{\psi_i}} \right),
\end{equation}

onde \( \hat{\psi_i} \) é o número médio de blocos digitados corretamente nos últimos cinco períodos observados no tratamento designado.

Seja \( R_i = 1(J_i = r) \) um indicador para a escolha do emprego arriscado. As atitudes em relação ao risco de um participante são identificadas sob as seguintes condições:

\begin{equation}
(\gamma_r - \gamma_s)_i < \gamma_r - \gamma_s \text{ e } R_i = 0 \Rightarrow \text{avesso ao risco} \tag{8}
\end{equation}

\begin{equation}
(\gamma_r - \gamma_s)_i > \gamma_r - \gamma_s \text{ e } R_i = 1 \Rightarrow \text{amante do risco} \tag{9}
\end{equation}

Note que o inverso das condições (8) e (9) não é verdadeiro, ou seja,

\textit{avesso ao risco} \( \Rightarrow (\gamma_r - \gamma_s)_i < \gamma_r - \gamma_s \text{ e } R_i = 0 \)

\textit{amante do risco} \( \Rightarrow (\gamma_r - \gamma_s)_i > \gamma_r - \gamma_s \text{ e } R_i = 1 \).

O complemento dos conjuntos de observações correspondentes às condições (8) e (9) é dado por:

\begin{equation}
(\gamma_r - \gamma_s)_i \leq \gamma_r - \gamma_s \text{ e } R_i = 1 \Rightarrow \text{atitude em relação ao risco não identificada} \tag{10}
\end{equation}

\begin{equation}
(\gamma_r - \gamma_s)_i \geq \gamma_r - \gamma_s \text{ e } R_i = 0 \Rightarrow \text{atitude em relação ao risco não identificada} \tag{11}
\end{equation}

A satisfação da condição (10) ou (11) é consistente com aversão ao risco, neutralidade ao risco ou amor ao risco.

\begin{table}[H]
\centering
\begin{tabular}{lcccccc}
\hline
           & \textbf{Homens} & \textbf{M.E.} & \textbf{Mulheres} & \textbf{M.E.} & \textbf{Diferença} & \textbf{M.E.} \\
\hline
\textbf{Pessoal}      &               &                &                 &                &                &               \\
Idade       & 23.35         & (0.32)        & 23.54          & (0.50)         & -0.19          & (0.57)         \\
Aversão ao Risco (HL) & 6.17          & (0.18)         & 6.08           & (0.23)         & 0.10           & (0.29)         \\
\textbf{Experimental} &               &                &                 &                &                &               \\
Produtividade  & 23.54         & (0.59)        & 22.89          & (0.55)         & 1.39*          & (0.82)         \\
Histórico de Desemprego & 0.31          & (0.01)         & 0.31           & (0.02)         & -0.01          & (0.02)         \\
Escolha Arriscada  & 0.75          & (0.04)         & 0.6            & (0.05)         & 0.15**         & (0.07)         \\
\hline
obs        & 103           &                & 89             &                &                &               \\
\hline
\end{tabular}
\caption{Estatísticas descritivas.}
\end{table}

\textbf{Notas:} * \( p < 0.1 \), ** \( p < 0.05 \), *** \( p < 0.01 \). Erros padrão entre parênteses. Testes t de duas caudas são realizados para a diferença entre homens e mulheres.

\subsection{\textbf{ Empirical Finding}}

Um total de 192 participantes participou do experimento: 97 no experimento de prêmio de risco de €0.07 (54 homens e 43 mulheres) e 95 no experimento de prêmio de risco de €0.06 (49 homens e 46 mulheres).

A Tabela 1 apresenta estatísticas descritivas selecionadas, separadas como \textit{Personal} (não associadas ao comportamento durante o experimento) e \textit{Experimental} (comportamento no experimento). Os sujeitos tinham em média 23 anos de idade e não houve diferença significativa na idade média entre os gêneros. Em média, a produtividade masculina excedeu a feminina por uma margem estatisticamente significativa de 1.39 palavras corretamente digitadas. As taxas de desemprego foram as mesmas para homens e mulheres, com 0.31, o que está muito próximo do valor de design experimental de 0.30. Finalmente, a proporção de homens que escolheu o emprego arriscado excedeu a das mulheres por uma margem estatisticamente significativa de 0.15.

As Tabelas 2-5 relatam os resultados de produtividade e salário para os experimentos de prêmio de risco de €0.07 e €0.06. Para ambos os experimentos de prêmio de risco, os resultados salariais reportados dentro dos empregos são por design proporcionais aos correspondentes resultados de produtividade. No experimento de €0.07 (Tabela 2), o desempenho de digitação dos homens foi virtualmente idêntico entre os tratamentos designados e o tratamento de escolha (23.21 palavras e 23.31 palavras, respectivamente). O desempenho de digitação dos homens também foi praticamente o mesmo entre os tratamentos designados e o tratamento de escolha (22.07 palavras e 21.67 palavras, respectivamente). Apesar dos desempenhos semelhantes para homens e mulheres ao comparar as performances de digitação entre os tratamentos designados e de escolha, as mulheres digitaram mais palavras corretamente do que os homens em ambos os tratamentos, o designado e o de escolha. Essas diferenças foram estatisticamente significativas e geraram disparidades salariais de gênero significativas (Tabela 3).

Resultados semelhantes aparecem nos experimentos de prêmio de risco de €0.06 (Tabela 4). O desempenho de digitação foi virtualmente o mesmo entre os tratamentos designados e de escolha tanto para homens quanto para mulheres.

\begin{table}[h!]
\centering
\caption{Produtividade (Número de palavras digitadas corretamente): Prêmio de Risco = 0.07.}
\resizebox{\textwidth}{!}{
\begin{tabular}{lcccccc}
\hline
           & \textbf{Atribuído} & \textbf{Atribuído} & \textbf{Diferença} & \textbf{Escolha} & \textbf{Escolha} & \textbf{Diferença} \\
           & \textbf{Geral} & \textbf{Arriscado} & \textbf{(\(\bar{P}_r - \bar{P}_s\))} & \textbf{Geral} & \textbf{Arriscado} & \textbf{(\(\bar{P}_r - \bar{P}_s\))} \\
\hline
Homens     & 23.21 (0.08)  & 23.10 (0.11) & -0.18 (0.15) & 23.31 (0.16) & 23.26 (0.19) & -0.16 (0.22) \\
Mulheres   & 21.67 (0.08)  & 21.54 (0.11) & -0.22 (0.17) & 22.07 (0.16) & 21.29 (0.19) & -1.71*** (0.23) \\
Obs        & 729           & 299          & 430          & 328          & 178          & 150 \\
\(\bar{P}_r - \bar{P}_s\) & 1.54** (0.11) & 1.56** (0.16) & 1.52*** (0.14) & 1.24** (0.16) & 1.97*** (0.19) & 0.42 (0.25) & 1.55** (0.32) \\
\hline
\end{tabular}}
\end{table}

\begin{table}[h!]
\centering
\caption{Salário: Prêmio de Risco = 0.07.}
\resizebox{\textwidth}{!}{
\begin{tabular}{lcccccc}
\hline
           & \textbf{Atribuído} & \textbf{Atribuído} & \textbf{Diferença} & \textbf{Escolha} & \textbf{Escolha} & \textbf{Diferença} \\
           & \textbf{Geral} & \textbf{Arriscado} & \textbf{(\(\bar{W}_r - \bar{W}_s\))} & \textbf{Geral} & \textbf{Arriscado} & \textbf{(\(\bar{W}_r - \bar{W}_s\))} \\
\hline
Homens     & 3.78 (0.01)  & 4.62 (0.02) & 3.03 (0.03) & 4.18 (0.02) & 4.65 (0.02) & 3.04 (0.03) & 1.61** (0.04) \\
Mulheres   & 3.45 (0.01)  & 4.31 (0.02) & 2.83 (0.03) & 3.68 (0.02) & 4.26 (0.02) & 2.99 (0.03) & 1.27** (0.04) \\
Obs        & 729           & 299          & 430          & 328          & 178          & 150 \\
\(\bar{W}_r - \bar{W}_s\) & 0.34** (0.01) & 0.30** (0.03) & 0.20** (0.01) & 0.50** (0.03) & 0.39** (0.03) & 0.05 (0.04) & 0.34** (0.05) \\
\hline
\end{tabular}}
\end{table}

\begin{table}[h!]
\centering
\caption{Produtividade (Número de palavras digitadas corretamente): Prêmio de Risco = 0.06.}
\resizebox{\textwidth}{!}{
\begin{tabular}{lcccccc}
\hline
           & \textbf{Atribuído} & \textbf{Atribuído} & \textbf{Diferença} & \textbf{Escolha} & \textbf{Escolha} & \textbf{Diferença} \\
           & \textbf{Geral} & \textbf{Arriscado} & \textbf{(\(\bar{P}_r - \bar{P}_s\))} & \textbf{Geral} & \textbf{Arriscado} & \textbf{(\(\bar{P}_r - \bar{P}_s\))} \\
\hline
Homens     & 23.94 (0.08)  & 24.09 (0.12) & 23.84 (0.16) & 24.18 (0.13) & 23.13 (0.16) & 26.18 (0.21) & -3.05*** (0.23) \\
Mulheres   & 22.71 (0.08)  & 22.89 (0.12) & 22.52 (0.17) & 22.88 (0.15) & 23.84 (0.22) & 21.95 (0.22) & 1.89*** (0.31) \\
Obs        & 819           & 329          & 490          & 378          & 248          & 130 \\
\(\bar{P}_r - \bar{P}_s\) & 1.23** (0.11) & 1.11** (0.16) & 1.32** (0.24) & -0.21 (0.15) & 1.38** (0.22) & -0.71*** (0.22) & 4.23*** (0.22) & -4.94*** (0.31) \\
\hline
\end{tabular}}
\end{table}

\textbf{Notas:} * \( p < 0.1 \), ** \( p < 0.05 \), *** \( p < 0.01 \). Os erros padrão nas diferenças são calculados como a raiz quadrada da soma dos erros quadrados do modelo de efeitos fixos.

Apesar dos padrões semelhantes, as mulheres digitaram, em média, menos palavras corretamente do que os homens, tanto nos tratamentos designados quanto nos de escolha. Como mostra a Tabela 5, essas diferenças no desempenho de digitação resultaram em disparidades salariais de gênero estatisticamente significativas, tanto nos tratamentos designados quanto nos de escolha.

De particular interesse são os padrões associados às escolhas de emprego arriscado versus seguro no tratamento de escolha. Nos experimentos de prêmio de risco de €0.07 (Tabela 2), o desempenho de digitação dos homens foi virtualmente idêntico entre aqueles que escolheram o emprego arriscado e aqueles que escolheram o emprego seguro. O salário médio dos homens foi maior para aqueles que escolheram o emprego arriscado (Tabela 3). Entre as mulheres, aquelas que escolheram o emprego seguro foram significativamente mais produtivas do que as que escolheram o emprego arriscado (23.00 vs 21.29), mas ainda assim ganharam menos do que as mulheres que escolheram o emprego arriscado. No entanto, o desempenho médio das mulheres que escolheram o emprego arriscado foi significativamente menor do que o dos homens que escolheram o emprego arriscado. Por outro lado, as mulheres que escolheram o emprego seguro tiveram um desempenho virtualmente igual ao dos homens que escolheram o emprego seguro.

Quando examinamos a escolha de emprego nos experimentos de prêmio de risco de €0.06, descobrimos que os homens que escolheram o emprego arriscado foram significativamente menos produtivos do que aqueles que escolheram o emprego seguro (Tabela 4), mas ainda assim ganharam um salário médio mais alto do que os homens que escolheram o emprego seguro (Tabela 5). Em contraste, as mulheres que escolheram o emprego arriscado foram significativamente mais produtivas do que aquelas que escolheram o emprego seguro e também ganharam um salário médio mais alto. Enquanto as mulheres que escolheram o emprego seguro foram menos produtivas do que os homens que escolheram o emprego seguro, as mulheres no emprego arriscado foram mais produtivas do que os homens no emprego arriscado.

Uma análise mais detalhada de nossas descobertas revela que a disparidade salarial de gênero no emprego seguro no tratamento de escolha de emprego desaparece porque a diferença de gênero no desempenho de digitação caiu de 4.23 palavras digitadas corretamente no experimento de prêmio de risco de €0.06 para 0.42 no experimento de prêmio de risco de €0.07. De acordo com o quadro teórico apresentado acima, pode-se esperar que o aumento do prêmio de risco induza os trabalhadores mais produtivos a escolher o emprego arriscado. Embora possamos esperar que trabalhadores menos produtivos acabem no emprego seguro tanto para homens quanto para mulheres em resposta a um prêmio de risco mais alto, isso por si só não produziria previsões sobre a disparidade salarial de gênero no emprego seguro. Consistente com as expectativas teóricas, o desempenho dos homens que escolheram o emprego arriscado em relação aos que escolheram o emprego seguro foi maior no tratamento RP 0.07. No entanto, este não foi o caso para as mulheres. Para as mulheres, a razão entre a produtividade no emprego arriscado e no emprego seguro foi menor no tratamento RP = 0.07.

\begin{table}[h!]
\centering
\caption{Salário: Prêmio de Risco = 0.06.}
\resizebox{\textwidth}{!}{
\begin{tabular}{lcccccc}
\hline
           & \textbf{Atribuído} & \textbf{Atribuído} & \textbf{Diferença} & \textbf{Escolha} & \textbf{Escolha} & \textbf{Diferença} \\
           & \textbf{Geral} & \textbf{Arriscado} & \textbf{(\(\overline{W}_r - \overline{W}_s\))} & \textbf{Geral} & \textbf{Arriscado} & \textbf{(\(\overline{W}_r - \overline{W}_s\))} \\
\hline
Homens     & 3.93 (0.01)  & 4.82 (0.02) & 3.34 (0.01) & 4.30 (0.02) & 4.63 (0.03) & 3.67 (0.02) & 0.96*** (0.04) \\
Mulheres   & 3.74 (0.01)  & 4.60 (0.03) & 3.15 (0.02) & 3.83 (0.02) & 4.77 (0.04) & 3.07 (0.02) & 1.70*** (0.04) \\
Obs        & 819           & 329          & 490          & 378          & 248          & 130 \\
\(\overline{W}_r - \overline{W}_s\) & 0.19*** (0.01) & 0.22*** (0.04) & 0.19*** (0.02) & 0.03 (0.04) & 0.47*** (0.03) & -0.14*** (0.05) & 0.60*** (0.03) & -0.74*** (0.06) \\
\hline
\end{tabular}}
\textbf{Notas:} * \( p < 0.1 \), ** \( p < 0.05 \), *** \( p < 0.01 \). Os erros padrão estão entre parênteses. Os erros padrão nas diferenças são calculados como a raiz quadrada da soma dos erros quadrados do modelo de efeitos fixos.
\end{table}

\begin{table}[h!]
\centering
\caption{Diferenças de Gênero nas Escolhas de Emprego.}
\resizebox{\textwidth}{!}{
\begin{tabular}{lcccccc}
\hline
           & \textbf{RP = 0.07} & \textbf{RP = 0.06} & \textbf{Combinado} \\
           & \textbf{n} & \textbf{\(\theta_r\)} & \textbf{n} & \textbf{\(\theta_r\)} & \textbf{n} & \textbf{\(\theta_r\)} \\
\hline
Homens     & 0.76 (0.06)  & 0.70 (0.06) & 0.73 (0.06) & 0.66 (0.07) & 0.75 (0.04) & 0.68 (0.05) \\
Mulheres   & 0.65 (0.07)  & 0.54 (0.07) & 0.54 (0.08) & 0.45 (0.08) & 0.60 (0.05) & 0.49 (0.05) \\
Obs        & 54           & 439          & 49          & 378          & 103          & 817 \\
Diferença  & 0.11 (0.09)  & 0.16 (0.10) & 0.19* (0.10) & 0.21** (0.10) & 0.15** (0.07) & 0.19*** (0.07) \\
\hline
\end{tabular}}
\textbf{Notas:} * \( p < 0.1 \), ** \( p < 0.05 \), *** \( p < 0.01 \).
\end{table}

A Tabela 6 fornece uma visão geral das diferenças de gênero na escolha de emprego. Relatamos duas medidas de escolha de emprego arriscado: a proporção de indivíduos dentro de cada grupo de gênero que selecionaram o emprego arriscado (\(n\)); e a proporção de pagamentos salariais dentro de cada grupo de gênero decorrente do emprego arriscado selecionado (\(\theta_r\)). Para ambas as medidas em ambos os experimentos de prêmio de risco, as proporções de risco para homens são maiores. O prêmio de risco mais alto atrai proporções mais elevadas tanto de homens quanto de mulheres para o emprego arriscado, mas o aumento da atração pelo emprego arriscado é proporcionalmente maior para as mulheres. O aumento do prêmio de risco atrai mais sujeitos a escolher o emprego arriscado de forma monotônica. No limite, quase certamente observaríamos nenhuma diferença de gênero na escolha de emprego, pois as recompensas financeiras induziriam até mesmo os sujeitos mais avessos ao risco a escolher o emprego arriscado. Espera-se que as mulheres respondam mais às mudanças no prêmio de risco porque as mulheres estão em um ponto de saturação mais baixo com relação à escolha de emprego arriscado. Consequentemente, as lacunas de proporção de risco de gênero associadas à escolha do emprego arriscado diminuem quando o prêmio de risco é maior. Por exemplo, a diferença de gênero na proporção de indivíduos que escolhem o emprego arriscado quando o prêmio de risco é de €0.06 é de 19 pontos percentuais. Essa diferença cai para 11 pontos percentuais com o prêmio de risco mais alto de €0.07. De forma semelhante, Petrie e Segal (2017) descobriram que a diferença de gênero na propensão a entrar em um torneio diminui com a magnitude do prêmio do torneio.

Descobrimos que as lacunas de gênero na escolha do emprego arriscado são estatisticamente significativas para o experimento de prêmio de risco mais baixo e marginalmente insignificantes para as proporções de pagamentos salariais no experimento de prêmio de risco mais alto. Com uma exceção, a hipótese de que as proporções de emprego arriscado são menores para os homens pode ser rejeitada em favor de proporções mais altas de emprego arriscado para os homens. A exceção para este teste unilateral ocorre no caso da proporção de indivíduos no experimento de prêmio de risco mais alto. A falha em rejeitar a nula é marginal neste caso.

A Tabela 7 relata os resultados de nossas classificações de atitudes de risco de acordo com as condições (8), (9), (10) e (11). Como se constatou, o prêmio de risco previsto sob neutralidade ao risco para cada sujeito foi inferior ao prêmio de risco experimental. Isso significa que as condições (9) e (11) nunca foram satisfeitas nos dados. Consequentemente, nenhum sujeito amante do risco foi identificado, e todos os sujeitos cujas atitudes em relação ao risco não foram identificadas satisfizeram a condição (10). A proporção de sujeitos identificados como avessos ao risco foi maior para as mulheres em ambos os experimentos de prêmio de risco. Prevísivelmente, as proporções para ambos os gêneros foram menores no experimento de prêmio de risco mais alto. Isso é consistente com os resultados relatados na Tabela 6. Além disso, proporcionalmente menos mulheres foram identificadas como avessas ao risco no experimento de prêmio de risco mais alto. Isso também é consistente com os resultados relatados na Tabela 6.

As decomposições salariais geradas pelas escolhas de emprego dos sujeitos são relatadas na Tabela 8. Em média, os salários foram mais altos para os homens em ambos os tratamentos de prêmio de risco. A diferença salarial de gênero no experimento de prêmio de risco de €0.07 foi de €0.50. Dependendo de qual fração de pagamento salarial do emprego arriscado (\(\theta_r^f\) para mulheres ou \(\theta_r^m\) para homens) é usada para ponderar as diferenças de gênero no desempenho, a propensão das mulheres a selecionar o emprego seguro explica €0.26 (52\%) ou €0.20 (40\%) da diferença salarial. Para o tratamento de prêmio de risco de €0.06, a diferença salarial de gênero foi ligeiramente menor, em €0.47. Novamente, dependendo de qual fração de pagamento salarial do emprego arriscado é usada para ponderar as diferenças de gênero no desempenho, a propensão das mulheres a selecionar o emprego seguro pode explicar €0.20 (43\%) ou até €0.36 (77\%) da diferença salarial. Quando os dois conjuntos de experimentos de prêmio de risco são combinados usando os pesos da amostra do experimento para homens (\(\Omega_m = 0.537\)) para ponderar as diferenças de gênero na produtividade dentro dos empregos, a diferença salarial geral foi em média de €0.48. Sob nossas duas alternativas de decomposições salariais gerais, a escolha de emprego explica em média €0.23 (48\%) ou €0.27 (56\%) da diferença salarial de gênero. O erro de arredondamento e a contribuição das diferenças nos pesos das amostras do experimento são menores e se compensam exatamente.

\begin{table}[h!]
\centering
\caption{Atitudes de Risco Inferidas.}
\resizebox{\textwidth}{!}{
\begin{tabular}{lccccccc}
\hline
           & \multicolumn{2}{c}{\textbf{RP = 0.07}} & \multicolumn{2}{c}{\textbf{RP = 0.06}} & \multicolumn{2}{c}{\textbf{Combinado}} \\
           & \textbf{Homens} & \textbf{Mulheres} & \textbf{Homens} & \textbf{Mulheres} & \textbf{Homens} & \textbf{Mulheres} \\
\hline
Identificados como Aversos ao Risco     & 24\%  & 35\% & 27\% & 46\% & 25\% & 40\% \\
Identificados como Amantes do Risco     & 0\%  & 0\% & 0\% & 0\% & 0\% & 0\% \\
Não Identificados     & 76\%  & 65\% & 73\% & 54\% & 75\% & 60\% \\
Total & 100\%  & 100\% & 100\% & 100\% & 100\% & 100\% \\
Obs & 54  & 43 & 49 & 46 & 103 & 89 \\
\hline
\end{tabular}}
\end{table}

\begin{table}[h!]
\centering
\caption{Decomposições Salariais por Escolhas de Emprego.}
\resizebox{\textwidth}{!}{
\begin{tabular}{lcccccc}
\hline
           & \multicolumn{2}{c}{\textbf{RP = 0.07}} & \multicolumn{2}{c}{\textbf{RP = 0.06}} & \multicolumn{2}{c}{\textbf{Combinado}} \\
           & \(\theta_r^f\) & \(\theta_r^m\) & \(\theta_r^f\) & \(\theta_r^m\) & \(\theta_r^f\) & \(\theta_r^m\) \\
\hline
Produtividade & 0.24 & 0.29 & 0.26 & 0.11 & 0.25 & 0.21 \\
Escolha de Emprego & 0.26 & 0.20 & 0.20 & 0.36 & 0.23 & 0.27 \\
Diferença de Peso Experimental & 0.00 & -0.01 & 0.00 & -0.01 & 0.00 & -0.01 \\
Erro de Arredondamento & 0.01 & 0.01 & 0.01 & 0.01 & 0.01 & 0.01 \\
Diferença Salarial & 0.50 & & 0.47 & & 0.48 & \\
\hline
\end{tabular}}
\textbf{Notas:} \(\theta_r^f\) e \(\theta_r^m\) são as proporções de pagamentos salariais para mulheres e homens decorrentes do emprego arriscado (selecionado). As decomposições combinadas usam \(\Omega_m\) para ponderar as decomposições resultantes dos dois experimentos de prêmio de risco.
\end{table}

\begin{table}[h!]
\centering
\caption{Equação de Seleção Probit para 'Emprego Arriscado'.}
\resizebox{\textwidth}{!}{
\begin{tabular}{lcccccc}
\hline
           & \textbf{Geral} & \textbf{RP = 0.07} & \textbf{RP = 0.06} \\
\hline
Mulher & -0.492** (0.20) & -0.396 (0.29) & -0.602** (0.28) \\
Idade & 0.007 (0.03) & -0.035 (0.04) & 0.040 (0.04) \\
Aversão ao Risco (HL) & -0.093** (0.05) & -0.115* (0.07) & -0.087 (0.07) \\
Produtividade & -0.023 (0.02) & -0.054 (0.04) & -0.014 (0.02) \\
Histórico de Desemprego & -2.635*** (0.69) & -2.472** (1.03) & -2.827*** (1.01) \\
Empregos Arriscados Antes da Escolha & 0.082 (0.20) & -0.155 (0.30) & 0.237 (0.29) \\
Prêmio de Risco (0.07) & 0.075 (0.20) & & \\
Constante & 2.419** (1.10) & 4.312** (1.98) & 1.461 (1.32) \\
chi2 & 25.65 & 11.97 & 15.35 \\
N & 192 & 97 & 95 \\
\hline
\end{tabular}}
\textbf{Notas:} * \( p < 0.1 \), ** \( p < 0.05 \), *** \( p < 0.01 \). Os erros padrão estão entre parênteses.
\end{table}

Para descobrir quais fatores adicionais além do gênero podem explicar as escolhas de emprego dos sujeitos, estimamos um modelo probit de escolha de emprego arriscado. A Tabela 9 relata os resultados desse exercício. Quando condicionamos outros fatores, ser mulher exibe um efeito consistentemente negativo na probabilidade de escolher o emprego arriscado. Esse efeito negativo é estatisticamente significativo no geral e para o tratamento de prêmio de risco mais baixo, mas não para o tratamento de prêmio de risco mais alto. A idade do sujeito e o desempenho médio de digitação do sujeito nos cinco períodos mais recentes no tratamento atribuído nunca são estatisticamente significativos. No entanto, o efeito da produtividade é consistentemente negativo. Isso é consistente com o raciocínio teórico que sugere que quanto maior a produtividade de alguém, maior deverá ser o prêmio de risco para induzir um agente neutro ao risco a escolher o emprego arriscado.

Curiosamente, a medida de aversão ao risco de Holt–Laury exibe consistentemente um efeito negativo na probabilidade de selecionar o emprego arriscado, embora seja estatisticamente significativa apenas no geral e não nos experimentos individuais de prêmio de risco. A Tabela 9 nos diz que, ao controlar pela aversão ao risco do tipo Holt e Laury, as diferenças de gênero ainda permanecem em termos de escolha do emprego arriscado. Nosso desenho experimental diferencia apenas entre empregos seguros e empregos arriscados com relação ao nível de risco de desemprego. Portanto, em termos de escolha entre os empregos seguros e arriscados, o único fator (além da produtividade do sujeito) que determina a escolha deve ser as atitudes em relação ao risco. Portanto, é razoável obter coeficientes significativamente negativos para a aversão ao risco de HL na escolha de empregos arriscados. No entanto, ainda observamos uma diferença de gênero significativa entre homens e mulheres. Acreditamos que essa lacuna de gênero pode capturar uma dimensão diferente da preferência por risco.

Estudos recentes sugerem que as diferenças de gênero na propensão ao risco podem ser muito diferentes dependendo dos métodos de elicitação de preferência por risco (por exemplo, ver Filippin (2016)), e o risco pode ter várias dimensões (ver Dohmen e Falk (2011)). Como mostrado na Tabela 1, a aversão ao risco de HL não parece diferir estatisticamente entre os gêneros. Isso não significa necessariamente, no entanto, que a medida de HL não esteja funcionando bem. Os resultados na Tabela 9 mostram claramente que a aversão ao risco de HL retém algum poder explicativo. Como observamos que as mulheres escolhem empregos arriscados com menos frequência em comparação aos homens, certamente rejeitaríamos a proposição de que não há diferença de gênero nas atitudes em relação ao risco. Podemos concluir que a diferença de gênero nas atitudes em relação ao risco existe em nossa amostra, mas não é bem capturada pela medida de aversão ao risco de HL. Consequentemente, o fato de não haver uma diferença de gênero estatisticamente significativa na medida de HL, apesar da clara diferença de gênero na escolha de emprego, sugere as limitações dessa medida para prever diferenças de gênero na escolha de empregos arriscados.

Os resultados mais fortes em relação à escolha de emprego arriscado decorrem dos efeitos da experiência real de desemprego do sujeito no tratamento atribuído. Quanto mais frequentemente um sujeito realmente experimentava o desemprego nos tratamentos atribuídos, menos provável era que ele escolhesse o emprego arriscado. Esse efeito é estatisticamente significativo no geral e nos experimentos separados de prêmio de risco. A significância estatística é exibida apesar do fato de os sujeitos estarem totalmente informados sobre a probabilidade de enfrentar um período de desemprego. Um possível efeito de ordem do tipo de empregos que o sujeito experimenta primeiro nos dois primeiros tratamentos atribuídos em termos de escolher empregos arriscados no tratamento de escolha, não parece existir. Os coeficientes para a variável dummy de experimentar o tipo de emprego arriscado logo antes da decisão de escolha nunca são significativos. Embora o experimento de prêmio de risco mais alto tenha atraído mais sujeitos para o emprego arriscado, o efeito positivo do prêmio de risco mais alto não é estatisticamente significativo.

\begin{table}[h!]
\centering
\caption{Desempenho: Emprego Arriscado vs. Emprego Seguro (variável dependente: log do desempenho).}
\resizebox{\textwidth}{!}{
\begin{tabular}{lcccccccc}
\hline
           & \multicolumn{4}{c}{\textbf{Prêmio de Risco = 0.07}} & \multicolumn{4}{c}{\textbf{Prêmio de Risco = 0.06}} \\
           & \multicolumn{2}{c}{\textbf{Atribuído}} & \multicolumn{2}{c}{\textbf{Escolha}} & \multicolumn{2}{c}{\textbf{Atribuído}} & \multicolumn{2}{c}{\textbf{Escolha}} \\
           & \textbf{Arriscado} & \textbf{Seguro} & \textbf{Arriscado} & \textbf{Seguro} & \textbf{Arriscado} & \textbf{Seguro} & \textbf{Arriscado} & \textbf{Seguro} \\
\hline
Mulher & -0.055 (0.041) & -0.065 (0.040) & -0.120*** (0.043) & 0.055 (0.079) & -0.050 (0.057) & -0.047 (0.056) & 0.004 (0.070) & -0.116 (0.107) \\
Aversão ao Risco (HL) & -0.019* (0.010) & -0.021** (0.010) & -0.029*** (0.010) & 0.018 (0.025) & -0.003 (0.014) & -0.003 (0.014) & -0.008 (0.016) & 0.025 (0.032) \\
Idade & -0.028*** (0.005) & -0.028*** (0.005) & -0.027*** (0.006) & -0.033*** (0.007) & -0.011 (0.008) & -0.012 (0.008) & -0.012 (0.009) & -0.018 (0.021) \\
Período & 0.008*** (0.001) & 0.005*** (0.001) & -0.000 (0.002) & 0.001 (0.002) & 0.006*** (0.001) & 0.006*** (0.001) & 0.002 (0.002) & -0.001 (0.002) \\
Constante & 3.849*** (0.138) & 3.894*** (0.133) & 3.948*** (0.168) & 3.752*** (0.229) & 3.361*** (0.210) & 3.386*** (0.206) & 3.412*** (0.232) & 3.427*** (0.478) \\
N & 684 & 970 & 487 & 280 & 640 & 950 & 418 & 340 \\
\(\beta_r^F - \beta_s^F\) & 0.010 (0.057) & & & -0.175** (0.090) & -0.003 (0.080) & & & 0.120 (0.128) \\
\(\beta_r^R - \beta_s^R\) & 0.002 (0.014) & & & -0.047* (0.027) & 0.003 (0.049) & & & -0.033 (0.036) \\
\hline
\end{tabular}}
\textbf{Notas:} Resultados obtidos a partir de Estimação de Efeitos Aleatórios. * \( p < 0.1 \), ** \( p < 0.05 \), *** \( p < 0.01 \). Erros padrão entre parênteses. \(\beta_r^F - \beta_s^F\) testa se os coeficientes de 'Mulher' em ambos os empregos são os mesmos. \(\beta_r^R - \beta_s^R\) testa se os coeficientes de 'Aversão ao Risco (HL)' em ambos os empregos são os mesmos.
\end{table}

Exploramos os determinantes do desempenho na digitação além do gênero em um modelo de efeitos aleatórios de desempenho diferenciado por prêmio de risco e emprego arriscado vs. seguro. O desempenho é medido a cada período como o número de blocos de 5 letras digitados corretamente. No emprego arriscado, as observações são omitidas quando o sujeito enfrenta um período de desemprego. Os resultados são relatados na Tabela 10. Após condicionarmos a medida de risco Holt-Laury, idade e período, os efeitos de ser mulher sobre o desempenho na digitação são negativos em 6 dos 8 casos. No entanto, esse efeito é estatisticamente significativo apenas entre aqueles que selecionaram o emprego arriscado no experimento de prêmio de risco mais alto. O efeito de ser mulher sobre o desempenho na digitação é positivo, mas não estatisticamente significativo, para o emprego seguro quando escolhido no experimento de prêmio de risco mais alto e para o emprego arriscado escolhido no tratamento de prêmio de risco mais baixo.

A medida de risco Holt-Laury é estatisticamente significativa em três casos. Todos esses três casos estão no experimento de prêmio de risco mais alto e mostram uma associação negativa com o desempenho para o emprego arriscado atribuído, o emprego seguro atribuído e o emprego arriscado selecionado. A idade do sujeito apresenta uniformemente um efeito negativo sobre o desempenho, mas é estatisticamente significativo apenas nos tratamentos de prêmio de risco mais alto. Talvez o mais interessante sejam os efeitos do período. Em ambos os experimentos de prêmio de risco, a variável 'Período' teve um efeito positivo e estatisticamente significativo sobre o desempenho apenas nos tratamentos atribuídos. Isso é consistente com os sujeitos aprendendo a tarefa de digitação durante o tratamento atribuído, sem que ocorra mais aprendizado em média no momento em que os sujeitos são solicitados a escolher entre o emprego arriscado e o seguro.

Também examinamos se os efeitos de gênero e os efeitos da medida de risco Holt-Laury são significativamente diferentes entre o emprego arriscado e o emprego seguro, ou seja, \(\beta_r^F - \beta_s^F\), e \(\beta_r^R - \beta_s^R\). Essas diferenças são examinadas separadamente, mas não em conjunto, e são estatisticamente significativas apenas para a escolha de emprego no experimento de prêmio de risco mais alto. Em relação aos homens, as mulheres que escolheram o emprego arriscado no experimento de prêmio de risco mais alto não tiveram um desempenho tão bom quanto aquelas que selecionaram o emprego seguro. O desempenho na digitação está negativamente associado à medida de risco Holt-Laury para aqueles que escolheram o emprego arriscado em comparação com aqueles que selecionaram o emprego seguro.


\section{\textbf{Discussion}}

Ao comparar o comportamento dos sujeitos no experimento de prêmio de risco mais alto (\(0.07\)) com o comportamento no experimento de prêmio de risco mais baixo (\(0.06\)), estamos nos baseando em um desenho entre sujeitos, uma vez que os sujeitos experienciaram apenas um dos dois prêmios de risco. Como os sujeitos são sorteados do mesmo grupo, podemos interpretar as diferenças de comportamento entre os dois experimentos como efeitos de tratamento de prêmio de risco. Incondicionalmente, o prêmio de risco mais alto aumenta a propensão a selecionar o emprego arriscado. Quando condicionamos outros fatores, esse efeito persiste, mas perde precisão em um modelo probit de escolha de emprego.

Pode-se especular se um incentivo adicional para escolher o emprego arriscado é a utilidade obtida pelo ganho de tempo de lazer associado ao passar por um período de desemprego. Nossos resultados revelam fortemente o impacto negativo do desemprego experimentado durante a fase de tratamento atribuída na probabilidade subsequente de escolher o emprego arriscado. Isso é indicativo de uma aversão à perda de renda ocasionada pelo desemprego, em oposição a qualquer utilidade positiva do lazer.

Na busca por desvincular os efeitos independentes da produtividade ex-ante na escolha de emprego, utilizamos a média dos cinco períodos mais recentemente observados no emprego designado como base. Nossa análise dos dados apoia fortemente a noção de que essa medida é um proxy razoável para as crenças dos sujeitos sobre suas habilidades de digitação esperadas. Nas rodadas de tratamento de emprego designado do experimento, o aprendizado é evidenciado pelo efeito positivo e estatisticamente significativo do Período sobre o desempenho de digitação (em logaritmo). Enquanto nas rodadas subsequentes de tratamento de escolha de emprego, há uma completa falta de significância estatística do Período sobre o desempenho de digitação. Assim, nos tratamentos de emprego designado, há uma tendência positiva no desempenho de digitação, mas no tratamento de escolha de emprego não há tendência no desempenho de digitação. Portanto, concluímos que o aprendizado foi praticamente concluído até o momento em que os sujeitos são expostos ao tratamento de escolha de emprego.

Embora nosso desenho experimental não tenha sido projetado para examinar as diferenças de gênero no excesso de confiança, a possível existência de excesso de confiança masculino nas habilidades tem implicações para nossas descobertas. Como mostrado acima, o modelo teórico desenvolvido no artigo (Eq. (4)) mostra que, ceteris paribus, o prêmio de risco compensatório aumenta com o nível de habilidade. Portanto, para qualquer prêmio de risco dado, um aumento no nível de habilidade presumivelmente reduziria a probabilidade de selecionar o emprego arriscado. Em outras palavras, quanto maior o nível de habilidade, maior terá que ser o prêmio de risco para que se coloque suas habilidades em risco. Embora a Tabela 9 mostre que a produtividade não é estatisticamente significativa para explicar a escolha do emprego arriscado, ela é consistentemente negativa. Tudo o mais sendo igual, poderia se pensar que se os homens são relativamente mais confiantes em suas habilidades, eles tenderiam a ser menos propensos a selecionar o emprego arriscado. Na medida em que isso seja verdade, estamos sistematicamente subestimando a diferença de escolha atribuível às puras diferenças nas preferências por risco.

Evidências adicionais vêm de regressões OLS e de Efeitos Aleatórios do desempenho de digitação no tratamento de escolha de emprego sobre nossa medida de produtividade ex-ante dos tratamentos de emprego designado. Os coeficientes estimados em nossa medida de produtividade ex-ante são muito próximos de 1.00 e altamente significativos estatisticamente. Por outro lado, o termo constante e o gênero dos sujeitos nunca são estatisticamente significativos. O R simples entre o desempenho de digitação no tratamento de escolha de emprego e a medida de produtividade ex-ante varia de 0.90 a 0.95.

Claramente, não existe uma medida universalmente aceita de atitudes em relação ao risco. No entanto, a medida de Holt-Laury baseada em escolhas de loteria tem desfrutado de algum status como uma medida amplamente aceita ou proxy para preferências por risco. Embora não seja o tema central deste artigo, foi relativamente barato investigar a associação entre esta medida de preferências por risco e a escolha de emprego. A medida HL de aversão ao risco é consistentemente negativa em seu efeito sobre a probabilidade de escolher o emprego arriscado, embora seja estatisticamente significativa apenas com os dados combinados dos experimentos de prêmio de risco. Nossa visão é que a medida HL tem alguma validade, mas está longe de ser o único ou mesmo o principal determinante da escolha entre emprego arriscado e seguro.

Embora a taxa de desemprego esperada no emprego arriscado seja definida como \( \phi = 0.3 \), as taxas de desemprego realizadas para homens e mulheres que escolhem o emprego arriscado geralmente não serão iguais à taxa de desemprego esperada em amostras finitas. As diferenças de gênero nos erros de amostragem podem afetar os pesos \( \theta \), e, portanto, a magnitude do componente de escolha de emprego da decomposição salarial. No apêndice técnico, modificamos as decomposições salariais para levar em conta explicitamente as diferenças de gênero nas taxas de desemprego realizadas. No tratamento \( RP = 0.06 \), as taxas de desemprego realizadas para homens e mulheres são virtualmente idênticas entre si e à taxa de desemprego esperada (0.31 para homens e 0.32 para mulheres). Por outro lado, a diferença de gênero nas taxas de desemprego realizadas é bastante pronunciada no tratamento \( RP = 0.07 \) (0.25 para homens e 0.36 para mulheres). Embora não seja estatisticamente significativa, essa diferença é, no entanto, considerável em magnitude. Nossa decomposição modificada para o tratamento \( RP = 0.07 \) mostra que, dependendo dos pesos de decomposição, o componente de escolha de emprego da decomposição salarial cai de 0.20 (40\%) para 0.15 (30\%) ou de 0.26 (52\%) para 0.19 (38\%). Embora esses resultados ainda sejam uma consequência da decisão de escolher o emprego arriscado, eles iluminam um fator potencialmente importante subjacente às disparidades salariais de gênero decorrentes das diferenças de gênero na escolha de emprego.

Deve-se notar que testamos as diferenças de gênero nas preferências por risco em apenas um tipo de ambiente, a saber, o risco financeiro associado a períodos involuntários de desemprego. Existem, é claro, outros tipos de riscos que influenciam as escolhas de emprego, por exemplo, riscos à saúde e riscos de lesões corporais. As diferenças de gênero nesses outros tipos de preferências por risco claramente poderiam existir no mercado de trabalho natural e poderiam contribuir ainda mais para as diferenças salariais de gênero observadas.


\section{\textbf{Summary and Conclusions}}

A pesquisa relatada neste artigo explora o controle rigoroso do ambiente de laboratório para examinar o potencial das diferenças de gênero nas atitudes em relação ao risco contribuírem para as disparidades salariais de gênero no mercado de trabalho. O mecanismo examinado aqui é a seleção entre um emprego arriscado e um emprego seguro, definidos por uma probabilidade conhecida de desemprego no primeiro e a ausência de desemprego no segundo. O emprego arriscado envolve um prêmio de risco na taxa por peça para blocos de letras digitados com precisão. Ao contrário do ambiente de campo, não há discriminação salarial, poder de monopsônio, segregação ocupacional imposta ou competição em nosso ambiente de laboratório. Qualquer disparidade salarial de gênero que surja só pode surgir de duas fontes: desempenho no trabalho e escolha de emprego.

Em nossos experimentos, surgiu uma disparidade salarial de gênero em ambos os tratamentos de prêmio de risco, e essas disparidades favoreceram os homens. A diferença salarial foi de 13,6\% do salário médio das mulheres no experimento de alto prêmio e 12,3\% do salário médio das mulheres no experimento de baixo prêmio de risco. As mulheres exibiram uma maior propensão a escolher o emprego seguro, mas de menor remuneração. Essas escolhas explicam entre 40\% e 77\% da diferença salarial de gênero, dependendo do prêmio de risco e de qual fração de pagamento salarial do emprego arriscado ($\theta_r^f$ para as mulheres ou $\theta_r^m$ para os homens) é usada para ponderar as diferenças de desempenho no trabalho entre os gêneros. Embora se possa esperar que as magnitudes das disparidades salariais de gênero sejam diferentes no campo devido a uma série de fatores adicionais em jogo, as evidências experimentais em nosso caso apontam para o potencial das atitudes em relação ao risco contribuírem para as disparidades salariais de gênero observadas em mercados de trabalho naturalmente existentes.

Outras potenciais linhas de extensão do nosso trabalho incluem a) condicionar a probabilidade de desemprego ao desempenho individual no trabalho, e b) a relação entre aversão ao risco e aversão à competição. No entanto, é necessário um aviso de cautela. Como a literatura mostra claramente, a noção de preferência por risco pode ser um conceito evasivo. Atitudes em relação a escolhas arriscadas são realisticamente vistas como um conjunto de elementos comportamentais, como excesso de confiança e otimismo, além da noção teórica estrita baseada na curvatura das funções de utilidade em relação à renda. A composição desse conjunto pode depender do contexto particular. Pode ser bastante difícil, senão impossível, desvendar essas dimensões das escolhas arriscadas em qualquer situação de escolha dada. Pelo menos, o presente estudo investiga sistematicamente as potenciais consequências para as disparidades salariais de gênero que resultam das diferenças de gênero nas escolhas entre empregos financeiramente arriscados e seguros em um ambiente que abstrai de uma variedade de influências confusas nos mercados de trabalho naturalmente existentes.


\end{document}