\documentclass[a4paper,12pt]{article}[abntex2]
\bibliographystyle{abntex2-alf}
\usepackage{siunitx} % Fornece suporte para a tipografia de unidades do Sistema Internacional e formatação de números
\usepackage{booktabs} % Melhora a qualidade das tabelas
\usepackage{tabularx} % Permite tabelas com larguras de colunas ajustáveis
\usepackage{graphicx} % Suporte para inclusão de imagens
\usepackage{newtxtext} % Substitui a fonte padrão pela Times Roman
\usepackage{ragged2e} % Justificação de texto melhorada
\usepackage{setspace} % Controle do espaçamento entre linhas
\usepackage[a4paper, left=3.0cm, top=3.0cm, bottom=2.0cm, right=2.0cm]{geometry} % Personalização das margens do documento
\usepackage{lipsum} % Geração de texto dummy 'Lorem Ipsum'
\usepackage{fancyhdr} % Customização de cabeçalhos e rodapés
\usepackage{titlesec} % Personalização dos títulos de seções
\usepackage[portuguese]{babel} % Adaptação para o português (nomes e hifenização
\usepackage{hyperref} % Suporte a hiperlinks
\usepackage{indentfirst} % Indentação do primeiro parágrafo das seções
\sisetup{
  output-decimal-marker = {,},
  inter-unit-product = \ensuremath{{}\cdot{}},
  per-mode = symbol
}
\setlength{\headheight}{14.49998pt}

\DeclareSIUnit{\real}{R\$}
\newcommand{\real}[1]{R\$#1}
\usepackage{float} % Melhor controle sobre o posicionamento de figuras e tabelas
\usepackage{footnotehyper} % Notas de rodapé clicáveis em combinação com hyperref
\hypersetup{
    colorlinks=true,
    linkcolor=black,
    filecolor=magenta,      
    urlcolor=cyan,
    citecolor=black,        
    pdfborder={0 0 0},
}
\usepackage[normalem]{ulem} % Permite o uso de diferentes tipos de sublinhados sem alterar o \emph{}
\makeatletter
\def\@pdfborder{0 0 0} % Remove a borda dos links
\def\@pdfborderstyle{/S/U/W 1} % Estilo da borda dos links
\makeatother
\onehalfspacing

\begin{document}

\begin{titlepage}
    \centering
    \vspace*{1cm}
    \Large\textbf{INSPER – INSTITUTO DE ENSINO E PESQUISA}\\
    \Large ECONOMIA\\
    \vspace{1.5cm}
    \Large\textbf{Resolução Guia de Estudos - HPE}\\
    \vspace{1.5cm}
    Prof. Pedro Duarte\\
    Prof. Auxiliar Guilherme Mazer\\
    \vfill
    \normalsize
    Hicham Munir Tayfour, \href{mailto:hichamt@al.insper.edu.br}{hichamt@al.insper.edu.br}\\
    4º Período - Economia B\\
    \vfill
    São Paulo\\
    Março/2024
\end{titlepage}

\newpage
\tableofcontents
\thispagestyle{empty} % This command removes the page number from the table of contents page
\newpage
\setcounter{page}{1} % This command sets the page number to start from this page
\justify
\onehalfspacing

\pagestyle{fancy}
\fancyhf{}
\rhead{\thepage}

\section{\textbf{Mentor Matemárico}}
Link : \href{https://chatgpt.com/g/g-85ZEueC6o-mentor-matematico}{Mentor Matemático}

\subsection{Fundamentos da Matemática Elementar; Volume 1 ; Conjuntos e Funções}
\subsubsection*{CAPÍTULO I — Noções de lógica}

\begin{itemize}
\item Proposição
\item Negação
\item Proposição composta — Conectivos
\item Condicionais
\item Tautologias
\item Proposições logicamente falsas
\item Relação de implicação
\item Relação de equivalência
\item Sentenças abertas, quantificadores
\item Como negar proposições
\end{itemize}
\subsubsection*{CAPÍTULO II — Conjuntos}

\begin{itemize}
\item Conjunto — Elemento — Pertinência
\item Descrição de um conjunto
\item Conjunto unitário — Conjunto vazio
\item Conjunto universo
\item Conjuntos iguais
\item Subconjuntos
\item Reunião de conjuntos
\item Interseção de conjuntos
\item Propriedades
\item Diferença de conjuntos
\item Complementar de B em A
\item Leitura: Cantor e a teoria dos conjuntos
\end{itemize}
\subsubsection*{CAPÍTULO III — Conjuntos numéricos}

\begin{itemize}
\item Conjunto dos números naturais
\item Conjunto dos números inteiros
\item Conjunto dos números racionais
\item Conjunto dos números reais
\item Intervalos
\item Conjunto dos números complexos
\item Resumo
\item Apêndice: Princípio da indução finita
\item Leitura: Eudóxio e os incomensuráveis
\end{itemize}
\subsubsection*{CAPÍTULO IV — Relações}

\begin{itemize}
\item Par ordenado
\item Representação gráfica
\item Produto cartesiano
\item Relação binária
\item Domínio e imagem
\item Relação inversa
\item Propriedades das relações
\end{itemize}
\subsubsection*{CAPÍTULO V — Introdução às funções}

\begin{itemize}
\item Conceito de função
\item Definição de função
\item Notação das funções
\item Domínio e imagem
\item Funções iguais
\item Leitura: Stevin e as frações decimais
\end{itemize}
\subsubsection*{CAPÍTULO VI — Função constante — Função afim}

\begin{itemize}
\item Função constante
\item Função identidade
\item Função linear
\item Função afim
\item Gráfico
\item Imagem
\item Coeficientes da função afim
\item Zero da função afim
\item Funções crescentes e decrescentes
\item Crescimento/decrescimento da função afim
\item Sinal de uma função
\item Sinal da função afim
\item Inequações
\item Inequações simultâneas
\item Inequações-produto
\item Inequações-quociente
\end{itemize}
\subsubsection*{CAPÍTULO VII — Funções quadráticas}

\begin{itemize}
\item Definição
\item Gráfico
\item Concavidade
\item Forma canônica
\item Zeros
\item Máximo e mínimo
\item Vértice da parábola
\item Imagem
\item Eixo de simetria
\item Informações que auxiliam a construção do gráfico
\item Sinal da função quadrática
\item Inequação do 2º grau
\item Comparação de um número real com as raízes da equação do 2º grau
\item Sinais das raízes da equação do 2º grau
\item Leitura: Dedekind e os números reais
\end{itemize}
\subsubsection*{CAPÍTULO VIII — Função modular}

\begin{itemize}
\item Função definida por várias sentenças abertas
\item Módulo
\item Função modular
\item Equações modulares
\item Inequações modulares
\item Leitura: Boole e a álgebra do pensamento
\end{itemize}
\subsubsection*{CAPÍTULO IX — Outras funções elementares}

\begin{itemize}
\item Função f(x) = x3
\item Função recíproca
\item Função máximo inteiro
\end{itemize}
\subsubsection*{CAPÍTULO X — Função composta — Função inversa}

\begin{itemize}
\item Função composta
\item Função sobrejetora
\item Função injetora
\item Função bijetora
\item Função inversa
\item Leitura: Bertrand Russell e o Logicismo
\end{itemize}
\subsubsection*{APÊNDICE I — Equações irracionais}

\subsubsection*{APÊNDICE II — Inequações irracionais}

\subsubsection*{Respostas dos exercícios}

\subsubsection*{Questões de vestibulares}

\subsubsection*{Respostas das questões de vestibulares}
 
 \subsection{Fundamentos da Matemática Elementar; Volume 2 ; Logaritmos}

\subsubsection*{CAPÍTULO I — Potências e raízes}

\begin{itemize}
\item Potência de expoente natural
\item Potência de expoente inteiro negativo
\item Raiz enésima aritmética
\item Potência de expoente racional
\item Potência de expoente irracional
\item Potência de expoente real
\item Leitura: Stifel, Bürgi e a criação dos logaritmos
\end{itemize}
\subsubsection*{CAPÍTULO II — Função exponencial}

\begin{itemize}
\item Definição
\item Propriedades
\item Imagem
\item Gráfico
\item Equações exponenciais
\item Inequações exponenciais
\item Leitura: Os logaritmos segundo Napier
\end{itemize}
\subsubsection*{CAPÍTULO III — Logaritmos}

\begin{itemize}
\item Conceito de logaritmo
\item Antilogaritmo
\item Consequências da definição
\item Sistemas de logaritmos
\item Propriedades dos logaritmos
\item Mudança de base
\item Leitura: Lagrange: a grande pirâmide da Matemática
\end{itemize}
\subsubsection*{CAPÍTULO IV — Função logarítmica}

\begin{itemize}
\item Definição
\item Propriedades
\item Imagem
\item Gráfico
\end{itemize}
\subsubsection*{CAPÍTULO V — Equações exponenciais e logarítmicas}

\begin{itemize}
\item Equações exponenciais
\item Equações logarítmicas
\item Leitura: Gauss: o universalista por excelência
\end{itemize}
\subsubsection*{CAPÍTULO VI — Inequações exponenciais e logarítmicas}

\begin{itemize}
\item Inequações exponenciais
\item Inequações logarítmicas
\item Leitura: A computação e o sonho de Babbage
\end{itemize}
\subsubsection*{CAPÍTULO VII — Logaritmos decimais}

\begin{itemize}
\item Introdução
\item Característica e mantissa
\item Regras da característica
\item Mantissa
\item Exemplos de aplicações da tábua de logaritmos
\end{itemize}
\subsubsection*{Respostas dos exercícios}

\subsubsection*{Questões de vestibulares}

\subsubsection*{Respostas das questões de vestibulares}

\subsubsection*{Significado das siglas de vestibulares}
 
 \subsection{Fundamentos da Matemática Elementar; Volume 3 ; Trigonometria}

\subsubsection*{1ª PARTE: Trigonometria no triângulo retângulo}

\subsubsection*{CAPÍTULO I — Revisão inicial de geometria}

\subsubsection*{CAPÍTULO II — Razões trigonométricas no triângulo retângulo}

\begin{itemize}
\item Triângulo retângulo: conceito, elementos, teorema de Pitágoras
\item Triângulo retângulo: razões trigonométricas
\item Relações entre seno, cosseno, tangente e cotangente
\item Seno, cosseno, tangente e cotangente de ângulos complementares
\item Razões trigonométricas especiais
\end{itemize}
\subsubsection*{2ª PARTE: Trigonometria na circunferência}

\subsubsection*{CAPÍTULO III — Arcos e ângulos}

\begin{itemize}
\item Arcos de circunferência
\item Medidas de arcos
\item Medidas de ângulos
\item Ciclo trigonométrico
\item Leitura: Hiparco, Ptolomeu e a Trigonometria
\end{itemize}
\subsubsection*{CAPÍTULO IV — Razões trigonométricas na circunferência}

\begin{itemize}
\item Noções gerais
\item Seno
\item Cosseno
\item Tangente
\item Cotangente
\item Secante
\item Cossecante
\end{itemize}
\subsubsection*{CAPÍTULO V — Relações fundamentais}

\begin{itemize}
\item Introdução
\item Relações fundamentais
\end{itemize}
\subsubsection*{CAPÍTULO VI — Arcos notáveis}

\begin{itemize}
\item Teorema
\item Aplicações
\item Leitura: Viète, a Notação Literal e a Trigonometria
\end{itemize}
\subsubsection*{CAPÍTULO VII — Redução ao 1º quadrante}

\begin{itemize}
\item Redução do 2º ao 1º quadrante
\item Redução do 3º ao 1º quadrante
\item Redução do 4º ao 1º quadrante
\item Redução de 3p/4, p/24 a 30, p/44
\end{itemize}
\subsubsection*{3ª PARTE: Funções trigonométricas}

\subsubsection*{CAPÍTULO VIII — Funções circulares}

\begin{itemize}
\item Noções básicas
\item Funções periódicas
\item Ciclo trigonométrico
\item Função seno
\item Função cosseno
\item Função tangente
\item Função cotangente
\item Função secante
\item Função cossecante
\item Funções pares e funções ímpares
\end{itemize}
\subsubsection*{CAPÍTULO IX — Transformações}

\begin{itemize}
\item Fórmulas de adição
\item Fórmulas de multiplicação
\item Fórmulas de divisão
\item É dada a tg x/2
\item Transformação em produto
\item Leitura: Fourier, o Som e a Trigonometria
\end{itemize}
\subsubsection*{CAPÍTULO X — Identidades}

\begin{itemize}
\item Demonstração de identidade
\item Identidades no ciclo trigonométrico
\end{itemize}
\subsubsection*{CAPÍTULO XI — Equações}

\begin{itemize}
\item Equações fundamentais
\item Resolução da equação sen a 5 sen b
\item Resolução da equação cos a 5 cos b
\item Resolução da equação tg a 5 tg b
\item Equações clássicas
\end{itemize}
\subsubsection*{CAPÍTULO XII — Inequações}

\begin{itemize}
\item Inequações fundamentais
\item Resolução de sen x . m
\item Resolução de sen x , m
\item Resolução de cos x . m
\item Resolução de cos x , m
\item Resolução de tg x . m
\item Resolução de tg x , m
\item Leitura: Euler e a incorporação da trigonometria à análise
\end{itemize}
\subsubsection*{CAPÍTULO XIII — Funções circulares inversas}

\begin{itemize}
\item Introdução
\item Função arco-seno
\item Função arco-cosseno
\item Função arco-tangente
\end{itemize}
\subsubsection*{4ª PARTE: Apêndices}

\subsubsection*{APÊNDICE A: Resolução de equações e inequações em intervalos determinados}

\begin{itemize}
\item Resolução de equações
\item Resolução de inequações
\end{itemize}
\subsubsection*{APÊNDICE B: Trigonometria em triângulos quaisquer}

\begin{itemize}
\item Lei dos cossenos
\item Lei dos senos
\item Propriedades geométricas
\end{itemize}
\subsubsection*{APÊNDICE C: Resolução de triângulos}

\begin{itemize}
\item Triângulos retângulos
\item Triângulos quaisquer
\end{itemize}
\subsubsection*{RESPOSTAS DOS EXERCÍCIOS}

\subsubsection*{QUESTÕES DE VESTIBULARES}

\subsubsection*{RESPOSTAS DAS QUESTÕES DE VESTIBULARES}

\subsubsection*{TABELA DE RAZÕES TRIGONOMÉTRICAS}

\subsubsection*{SIGNIFICADO DAS SIGLAS DE VESTIBULARES }

 \subsection{Fundamentos da Matemática Elementar; Volume 4 ; Sequências, Matrizes, Determinantes e Sistemas}

\subsubsection*{CAPÍTULO I — Sequências}

\begin{itemize}
\item Noções iniciais
\item Igualdade
\item Lei de formação
\end{itemize}
\subsubsection*{CAPÍTULO II — Progressão aritmética}

\begin{itemize}
\item Definição
\item Classificação
\item Notações especiais
\item Fórmula do termo geral
\item Interpolação aritmética
\item Soma
\item Leitura: Dirichlet e os números primos de uma progressão aritmética
\end{itemize}
\subsubsection*{CAPÍTULO III — Progressão geométrica}

\begin{itemize}
\item Definição
\item Classificação
\item Notações especiais
\item Fórmula do termo geral
\item Interpolação geométrica
\item Produto
\item Soma dos termos de P.G. finita
\item Limite de uma sequência
\item Soma dos termos de P.G. infinita
\end{itemize}
\subsubsection*{CAPÍTULO IV — Matrizes}

\begin{itemize}
\item Noção de matriz
\item Matrizes especiais
\item Igualdade
\item Adição
\item Produto de número por matriz
\item Produto de matrizes
\item Matriz transposta
\item Matrizes inversíveis
\item Leitura: Cayley e a teoria das matrizes
\end{itemize}
\subsubsection*{CAPÍTULO V — Determinantes}

\begin{itemize}
\item Introdução
\item Definição de determinante (n3)
\item Menor complementar e complemento algébrico
\item Definição de determinante por recorrência (caso geral)
\item Teorema fundamental (de Laplace)
\item Propriedades dos determinantes
\item Abaixamento de ordem de um determinante – Regra de Chió
\item Matriz de Vandermonde (ou das potências)
\item Apêndice: Cálculo da matriz inversa por meio de determinantes
\item Leitura: Sistemas lineares e determinantes: origens e desenvolvimento
\end{itemize}
\subsubsection*{CAPÍTULO VI — Sistemas lineares}

\begin{itemize}
\item Introdução
\item Teorema de Cramer
\item Sistemas escalonados
\item Sistemas equivalentes – Escalonamento de um sistema
\item Sistema linear homogêneo
\item Característica de uma matriz – Teorema de Rouché-Capelli
\item Leitura: Emmy Noether e a Álgebra Moderna
\end{itemize}
\subsubsection*{Respostas dos exercícios}

\subsubsection*{Questões de vestibulares}

\subsubsection*{Respostas das questões de vestibulares}

\subsubsection*{Significado das siglas de vestibulares}
 
 \subsection*{Fundamentos da Matemática Elementar; Volume 5 ; Combinatória e Probabilidade}

\subsubsection*{CAPÍTULO I — Análise Combinatória}

\begin{itemize}
\item Introdução
\item Princípio fundamental da contagem
\item Consequências do princípio fundamental da contagem
\item Arranjos com repetição
\item Arranjos
\item Permutações
\item Fatorial
\item Combinações
\item Permutações com elementos repetidos
\item Complementos
\item Leitura: Cardano: o intelectual jogador
\end{itemize}
\subsubsection*{CAPÍTULO II — Binômio de Newton}

\begin{itemize}
\item Introdução
\item Teorema binomial
\item Observações
\item Triângulo aritmético de Pascal (ou de Tartaglia)
\item Expansão multinomial
\item Leitura: Pascal e a teoria das probabilidades
\item Leitura: Os irmãos Jacques e Jean Bernoulli
\end{itemize}
\subsubsection*{CAPÍTULO III — Probabilidade}

\begin{itemize}
\item Experimentos aleatórios
\item Espaço amostral
\item Evento
\item Combinações de eventos
\item Frequência relativa
\item Definição de probabilidade
\item Teoremas sobre probabilidades em espaço amostral finito
\item Espaços amostrais equiprováveis
\item Probabilidade de um evento num espaco equiprovável
\item Probabilidade condicional
\item Teorema da multiplicação
\item Teorema da probabilidade total
\item Independência de dois eventos
\item Independência de três ou mais eventos
\item Lei binomial da probabilidade
\item Leitura: Laplace: a teoria das probabilidades chega ao firmamento
\end{itemize}
\subsubsection*{Respostas dos exercícios}

\subsubsection*{Questões de vestibulares}

\subsubsection*{Respostas das questões de vestibulares}

\subsubsection*{Significado das siglas de vestibulares}
 
 \subsection{Fundamentos da Matemática Elementar; Volume 6 ; Complexos, Polinômios e Equações}

\subsubsection*{CAPÍTULO I — Números complexos}

\begin{itemize}
\item Operações com pares ordenados
\item Forma algébrica
\item Forma trigonométrica
\item Potenciação
\item Radiciação
\item Equações binômias e trinômias
\item Leitura: Números complexos: de Cardano a Hamilton
\end{itemize}
\subsubsection*{CAPÍTULO II — Polinômios}

\begin{itemize}
\item Polinômios
\item Igualdade
\item Operações
\item Grau
\item Divisão
\item Divisão por binômios do 1º grau
\item Leitura: Tartaglia e as equações de grau três
\end{itemize}
\subsubsection*{CAPÍTULO III — Equações polinomiais}

\begin{itemize}
\item Introdução
\item Definições
\item Número de raízes
\item Multiplicidade de uma raiz
\item Relações entre coeficientes e raízes (Relações de Girard)
\item Raízes complexas
\item Raízes reais
\item Raízes racionais
\item Leitura: Abel e as equações de grau > 5
\end{itemize}
\subsubsection*{CAPÍTULO IV — Transformações}

\begin{itemize}
\item Transformações
\item Equações recíprocas
\end{itemize}
\subsubsection*{CAPÍTULO V — Raízes múltiplas e raízes comuns}

\begin{itemize}
\item Derivada de uma função polinomial
\item Raízes múltiplas
\item Máximo divisor comum
\item Raízes comuns
\item Mínimo múltiplo comum
\item Leitura: Galois: nasce a Álgebra Moderna
\end{itemize}
\subsubsection*{Respostas dos exercícios}

\subsubsection*{Questões de vestibulares}

\subsubsection*{Respostas das questões de vestibulares}

\subsubsection*{Significado das siglas de vestibulares}
 
 \subsection{Fundamentos da Matemática Elementar; Volume 7 ; Geometria analítica}
 \subsubsection*{CAPÍTULO I — Coordenadas cartesianas no plano}

\begin{itemize}
\item Noções básicas
\item Posições de um ponto em relação ao sistema
\item Distância entre dois pontos
\item Razão entre segmentos colineares
\item Coordenadas do terceiro ponto
\item Condição para alinhamento de três pontos
\item Complemento — Cálculo de determinantes
\end{itemize}
\subsubsection*{CAPÍTULO II — Equação da reta}

\begin{itemize}
\item Equação geral
\item Interseção de duas retas
\item Posições relativas de duas retas
\item Feixe de retas concorrentes
\item Feixe de retas paralelas
\item Formas da equação da reta
\item Leitura: Menaecmus, Apolônio e as seções cônicas
\end{itemize}
\subsubsection*{CAPÍTULO III — Teoria angular}

\begin{itemize}
\item Coeficiente angular
\item Cálculo de m
\item Equação de uma reta passando por P(x0, y0)
\item Condição de paralelismo
\item Condição de perpendicularismo
\item Ângulo de duas retas
\end{itemize}
\subsubsection*{CAPÍTULO IV — Distância de ponto a reta}

\begin{itemize}
\item Translação de sistema
\item Distância entre ponto e reta
\item Área do triângulo
\item Variação de sinal da função E(x, y) 5 ax 1 by 1 c
\item Inequações do 1º grau
\item Bissetrizes dos ângulos de duas retas
\item Complemento — Rotação de sistema
\item Leitura: Fermat, o grande amador da Matemática, e a geometria analítica
\end{itemize}
\subsubsection*{CAPÍTULO V — Circunferências}

\begin{itemize}
\item Equação reduzida
\item Equação normal
\item Reconhecimento
\item Ponto e circunferência
\item Inequações do 2º grau
\item Reta e circunferência
\item Duas circunferências
\end{itemize}
\subsubsection*{CAPÍTULO VI — Problemas sobre circunferências}

\begin{itemize}
\item Problemas de tangência
\item Determinação de circunferências
\item Complemento
\item Leitura: Descartes, o primeiro filósofo moderno, e a geometria analítica
\end{itemize}
\subsubsection*{CAPÍTULO VII — Cônicas}

\begin{itemize}
\item Elipse
\item Hipérbole
\item Parábola
\item Reconhecimento de uma cônica
\item Interseções de cônicas
\item Tangentes a uma cônica
\end{itemize}
\subsubsection*{CAPÍTULO VIII — Lugares geométricos}

\begin{itemize}
\item Equação de um lugar geométrico
\item Interpretação de uma equação do 2º grau
\end{itemize}
\subsubsection*{APÊNDICE — Demonstração de teoremas de Geometria Plana}

\begin{itemize}
\item Leitura: Monge e a consolidação da geometria analítica
\end{itemize}
\subsubsection*{Respostas dos exercícios}

\subsubsection*{Questões de vestibulares}

\subsubsection*{Respostas das questões de vestibulares}

\subsubsection*{Significado das siglas de vestibulares}
 
 \subsection{Fundamentos da Matemática Elementar; Volume 8 ;Limites, Derivadas e Noções de integral}
\subsubsection*{CAPÍTULO I — Funções}

\begin{itemize}
\item A noção de função
\item Principais funções elementares
\item Composição de funções
\item Funções inversíveis
\item Operações com funções
\end{itemize}
\subsubsection*{CAPÍTULO II — Limite}

\begin{itemize}
\item Noção intuitiva de limite
\item Definição de limite
\item Unicidade do limite
\item Propriedades do limite de uma função
\item Limite de uma função polinomial
\item Limites laterais
\item Leitura: Arquimedes, o grande precursor do Cálculo Integral
\end{itemize}
\subsubsection*{CAPÍTULO III — O infinito}

\begin{itemize}
\item Limites infinitos
\item Propriedades dos limites infinitos
\item Limites no infinito
\item Propriedades dos limites no infinito
\end{itemize}
\subsubsection*{CAPÍTULO IV — Complemento sobre limites}

\begin{itemize}
\item Teoremas adicionais sobre limites
\item Limites trigonométricos
\item Limites da função exponencial
\item Limites da função logarítmica
\item Limite exponencial fundamental
\item Leitura: Newton e o método dos fluxos
\end{itemize}
\subsubsection*{CAPÍTULO V — Continuidade}

\begin{itemize}
\item Noção de continuidade
\item Propriedades das funções contínuas
\item Limite da n√f(x)
\end{itemize}
\subsubsection*{CAPÍTULO VI — Derivadas}

\begin{itemize}
\item Derivada no ponto x0
\item Interpretação geométrica
\item Interpretação cinemática
\item Função derivada
\item Derivadas das funções elementares
\item Derivada e continuidade
\item Leitura: Leibniz e as diferenciais
\end{itemize}
\subsubsection*{CAPÍTULO VII — Regras de derivação}

\begin{itemize}
\item Derivada da soma
\item Derivada do produto
\item Derivada do quociente
\item Derivada de uma função composta (regra da cadeia)
\item Derivada da função inversa
\item Derivadas sucessivas
\end{itemize}
\subsubsection*{CAPÍTULO VIII — Estudo da variação das funções}

\begin{itemize}
\item Máximos e mínimos
\item Derivada — crescimento — decréscimo
\item Determinação dos extremantes
\item Concavidade
\item Ponto de inflexão
\item Variação das funções
\item Leitura: Cauchy e Weierstrass: o rigor chega ao Cálculo
\end{itemize}
\subsubsection*{CAPÍTULO IX — Noções de Cálculo Integral}

\begin{itemize}
\item Introdução — Área
\item A integral definida
\item O cálculo da integral
\item Algumas técnicas de integração
\item Uma aplicação geométrica: cálculo de volumes
\end{itemize}
\subsubsection*{Respostas dos exercícios}

\subsubsection*{Questões de vestibulares}

\subsubsection*{Respostas das questões de vestibulares}

\subsubsection*{Significados da siglas de vestibulares}

 \subsection{Fundamentos da Matemática Elementar; Volume 9 ; Geometria plana}

\subsubsection*{CAPÍTULO I — Noções e proposições primitivas}

\begin{itemize}
\item Noções primitivas
\item Proposições primitivas
\end{itemize}
\subsubsection*{CAPÍTULO II — Segmento de reta}

\begin{itemize}
\item Conceitos
\end{itemize}
\subsubsection*{CAPÍTULO III — Ângulos}

\begin{itemize}
\item Introdução
\item Definições
\item Congruência e comparação
\item Ângulo reto, agudo, obtuso — Medida
\end{itemize}
\subsubsection*{CAPÍTULO IV — Triângulos}

\begin{itemize}
\item Conceito — Elementos — Classificação
\item Congruência de triângulos
\item Desigualdades nos triângulos
\item Leitura: Euclides e a geometria dedutiva
\end{itemize}
\subsubsection*{CAPÍTULO V — Paralelismo}

\begin{itemize}
\item Conceitos e propriedades
\end{itemize}
\subsubsection*{CAPÍTULO VI — Perpendicularidade}

\begin{itemize}
\item Definições — Ângulo reto
\item Existência e unicidade da perpendicular
\item Projeções e distância
\end{itemize}
\subsubsection*{CAPÍTULO VII — Quadriláteros notáveis}

\begin{itemize}
\item Quadrilátero — Definição e elementos
\item Quadriláteros notáveis — Definições
\item Propriedades dos trapézios
\item Propriedades dos paralelogramos
\item Propriedades do retângulo, do losango e do quadrado
\item Consequências — Bases médias
\end{itemize}
\subsubsection*{CAPÍTULO VIII — Pontos notáveis do triângulo}

\begin{itemize}
\item Baricentro — Medianas
\item Incentro — Bissetrizes internas
\item Circuncentro — Mediatrizes
\item Ortocentro — Alturas
\item Leitura: Papus: o epílogo da geometria grega
\end{itemize}
\subsubsection*{CAPÍTULO IX — Polígonos}

\begin{itemize}
\item Definições e elementos
\item Diagonais — Ângulos internos — Ângulos externos
\end{itemize}
\subsubsection*{CAPÍTULO X — Circunferência e círculo}

\begin{itemize}
\item Definições — Elementos
\item Posições relativas de reta e circunferência
\item Posições relativas de duas circunferências
\item Segmentos tangentes — Quadriláteros circunscritíveis
\end{itemize}
\subsubsection*{CAPÍTULO XI — Ângulos na circunferência}

\begin{itemize}
\item Congruência, adição e desigualdade de arcos
\item Ângulo central
\item Ângulo inscrito
\item Ângulo de segmento ou ângulo semi-inscrito
\end{itemize}
\subsubsection*{CAPÍTULO XII — Teorema de Tales}

\begin{itemize}
\item Teorema de Tales
\item Teorema das bissetrizes
\item Leitura: Legendre: por uma geometria rigorosa e didática
\end{itemize}
\subsubsection*{CAPÍTULO XIII — Semelhança de triângulos e potência de ponto}

\begin{itemize}
\item Semelhança de triângulos
\item Casos ou critérios de semelhança
\item Potência de ponto
\end{itemize}
\subsubsection*{CAPÍTULO XIV — Triângulos retângulos}

\begin{itemize}
\item Relações métricas
\item Aplicações do teorema de Pitágoras
\end{itemize}
\subsubsection*{CAPÍTULO XV — Triângulos quaisquer}

\begin{itemize}
\item Relações métricas e cálculo de linhas notáveis
\end{itemize}
\subsubsection*{CAPÍTULO XVI — Polígonos regulares}

\begin{itemize}
\item Conceitos e propriedades
\item Leitura: Hilbert e a formalização da geometria
\end{itemize}
\subsubsection*{CAPÍTULO XVII — Comprimento da circunferência}

\begin{itemize}
\item Conceitos e propriedades
\end{itemize}
\subsubsection*{CAPÍTULO XVIII — Equivalência plana}

\begin{itemize}
\item Definições
\item Redução de polígonos por equivalência
\end{itemize}
\subsubsection*{CAPÍTULO XIX — Áreas de superfícies planas}

\begin{itemize}
\item Áreas de superfícies planas
\item Áreas de polígonos
\item Expressões da área do triângulo
\item Área do círculo e de suas partes
\item Razão entre áreas
\end{itemize}
\subsubsection*{Respostas dos exercícios}

\subsubsection*{Questões de vestibulares}

\subsubsection*{Respostas das questões de vestibulares}

\subsubsection*{Significado das siglas de vestibulares}

 \subsection{Fundamentos da Matemática Elementar; Volume 10 ; Geometria espacial : posição e métrica}

\subsubsection*{CAPÍTULO I — Introdução}

\begin{itemize}
\item Conceitos primitivos e postulados
\item Determinação de plano
\item Posições das retas
\item Interseção de planos
\end{itemize}
\subsubsection*{CAPÍTULO II — Paralelismo}

\begin{itemize}
\item Paralelismo de retas
\item Paralelismo entre retas e planos
\item Posições relativas de uma reta e um plano
\item Duas retas reversas
\item Paralelismo entre planos
\item Posições relativas de dois planos
\item Três retas reversas duas a duas
\item Ângulo de duas retas — Retas ortogonais
\end{itemize}
\subsubsection*{CAPÍTULO III — Perpendicularidade}

\begin{itemize}
\item Reta e plano perpendiculares
\item Planos perpendiculares
\end{itemize}
\subsubsection*{CAPÍTULO IV — Aplicações}

\begin{itemize}
\item Projeção ortogonal sobre um plano
\item Segmento perpendicular e segmentos oblíquos a um plano por um ponto
\item Distâncias geométricas
\item Ângulo de uma reta com um plano
\item Reta de maior declive de um plano em relação a outro
\item Lugares geométricos
\item Leitura: Tales, Pitágoras e a Geometria demonstrativa
\end{itemize}
\subsubsection*{CAPÍTULO V — Diedros}

\begin{itemize}
\item Definições
\item Seções
\item Diedros congruentes — Bissetor — Medida
\item Seções igualmente inclinadas — Congruência de diedros
\end{itemize}
\subsubsection*{CAPÍTULO VI — Triedros}

\begin{itemize}
\item Conceito e elementos
\item Relações entre as faces
\item Congruência de triedros
\item Triedros polares ou suplementares
\item Critérios ou casos de congruência entre triedros
\item Ângulos poliédricos convexos
\end{itemize}
\subsubsection*{CAPÍTULO VII — Poliedros convexos}

\begin{itemize}
\item Poliedros convexos
\item Poliedros de Platão
\item Poliedros regulares
\end{itemize}
\subsubsection*{CAPÍTULO VIII — Prisma}

\begin{itemize}
\item Prisma ilimitado
\item Prisma
\item Paralelepípedos e romboedros
\item Diagonal e área do cubo
\item Diagonal e área do paralelepípedo retângulo
\item Razão entre paralelepípedos retângulos
\item Volume de um sólido
\item Volume do paralelepípedo retângulo e do cubo
\item Área lateral e área total do prisma
\item Princípio de Cavalieri
\item Volume do prisma
\item Seções planas do cubo
\item Problemas gerais sobre prismas
\item Leitura: Cavalieri e os indivisíveis
\end{itemize}
\subsubsection*{CAPÍTULO IX — Pirâmide}

\begin{itemize}
\item Pirâmide ilimitada
\item Pirâmide
\item Volume da pirâmide
\item Área lateral e área total da pirâmide
\end{itemize}
\subsubsection*{CAPÍTULO X — Cilindro}

\begin{itemize}
\item Preliminar: noções intuitivas de geração de superfícies cilíndricas
\item Cilindro
\item Áreas lateral e total
\item Volume do cilindro
\end{itemize}
\subsubsection*{CAPÍTULO XI — Cone}

\begin{itemize}
\item Preliminar: noções intuitivas de geração de superfícies cônicas
\item Cone
\item Áreas lateral e total
\item Volume do cone
\end{itemize}
\subsubsection*{CAPÍTULO XII — Esfera}

\begin{itemize}
\item Definições
\item Área e volume
\item Fuso e cunha
\item Dedução das fórmulas das áreas do cilindro, do cone e da esfera
\item Leitura: Lobachevski e as geometrias não euclidianas
\end{itemize}
\subsubsection*{CAPÍTULO XIII — Sólidos semelhantes — Troncos}

\begin{itemize}
\item Seção de uma pirâmide por um plano paralelo à base
\item Tronco de pirâmide de bases paralelas
\item Tronco de cone de bases paralelas
\item Problemas gerais sobre sólidos semelhantes e troncos
\item Tronco de prisma triangular
\item Tronco de cilindro
\end{itemize}
\subsubsection*{CAPÍTULO XIV — Inscrição e circunscrição de sólidos}

\begin{itemize}
\item Esfera e cubo
\item Esfera e octaedro regular
\item Esfera e tetraedro regular
\item Inscrição e circunscrição envolvendo poliedros regulares
\item Prisma e cilindro
\item Pirâmide e cone
\item Prisma e pirâmide
\item Cilindro e cone
\item Cilindro e esfera
\item Esfera e cone reto
\item Esfera, cilindro equilátero e cone equilátero
\item Esfera e tronco de cone
\item Exercícios gerais sobre inscrição e circunscrição de sólidos
\end{itemize}
\subsubsection*{CAPÍTULO XV — Superfícies e sólidos de revolução}

\begin{itemize}
\item Superfícies de revolução
\item Sólidos de revolução
\end{itemize}
\subsubsection*{CAPÍTULO XVI — Superfícies e sólidos esféricos}

\begin{itemize}
\item Superfícies — Definições
\item Áreas das superfícies esféricas
\item Sólidos esféricos: definições e volumes
\item Deduções das fórmulas de volumes dos sólidos esféricos
\item Leitura: Riemann, o grande filósofo da Geometria
\end{itemize}
\subsubsection*{Respostas dos Exercícios}

\subsubsection*{Questões de vestibulares}

\subsubsection*{Respostas das questões de vestibulares}
 
 \subsection{Fundamentos da Matemática Elementar; Volume 11 ;  Matemática comercial ,Matemática financeira e Estatística descritiva}
 
 \subsubsection*{CAPÍTULO I — Matemática comercial}

\begin{itemize}
\item Razões e proporções
\item Grandezas diretamente e inversamente proporcionais
\item Porcentagem
\item Variação percentual
\item Taxas de inflação
\end{itemize}
\subsubsection*{CAPÍTULO II — Matemática financeira}

\begin{itemize}
\item Capital, juros, taxa de juros e montante
\item Regimes de capitalização
\item Juros simples
\item Descontos simples
\item Juros compostos
\item Juros compostos com taxa de juros variáveis
\item Valor atual de um conjunto de capitais
\item Sequência uniforme de pagamentos
\item Montante de uma sequência uniforme de depósitos
\item Leitura: Richard Price e a sequência uniforme de capitais
\end{itemize}
\subsubsection*{CAPÍTULO III — Estatística descritiva}

\begin{itemize}
\item Introdução
\item Variável
\item Tabelas de frequência
\item Representação gráfica
\item Gráfico de setores
\item Gráfico de barras
\item Histograma
\item Gráfico de linhas (poligonal)
\item Medidas de centralidade e variabilidade
\item Média aritmética
\item Média aritmética ponderada
\item Mediana
\item Moda
\item Variância
\item Desvio padrão
\item Medidas de centralidade e dispersão para dados agrupados
\item Outras medidas de separação de dados
\item Leitura: Florence Nightingale e os gráficos estatísticos
\item Leitura: Jerzy Neyman e os intervalos de confiança
\end{itemize}
\subsubsection*{APÊNDICE I — Média geométrica}

\subsubsection*{APÊNDICE II — Média harmônica}

\subsubsection*{Respostas dos exercícios}

\subsubsection*{Questões de vestibulares}

\subsubsection*{Respostas das questões de vestibulares}

\subsubsection*{Significado das siglas de vestibulares}

\newpage

\section{\textbf{Professor de História}}

Link : \href{https://chatgpt.com/g/g-0azinr48l-professor-historia}{Professor de História}

\subsection{História Geral e do Brasil ; Volume 1}
\subsubsection*{Bastidores da História}

\begin{itemize}
\item Estudar História: vários viajantes, múltiplos caminhos
\item Fontes históricas
\item As múltiplas formas de exercitar o poder
\item Leituras do tempo
\end{itemize}
\subsubsection*{UNIDADE 1 - Os Primeiros Agrupamentos Humanos}

\begin{itemize}
\item Discutindo a História
\item Progresso e atraso cultural
\end{itemize}
\subsubsection*{1 Em Busca de Nossos Ancestrais}

\begin{itemize}
\item Para pensar historicamente: espaço e tempo
\item África: nosso lugar de origem
\item A vida em grupo
\item O domínio sobre a natureza
\item Para recordar (esquema-resumo)
\item Exercícios de História
\end{itemize}
\subsubsection*{2 A Ocupação do Continente em Que Vivemos}

\begin{itemize}
\item Para pensar historicamente: ocupação e cultura
\item Descobrindo caminhos
\item Diversidade de culturas
\item Para recordar (esquema-resumo)
\item Exercícios de História
\item Questões e testes
\end{itemize}
\subsubsection*{UNIDADE 2 - Civilizações Antigas}

\begin{itemize}
\item Discutindo a História
\item Estudando civilizações antigas
\item Civilização
\item O Oriente Próximo e o Médio: Mesopotâmia, Egito e os Hebreus
\item O Extremo Oriente: Índia e China
\item América e África
\item Antiguidade clássica: Grécia e Roma
\end{itemize}
\subsubsection*{3 A Vida em Cidades}

\begin{itemize}
\item Para pensar historicamente: Cidades na História
\item Dos grupos nômades às cidades
\item Exercícios de História
\item Das cidades aos reinos e impérios
\item A civilização mesopotâmica
\item Exercícios de História
\item A civilização egípcia
\item Para recordar (esquema-resumo)
\item Exercícios de História
\item A civilização dos hebreus, fenícios e persas
\item Os hebreus
\item Os fenícios e os persas
\item Para recordar (esquema-resumo)
\item Exercícios de História
\item O Extremo Oriente: Índia e China
\item A Índia antiga
\item A China antiga
\item Para recordar (esquema-resumo)
\item Exercícios de História
\item América e África
\item Para recordar (esquema-resumo)
\item Exercícios de História
\end{itemize}
\subsubsection*{4 A Grécia Antiga}

\begin{itemize}
\item Para pensar historicamente: Democracia e cidadania
\item O legado grego
\item Geografia e História
\item Civilizações cretense e micênica
\item Do período homérico ao período arcaico (XII a.C.-VI a.C.)
\item Para recordar (esquema-resumo)
\item Exercícios de História
\item Períodos clássico e helenístico
\item A cultura grega
\item Para recordar (esquema-resumo)
\item Exercícios de História
\end{itemize}
\subsubsection*{5 A Civilização Romana}

\begin{itemize}
\item Para pensar historicamente: Roma e nós
\item As fontes históricas para o estudo de Roma
\item Monarquia (da fundação de Roma ao século VI a.C.)
\item República (séculos VI a.C.-I a.C.)
\item Para recordar (esquema-resumo)
\item Exercícios de História
\item O Alto Império (séculos I a.C.-III d.C.)
\item O Baixo Império (séculos III d.C.-V d.C.)
\item A cultura romana
\item Para recordar (esquema-resumo)
\item Exercícios de História
\item Questões e testes
\end{itemize}
\subsubsection*{UNIDADE 3 - A Europa, Periferia do Mundo}

\begin{itemize}
\item Discutindo a História
\item Por que Idade Média?
\item Idade Média — Idade das Trevas?
\item Idade Média — onde?
\item Por que estudar a Idade Média?
\end{itemize}
\subsubsection*{6 O Império Bizantino, o Islã e o Panorama Mundial}

\begin{itemize}
\item Para pensar historicamente: Permanências e mudanças
\item O Império Romano com capital em Bizâncio
\item Exercícios de História
\item E quem não estava no século V d.C.?
\item Na África
\item Os árabes e o islamismo
\item Na China
\item Na América
\item Para recordar (esquema-resumo)
\item Exercícios de História
\end{itemize}
\subsubsection*{7 O Surgimento da Europa}

\begin{itemize}
\item Para pensar historicamente: O espaço como construção social e histórica
\item A Alta Idade Média
\item Para recordar (esquema-resumo)
\item Exercícios de História
\item Baixa Idade Média
\item Para recordar (esquema-resumo)
\item Exercícios de História
\end{itemize}
\subsubsection*{8 Economia, Sociedade e Cultura Medieval}

\begin{itemize}
\item Para pensar historicamente: Subordinação e dominação
\item Islâmicos e bizantinos na contramão da Europa feudal
\item A Igreja medieval
\item Para recordar (esquema-resumo)
\item Exercícios de História
\item A cultura na época medieval
\item Baixa Idade Média: dinamização cultural
\item O conhecimento em todo o mundo
\item Para recordar (esquema-resumo)
\item Exercícios de História
\end{itemize}
\subsubsection*{9 O Mundo às Vésperas do Século XVI}

\begin{itemize}
\item Para pensar historicamente: As origens dos Estados modernos
\item Formação das monarquias centralizadas na Europa
\item Uma volta ao mundo antes de 1500…
\item Para recordar (esquema-resumo)
\item Exercícios de História
\item Questões e testes
\end{itemize}
\subsubsection*{Sugestões de Leitura para o Aluno}

\subsubsection*{Bibliografia}

\subsubsection*{Respostas dos Testes}

\subsubsection*{Índice Remissivo}

\subsection{História Geral e do Brasil ; Volume 2}
\subsubsection*{UNIDADE 1 - Europa, O Centro do Mundo}

\begin{itemize}
\item Discutindo a História
\item História do Brasil e História geral: duas histórias?
\item Idade Moderna: Europa como centro do mundo
\end{itemize}
\subsubsection*{1 A Expansão Europeia}

\begin{itemize}
\item Para pensar historicamente: A unificação do mundo
\item Emergindo da Idade Média
\item O Estado moderno
\item As navegações portuguesas
\item Para recordar (esquema-resumo)
\item Exercícios de História
\item Uma questão histórica: por que a China não descobriu a Europa?
\item As navegações espanholas
\item O mercantilismo
\item Para recordar (esquema-resumo)
\item Exercícios de História
\end{itemize}
\subsubsection*{2 A Colônia Portuguesa na América}

\begin{itemize}
\item Para pensar historicamente: Projeto colonial para servir a quem?
\item A gradativa tomada de posse
\item O projeto agrícola da exploração colonial portuguesa
\item As capitanias hereditárias e os governos-gerais
\item A União Ibérica e a América colonial (1580-1640)
\item A administração colonial portuguesa e os poderes locais
\item Para recordar (esquema-resumo)
\item Exercícios de História
\end{itemize}
\subsubsection*{3 A Diáspora Africana}

\begin{itemize}
\item Para pensar historicamente: Deslocamentos populacionais forçados
\item Povos africanos na época moderna
\item A inserção do escravismo no sistema econômico mundial
\item Para recordar (esquema-resumo)
\item Exercícios de História
\end{itemize}
\subsubsection*{4 Arte e Tecnologia}

\begin{itemize}
\item Para pensar historicamente: Cultura, arte e tecnologia
\item A efervescência cultural europeia: o Renascimento
\item Cidades italianas: origem do Renascimento
\item O Renascimento em outras regiões da Europa
\item Renascimento além da arte
\item Arte e tecnologia na Índia após as grandes navegações europeias
\item Para recordar (esquema-resumo)
\item Exercícios de História
\end{itemize}
\subsubsection*{5 O Cristianismo em Transformação}

\begin{itemize}
\item Para pensar historicamente: Vida material e mentalidade
\item O contexto da Reforma
\item A Reforma Católica
\item Guerras religiosas
\item Efeitos das Reformas na América Ibérica
\item Para recordar (esquema-resumo)
\item Exercícios de História
\end{itemize}
\subsubsection*{6 O Caminho das Monarquias Europeias}

\begin{itemize}
\item Para pensar historicamente: Política e moral
\item Pensadores do Estado moderno
\item A monarquia francesa
\item A monarquia inglesa
\item A monarquia espanhola: o caso de Felipe II
\item Para recordar (esquema-resumo)
\item Exercícios de História
\end{itemize}
\subsubsection*{7 América Portuguesa: Expansão e Diversidade Econômica}

\begin{itemize}
\item Para pensar historicamente: Economia e sociedade
\item As invasões de nações europeias
\item Outras atividades e expansão territorial
\item Para recordar (esquema-resumo)
\item Exercícios de História
\end{itemize}
\subsubsection*{8 A América Espanhola e A América Inglesa}

\begin{itemize}
\item Para pensar historicamente: Cultura, dominação e refugiados
\item América espanhola: a conquista das civilizações pré-colombianas
\item A exploração da América espanhola
\item Exercícios de História
\item A América inglesa
\item As treze colônias inglesas
\item Para recordar (esquema-resumo)
\item Exercícios de História
\end{itemize}
\subsubsection*{9 Apogeu e Desagregação do Sistema Colonial}

\begin{itemize}
\item Para pensar historicamente: Negociação e enfrentamento
\item A atividade mineradora: interiorização e urbanização
\item A crise portuguesa e o reforço do controle colonial
\item Os confrontos coloniais: alguns destaques
\item Para recordar (esquema-resumo)
\item Exercícios de História
\end{itemize}
\subsubsection*{10 O Iluminismo e A Independência das Colônias Inglesas da América do Norte}

\begin{itemize}
\item Para pensar historicamente: Liberalismo político e democracia
\item A emergência do Iluminismo
\item A queda do Antigo Regime e a era das revoluções
\item Exercícios de História
\item A fundação dos Estados Unidos da América
\item Para recordar (esquema-resumo)
\item Exercícios de História
\item Questões e testes
\end{itemize}
\subsubsection*{UNIDADE 2 - Para Entender Nosso Tempo: O Século XIX}

\begin{itemize}
\item Discutindo a História
\item O longo século XIX
\item Revolução Francesa – Leituras
\item As independências do Brasil
\item Revolução Industrial
\end{itemize}
\subsubsection*{11 Uma Era de Revoluções}

\begin{itemize}
\item Para pensar historicamente: Processos revolucionários
\item Revolução Inglesa, Revolução Industrial
\item Exercícios de História
\item Revolução Francesa
\item Balanço das revoluções
\item Para recordar (esquema-resumo)
\item Exercícios de História
\end{itemize}
\subsubsection*{12 “Colando os Cacos” do Poder Monárquico}

\begin{itemize}
\item Para pensar historicamente: Projetos políticos em confronto
\item A ascensão de Napoleão Bonaparte
\item Napoleão e o Império (1804–1815)
\item Rio de Janeiro, sede da monarquia portuguesa (1808-1821)
\item O fim do Império napoleônico
\item Da Revolução Francesa à Revolução Haitiana
\item O congresso de Viena
\item Para recordar (esquema-resumo)
\item Exercícios de História
\end{itemize}
\subsubsection*{13 Brasil: Surge um País}

\begin{itemize}
\item Para pensar historicamente: Os limites da independência
\item Conspirações contra a ordem colonial
\item O período Joanino e a Independência
\item Para recordar (esquema-resumo)
\item Exercícios de História
\end{itemize}
\subsubsection*{14 As Independências na América Espanhola}

\begin{itemize}
\item Para pensar historicamente: Américas e seus processos de independência
\item Preparando o cenário das independências
\item As guerras de independência
\item Para recordar (esquema-resumo)
\item Exercícios de História
\end{itemize}
\subsubsection*{15 Novos Projetos Políticos: Liberalismo, Socialismo e Nacionalismo}

\begin{itemize}
\item Para pensar historicamente: A conquista do futuro
\item O pensamento liberal
\item As doutrinas socialistas
\item O nacionalismo
\item As lutas trabalhistas e as internacionais operárias
\item Para recordar (esquema-resumo)
\item Exercícios de História
\end{itemize}
\subsubsection*{16 Europa e Estados Unidos no Século XIX}

\begin{itemize}
\item Para pensar historicamente: Liberalismo, socialismo, nacionalismo e imperialismo
\item Um mundo em movimento
\item A Segunda Revolução Industrial
\item Inglaterra e a Era Vitoriana
\item A França no século XIX
\item Portugal e Espanha
\item Os Estados Unidos no século XIX
\item Para recordar (esquema-resumo)
\item Exercícios de História
\end{itemize}
\subsubsection*{17 A Construção do Estado Brasileiro}

\begin{itemize}
\item Para pensar historicamente: Projetos para o Brasil
\item A consolidação de um projeto (1822-1831)
\item O período Regencial
\item Outros projetos: as rebeliões
\item Para recordar (esquema-resumo)
\item Exercícios de História
\end{itemize}
\subsubsection*{18 África e Ásia no Século XIX}

\begin{itemize}
\item Para pensar historicamente: O “outro” na expansão imperialista
\item Práticas imperialistas
\item A marca do colonialismo na África
\item O colonialismo europeu na Ásia
\item Para recordar (esquema-resumo)
\item Exercícios de História
\end{itemize}
\subsubsection*{19 O Segundo Reinado no Brasil}

\begin{itemize}
\item Para pensar historicamente: Sociedade escravista em ebulição
\item Economia e sociedade
\item A evolução política do Segundo Reinado
\item A política externa e o declínio do império oligárquico
\item O fim do Império
\item Para recordar (esquema-resumo)
\item Exercícios de História
\item Questões e testes
\end{itemize}
\subsubsection*{Sugestões de Leitura para o Aluno}

\subsubsection*{Bibliografia}

\subsubsection*{Respostas dos Testes}

\subsubsection*{Índice Remissivo}

\subsection{História Geral e do Brasil ; Volume 3}
\subsubsection*{UNIDADE 1 - Para Entender Nosso Tempo: O Século XX}

\begin{itemize}
\item Discutindo a História
\item Século XX – A História se acelera?
\item O socialismo e as guerras
\item História em múltiplos focos
\end{itemize}
\subsubsection*{1 O Brasil, Uma República (1889-1914)}

\begin{itemize}
\item Para pensar historicamente: O nascimento da república e a inclusão social
\item Diferentes projetos republicanos
\item O governo provisório de Deodoro da Fonseca (1889-1891)
\item A “República da Espada”
\item Transição para o poder civil
\item O apogeu da ordem oligárquica (1898-1914)
\item As lutas sociais
\item Mecanismos políticos do poder oligárquico
\item Para recordar (esquema-resumo)
\item Exercícios de História
\end{itemize}
\subsubsection*{2 Um Mundo em Guerra (1914-1918)}

\begin{itemize}
\item Para pensar historicamente: Um século inaugurado pela guerra
\item A política de alianças
\item A questão balcânica
\item O desenvolvimento do conflito
\item Para recordar (esquema-resumo)
\item Exercícios de História
\end{itemize}
\subsubsection*{3 A Revolução Russa}

\begin{itemize}
\item Para pensar historicamente: Reflexões sobre as experiências históricas
\item A corrosão do czarismo russo
\item O colapso do czarismo
\item A Revolução Menchevique
\item A Revolução Bolchevique
\item O governo de Josef Stálin (1924-1953)
\item Para recordar (esquema-resumo)
\item Exercícios de História
\end{itemize}
\subsubsection*{4 Uma Jovem República Velha (1914-1930)}

\begin{itemize}
\item Para pensar historicamente: Rompimento e permanência
\item Crise política
\item As transformações sociais e econômicas
\item Novos sujeitos na cena histórica
\item O Tenentismo
\item E crescem os confrontos…
\item A Revolução de 1930
\item Para recordar (esquema-resumo)
\item Exercícios de História
\end{itemize}
\subsubsection*{5 A Crise de 1929 e o Nazifascismo}

\begin{itemize}
\item Para pensar historicamente: Economia e política
\item A crise da Bolsa de Nova York e a Grande Depressão
\item O ideário nazifascista
\item Para recordar (esquema-resumo)
\item Exercícios de História
\end{itemize}
\subsubsection*{UNIDADE 2 - Do Pós-Guerra ao Século XXI}

\begin{itemize}
\item Discutindo a História
\item A crise do eurocentrismo
\end{itemize}
\subsubsection*{6 Vargas de 1930 a 1945}

\begin{itemize}
\item Para pensar historicamente: A construção de um mito
\item O governo provisório (1930-1934)
\item O governo constitucional (1934-1937)
\item O Estado Novo (1937-1945)
\item Para recordar (esquema-resumo)
\item Exercícios de História
\end{itemize}
\subsubsection*{7 A Segunda Guerra Mundial (1939-1945)}

\begin{itemize}
\item Para pensar historicamente: Justificando a guerra
\item A guerra reaparece no horizonte
\item O desenvolvimento do conflito
\item Balanço da guerra
\item A fundação da ONU
\item A Europa nos primeiros anos do pós-guerra
\item Para recordar (esquema-resumo)
\item Exercícios de História
\item Questões e testes
\end{itemize}
\subsubsection*{8 O Período Liberal Democrático (1945-1964)}

\begin{itemize}
\item Para pensar historicamente: Uma experiência de democracia
\item Novos ares na política
\item Liberalismo – nacionalismo: projetos para o desenvolvimento
\item O segundo governo de Getúlio Vargas (1951-1954)
\item Exercícios de História
\item O governo de Café Filho (1954-1955)
\item O desenvolvimentismo de Juscelino Kubitschek (1956-1961)
\item O governo de Jânio Quadros (1961)
\item O governo de João Goulart (1961-1964)
\item Para recordar (esquema-resumo)
\item Exercícios de História
\end{itemize}
\subsubsection*{9 O Pós-Guerra e a Guerra Fria}

\begin{itemize}
\item Para pensar historicamente: Centro e periferia
\item A consolidação da Guerra Fria
\item Revolução Chinesa
\item A Guerra da Coreia (1950-1953)
\item Estados Unidos e União Soviética durante a Guerra Fria
\item Os soviéticos até 1964
\item O socialismo na China e em Cuba
\item Para recordar (esquema-resumo)
\item Exercícios de História
\end{itemize}
\subsubsection*{10 Descolonização e Lutas Sociais no “Terceiro Mundo”}

\begin{itemize}
\item Para pensar historicamente: Outros sujeitos na política mundial
\item A descolonização africana e asiática
\item A América Latina e as lutas sociais
\item Para recordar (esquema-resumo)
\item Exercícios de História
\end{itemize}
\subsubsection*{11 O Regime Militar}

\begin{itemize}
\item Para pensar historicamente: Autoritarismo e dependência econômica
\item Regimes militares
\item A montagem da ditadura
\item A ditadura total (1968-1977)
\item A abertura (1977-1985)
\item Para recordar (esquema-resumo)
\item Exercícios de História
\end{itemize}
\subsubsection*{12 O Fim da Guerra Fria e a Nova Ordem Mundial}

\begin{itemize}
\item Para pensar historicamente: Os desafios da globalização
\item O fim da Guerra Fria
\item A nova ordem internacional
\item Para recordar (esquema-resumo)
\item Exercícios de História
\end{itemize}
\subsubsection*{13 O Brasil no Século XXI}

\begin{itemize}
\item Para pensar historicamente: Democracia e neoliberalismo
\item O Brasil e a globalização capitalista
\item O governo de José Sarney (1985-1990)
\item O governo de Fernando Collor de Mello (1990-1992)
\item O governo de Itamar Franco (1992-1995)
\item O governo de Fernando Henrique Cardoso (1995-2002)
\item Primeiro e segundo governos de Luiz Inácio Lula da Silva (2003-2010)
\item O governo de Dilma Rousseff (2011- …)
\item Para recordar (esquema-resumo)
\item Exercícios de História
\item Questões e testes
\end{itemize}
\subsubsection*{Sugestões de Leitura para o Aluno}

\subsubsection*{Bibliografia}

\subsubsection*{Respostas dos Testes}

\subsubsection*{Índice Remissivo}

\subsection{História Concisa do Brasil (Boris Fausto)}
\subsubsection*{1. O Brasil Colonial (1500-1822)}

\begin{itemize}
\item A Expansão Marítima e a Chegada dos Portugueses ao Brasil
\item Os Índios
\item A Colonização
\item A Sociedade Colonial
\item As Atividades Econômicas
\item A União Ibérica e seus Reflexos no Brasil
\item A Colonização da Periferia
\item As Bandeiras e a Sociedade Paulista
\item Um Balanço da Economia Colonial: O Mercado Interno
\item A Crise do Sistema Colonial
\item Movimentos de Rebeldia e Consciência Nacional
\item O Brasil no Fim do Período Colonial
\end{itemize}
\subsubsection*{2. O Brasil Monárquico (1822-1889)}

\begin{itemize}
\item A Consolidação da Independência e a Construção do Estado
\item O Segundo Reinado
\item A Estrutura Socioeconômica e a Escravidão
\item A Modernização e a Expansão Cafeeira
\item O Início da Grande Imigração
\item A Guerra do Paraguai
\item A Crise do Segundo Reinado
\item O Republicanismo
\item A Queda da Monarquia
\item Economia e Demografia
\end{itemize}
\subsubsection*{3. A Primeira República (1889-1930)}

\begin{itemize}
\item Os Anos de Consolidação
\item As Oligarquias e os Coronéis
\item Relações entre a União e os Estados
\item As Mudanças Socioeconômicas
\item Os Movimentos Sociais
\item O Processo Político nos Anos 1920
\item A Revolução de 1930
\end{itemize}
\subsubsection*{4. O Estado Getulista (1930-1945)}

\begin{itemize}
\item A Ação Governamental
\item O Processo Político
\item O Estado Novo
\item O Fim do Estado Novo
\item O Quadro Socioeconômico
\end{itemize}
\subsubsection*{5. A Experiência Democrática (1945-1964)}

\begin{itemize}
\item As Eleições e a Nova Constituição
\item O Retorno de Getúlio
\item A Queda de Getúlio
\item Do Nacionalismo ao Desenvolvimentismo
\item A Crise do Regime e o Golpe de 1964
\end{itemize}
\subsubsection*{6. O Regime Militar e a Transição para a Democracia (1964-1990)}

\begin{itemize}
\item A Modernização Conservadora
\item O Fechamento Político e a Luta Armada
\item O Processo de Abertura Política
\item O Quadro Estrutural de 1950 em Diante
\item Gráfico 1. Taxa de Fecundidade Total (1940-2000)
\item Gráfico 2. Taxas de Analfabetismo da População de 15 Anos ou mais de Idade (1900-2020)
\item Gráfico 3. Taxas de Analfabetismo da População de 15 Anos ou mais de Idade por Países Selecionados (1995)
\item Tabela 1. Taxas de Escolarização das Pessoas de 7 a 14 Anos por Situação do Domicílio (1997)
\item Gráfico 4. Gastos Totais com Educação como Percentual do Produto Nacional Bruto, por Países Selecionados (1996)
\item Gráfico 5. Esperança de Vida ao Nascer (1930-2000)
\item Gráfico 6. Qualidade de Vida. Infraestrutura. Percentual de Domicílios Brasileiros Atendidos por Alguns Serviços
\item Gráfico 7. Concentração de Renda no Brasil. Rendimentos sobre o Total da Renda do País, em Porcentagem
\item Conclusão
\item Descrições dos Gráficos para Acessibilidade
\end{itemize}
\subsubsection*{7. A Modernização pela Via Democrática (1990-2010) • SÉRGIO FAUSTO}

\begin{itemize}
\item Introdução
\item O Governo Sarney: Democratização e Descontrole Inflacionário
\item A Crise Política e a Mudança Econômica: O Breve Mandato de Fernando Collor
\item O Governo Itamar Franco: Da Crise Política ao Plano Real
\item O Governo FHC : Estabilização Econômica e Reformas Estruturais
\item O Encerramento da Era Vargas
\item A Batalha das Privatizações
\item As Batalhas no Congresso
\item A Sinuosa Estrada da Estabilização
\item O Debate Sobre a Política Econômica
\item Os Efeitos Colaterais da Queda da Inflação
\item O Fim do Gradualismo e o Teste da Desvalorização Cambial
\item A Estabilização e a Modernização da Economia: Breve Balanço
\item A Democracia e os Direitos Sociais
\item A Crise das Metrópoles: (In)Segurança e (Des)Emprego
\item A Política Externa: A Autonomia pela Inserção
\item Uma Herança Maldita ou Bendita?
\item A Alternância no Poder: A Vitória de Lula
\item A Formação do Novo Governo
\item Lula e o Presidencialismo de Coalizão
\item A Aceleração do Crescimento e a Distribuição de Renda
\item A Política Econômica no Primeiro Mandato
\item O Surgimento de uma Nova Classe Média
\item A Política Econômica no Segundo Mandato
\item A Política Social: Mais Continuidade do que Ruptura
\item A Crise Política e a Crise da Política: Sucessão de Escândalos
\item O Mensalão
\item Um Novo Modelo de Desenvolvimento?
\item A Controvérsia sobre o Pré-Sal
\item O Crescimento e o Meio Ambiente
\item A Posição do Brasil nas Negociações sobre o Clima
\item A Identidade da Política Externa
\item A América do Sul no Centro da Agenda
\item As Relações com os Estados Unidos
\item A Sucessão de Lula
\item A Democracia sob Lula
\item Conclusão
\end{itemize}

\subsection{História do Brasil (Boris Fausto)}
\subsubsection*{Introdução}

\subsubsection*{1. As Causas da Expansão Marítima e a Chegada dos Portugueses ao Brasil}

\begin{itemize}
\item O Gosto pela Aventura
\item O Desenvolvimento das Técnicas de Navegação. A Nova Mentalidade
\item A Atração pelo Ouro e pelas Especiarias
\item A Ocupação da Costa Africana e as Feitorias
\item A Ocupação das Ilhas do Atlântico
\item A Chegada ao Brasil
\end{itemize}
\subsubsection*{2. O Brasil Colonial (1500-1822)}

\begin{itemize}
\item Os Índios
\item Os Períodos do Brasil Colonial
\item Tentativas Iniciais de Exploração
\item Início da Colonização - As Capitanias Hereditárias
\item O Governo Geral
\item A Colonização se Consolida
\item O Trabalho Compulsório
\item A Escravidão - Índios e Negros
\item O Mercantilismo
\item O “Exclusivo” Colonial
\item A Grande Propriedade Agroexportadora e a Acumulação Urbana
\item Estado e Igreja
\item O Estado Absolutista e o “Bem Comum”
\item As Instituições da Administração Colonial
\item As Divisões Sociais
\item A Pureza de Sangue
\item Livres e Escravos
\item Escravos e Escravos
\item Livres e Libertos
\item Nobreza, Clero e Povo
\item Hierarquia das Profissões
\item Os que Mandam
\item Discriminação Religiosa
\item Discriminação Sexual
\item Cidade e Campo
\item Estado e Sociedade
\item As Primeiras Atividades Econômicas
\item O Açúcar
\item O Fumo
\item A Pecuária
\item As Invasões Holandesas
\item A Colonização do Norte
\item A Colonização do Sudeste e do Centro-sul
\item A Expansão da Agropecuária
\item As Bandeiras e a Sociedade Paulista
\item Ouro e Diamantes
\item A Coroa e o Controle das Minas
\item A Sociedade das Minas
\item A Crise do Antigo Regime
\item O Pensamento Ilustrado e o Liberalismo
\item A Crise do Sistema Colonial
\item A Administração Pombalina
\item O Reinado de Dona Maria
\item Os Movimentos de Rebeldia
\item A Inconfidência Mineira
\item A Conjuração dos Alfaiates
\item A Vinda da Família Real para o Brasil
\item A Abertura dos Portos
\item A Corte no Rio de Janeiro
\item A Revolução Pernambucana de 1817
\item A Independência
\item O Brasil no Fim do Período Colonial
\end{itemize}
\subsubsection*{3. O Primeiro Reinado (1822-1831)}

\begin{itemize}
\item A Consolidação da Independência
\item Uma Transição sem Abalos
\item A Constituinte
\item A Constituição de 1824
\item A Confederação do Equador
\item A Abdicação de Dom Pedro I
\end{itemize}
\subsubsection*{4. A Regência (1831-1840)}

\begin{itemize}
\item As Reformas Institucionais
\item As Revoltas Provinciais
\item As Revoltas no Norte e no Nordeste
\item A Guerra dos Farrapos
\item A Política no Período Regencial
\end{itemize}
\subsubsection*{5. O Segundo Reinado (1840-1889)}

\begin{itemize}
\item O “Regresso”
\item A Luta Contra o Império Centralizado
\item O Acordo das Elites e o “Parlamentarismo”
\item Os Partidos: Semelhanças e Diferenças
\item A Preservação da Unidade Territorial
\item A Estrutura Socioeconômica e a Escravidão
\item A Economia Cafeeira
\item O Tráfico de Escravos e sua Extinção
\item Em Busca da Modernização Capitalista
\item A Expansão Cafeeira no Oeste Paulista
\item O Início da Grande Imigração
\item A Guerra do Paraguai
\item A Crise do Segundo Reinado (1870-1889)
\item O Fim da Escravidão
\item As Controvérsias Sobre a Escravidão
\item O Republicanismo
\item As Tensões Entre Estado e Igreja
\item O Papel dos Militares
\item O Positivismo
\item O Reformismo do Império
\item Problemas com os Militares
\item A Queda da Monarquia
\item Balanço Econômico e Populacional
\end{itemize}
\subsubsection*{6. A Primeira República (1889-1930)}

\begin{itemize}
\item A Primeira Constituição Republicana
\item O Encilhamento
\item Deodoro na Presidência
\item Floriano Peixoto
\item A Revolução Federalista
\item Prudente de Morais
\item Canudos
\item Campos Sales
\item A Política dos Governadores
\item Os Problemas Financeiros
\item Características Políticas da Primeira República
\item As Oligarquias
\item Os Coronéis
\item Relações Entre a União e os Estados
\item O Estado e a Burguesia do Café
\item Principais Mudanças Socioeconômicas - 1890 a 1930
\item A Imigração
\item O Peso das Atividades Agrícolas
\item A Urbanização
\item A Industrialização
\item A Diversificação Econômica e o Rio Grande do Sul
\item A Borracha Amazônica
\item Relações Financeiras Internacionais
\item Os Movimentos Sociais
\item Movimentos Sociais no Campo
\item Movimentos Sociais Urbanos
\item O PCB
\item O Processo Político nos Anos 20
\item O Tenentismo
\item As Elites Civis
\item A Revolução de 1930
\item Os Jovens Políticos e os Tenentes
\item O Estopim da Revolução
\item As Ações Militares
\item Uma Complexa Base Social e Política
\end{itemize}
\subsubsection*{7. O Estado Getulista (1930-1945)}

\begin{itemize}
\item A Colaboração Entre o Estado e a Igreja
\item A Centralização
\item A Política do Café
\item A Política Trabalhista
\item A Educação
\item O Processo Político (1930-1934)
\item O Tenentismo e a Luta Contra as Oligarquias
\item A Revolução de 1932
\item A Constitucionalização
\item A Gestação do Estado Novo
\item O Integralismo
\item O Autoritarismo e a Modernização Conservadora
\item O Fortalecimento do Exército
\item O Processo Político (1934-1937)
\item O Estado Novo
\item A Carta de 1937 e a Centralização
\item Estado e Sociedade
\item O Aparelho do Estado
\item O Fim do Estado Novo
\item As Mudanças Ocorridas no Brasil Entre 1920 e 1940
\item População
\item Urbanização
\item Economia
\item Educação
\end{itemize}
\subsubsection*{8. O Período Democrático (1945-1964)}

\begin{itemize}
\item A Eleição de Dutra
\item A Constituição de 1946
\item O Governo Dutra
\item Liberalismo ou Controle Estatal?
\item A Sucessão de Dutra
\item O Novo Governo Vargas
\item Divisões no Exército: Nacionalistas Versus “Entreguistas”
\item O Quadro Econômico-Financeiro
\item A Política Trabalhista e as Greves
\item O Janismo
\item A Oposição
\item A Queda de Getúlio Vargas
\item A Eleição de Juscelino Kubitschek
\item O Golpe Preventivo do General Lott
\item O Governo JK
\item As Forças Armadas e os Partidos
\item O Programa de Metas
\item O Movimento Operário e a Organização Sindical
\item As Dificuldades do Governo
\item A Sucessão Presidencial
\item O Governo Jânio Quadros
\item Política Externa
\item Política Financeira
\item A Renúncia
\item A Sucessão de Jânio
\item O Governo João Goulart
\item As Ligas Camponesas
\item Os Estudantes
\item A Igreja Católica
\item As Reformas de Base e o Movimento Operário
\item A Política
\item As Forças Armadas
\item O Período Parlamentarista
\item A Volta do Presidencialismo
\item O Golpe de 1964
\end{itemize}\subsubsection*{9. O Regime Militar (1964-1985)}

\begin{itemize}
\item O Ato Institucional 1 e a Repressão
\item O Governo Castelo Branco
\item O PAEG
\item A Política
\item O Governo Costa e Silva
\item A Oposição se Rearticula
\item Início da Luta Armada
\item O AI-5
\item A Junta Militar
\item O Governo Médici
\end{itemize}
\subsubsection*{10. Completa-se a Transição: o Governo Sarney (1985-1989)}

\begin{itemize}
\item Política Econômica
\item O Plano Cruzado
\item As Eleições de 1986
\item A Assembléia Nacional Constituinte
\item A Transição Avaliada
\end{itemize}
\subsubsection*{11. Principais Mudanças Ocorridas no Brasil entre 1950 e 1980}

\begin{itemize}
\item População
\item Distribuição Regional da População
\item Urbanização
\item Economia
\item A Agricultura e a Agroindústria
\item A Industrialização
\item Indicadores Sociais
\item Educação
\item Outros Indicadores
\end{itemize}
\subsubsection*{12. A Nova Ordem Mundial e o Brasil}

\subsubsection*{Cronologia Histórica}

\begin{itemize}
\item Brasil 1500-1993
\item Mundo 1500-1993
\end{itemize}

\newpage
\section{Gramática Jovem}
\href{https://chatgpt.com/g/g-U5U26T3Vo}{Gramática Jovem}

\subsection{Língua Portuguesa (Adriano Alves)}
\subsubsection*{Unidade I – Teoria da Comunicação}
\begin{itemize}
    \item Capítulo 1 – Linguagem, Língua, Fala, Signo Linguístico, Linguagem Verbal e Linguagem não Verbal
    \item Capítulo 2 – Variações Linguísticas
    \item Capítulo 3 – Elementos da Comunicação e Funções da Linguagem
\end{itemize}

\subsubsection*{Unidade II – Introdução à Semântica}
\begin{itemize}
    \item Capítulo 4 – Significação das Palavras
    \item Capítulo 5 – Polissemia e Ambiguidade
    \item Capítulo 6 – Implícitos
    \item Capítulo 7 – Ideia Principal; Ideias Periféricas e Inferência
\end{itemize}

\subsubsection*{Unidade III – O Texto}
\begin{itemize}
    \item Capítulo 8 – Interdiscursividade e Intertextualidade
    \item Capítulo 9 – Coesão
    \item Capítulo 10 – Operadores Argumentativos, Modalizadores e Elementos Dêiticos
    \item Capítulo 11 – Paralelismo
    \item Capítulo 12 – Coerência
    \item Capítulo 13 – Tipologia Textual
    \item Capítulo 14 – Discursos
\end{itemize}

\subsubsection*{Unidade IV – Estilística}
\begin{itemize}
    \item Capítulo 15 – Estilística I (Denotação/Conotação; Comparação/ Metáfora)
    \item Capítulo 16 – Estilística II (Desdobramentos da Metáfora)
    \item Capítulo 17 – Estilística III (Outras Figuras Ligadas à Semântica)
    \item Capítulo 18 – Estilística IV (Outras Figuras Ligadas à Semântica)
    \item Capítulo 19 – Estilística V (Figuras Ligadas ao Som)
    \item Capítulo 20 – Estilística VI (Figuras Ligadas à Sintaxe)
    \item Capítulo 21 – Estilística VII (Outras Figuras Ligadas à Sintaxe)
    \item Capítulo 22 – Vícios de Linguagem
\end{itemize}

\subsection{Nova Gramático Português Contemporâneo}

\subsubsection*{Introdução – CONCEITOS GERAIS}
\begin{itemize}
    \item Linguagem, língua, discurso, estilo
    \item Língua e sociedade: variação e conservação linguística
    \item Diversidade geográfica da língua: dialeto e falar
    \item A noção de correto
\end{itemize}

\subsubsection*{Capítulo 1 – DO LATIM AO PORTUGUÊS ATUAL}
\begin{itemize}
    \item O latim e a expansão romana
    \item Latim literário e latim vulgar
    \item As línguas românicas
    \item A romanização da Península
    \item O domínio visigótico
    \item O domínio árabe
    \item O português primitivo
    \item Períodos evolutivos da língua portuguesa
\end{itemize}

\subsubsection*{Capítulo 2 – DOMÍNIO ATUAL DA LÍNGUA PORTUGUESA}
\begin{itemize}
    \item Unidade e diversidade
    \item As variedades do português
    \item Os dialetos do português europeu
    \item Os dialetos das ilhas atlânticas
    \item Os dialetos brasileiros
    \item O português de África, da Ásia e da oceânia
\end{itemize}

\subsubsection*{Capítulo 3 – FONÉTICA E FONOLOGIA}
\begin{itemize}
    \item Os sons da fala
    \item O aparelho fonador
    \item Funcionamento do aparelho fonador
    \item Som e fonema
    \item Descrição fonética e fonológica
    \item Transcrição fonética e fonológica
    \item Alfabeto fonético utilizado
    \item Classificação dos sons linguísticos
    \item Vogais e consoantes
    \item Semivogais
    \item Classificação das vogais
    \item Articulação
    \item Timbre
    \item Intensidade e acento
    \item Vogais orais e vogais nasais
    \item Vogais tônicas orais
    \item Vogais tônicas nasais
    \item Vogais átonas orais
    \item Classificação das consoantes
    \item Modo de articulação
    \item O ponto ou zona de articulação
    \item O papel das cordas vocais
    \item O papel das cavidades bucal e nasal
    \item Quadro das consoantes
    \item A posição das consoantes
    \item Encontros vocálicos
    \item Ditongos
    \item Ditongos decrescentes e crescentes
    \item Ditongos orais e nasais
    \item Tritongos
    \item Hiatos
    \item Encontros intraverbais e interverbais
    \item Encontros consonantais
    \item Dígrafos
    \item Sílaba
    \item Sílabas abertas e sílabas fechadas
    \item Classificação das palavras quanto ao número de sílabas
    \item Acento tônico
    \item Classificação das palavras quanto ao acento tônico
    \item Observações sobre a pronúncia culta
    \item Valor distintivo do acento tônico
    \item Acento principal e acento secundário
    \item Grupo acentual (ou de intensidade)
    \item Ênclise e próclise
    \item Acento de insistência
    \item Acento afetivo
    \item Acento intelectual
    \item Distinções fundamentais
\end{itemize}

\subsubsection*{Capítulo 4 – ORTOGRAFIA}
\begin{itemize}
    \item Letra e alfabeto
    \item Notações léxicas
    \item O acento
    \item O til
    \item O trema
    \item O apóstrofo
    \item A cedilha
    \item O hífen
    \item Emprego do hífen nos compostos
    \item Emprego do hífen nas formações por prefixação, recomposição e sufixação
    \item Emprego do hífen na ênclise, na mesóclise e com as formas do verbo haver
    \item Partição das palavras no fim da linha
    \item Regras de acentuação
    \item Divergências entre as ortografias oficialmente adotadas em Portugal e no Brasil
\end{itemize}

\subsubsection*{Capítulo 5 – CLASSE, ESTRUTURA E FORMAÇÃO DE PALAVRAS}
\begin{itemize}
    \item Palavra e morfema
    \item Tipos de morfemas
    \item Classes de palavras
    \item Estrutura das palavras
    \item Radical
    \item Desinência
    \item Afixo
    \item Vogal temática
    \item Vogal e consoante de ligação
    \item Formação de palavras
    \item Palavras primitivas e derivadas
    \item Palavras simples e compostas
    \item Famílias de palavras
\end{itemize}

\subsubsection*{Capítulo 6 – DERIVAÇÃO E COMPOSIÇÃO}
\begin{itemize}
    \item Formação de palavras
    \item Derivação prefixal
    \item Prefixos de origem latina
    \item Prefixos de origem grega
    \item Derivação sufixal
    \item Sufixos nominais
    \item Sufixos verbais
    \item Sufixo adverbial
    \item Derivação parassintética
    \item Derivação regressiva
    \item Derivação imprópria
    \item Formação de palavras por composição
    \item Tipos de composição
    \item Compostos eruditos
    \item Radicais latinos
    \item Radicais gregos
    \item Recomposição
    \item Pseudoprefixos
    \item Hibridismo
    \item Onomatopeia
    \item Abreviação vocabular
    \item Siglas
\end{itemize}

\subsubsection*{Capítulo 7 – FRASE, ORAÇÃO, PERÍODO}
\begin{itemize}
    \item A frase e sua constituição
    \item Frase e oração
    \item Oração e período
    \item A oração e os seus termos essenciais
    \item Sujeito e predicado
    \item Sintagma nominal e verbal
    \item O Sujeito
    \item Representação do sujeito
    \item Sujeito simples e sujeito composto
    \item Sujeito oculto (determinado)
    \item Sujeito indeterminado
    \item Oração sem sujeito
    \item Da atitude do sujeito
    \item Com os verbos de ação
    \item Com os verbos de estado
    \item O predicado
    \item Predicado nominal
    \item Predicado verbal
    \item Verbos intransitivos
    \item Verbos transitivos
    \item Predicado verbo-nominal
    \item Variabilidade da predicação verbal
    \item A oração e os seus termos integrantes
    \item Complemento nominal
    \item Complementos verbais
    \item Objeto direto
    \item Objeto direto preposicionado
    \item Objeto direto pleonástico
    \item Objeto indireto
    \item Objeto indireto pleonástico
    \item Predicativo do objeto
    \item Agente da passiva
    \item Transformação da oração ativa em passiva
    \item A oração e os seus termos acessórios
    \item Adjunto adnominal
    \item Adjunto adverbial
    \item Classificação dos adjuntos adverbiais
    \item Aposto
    \item Valor sintático do aposto
    \item Aposto predicativo
    \item Vocativo
    \item Colocação dos termos na oração
    \item Ordem direta e ordem inversa
    \item Inversões de natureza estilística
    \item Inversões de natureza gramatical
    \item Inversão verbo + sujeito
    \item Inversão predicativo + verbo
    \item Entoação oracional
    \item Grupo acentual e grupo fônico
    \item O grupo fônico, unidade melódica
    \item O grupo fônico e a oração
    \item Oração declarativa
    \item Oração interrogativa
    \item Oração exclamativa
    \item Conclusão
\end{itemize}

\subsubsection*{Capítulo 8 – SUBSTANTIVO}
\begin{itemize}
    \item Classificação dos substantivos
    \item Substantivos concretos e abstratos
    \item Substantivos próprios e comuns
    \item Substantivos coletivos
    \item Flexões dos substantivos
    \item Número
    \item Formação do plural
    \item Gênero
    \item Quanto à significação
    \item Quanto à terminação
    \item Formação do feminino
    \item Substantivos uniformes
    \item Mudança de sentido na mudança de gênero
    \item Substantivos masculinos terminados em -a
    \item Substantivos de gênero vacilante
    \item Grau
    \item Valor das formas aumentativas e diminutivas
    \item Especialização de formas
    \item Emprego do substantivo
    \item Funções sintáticas do substantivo
    \item Substantivo como adjunto adnominal
    \item Substantivo caracterizador de adjetivo
    \item Substantivo caracterizado por um nome
    \item O substantivo como núcleo das frases sem verbo
\end{itemize}

\subsubsection*{Capítulo 9 – ARTIGO}
\begin{itemize}
    \item Artigo definido e indefinido
    \item Formas do artigo
    \item Formas simples
    \item Formas combinadas do artigo definido
    \item Formas combinadas do artigo indefinido
    \item Valores do artigo
    \item A determinação
    \item Emprego do artigo definido
    \item Com os substantivos comuns
    \item Empregos particulares
    \item Emprego genérico
    \item Emprego em expressões de tempo
    \item Emprego com expressões de peso e medida
    \item Com a palavra casa
    \item Com a palavra palácio
    \item Emprego com o superlativo relativo
    \item Com os nomes próprios
    \item Com os nomes de pessoas
    \item Com os nomes geográficos
    \item Com os nomes de obras literárias e artísticas
    \item Casos especiais
    \item Antes da palavra outro
    \item Depois das palavras ambos e todo
    \item Repetição do artigo definido
    \item Com substantivos
    \item Com adjetivos
    \item Omissão do artigo definido
    \item Emprego do artigo indefinido
    \item Com os substantivos comuns
    \item Com os nomes próprios
    \item Omissão do artigo indefinido
    \item Em expressões de identidade
    \item Em expressões comparativas
    \item Em expressões de quantidade
    \item Com substantivo denotador da espécie
    \item Outros casos de omissão do artigo indefinido
\end{itemize}

\subsubsection*{Capítulo 10 – PRONOMES}
\begin{itemize}
    \item Pronomes substantivos e pronomes adjetivos
    \item Pronomes pessoais
    \item Formas dos pronomes pessoais
    \item Formas o, lo e no do pronome oblíquo
    \item Pronomes reflexivos e recíprocos
    \item Emprego dos pronomes retos
    \item Funções dos pronomes retos
    \item Omissão do pronome sujeito
    \item Presença do pronome sujeito
    \item Extensão de emprego dos pronomes retos
    \item Realce do pronome sujeito
    \item Precedência dos pronomes sujeitos
    \item Equívocos e incorreções
    \item Contração das preposições de e em com o pronome reto da 3.ª pessoa
    \item Pronomes de tratamento
    \item Emprego dos pronomes de tratamento da 2.ª pessoa
    \item Fórmulas de representação da 1.ª pessoa
    \item Emprego dos pronomes oblíquos
    \item Formas tônicas
    \item Emprego enfático do pronome oblíquo tônico
    \item Pronomes precedidos de preposição
    \item Formas átonas
    \item O pronome oblíquo átono sujeito de um infinitivo
    \item Emprego enfático do pronome oblíquo átono
    \item O pronome de interesse
    \item Pronome átono com valor possessivo
    \item Pronomes complementos de verbos de regência distinta
    \item Valores e empregos do pronome se
    \item Combinações e contrações dos pronomes átonos
    \item Colocação dos pronomes átonos
\end{itemize}

\subsubsection*{Capítulo 11 – NUMERAIS}
\begin{itemize}
    \item Espécies de numerais
    \item Numerais coletivos
    \item Flexão dos numerais
    \item Cardinais
    \item Ordinais
    \item Multiplicativos
    \item Fracionários
    \item Numerais coletivos
    \item Valores e empregos dos cardinais
    \item Cardinal como indefinido
    \item Emprego da conjunção e com os cardinais
    \item Valores e empregos dos ordinais
    \item Emprego dos cardinais pelos ordinais
    \item Emprego dos multiplicativos
    \item Emprego dos fracionários
    \item Quadro dos numerais
    \item Numerais cardinais e ordinais
    \item Numerais multiplicativos e fracionários
\end{itemize}

\subsubsection*{Capítulo 12 – VERBO}
\begin{itemize}
    \item Noções preliminares
    \item Flexões do verbo
    \item Números
    \item Pessoas
    \item Modos
    \item Tempos
    \item Aspectos
    \item Vozes
    \item Formas rizotônicas e arrizotônicas
    \item Classificação do verbo
    \item Conjugações
    \item Tempos simples
    \item Estrutura do verbo
    \item Formação dos tempos simples
    \item Verbos auxiliares e o seu emprego
    \item Distinção importante
    \item Conjugação dos verbos ter, haver, ser e estar
    \item Modo indicativo
    \item Modo subjuntivo
    \item Modo imperativo
    \item Formas nominais
    \item Formação dos tempos compostos
    \item Modo indicativo
    \item Modo subjuntivo
    \item Formas nominais
    \item Conjugação dos verbos regulares
    \item Conjugação da voz passiva
    \item Modo indicativo
    \item Modo subjuntivo
    \item Formas nominais
    \item Voz reflexiva
    \item Verbo reflexivo e verbo pronominal
    \item Conjugação de um verbo reflexivo
    \item Modo indicativo
    \item Modo subjuntivo
    \item Modo imperativo
    \item Formas nominais
    \item Conjugação dos verbos irregulares
    \item Irregularidade verbal
    \item Irregularidade verbal e discordância gráfica
    \item Verbos com alternância vocálica
    \item Outros tipos de irregularidade
    \item Verbos terminados em -uzir
    \item Verbos de particípio irregular
    \item Verbos abundantes
    \item Verbos impessoais, unipessoais e defectivos
    \item Sintaxe dos modos e dos tempos
    \item Modo indicativo
    \item Emprego dos tempos do indicativo
    \item Modo subjuntivo
    \item Indicativo e subjuntivo
    \item Emprego do subjuntivo
    \item Subjuntivo independente
    \item Subjuntivo subordinado
    \item Substitutos do subjuntivo
    \item Tempos do subjuntivo
    \item Modo imperativo
    \item Formas do imperativo
    \item Emprego do modo imperativo
    \item Substitutos do imperativo
    \item Reforço ou atenuação da ordem
    \item Emprego das formas nominais
    \item Características gerais
    \item Emprego do infinitivo
    \item Emprego do gerúndio
    \item Emprego do particípio
    \item Concordância verbal
    \item Regras gerais
    \item Com um só sujeito
    \item Com mais de um sujeito
    \item Casos particulares
    \item Com um só sujeito
    \item Com mais de um sujeito
    \item Regência verbal
    \item Regência
    \item Regência verbal
    \item Diversidade e igualdade de regência
    \item Regência de alguns verbos
    \item Sintaxe do verbo haver
\end{itemize}

\subsubsection*{Capítulo 13 – ADJETIVO}
\begin{itemize}
    \item Nome substantivo e nome adjetivo
    \item Substantivação do adjetivo
    \item Substitutos do adjetivo
    \item Morfologia dos adjetivos
    \item Adjetivos pátrios
    \item Pátrios brasileiros
    \item Pátrios portugueses
    \item Pátrios africanos
    \item Adjetivos pátrios compostos
    \item Radicais latinos
    \item Radicais gregos
    \item Flexões dos adjetivos
    \item Número
    \item Formação do plural
    \item Gênero
    \item Formação do feminino
    \item Adjetivos uniformes
    \item Mudança de sentido na mudança de gênero
    \item Substantivos masculinos terminados em -a
    \item Substantivos de gênero vacilante
    \item Grau
    \item Valor das formas aumentativas e diminutivas
    \item Especialização de formas
    \item Emprego do adjetivo
    \item Funções sintáticas do adjetivo
    \item Substantivo como adjunto adnominal
    \item Substantivo caracterizador de adjetivo
    \item Substantivo caracterizado por um nome
    \item O substantivo como núcleo das frases sem verbo
\end{itemize}

\subsubsection*{Capítulo 14 – PRONOMES}
\begin{itemize}
    \item Pronomes substantivos e pronomes adjetivos
    \item Pronomes pessoais
    \item Formas dos pronomes pessoais
    \item Formas o, lo e no do pronome oblíquo
    \item Pronomes reflexivos e recíprocos
    \item Emprego dos pronomes retos
    \item Funções dos pronomes retos
    \item Omissão do pronome sujeito
    \item Presença do pronome sujeito
    \item Extensão de emprego dos pronomes retos
    \item Realce do pronome sujeito
    \item Precedência dos pronomes sujeitos
    \item Equívocos e incorreções
    \item Contração das preposições de e em com o pronome reto da 3.ª pessoa
    \item Pronomes de tratamento
    \item Emprego dos pronomes de tratamento da 2.ª pessoa
    \item Fórmulas de representação da 1.ª pessoa
    \item Emprego dos pronomes oblíquos
    \item Formas tônicas
    \item Emprego enfático do pronome oblíquo tônico
    \item Pronomes precedidos de preposição
    \item Formas átonas
    \item O pronome oblíquo átono sujeito de um infinitivo
    \item Emprego enfático do pronome oblíquo átono
    \item O pronome de interesse
    \item Pronome átono com valor possessivo
    \item Pronomes complementos de verbos de regência distinta
    \item Valores e empregos do pronome se
    \item Combinações e contrações dos pronomes átonos
    \item Colocação dos pronomes átonos
\end{itemize}

\subsubsection*{Capítulo 15 – NUMERAIS}
\begin{itemize}
    \item Espécies de numerais
    \item Numerais coletivos
    \item Flexão dos numerais
    \item Cardinais
    \item Ordinais
    \item Multiplicativos
    \item Fracionários
    \item Numerais coletivos
    \item Valores e empregos dos cardinais
    \item Cardinal como indefinido
    \item Emprego da conjunção e com os cardinais
    \item Valores e empregos dos ordinais
    \item Emprego dos cardinais pelos ordinais
    \item Emprego dos multiplicativos
    \item Emprego dos fracionários
    \item Quadro dos numerais
    \item Numerais cardinais e ordinais
    \item Numerais multiplicativos e fracionários
\end{itemize}

\subsubsection*{Capítulo 16 – VERBO}
\begin{itemize}
    \item Noções preliminares
    \item Flexões do verbo
    \item Números
    \item Pessoas
    \item Modos
    \item Tempos
    \item Aspectos
    \item Vozes
    \item Formas rizotônicas e arrizotônicas
    \item Classificação do verbo
    \item Conjugações
    \item Tempos simples
    \item Estrutura do verbo
    \item Formação dos tempos simples
    \item Verbos auxiliares e o seu emprego
    \item Distinção importante
    \item Conjugação dos verbos ter, haver, ser e estar
    \item Modo indicativo
    \item Modo subjuntivo
    \item Modo imperativo
    \item Formas nominais
    \item Formação dos tempos compostos
    \item Modo indicativo
    \item Modo subjuntivo
    \item Formas nominais
    \item Conjugação dos verbos regulares
    \item Conjugação da voz passiva
    \item Modo indicativo
    \item Modo subjuntivo
    \item Formas nominais
    \item Voz reflexiva
    \item Verbo reflexivo e verbo pronominal
    \item Conjugação de um verbo reflexivo
    \item Modo indicativo
    \item Modo subjuntivo
    \item Modo imperativo
    \item Formas nominais
    \item Conjugação dos verbos irregulares
    \item Irregularidade verbal
    \item Irregularidade verbal e discordância gráfica
    \item Verbos com alternância vocálica
    \item Outros tipos de irregularidade
    \item Verbos terminados em -uzir
    \item Verbos de particípio irregular
    \item Verbos abundantes
    \item Verbos impessoais, unipessoais e defectivos
    \item Sintaxe dos modos e dos tempos
    \item Modo indicativo
    \item Emprego dos tempos do indicativo
    \item Modo subjuntivo
    \item Indicativo e subjuntivo
    \item Emprego do subjuntivo
    \item Subjuntivo independente
    \item Subjuntivo subordinado
    \item Substitutos do subjuntivo
    \item Tempos do subjuntivo
    \item Modo imperativo
    \item Formas do imperativo
    \item Emprego do modo imperativo
    \item Substitutos do imperativo
    \item Reforço ou atenuação da ordem
    \item Emprego das formas nominais
    \item Características gerais
    \item Emprego do infinitivo
    \item Emprego do gerúndio
    \item Emprego do particípio
    \item Concordância verbal
    \item Regras gerais
    \item Com um só sujeito
    \item Com mais de um sujeito
    \item Casos particulares
    \item Com um só sujeito
    \item Com mais de um sujeito
    \item Regência verbal
    \item Regência
    \item Regência verbal
    \item Diversidade e igualdade de regência
    \item Regência de alguns verbos
    \item Sintaxe do verbo haver
\end{itemize}

\subsubsection*{Capítulo 17 – ADVÉRBIO}
\begin{itemize}
    \item Classificação dos advérbios
    \item Advérbios interrogativos
    \item Advérbio relativo
    \item Locução adverbial
    \item Colocação dos advérbios
    \item Repetição de advérbios em –mente
    \item Gradação dos advérbios
    \item Grau comparativo
    \item Grau superlativo
    \item Outras formas de comparativo e superlativo
    \item Repetição do advérbio
    \item Diminutivo com valor superlativo
    \item Advérbios que não se flexionam em grau
\end{itemize}

\subsubsection*{Capítulo 18 – PREPOSIÇÃO}
\begin{itemize}
    \item Função das preposições
    \item Forma das preposições
    \item Preposições simples
    \item Locuções prepositivas
    \item Significação das preposições
    \item Conteúdo significativo e função relacional
    \item Relações fixas
    \item Relações necessárias
    \item Relações livres
    \item Valores das preposições
    \item A
    \item Ante
    \item Após
    \item Até
    \item Com
    \item Contra
    \item De
    \item Desde
    \item Em
    \item Entre
    \item Para
    \item Perante
    \item Por (per)
    \item Sem
    \item Sob
    \item Sobre
    \item Trás
\end{itemize}

\subsubsection*{Capítulo 19 – CONJUNÇÃO}
\begin{itemize}
    \item Conjunção coordenativa e subordinativa
    \item Conjunções coordenativas
    \item Posição das conjunções coordenativas
    \item Valores particulares
    \item Conjunções subordinativas
    \item Conjunções conformativas e proporcionais
    \item Polissemia conjuncional
    \item Locução conjuntiva
\end{itemize}

\subsubsection*{Capítulo 20 – INTERJEIÇÃO}
\begin{itemize}
    \item Classificação das interjeições
    \item Locução interjectiva
\end{itemize}

\subsubsection*{Capítulo 21 – O PERÍODO E SUA CONSTRUÇÃO}
\begin{itemize}
    \item Período simples e período composto
    \item Composição do período
    \item Características da oração principal
    \item Conclusão
    \item Coordenação
    \item Orações coordenadas sindéticas e assindéticas
    \item Orações coordenadas sindéticas
    \item Subordinação
    \item A oração subordinada como termo de outra oração
    \item Classificação das orações subordinadas
    \item Orações subordinadas substantivas
    \item Orações subordinadas adjetivas
    \item Orações subordinadas adverbiais
    \item Orações reduzidas
\end{itemize}

\subsubsection*{Capítulo 22 – FIGURAS DE SINTAXE}
\begin{itemize}
    \item Elipse
    \item A elipse como processo gramatical
    \item A elipse como processo estilístico
    \item Zeugma
    \item Pleonasmo
    \item Pleonasmo vicioso
    \item Pleonasmo e epíteto de natureza
    \item Objeto pleonástico
    \item Hipérbato
    \item Anástrofe
    \item Prolepse
    \item Sínquise
    \item Assíndeto
    \item Polissíndeto
    \item Anacoluto
    \item Silepse
    \item Silepse de número
    \item Silepse de gênero
    \item Silepse de pessoa
\end{itemize}

\subsubsection*{Capítulo 23 – DISCURSO DIRETO, DISCURSO INDIRETO E DISCURSO INDIRETO LIVRE}
\begin{itemize}
    \item Estruturas de reprodução de enunciações
    \item Discurso direto
    \item Características do discurso direto
    \item Discurso indireto
    \item Características do discurso indireto
    \item Transposição do discurso direto para o indireto
    \item Discurso indireto livre
    \item Características do discurso indireto livre
\end{itemize}

\subsubsection*{Capítulo 24 – PONTUAÇÃO}
\begin{itemize}
    \item Sinais pausais e sinais melódicos
    \item Sinais que marcam sobretudo a pausa
    \item A vírgula
    \item O ponto
    \item O ponto e vírgula
    \item Valor melódico dos sinais pausais
    \item Sinais que marcam sobretudo a melodia
    \item Os dois-pontos
    \item O ponto de interrogação
    \item O ponto de exclamação
    \item As reticências
    \item As aspas
    \item Os parênteses
    \item Os colchetes
    \item O travessão
\end{itemize}

\subsubsection*{Capítulo 25 – NOÇÕES DE VERSIFICAÇÃO}
\begin{itemize}
    \item Estrutura do verso
    \item Ritmo e verso
    \item Os limites do verso
    \item As ligações rítmicas
    \item Sinalefa, elisão e crase
    \item Ectlipse
    \item O hiato intervocabular
    \item A medida das palavras
    \item Sinérese
    \item Diérese
    \item Crase, aférese, síncope e apócope
    \item A cesura e a pausa final
    \item Cavalgamento (enjambement)
    \item O cavalgamento e a pausa final
    \item Tipos de verso
    \item Os versos tradicionais
    \item Monossílabos
    \item Dissílabos
    \item Trissílabos
    \item Tetrassílabos
    \item Pentassílabos
    \item Hexassílabos
    \item Heptassílabos
    \item Octossílabos
    \item Eneassílabos
    \item Decassílabos
    \item Hendecassílabos
    \item Dodecassílabos
    \item Isossilabismo e versificação flutuante
    \item O verso livre
    \item A rima
    \item A rima e o acento
    \item Rima perfeita e rima imperfeita
    \item Rima pobre e rima rica
    \item Combinações de rimas
    \item Rima interior
    \item Indicação esquemática das rimas
    \item Versos sem rima
    \item Estrofação
    \item O dístico
    \item O terceto
    \item A quadra
    \item A quintilha
    \item A sextilha
    \item A estrofe de sete versos
    \item A oitava
    \item A estrofe de nove versos
    \item A décima
    \item Estrofe simples e composta
    \item Estrofe livre
    \item Poemas de forma fixa
    \item O soneto
\end{itemize}

\end{document}