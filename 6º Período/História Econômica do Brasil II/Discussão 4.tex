\documentclass[a4paper,12pt]{article}[abntex2]
\bibliographystyle{abntex2-alf}


% Definições de layout e formatação
\usepackage[a4paper, left=3.0cm, top=3.0cm, bottom=2.0cm, right=2.0cm]{geometry} % Personalização das margens do documento
\usepackage{setspace} % Controle do espaçamento entre linhas
\onehalfspacing % Espaçamento entre linhas de 1,5
\usepackage{indentfirst} % Indentação do primeiro parágrafo das seções
\usepackage{newtxtext} % Substitui a fonte padrão pela Times Roman
\usepackage{titlesec} % Personalização dos títulos de seções
\usepackage{ragged2e} % Melhor controle de justificação do texto
\usepackage[portuguese]{babel} % Adaptação para o português (nomes e hifenização)

% Pacotes de cabeçalho, rodapé e títulos
\usepackage{fancyhdr} % Customização de cabeçalhos e rodapés
\setlength{\headheight}{14.49998pt} % Altura do cabeçalho
\pagestyle{fancy}
\fancyhf{} % Limpa cabeçalho e rodapé
\rhead{\thepage} % Página no canto direito do cabeçalho

% Pacotes para tabelas
\usepackage{booktabs} % Melhora a qualidade das tabelas
\usepackage{tabularx} % Permite tabelas com larguras de colunas ajustáveis
\usepackage{float} % Melhor controle sobre o posicionamento de figuras e tabelas

% Pacotes para gráficos e imagens
\usepackage{graphicx} % Suporte para inclusão de imagens

\usepackage[utf8]{inputenc}
\usepackage{listingsutf8}

\lstset{
    language=R,                      
    basicstyle=\ttfamily\scalefont{1.0},
    keywordstyle=\color{blue},       
    stringstyle=\color{red},         
    commentstyle=\color{green},      
    numbers=left,                    
    numberstyle=\tiny\color{gray},   
    stepnumber=1,                    
    numbersep=5pt,                   
    backgroundcolor=\color{lightgray!10}, 
    frame=single,                    
    breaklines=true,                 
    captionpos=b,                    
    keepspaces=true,                 
    showspaces=false,                
    showstringspaces=false,          
    showtabs=false,                  
    tabsize=2,
     literate={á}{{\'a}}1
             {é}{{\'e}}1
             {í}{{\'i}}1
             {ó}{{\'o}}1
             {ú}{{\'u}}1
             {Ú}{{\'U}}1
             {â}{{\^a}}1
             {ê}{{\^e}}1
             {î}{{\^i}}1
             {ô}{{\^o}}1
             {û}{{\^u}}1
             {ã}{{\~a}}1
             {õ}{{\~o}}1
             {ç}{{\c{c}}}1,
}


% Pacotes para unidades e formatação numérica
\usepackage{siunitx} % Tipografia de unidades do Sistema Internacional e formatação de números
\sisetup{
  output-decimal-marker = {,},
  inter-unit-product = \ensuremath{{}\cdot{}},
  per-mode = symbol
}
\DeclareSIUnit{\real}{R\$}
\newcommand{\real}[1]{R\$#1}

% Pacotes para hiperlinks e referências
\usepackage{hyperref} % Suporte a hiperlinks
\usepackage{footnotehyper} % Notas de rodapé clicáveis em combinação com hyperref
\hypersetup{
    colorlinks=true,
    linkcolor=black,
    filecolor=magenta,      
    urlcolor=cyan,
    citecolor=black,        
    pdfborder={0 0 0},
}
\makeatletter
\def\@pdfborder{0 0 0} % Remove a borda dos links
\def\@pdfborderstyle{/S/U/W 1} % Estilo da borda dos links
\makeatother

% Pacotes para texto e outros
\usepackage{lipsum} % Geração de texto dummy 'Lorem Ipsum'
\usepackage[normalem]{ulem} % Permite o uso de diferentes tipos de sublinhados sem alterar o \emph{}

\begin{document}

\begin{titlepage}
    \centering
    \vspace*{1cm}
    \Large\textbf{INSPER – INSTITUTO DE ENSINO E PESQUISA}\\
    \Large ECONOMIA\\
    \vspace{1.5cm}
    \Large\textbf{Discussão 4}\\
    \textbf{História Econômica do Brasil II}\\
    \vspace{1.5cm}
    Prof. Heleno Piazentini Vieira \\
    Prof. Auxiliar  \\
    \vfill
    \normalsize
    Andreas Azambuja Barbisan, \href{mailto:andreasab@al.insper.edu.br}{andreasab@al.insper.edu.br}\\
    Bruno Frasao Brazil Leiros, \href{mailto:brunofbl@al.insper.edu.br}{brunofbl@al.insper.edu.br}\\
    Érika Kaori Fuzisaka, \href{mailto:erikakf1@al.insper.edu.br}{erikakf1@al.insper.edu.br}\\
    Hicham Munir Tayfour, \href{mailto:hichamt@al.insper.edu.br}{hichamt@al.insper.edu.br}\\
    Lorena Liz Giusti e Santos,\href{mailto:lorenalgs@al.insper.edu.br}{lorenalgs@al.insper.edu.br}\\
    Nicolas Pedro Diniz Brito, \href{mailto:nicolasb2@al.insper.edu.br}{nicolasb2@al.insper.edu.br}\\
    Sarah de Araújo Nascimento Silva, \href{mailto:sarahans@al.insper.edu.br}{sarahans@al.insper.edu.br}\\


    \vfill
    São Paulo\\
    Abril/2025
\end{titlepage}

\newpage
\tableofcontents
\thispagestyle{empty} % Esse comando remove a numeração de pagina da tabela de conteúdo

\newpage
\setcounter{page}{1} % Inicia a contagem de páginas a partir desta página
\justify
\onehalfspacing

\section{\textbf{Plano Real}}

O Plano Real foi concebido e implementado em 1994 como uma resposta estruturada e inovadora ao regime crônico de inflação que dominava a economia brasileira desde o final dos anos 1970. Com base em lições aprendidas ao longo de sucessivas tentativas fracassadas de estabilização, o plano representou uma ruptura com a lógica dos choques heterodoxos e das intervenções abruptas, ao apostar em uma estratégia gradual, técnica, transparente e articulada em torno da confiança dos agentes econômicos. Seu sucesso decorreu de um arranjo institucional cuidadosamente desenhado e do equilíbrio entre medidas ortodoxas de política econômica e soluções heterodoxas de transição monetária.

O contexto político da implementação foi extremamente delicado. O presidente Itamar Franco assumiu o governo após o impeachment de Fernando Collor e acumulava desgaste por sucessivas trocas de ministros da Fazenda. A nomeação de Fernando Henrique Cardoso, em maio de 1993, trouxe estabilidade à condução da política econômica e permitiu a formação de uma equipe técnica de alto nível, que operou sob autonomia relativa e regras internas de deliberação técnica, sem interferência hierárquica. O grupo, liderado por Cardoso, concebeu um plano dividido em três fases: ajuste fiscal, desindexação via Unidade Real de Valor (URV) e introdução de nova moeda com ancoragem nominal.

Na primeira fase, o objetivo era sinalizar responsabilidade fiscal. Com base na análise de Edmar Bacha, diagnosticou-se a existência de um “déficit potencial” — um desequilíbrio oculto que não aparecia nas estatísticas por conta da indexação das receitas e da corrosão inflacionária das despesas públicas. Essa lógica ficou conhecida como o “efeito Tanzi às avessas”. Para enfrentá-lo, o governo lançou o Programa de Ação Imediata (PAI) e aprovou o Fundo Social de Emergência (FSE), que permitia a desvinculação de receitas obrigatórias. Embora essas medidas tenham gerado superávits primários relevantes — 2,6\% do PIB em 1993 e 5,1\% em 1994 —, os textos analisados apontam que sua eficácia estrutural foi limitada. Reformas mais amplas, como a da Previdência e do sistema tributário, foram reconhecidas como necessárias, mas não avançaram.

A segunda fase do Plano Real foi a criação da URV, uma unidade de conta indexada, inspirada na proposta “Larida” de 1984, formulada por Persio Arida e André Lara Resende. A proposta original previa a convivência entre a nova unidade e a moeda antiga, com adesão voluntária e transição gradual. No entanto, o Plano Real introduziu inovações importantes: a URV não circulava como moeda, mas como indexador; sua cotação diária era definida com base em três índices de preços; e os pagamentos continuaram sendo feitos em cruzeiros reais até o lançamento oficial do real. Além disso, a política monetária adotada foi ativa — e não passiva como na proposta original —, com elevação dos juros e do recolhimento compulsório para conter o consumo pós-estabilização. A separação funcional entre a URV (unidade de conta) e o cruzeiro real (meio de pagamento) permitiu preservar os preços relativos e evitou o colapso da moeda antiga, como ocorrera na experiência húngara de 1945.

A terceira fase iniciou-se em 1º de julho de 1994, com a conversão da URV em real (R\$1,00 = US\$1,00 = 2.750 cruzeiros reais). Nesse momento, o plano passou a operar com três âncoras: fiscal, cambial e monetária. A âncora cambial fixava o valor do real em relação ao dólar dentro de uma banda assimétrica, com livre valorização e teto fixo para desvalorização. A âncora monetária previa metas para a base monetária e controle da liquidez. Posteriormente, com o abandono dessas metas, os juros reais elevados tornaram-se a principal âncora de contenção da inflação. A âncora fiscal baseava-se na arrecadação ampliada via FSE, embora sem reformas estruturais profundas.

O resultado foi uma queda vertiginosa da inflação: de mais de 40\% ao mês no primeiro semestre de 1994 para menos de 1\% ao mês no final do ano. A URV teve papel central ao desindexar a economia sem necessidade de congelamentos ou tabelamentos. A estratégia do Plano Real se diferenciou nitidamente do Plano Cruzado (1986), que havia adotado medidas autoritárias, como congelamento de preços e salários. O Real respeitou os contratos, operou com adesão voluntária dos agentes econômicos e apostou em comunicação clara e antecipada com a sociedade, desarmando expectativas defensivas e movimentos especulativos antecipatórios.

Contudo, os custos da estratégia adotada não foram desprezíveis. A combinação entre câmbio valorizado e juros elevados teve impactos negativos sobre a competitividade da economia brasileira, contribuiu para a deterioração das contas externas e pressionou o setor produtivo nacional. O saldo comercial caiu de US\$ 13,3 bilhões (1993) para US\$ 10,5 bilhões (1994), e a dependência de capitais externos aumentou. A âncora dos juros, mantida por todo o período 1994–1998, acabou contribuindo para a ampliação do déficit público e o crescimento da dívida, especialmente na ausência de reformas estruturais. Ainda assim, o sucesso do Plano Real foi inegável ao conseguir quebrar a inércia inflacionária e encerrar o ciclo de alta inflação que havia marcado o Brasil nas décadas anteriores.


\newpage
\section{\textbf{Resumo dos Textos da 4º Discussão}}
\subsection{\textbf{Economia Brasileira Contemporânea, Cap. 6, item: O Plano Real: concepção e prática}}

O Plano Real (1994) foi concebido como uma estratégia gradual de estabilização monetária, estruturada em três fases: ajuste fiscal, desindexação via URV e estabilização com nova moeda e âncora nominal. Sua originalidade e eficácia residiram na ruptura com as experiências anteriores, especialmente na forma de tratar a inflação inercial. A seguir, os elementos fundamentais da concepção e implementação do Plano Real:

\begin{itemize}

    \item \textbf{Diagnóstico e concepção inicial:}
    \begin{itemize}
        \item A inflação era atribuída ao desequilíbrio fiscal estrutural, com ênfase em um “déficit potencial”, apesar do déficit operacional aparentemente baixo.
        \item Propôs-se um programa em três fases: ajuste fiscal (PAI e FSE), criação da URV e lançamento do real com medidas de sustentação monetária e cambial.
        \item A URV surgiu como alternativa à indexação e congelamentos anteriores, buscando a eliminação da memória inflacionária sem choques abruptos.
    \end{itemize}

    \item \textbf{Fase I — Ajuste Fiscal:}
    \begin{itemize}
        \item O PAI (1993) visava a redefinição da relação da União com os Estados e combate à sonegação.
        \item O FSE (1994) permitiu a desvinculação de receitas, flexibilizando os gastos públicos sob rigidez da Constituição de 1988.
        \item Edmar Bacha propôs o “Efeito Tanzi às avessas”, em que a inflação ajudava a disfarçar desequilíbrios fiscais, ao corroer despesas não indexadas mais rapidamente que as receitas.
        \item Apesar da aprovação do FSE, o equilíbrio fiscal não foi alcançado em 1995 e reformas estruturais previstas não foram aprovadas.
        \item A inflação caiu mesmo sem ajuste fiscal duradouro, questionando a centralidade da política fiscal no controle da inflação.
    \end{itemize}

    \item \textbf{Fase II — Desindexação com a URV:}
    \begin{itemize}
        \item A URV (Unidade Real de Valor) foi lançada em março de 1994 como unidade de conta paralela, mantendo o cruzeiro real como meio de pagamento.
        \item Inspirada na proposta “Larida” (1984), introduziu uma quase-moeda estável, com adesão voluntária dos agentes econômicos.
        \item A URV teve cotação diária baseada em três índices (IGP-M, IPCA-E, IPC-Fipe), garantindo maior neutralidade setorial.
        \item Preços e salários foram convertidos gradualmente, com correção diária dos salários em URV (pagamento pela URV do dia).
        \item Medidas prudenciais evitaram hiperinflação: preços obrigatoriamente cotados em cruzeiros, política monetária contracionista (juros altos e compulsórios elevados) e pagamento em regime de caixa.
        \item A desindexação teve sucesso ao eliminar a memória inflacionária sem provocar explosão de preços, ao contrário da experiência húngara de 1945.
    \end{itemize}

    \item \textbf{Fase III — Lançamento do Real e âncora nominal:}
    \begin{itemize}
        \item Em julho de 1994, a URV foi convertida no real (R\$), nova moeda nacional.
        \item A Medida Provisória 542 introduziu:
        \begin{itemize}
            \item Lastreamento da base monetária em reservas cambiais (1 R\$ = 1 US\$);
            \item Metas para base monetária;
            \item Propostas de maior autonomia ao Banco Central.
        \end{itemize}
        \item Adotou-se banda cambial assimétrica: câmbio livre para valorização e fixo para desvalorização.
        \item Forte aperto monetário: compulsórios sobre depósitos à vista atingiram 100\%.
        \item Juros reais elevados (média de 21\% ao ano entre 1994–1998) tornaram-se nova âncora de preços, substituindo o câmbio.
    \end{itemize}

    \item \textbf{Impactos macroeconômicos do Plano Real:}
    \begin{itemize}
        \item \textbf{Inflação:} queda contínua da inflação até 1999, mesmo sem ajuste fiscal robusto.
        \item \textbf{Atividade econômica:} boom inicial com expansão do consumo e crédito, especialmente de bens duráveis.
        \item \textbf{Salários reais:} ganhos no início do plano, corroídos parcialmente até novo reajuste com o lançamento do real.
        \item \textbf{Contas externas:} rápida deterioração da conta-corrente, financiada por entrada de capitais, seguida por crises (México, Ásia, Rússia).
        \item \textbf{Déficit público:} aumento do déficit operacional devido aos altos juros; o ajuste fiscal esperado não se concretizou.
        \item \textbf{Juros:} permaneceram elevados e voláteis, tornando-se o principal instrumento de combate à inflação, diferente da experiência internacional com âncora cambial.
    \end{itemize}

    \item \textbf{Conclusão:}
    \begin{itemize}
        \item O Plano Real rompeu com o padrão de estabilização via choques e congelamentos.
        \item A URV foi o principal instrumento técnico para conter a inflação inercial.
        \item A âncora cambial inicial foi substituída pela âncora dos juros, com impacto significativo sobre crescimento e dívida pública.
        \item Apesar dos altos custos macroeconômicos, o plano obteve êxito em estabilizar preços e encerrar um ciclo prolongado de inflação crônica.
    \end{itemize}

\end{itemize}

\subsection{\textbf{Ordem e Progresso, Cap. 15, item: Concepção e implementação do Plano Real}}

A concepção e implementação do Plano Real ocorreu em um contexto de descrédito político, instabilidade econômica e sucessivos fracassos de planos anteriores. O novo plano precisou ser cuidadosamente arquitetado para romper com o passado, obter apoio institucional e conquistar a confiança da sociedade. A seguir, destacam-se os principais elementos dessa trajetória:

\begin{itemize}

    \item \textbf{Contexto político-institucional instável:}
    \begin{itemize}
        \item O presidente Itamar Franco assumiu em meio à crise gerada pelo impeachment de Collor, sendo instável na condução da política econômica, nomeando quatro ministros da Fazenda em sete meses.
        \item A escolha de Fernando Henrique Cardoso para o Ministério da Fazenda, em maio de 1993, marcou a virada na estratégia de estabilização.
    \end{itemize}

    \item \textbf{Desafios à formulação de um novo plano:}
    \begin{itemize}
        \item A inflação ultrapassava 30\% ao mês, e o mandato de Itamar tinha apenas 19 meses restantes.
        \item O histórico de fracassos (cinco planos desde 1986) exigia aprendizado institucional e inovação.
        \item Havia resistência à dolarização (modelo argentino), e as alternativas nacionais ainda estavam em elaboração.
    \end{itemize}

    \item \textbf{Construção da equipe econômica e liberdade técnica:}
    \begin{itemize}
        \item FHC formou uma equipe técnica de alto nível, com liberdade para debater ideias sem hierarquias rígidas.
        \item O pacto implícito previa não interferência política direta nas decisões técnicas do grupo.
    \end{itemize}

    \item \textbf{Atuação política de FHC:}
    \begin{itemize}
        \item Atuou como negociador no governo e no Congresso, garantindo apoio institucional para o plano.
        \item Enfrentou resistências internas e chegou a ameaçar demissão para proteger a integridade do projeto.
        \item Buscou engajar seu partido (PSDB), que inicialmente demonstrava ceticismo.
    \end{itemize}

    \item \textbf{Mobilização da opinião pública:}
    \begin{itemize}
        \item O plano foi planejado com ampla antecedência, de forma transparente, evitando congelamentos ou medidas arbitrárias.
        \item FHC comunicou eficazmente as etapas do plano, desarmando expectativas inflacionárias defensivas.
    \end{itemize}

    \item \textbf{Estrutura técnica do plano:}
    \begin{itemize}
        \item A estratégia de estabilização foi anunciada em três etapas: ajuste fiscal, criação da URV e reforma monetária com introdução do real.
        \item Em julho de 1993, houve reforma monetária inicial (corte de três zeros, criação do cruzeiro real).
    \end{itemize}

    \item \textbf{Pilar fiscal do Plano:}
    \begin{itemize}
        \item Medidas de ajuste foram viabilizadas pelo aumento de impostos sobre o setor financeiro e pela criação do Fundo Social de Emergência.
        \item Reformas mais profundas (Previdência e tributos) foram reconhecidas como necessárias, mas postergadas.
        \item Os superávits primários foram de 2,6\% do PIB (1993) e 5,1\% (1994), em parte pela corrosão inflacionária dos gastos públicos.
    \end{itemize}

    \item \textbf{Implementação da URV:}
    \begin{itemize}
        \item De março a junho de 1994, a URV conviveu com o cruzeiro real; sua cotação diária era corrigida conforme índices de preços.
        \item Contratos novos eram obrigatoriamente em URV; contratos antigos podiam ser convertidos voluntariamente.
        \item Essa conversão descentralizada permitiu transição suave, com preservação dos preços relativos.
    \end{itemize}

    \item \textbf{Criação do real:}
    \begin{itemize}
        \item Em julho de 1994, a URV foi convertida em real (1 real = 2.750 cruzeiros reais), com paridade inicial de 1 real = 1 dólar.
        \item A inflação caiu de 40--50\% ao mês (em cruzeiros reais) para abaixo de 10\% em julho e 1\% no fim do ano.
    \end{itemize}

    \item \textbf{As três âncoras do plano:}
    \begin{itemize}
        \item \textbf{Âncora cambial}: paridade inicial com o dólar, sustentada por reservas de US\$ 40 bilhões.
        \item \textbf{Âncora fiscal}: viabilizada pelo FSE e superávits primários.
        \item \textbf{Âncora monetária}: taxas de juros reais mantidas elevadas para conter demanda.
    \end{itemize}

    \item \textbf{Ajustes diante do superaquecimento:}
    \begin{itemize}
        \item No segundo semestre de 1994, o setor automobilístico mostrou forte expansão, com pressões salariais.
        \item O ministro Ciro Gomes reduziu tarifas de importação de automóveis (de 35\% para 20\%) para conter aumentos de preços.
        \item Com a crise do México (fim de 1994), o cenário externo piorou, e as críticas à liberalização comercial ganharam força.
    \end{itemize}

    \item \textbf{Desempenho macroeconômico:}
    \begin{itemize}
        \item Crescimento do PIB: 4,7\% em 1993 e 5,3\% em 1994.
        \item Formação bruta de capital fixo atingiu 20,7\% do PIB em 1994.
        \item O saldo comercial caiu de US\$ 13,3 bi (1993) para US\$ 10,5 bi (1994), com aumento de importações.
        \item Reservas internacionais subiram de US\$ 23,8 bi (1992) para US\$ 38,8 bi (1994).
    \end{itemize}

    \item \textbf{Relação com o FMI:}
    \begin{itemize}
        \item O FMI não apoiou o Plano Real, por ceticismo em relação ao seu sucesso.
        \item Esse distanciamento, após 12 anos de colaboração com planos fracassados, tornou-se um fato irônico diante do êxito do plano.
    \end{itemize}

\end{itemize}

\newpage
\section{\textbf{Respondendo as perguntas da discussão}}
\subsection{\textbf{Bloco 1 A questão fiscal no Real :Plano Real, a questão fiscal e a contribuição de Edmar Bacha. A 1ª fase do plano procurava corrigir o problema fiscal do Brasil no curto prazo; A presença do efeito Tanzi exigia um ajuste fiscal prévio a estabilização da moeda; A 1ª fase do plano Real teria sido irrelevante, já que a queda da inflação foi atingida sem o ajuste fiscal completo. Vocês concordam? Expliquem a resposta.}}

Afirmar que a 1ª fase do Plano Real foi “irrelevante” é um exagero. Embora sua eficácia no combate direto à inflação tenha sido inferior à esperada, ela cumpriu papéis políticos, simbólicos e institucionais importantes no processo de estabilização. Abaixo detalhamos os argumentos, com base exclusiva nos dois textos analisados.

\begin{itemize}

    \item \textbf{Contribuição de Edmar Bacha e o efeito Tanzi às avessas:}
    \begin{itemize}
        \item Bacha argumentava que a inflação elevada no Brasil coexistia com um déficit operacional aparentemente baixo devido a distorções geradas pela indexação das receitas e à subestimação da inflação nos orçamentos.
        \item Essa dinâmica produzia um \textit{“efeito Tanzi às avessas”}, pois a inflação corroía os gastos nominais do governo mais rapidamente que as receitas indexadas, reduzindo o déficit \textit{ex post} — mascarando, assim, um forte desequilíbrio fiscal \textit{ex ante}.
        \item Para Bacha, sem ajuste fiscal real, qualquer tentativa de estabilização puramente monetária estava fadada ao fracasso, pois a inflação tenderia a voltar para financiar o déficit oculto.
    \end{itemize}

    \item \textbf{A estrutura e os limites da 1ª fase do Plano Real:}
    \begin{itemize}
        \item O ajuste fiscal proposto envolvia o Programa de Ação Imediata (PAI) e a criação do Fundo Social de Emergência (FSE), cujo objetivo era ampliar a flexibilidade orçamentária.
        \item Segundo o texto do Cap. 6, essas medidas foram importantes no plano político, mas não produziram um equilíbrio fiscal sustentável. Reformas estruturais, como da Previdência e do sistema tributário, foram reconhecidas como necessárias, mas não avançaram.
        \item Ainda assim, os efeitos combinados da inflação sobre os gastos e a elevação da carga sobre o setor financeiro geraram superávits primários relevantes (2,6\% do PIB em 1993 e 5,1\% em 1994), embora sem resolver o problema estrutural.
    \end{itemize}

    \item \textbf{A inflação caiu sem o ajuste fiscal completo — e por quê:}
    \begin{itemize}
        \item Como ambos os textos deixam claro, a queda da inflação se deveu, majoritariamente, à introdução da URV, que rompeu com a memória inflacionária e preparou o sistema de preços para a nova moeda.
        \item A URV foi implementada sem medidas arbitrárias e com grande transparência, conquistando a confiança da sociedade — o que ajudou a evitar os efeitos desestabilizadores comuns aos planos anteriores.
        \item A âncora cambial e, depois, a elevação das taxas de juros reais, também tiveram papel fundamental na contenção da inflação, conforme discutido no Cap. 6.
    \end{itemize}

    \item \textbf{Conclusão:}
    \begin{itemize}
        \item A 1ª fase do plano teve função relevante ao sinalizar responsabilidade fiscal e construir as bases políticas e institucionais para a estabilização.
        \item Entretanto, sua capacidade técnica de conter a inflação foi limitada. O sucesso da estabilização derivou, principalmente, da segunda fase (desindexação via URV) e da combinação de âncoras nominal e cambial.
        \item Portanto, a 1ª fase não foi “irrelevante”, mas tampouco foi determinante para a queda da inflação. Sua principal contribuição foi preparar o terreno político e institucional para que as fases seguintes pudessem funcionar com credibilidade.

    \end{itemize}

\end{itemize}

\subsection{\textbf{Bloco 2 Medidas do Real : No geral, as medidas introduzidas com o plano Real: Tinham mentalidade nas linhas ortodoxa e heterodoxa; Diferentemente do plano Cruzado, não procuravam interferir na indexação e no processo de formação de preços; Estabeleceram conjuntamente com sucesso as âncoras monetária e cambial. Vocês concordam? Expliquem a resposta.}}

Concordamos com a afirmação de que o Plano Real combinou estratégias ortodoxas e heterodoxas de forma inovadora, rompeu com a tradição intervencionista dos planos anteriores (como o Cruzado) e conseguiu, com sucesso, estabelecer âncoras monetária e cambial articuladas. Essa combinação foi essencial para o êxito inicial da estabilização da inflação.

\begin{itemize}

    \item \textbf{Composição ortodoxa e heterodoxa:}
    \begin{itemize}
        \item O Plano Real utilizou elementos \textbf{ortodoxos}, como metas monetárias, contenção de liquidez via compulsórios elevados e manutenção de juros reais altos. Essas medidas refletiam uma preocupação com o controle da demanda agregada e a preservação do valor da nova moeda.
        \item Simultaneamente, incorporou componentes \textbf{heterodoxos}, como a criação da URV, uma unidade de conta paralela, inspirada na proposta Larida. Sua função era desindexar a economia sem congelar preços, por meio de uma transição gradual e voluntária.
        \item Essa abordagem híbrida foi uma resposta aos sucessivos fracassos dos planos anteriores e buscava resgatar a confiança dos agentes econômicos sem recorrer a choques abruptos.
    \end{itemize}

    \item \textbf{Diferença fundamental em relação ao Plano Cruzado:}
    \begin{itemize}
        \item O Plano Cruzado adotou congelamentos generalizados de preços e salários, além da famosa ``tablita'' de indexação decrescente, medidas que causaram distorções severas no sistema de preços relativos.
        \item O Plano Real, por outro lado, \textbf{não interferiu diretamente na formação de preços}: a URV foi uma unidade de conta com adesão descentralizada e negociada, permitindo que fornecedores e consumidores ajustassem voluntariamente seus contratos.
        \item Essa estratégia evitou os efeitos colaterais típicos dos congelamentos — como escassez, distorções e explosão inflacionária posterior — e manteve a funcionalidade do sistema de preços.
        \item Como destaca o texto de “Ordem e Progresso”, a credibilidade da URV derivava justamente de sua transparência e previsibilidade, sem surpresas nem medidas autoritárias.
    \end{itemize}

    \item \textbf{Estabelecimento das âncoras: cambial e monetária:}
    \begin{itemize}
        \item O plano foi estruturado para operar com três âncoras: \textbf{cambial}, \textbf{fiscal} e \textbf{monetária}.
        \item A âncora cambial foi estabelecida com a paridade inicial de 1 real = 1 dólar, sustentada por reservas internacionais robustas (US\$ 40 bilhões). Essa fixação de referência ajudou a coordenar expectativas inflacionárias, especialmente no setor de bens comercializáveis.
        \item A âncora monetária envolveu metas para a base monetária, reforçadas por compulsórios sobre depósitos (até 100\%) e juros reais elevados, como forma de impedir uma explosão de consumo pós-estabilização.
        \item Ainda que as metas monetárias tenham sido abandonadas poucos meses após o lançamento do real, os juros elevados foram mantidos como instrumento central de contenção inflacionária, tornando-se a nova âncora de preços a partir de 1995.
    \end{itemize}

    \item \textbf{Efetividade e limitações das âncoras:}
    \begin{itemize}
        \item A estratégia teve sucesso na estabilização da inflação: o índice mensal caiu de cerca de 50\% em junho de 1994 para menos de 10\% em julho e 1\% no final do ano.
        \item Por outro lado, as âncoras, sobretudo a cambial, trouxeram efeitos colaterais: apreciação do real, aumento das importações, perda de competitividade externa e rápida deterioração da conta-corrente.
        \item Como descrito no Capítulo 6, a combinação de câmbio valorizado com juros elevados causou distorções no setor externo e agravou o déficit público ao longo do tempo, embora tenha sido decisiva no curto prazo.
    \end{itemize}

    \item \textbf{Conclusão:}
    \begin{itemize}
        \item O Plano Real representou um avanço importante ao evitar os erros de planos anteriores, apostando em uma engenharia institucional mais sofisticada, baseada em coordenação gradual e transparência.
        \item As medidas não buscaram intervir autoritariamente nos preços, mas sim reestruturar as bases da economia por meio da URV e da confiança nos mecanismos de mercado.
        \item A dupla ancoragem (monetária e cambial), apesar de seus custos, foi efetiva na rápida desaceleração da inflação — e seu sucesso deve ser compreendido dentro da lógica híbrida do plano.
    \end{itemize}

\end{itemize}

\subsection{\textbf{Bloco 3 Inovações na proposta Larida : A proposta Larida que já estava disponível desde a década de 1980 como estratégia para promover a desindexação passou por inovações dentro do plano Real: A política monetária teria de ser passiva; A URV circularia paralelamente com a moeda atual; O arranjo ideal para a taxa de câmbio auxiliar na estabilização da moeda seria o flutuante. Vocês concordam? Expliquem a resposta.}}

Discordamos, em parte, das afirmações apresentadas. As inovações promovidas no interior do Plano Real em relação à proposta Larida foram substanciais, especialmente nos seguintes pontos: a política monetária adotada foi ativa (e não passiva), a URV não circulou como moeda, e a taxa de câmbio adotada inicialmente foi fixa (não flutuante), dentro de uma banda cambial assimétrica. Abaixo detalhamos as evidências com base nos textos analisados.

\begin{itemize}

    \item \textbf{A proposta Larida como base teórica:}
    \begin{itemize}
        \item Elaborada por Persio Arida e André Lara Resende em 1984, a proposta sugeria a criação de uma nova unidade de conta indexada (como a ORTN) que conviveria com a moeda antiga e permitiria a transição para a estabilização via desindexação voluntária.
        \item Sua principal inovação era reduzir a memória inflacionária sem recorrer a congelamentos, confiando na escolha livre dos agentes para migrarem seus contratos para a nova unidade.
    \end{itemize}

    \item \textbf{Inovações no Plano Real:}
    \begin{itemize}
        \item No Plano Real, a URV não foi criada como uma nova moeda em circulação, mas como uma unidade de conta: todos os pagamentos continuaram sendo realizados em cruzeiros reais até 1º de julho de 1994.
        \item Essa medida foi desenhada para evitar os erros da experiência húngara de 1945–46, em que a moeda indexada gerou hiperinflação na moeda antiga e contaminou a nova.
        \item A URV teve sua paridade com o cruzeiro real corrigida diariamente, com base em três índices (IGP-M, IPCA-E, IPC-Fipe), funcionando como um indexador neutro e tecnicamente sofisticado.
        \item A legislação determinou que novos contratos fossem firmados em URV e que os contratos antigos pudessem ser convertidos por negociação. Com a criação do real, em julho, todos os contratos foram convertidos automaticamente para a nova moeda.
    \end{itemize}

    \item \textbf{Política monetária: ativa, não passiva:}
    \begin{itemize}
        \item Ao contrário da proposta Larida original, que previa uma condução passiva da política monetária durante a transição, o Plano Real adotou uma postura ativa.
        \item Persio Arida, inclusive, reviu sua posição e defendeu que, mesmo em um processo de desindexação, seria necessário elevar os juros no curto prazo para conter o impulso de consumo gerado pela queda abrupta da inflação.
        \item O governo elevou compulsórios bancários (até 100\%) e manteve juros reais elevados (média de 21\% ao ano entre 1994–98), o que contribuiu decisivamente para sustentar a estabilidade de preços.
    \end{itemize}

    \item \textbf{URV não circulou como moeda paralela:}
    \begin{itemize}
        \item A URV não foi uma nova moeda em circulação, mas uma unidade de conta de transição.
        \item O cruzeiro real continuou exercendo a função de meio de pagamento. A URV funcionou apenas como base para cálculo de contratos e reajustes.
        \item Essa separação funcional (conta vs. pagamento) foi crucial para evitar a explosão da velocidade de circulação da moeda antiga e garantir uma transição ordenada.
        \item Como aponta o texto de “Ordem e Progresso”, essa arquitetura institucional foi desenhada de forma transparente e antecipada, contribuindo para a credibilidade e aceitação do plano.
    \end{itemize}

    \item \textbf{Taxa de câmbio: fixa, não flutuante:}
    \begin{itemize}
        \item O Plano Real adotou inicialmente uma \textbf{âncora cambial} baseada em paridade fixa: 1 real = 1 dólar.
        \item A política cambial operava com banda assimétrica: o câmbio poderia se valorizar livremente, mas havia um teto rígido para a desvalorização, garantindo previsibilidade.
        \item Esse regime foi escolhido para ancorar expectativas inflacionárias e estabilizar os preços dos bens comercializáveis.
        \item A taxa de câmbio flutuante foi descartada no início, pois não forneceria um referencial estável para preços e contratos na fase inicial de confiança frágil.
    \end{itemize}

    \item \textbf{Conclusão:}
    \begin{itemize}
        \item O Plano Real se apropriou da base teórica da proposta Larida, mas promoveu ajustes fundamentais:
        \begin{itemize}
            \item Rejeitou a ideia de política monetária passiva e adotou postura ativa com juros elevados;
            \item Impediu a circulação de uma segunda moeda ao manter a URV apenas como unidade de conta;
            \item Utilizou um câmbio administrado com paridade fixa como âncora nominal no lugar de flutuação.
        \end{itemize}
        \item Essas inovações foram fundamentais para evitar os erros de planos anteriores e garantir uma transição gradual, crível e bem-sucedida para a nova moeda.
    \end{itemize}

\end{itemize}

\newpage
\section{\textbf{Respondendo a pergunta para entregar}}

\end{document}
