\documentclass{sciposter}
\fontsize{6pt}{7.2pt}\selectfont
\usepackage{lipsum}
\usepackage{epsfig}
\usepackage{amsmath}
\usepackage{amssymb}
\usepackage{multicol}
\usepackage{graphicx,url}
\usepackage[portuges, brazil]{babel}   
\usepackage[utf8]{inputenc}

\newtheorem{Def}{Definição}

\title{História Econômica do Brasil II}
\author{Hicham Munir Tayfour}
% Título do projeto

\institute 
{Bacharelado em Economia\\
Insper - Instituto de Ensino e Pesquisa\\
São Paulo, Brasil}
% Nome e endereço da Instituição

\rightlogo[1]{Imagens/logo-insper.png}  % Substitua pelo logo do Insper

\begin{document}

\conference{{\bf História Econômica do Brasil II}, Curso de Economia - Insper, 2025, São Paulo, Brasil}

\maketitle

%%% Início do ambiente Multicolunas
\begin{multicols}{3}
\fontsize{17pt}{7pt}\selectfont

%%% Resumo
\begin{abstract}
Este documento fornece um modelo de consulta para a Prova Final de História Econômica do Brasil II. Ele organiza os principais conceitos e tópicos relevantes para facilitar a revisão do curso.
\end{abstract}

%%% Começa aqui
\section{\textbf{Explique a importância das reformas econômicas estruturais implementadas na primeira metade da década de 1990.}}

As reformas econômicas estruturais da primeira metade dos anos 1990 foram essenciais para reorganizar institucionalmente o Brasil e preparar o ambiente econômico para a estabilização monetária que viria com o Plano Real. A abertura comercial desempenhou um papel significativo ao reduzir barreiras tarifárias, incentivando uma competição saudável no mercado interno e estimulando a eficiência produtiva. As privatizações contribuíram para diminuir o papel do Estado na economia, atraindo investimentos estrangeiros e melhorando a gestão das empresas anteriormente estatais. Os ajustes fiscais ajudaram a controlar os desequilíbrios orçamentários, fortalecendo a credibilidade fiscal e criando um cenário favorável para combater a inflação crônica que afligia o país.

\section{\textbf{Explique as três fases do Plano Real e qual foi o principal objetivo de cada uma delas.}}

O Plano Real foi estruturado em três fases interdependentes e sequenciais. A primeira fase, o ajuste fiscal inicial, buscava demonstrar compromisso com a responsabilidade fiscal e reduzir imediatamente o déficit público. Para isso, foram implementados o Programa de Ação Imediata (PAI) e o Fundo Social de Emergência (FSE), que visavam combater o "efeito Tanzi às avessas", onde a alta inflação escondia déficits fiscais reais.

Na segunda fase, foi introduzida a Unidade Real de Valor (URV), com o objetivo de eliminar a indexação generalizada que impulsionava a inflação inercial. A URV funcionou como uma unidade de conta que permitiu reajustes graduais dos preços e contratos sem necessidade de congelamentos autoritários, preservando assim o funcionamento saudável do mercado.

A terceira fase consistiu na introdução do Real como nova moeda nacional em 1º de julho de 1994. Esta fase tinha como objetivo consolidar a estabilidade econômica conquistada pelas fases anteriores. O Real foi introduzido com o apoio de três âncoras fundamentais: cambial, monetária e fiscal, garantindo estabilidade monetária duradoura e credibilidade ao plano.

\section{\textbf{Faça uma síntese das contribuições de Edmar Bacha, Pérsio Arida, Gustavo Franco e Mario Henrique Simonsen, que foram fundamentais para o sucesso do Plano Real.}}

Edmar Bacha destacou-se pela formulação e defesa do ajuste fiscal prévio, elucidando o chamado "efeito Tanzi às avessas". Sua contribuição foi fundamental ao expor que a inflação elevada ocultava déficits fiscais reais, justificando a necessidade urgente de ajustes estruturais no orçamento.

Pérsio Arida, juntamente com André Lara Resende, criou a proposta Larida, que inspirou diretamente o mecanismo da URV, proporcionando uma solução técnica e gradual para desindexar a economia brasileira sem causar choques econômicos abruptos.

Gustavo Franco desempenhou papel crucial na gestão da política cambial e monetária do Plano Real, coordenando a estratégia de âncora cambial e garantindo níveis adequados de reservas internacionais, essenciais para sustentar a credibilidade externa e interna do novo regime monetário.

Mario Henrique Simonsen foi um intelectual que forneceu importantes contribuições teóricas sobre inflação inercial e estratégias monetárias, influenciando diretamente a abordagem e a estratégia adotadas no Plano Real, especialmente no tocante ao gerenciamento das expectativas inflacionárias.

\section{\textbf{Qual o aspecto fundamental do Plano Real diferencia-se dos planos anteriores e, sendo assim, é o mais importante para explicar o sucesso em promover a estabilização de preços brasileira? Explique sua resposta.}}

O aspecto fundamental que diferenciou o Plano Real dos planos anteriores foi a criação da URV como um mecanismo de desindexação gradual e voluntária. Enquanto planos anteriores recorreram a congelamentos arbitrários e autoritários de preços e salários, resultando em distorções econômicas e escassez de bens, a URV permitiu ajustes naturais e consensuais dos preços relativos. Isso manteve o funcionamento do mercado eficiente, evitando choques abruptos e preservando a credibilidade econômica necessária para a estabilização efetiva dos preços.

\section{\textbf{Explique a importância da âncora cambial e as dificuldades a partir dessa opção no contexto econômico brasileiro de 1995 até 1998.}}

A âncora cambial foi decisiva no período inicial pós-Real para reduzir rapidamente as expectativas inflacionárias ao estabelecer uma referência clara e estável entre o Real e o Dólar. Contudo, essa estratégia trouxe desafios significativos para a economia brasileira. A valorização excessiva do Real resultante da fixação cambial prejudicou a competitividade das exportações brasileiras. Além disso, o país tornou-se altamente dependente da entrada de capital estrangeiro para manter essa estabilidade, tornando-se vulnerável a crises internacionais como as que ocorreram no México, Ásia e Rússia. Ademais, o agravamento do déficit em conta corrente e a necessidade de manter taxas de juros altas para sustentar a âncora cambial resultaram no aumento da dívida pública.

\section{\textbf{Qual foi o fator mais importante para entendermos que o repasse cambial em 1999 foi menor do que o esperado pela maioria das previsões dos analistas? Explique.}}

O fator crucial que explica o baixo repasse cambial em 1999 foi a implementação efetiva do regime de metas de inflação. Este regime, gerido com firmeza pelo COPOM, utilizou elevadas taxas de juros reais para limitar o repasse automático da desvalorização cambial aos preços. Além disso, a desindexação promovida anteriormente pelo Plano Real havia reduzido significativamente a memória inflacionária da população e dos agentes econômicos, contribuindo para estabilizar as expectativas inflacionárias, apesar da forte desvalorização.


\section{\textbf{Explique o sucesso da gestão do Presidente Lula na superação das consequências da crise cambial de 2002.}}

A despeito da incerteza que marcou o processo eleitoral de 2002, a equipe econômica anunciada na transição – Palocci na Fazenda e Meirelles no BC – sinalizou continuidade institucional. Logo em janeiro de 2003 a meta de superávit primário foi elevada para 4{,}25 \% do PIB, mesmo sem exigência do FMI, acompanhada de alta da Selic e renovação preventiva do acordo com o Fundo. Tais medidas reforçaram o \emph{tripé} (superávit, metas de inflação e câmbio flutuante), derrubaram o risco-país (de 2 400 pb para \(<\) 800 pb) e permitiram rápida apreciação do real abaixo de R\$ 3,00. 

\begin{itemize}
  \item \textbf{Ajuste externo rápido} – o déficit em conta-corrente de –1,7 \% do PIB em 2002 virou superávit de +2 \% em 2004 graças à forte desvalorização de 2002 e ao \emph{boom} de commodities. 
  \item \textbf{Reformas micro} – crédito consignado, nova Lei de Falências e revisão da COFINS ampliaram o mercado de crédito e diminuíram spreads. 
\end{itemize}

\section{\textbf{Como podemos entender as “reviravoltas” no discurso e na prática da política econômica durante os mandatos de Lula?}}

O primeiro mandato (2003-06) foi dominado pela ortodoxia de Palocci: superávits elevados, Selic alta e disciplina com o FMI. A crise do Mensalão (2005) minou esse arranjo político; a saída de Dirceu – fiador da coesão partidária – abriu espaço para pressões por gasto e levou à famosa máxima de Dilma: “gasto público é vida”. A partir daí, já no segundo mandato, o governo adotou o \emph{PAC}, expandiu o BNDES e recorreu a políticas anticíclicas, especialmente após a crise de 2008. 

\begin{itemize}
  \item \emph{Lula I}: consolidação do tripé, dívida/PIB em queda. 
  \item \emph{Pós-Mensalão}: recuo da ortodoxia, valorização do gasto corrente. 
  \item \emph{Lula II}: PAC, BNDES e política fiscal anticíclica. 
\end{itemize}

\section{\textbf{Efeitos esperados sobre o \emph{produto potencial} da consolidação do regime de metas de inflação e das reformas de 2003}}

A credibilidade do regime de metas, reforçada por superávits robustos e câmbio flutuante, reduziu prêmios de risco e possibilitou declínio sustentado dos juros reais. Em paralelo, reformas microeconômicas (consignado, falências, COFINS valor-adicionado) diminuíram o custo de capital e ampliaram a profundidade financeira. Esses fatores elevaram a taxa de investimento e a produtividade total dos fatores (PTF), deslocando para cima o produto potencial. 

\begin{itemize}
  \item \textbf{Menores juros de equilíbrio} $\;\Rightarrow\;$ mais investimento privado.
  \item \textbf{Mercados de crédito mais amplos} $\;\Rightarrow\;$ inovação e TFP maior.
\end{itemize}

\section{\textbf{Recuperação pós-crise de 2008: por que foi rápida?}}

Em 2008 o Brasil acumulava ~US\$ 200 bi de reservas, sistema bancário sólido (fruto do PROER) e contas externas equilibradas. O governo lançou pacotes fiscais e creditícios (BNDES, redução de impostos) que sustentaram a demanda interna, enquanto o ciclo de commodities voltou a favorecer os termos de troca em 2009. O resultado: queda marginal do PIB em 2009 (–0,1 \%) e crescimento de 7,5 \% em 2010, com inflação declinante e desemprego em baixa. 

\section{\textbf{Comparação fiscal e dinâmica da dívida: FHC vs. Lula}}

\paragraph{FHC (1995-2002).} Crises externas e juros muito altos mantiveram a dívida/PIB em ascensão, mesmo após a Lei de Responsabilidade Fiscal; a desvalorização de 1999 agravou o passivo.   

\paragraph{Lula I (2003-06).} Política fiscal inicialmente contracionista e forte apreciação cambial reduziram consistentemente a dívida/PIB — dois terços da queda devem-se ao efeito-câmbio.  

\paragraph{Lula II (2007-10).} Gasto corrente volta a crescer, mas o indicador fica estável porque o denominador (PIB nominal) avança em ritmo acelerado e o ambiente externo permanece benigno.

\section{\textbf{Erros de política na gestão Dilma – concorda?}}

Sim. A \emph{Nova Matriz Econômica} (NME) cortou a Selic em 500 pb, concedeu desonerações de ~1 \% do PIB em 2012 e impôs controles de preços em energia e combustíveis. A combinação minou expectativas, elevou inflação e transformou o superávit primário em déficit, culminando na perda do \emph{investment grade} em 2015. 

\begin{itemize}
  \item Juros reais \(<\) 2 \% sem âncora fiscal. 
  \item Desonerações setoriais de baixo retorno. 
  \item Intervenções em energia e Petrobras. 
\end{itemize}

\section{\textbf{Estrutura de oferta em 2011/12 – problema e desafios}}

Com desemprego historicamente baixo e utilização elevada da capacidade industrial, a economia já operava próximo ao pleno emprego. Gargalos logísticos, infraestrutura deficiente e câmbio valorizado comprimiam a competitividade manufatureira; pressões salariais mantinham a inflação resistente, limitando espaço para estímulos adicionais. O desafio era destravar investimento em infraestrutura e elevar PTF antes de sustentar um novo ciclo de demanda. 

\section{\textbf{O que foi a “nova matriz econômica”?}}

Visava “corrigir” dois preços-chave – juros altos e câmbio apreciado – por meio de cortes agressivos na Selic, intervenções cambiais pró-depreciação, crédito subsidiado (BNDES) e desonerações de R\$ 45 bi em 2012. Abandonou, assim, o predomínio das metas de inflação e do superávit como âncoras, substituindo-os por ação discricionária do Estado na alocação de preços e crédito. 

\begin{itemize}
  \item \textbf{Meta implícita}: elevar investimento a taxas \(>\) PIB. 
  \item \textbf{Contraste}: substitui o \emph{tripé} por política industrial-financeira ativa.
\end{itemize}

\section{\textbf{Como a NME explica a recessão de 2014/15?}}

A perda de credibilidade fiscal (“pedaladas”) elevou o prêmio de risco e encareceu o custo do capital; controles de preços represaram inflação, gerando forte choque quando liberados em 2015; crédito subsidiado induziu sobre-investimento em setores como caminhões, que se tornaram ociosos na contração. A soma desses fatores derrubou a formação bruta de capital e precipitou a pior recessão desde 1929. 

\section{\textbf{O setor externo foi o principal fator da recessão Dilma?}}

Não. Embora o fim do super-ciclo de commodities tenha deteriorado termos de troca, o país detinha reservas elevadas e posição credora líquida. O fator dominante foi doméstico: política fiscal expansionista, intervenções microeconômicas e quebra de confiança. 

\section{\textbf{Por que o ajuste de 2015 não recuperou o crescimento?}}

O pacote de Joaquim Levy combinou forte contração fiscal, realinhamento tarifário e aperto monetário, mas foi lançado em recessão adiantada e sem sustentação política. Em poucos meses, o Congresso esvaziou medidas-chave, a inflação saltou (pela liberação de preços administrados) e o risco de impeachment bloqueou investimento privado. 

\begin{itemize}
  \item \textbf{Ajuste simultâneo} $\rightarrow$ choque de oferta + choque de demanda.
  \item \textbf{Incerteza política} $\rightarrow$ prêmio de risco elevado persistente.
\end{itemize}

\section{\textbf{Políticas sociais: FHC, Lula e Dilma}}

\paragraph{FHC.} Programas condicionais (Bolsa Escola, Alimentação) criaram a infraestrutura, mas cobriam poucos beneficiários. 

\paragraph{Lula.} Priorização e massificação do Bolsa Família – carro-chefe com rápida redução da pobreza – além de Luz para Todos e Um Milhão de Cisternas.   

\paragraph{Dilma.} Manutenção do Bolsa Família, lançamento do Brasil Sem Miséria (retirou 8 mi de crianças da extrema pobreza) e do Mais Médicos para municípios carentes; eficácia, porém, limitada pela crise fiscal pós-2014. 

\section{\textbf{O governo Temer trouxe uma agenda reformista?}}

Sim. Em dois anos foram aprovados: (i) PEC 241/55 – Teto de Gastos; (ii) reforma trabalhista (nov./2017); (iii) instituição da TLP no BNDES; (iv) agenda BC+. Estas medidas restabeleceram a âncora fiscal-institucional, derrubaram inflação (IPCA 2,9 \% em 2017) e permitiram corte da Selic de 14 \% para 6,5 \%. 

\begin{itemize}
  \item \textbf{Teto} – limite real de gasto por 20 anos, sinalizando estabilização da dívida. 
  \item \textbf{Reforma trabalhista} – maior flexibilidade e possível queda da NAIRU. 
  \item \textbf{TLP \& BC+} – redução de subsídios e modernização financeira. 
\end{itemize}

\section{\textbf{Como a agenda 2016/17 melhorou o ambiente macro em 2017/18?}}

O Teto ancorou a trajetória esperada da dívida/PIB, as reformas micro sinalizaram ganhos de produtividade e o BC, com expectativas de inflação contidas, iniciou forte ciclo de corte de juros. O risco-país caiu, o câmbio se apreciou e o PIB voltou a crescer (~1 \% a.a.) em 2017-18, encerrando formalmente a recessão. 

\end{multicols}
\end{document}