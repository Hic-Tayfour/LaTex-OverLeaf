\documentclass{sciposter}
\fontsize{6pt}{7.2pt}\selectfont
\usepackage{lipsum}
\usepackage{epsfig}
\usepackage{amsmath}
\usepackage{amssymb}
\usepackage{multicol}
\usepackage{graphicx,url}
\usepackage[portuges, brazil]{babel}   
\usepackage[utf8]{inputenc}

\newtheorem{Def}{Definição}

\title{História Econômica do Brasil II}
\author{Hicham Munir Tayfour}
% Título do projeto

\institute 
{Bacharelado em Economia\\
Insper - Instituto de Ensino e Pesquisa\\
São Paulo, Brasil}
% Nome e endereço da Instituição

\rightlogo[1]{Imagens/logo-insper.png}  % Substitua pelo logo do Insper

\begin{document}

\conference{{\bf História Econômica do Brasil II}, Curso de Economia - Insper, 2025, São Paulo, Brasil}

\maketitle

%%% Início do ambiente Multicolunas
\begin{multicols}{3}
\fontsize{17pt}{7pt}\selectfont

%%% Resumo
\begin{abstract}
Este documento fornece um modelo de consulta para a Prova Intermediária de História Econômica do Brasil II. Ele organiza os principais conceitos e tópicos relevantes para facilitar a revisão do curso.
\end{abstract}

%%% Começa aqui


\section{\textbf{Considerando as múltiplas funções do Banco do Brasil na década de 1950: por que havia uma tendência de acelerar a inflação?}}

A tendência de aceleração inflacionária na década de 1950, em especial no contexto do Plano de Metas do governo Juscelino Kubitschek, está fortemente relacionada ao papel multifuncional do Banco do Brasil (BB) dentro do Sistema Financeiro Brasileiro (SFB) da época. O BB atuava simultaneamente como banco comercial e como agente de política econômica do governo, o que gerava sérios conflitos com a condução de uma política monetária restritiva. A seguir, destacam-se os principais fatores que explicam essa tendência:

\begin{itemize}
    \item \textbf{Vazamentos da política monetária:} 
    Dois canais principais permitiam a expansão da base monetária:
    \begin{enumerate}
        \item \textit{Emissão de papel-moeda:} Embora a emissão fosse feita formalmente pela Caixa de Amortização, o BB tinha o poder de colocar essa moeda em circulação, frequentemente por meio de concessão de crédito subsidiado ao setor agrícola, industrial ou mesmo a pessoas físicas.
        \item \textit{Transformação de moeda escritural em base monetária:} O BB, ao acolher os depósitos compulsórios dos demais bancos, podia expandir crédito de forma praticamente ilimitada, funcionando como criador autônomo de moeda.
    \end{enumerate}

    \item \textbf{Pressões políticas e sociais:} O Congresso Nacional, pressionado por diferentes grupos de interesse, frequentemente ampliava os limites de financiamento do BB, agravando a expansão monetária. O banco não era um agente irracional, mas sim instrumento de um sistema político-econômico no qual diversos atores demandavam continuamente crédito subsidiado.

    \item \textbf{Financiamento do Plano de Metas:} Os recursos necessários para a execução do plano vinham, em boa parte, de uma ``poupança forçada'' via inflação. Ou seja, o aumento da emissão monetária permitia financiar investimentos públicos e privados estratégicos, mesmo sem o correspondente aumento da arrecadação fiscal.

    \item \textbf{Instituições extrativistas:} De acordo com a interpretação de Acemoglu e Robinson, o modelo institucional vigente no Brasil era extrativista, favorecendo grupos específicos com acesso privilegiado a crédito e subsídios, em detrimento do equilíbrio macroeconômico de longo prazo.
\end{itemize}

Em síntese, a atuação do Banco do Brasil como elo entre a política monetária e os interesses do governo e da sociedade organizada produziu vazamentos monetários difíceis de conter. Isso estruturou uma dinâmica inflacionária persistente, que financiava o crescimento mas ao custo de graves desequilíbrios macroeconômicos.

\section{\textbf{Como podemos caracterizar o Plano de Metas? Quais foram as consequências produtivas e macroeconômicas a partir desse plano?}}

O Plano de Metas (1957--1961), conduzido durante o governo de Juscelino Kubitschek, pode ser caracterizado como uma estratégia de crescimento econômico baseada na industrialização acelerada, com forte protagonismo do Estado e articulação com o setor privado e o capital estrangeiro. Inspirado pelo pensamento estruturalista da CEPAL, o plano adotou uma lógica de substituição de importações (PSI) com foco em setores estratégicos.

\begin{itemize}
    \item \textbf{Características principais:}
    \begin{enumerate}
        \item \textit{Planejamento centralizado:} Elaboração de 30 metas distribuídas em cinco grandes áreas: energia, transporte, indústrias de base, alimentação e educação; além da construção de Brasília.
        \item \textit{Participação estatal:} O Estado liderou os investimentos por meio do Banco do Brasil, BNDE e Tesouro Nacional.
        \item \textit{Escolha de setores estratégicos:} Forte estímulo à indústria automobilística, construção naval e equipamentos elétricos, com subsídios, proteção tarifária e crédito direcionado.
        \item \textit{Articulação institucional:} Criação do Conselho de Desenvolvimento e do Conselho de Política Aduaneira; adoção da Lei do Similar Nacional e da Regra de Conteúdo Local (RCL).
    \end{enumerate}

    \item \textbf{Consequências produtivas:}
    \begin{itemize}
        \item Modernização da indústria nacional, com destaque para a consolidação da cadeia automobilística e de autopeças.
        \item Expansão da infraestrutura logística e energética.
        \item Nacionalização produtiva de setores até então fortemente dependentes de importações.
        \item Atração de Investimento Direto Estrangeiro (IDE) e urbanização acelerada.
    \end{itemize}

    \item \textbf{Consequências macroeconômicas:}
    \begin{itemize}
        \item \textit{Inflação crescente:} Resultado da “poupança forçada” viabilizada por emissão monetária, principalmente via Banco do Brasil.
        \item \textit{Desequilíbrio fiscal e externo:} A expansão dos gastos públicos e das importações pressionou as contas públicas e o balanço de pagamentos.
        \item \textit{Instituições extrativistas:} O modelo reforçou a concessão de benefícios seletivos a grupos específicos (via crédito, subsídios e proteção), em detrimento de medidas horizontais e universais.
        \item \textit{Crescimento com prazo de validade:} Apesar do forte dinamismo produtivo, os desequilíbrios acumulados geraram instabilidade e crises no início da década de 1960.
    \end{itemize}
\end{itemize}

Em síntese, o Plano de Metas representou um impulso significativo ao processo de industrialização brasileira, com ganhos estruturais importantes. No entanto, sua condução verticalizada e dependente de financiamento inflacionário comprometeu a sustentabilidade macroeconômica, contribuindo para a crise que se seguiu nos anos posteriores.

\section{\textbf{Explique o Plano Trienal (1962): qual era seu diagnóstico para a aceleração inflacionária? Quais foram as consequências a partir de suas medidas?}}

O Plano Trienal de Desenvolvimento Econômico e Social, lançado no final de 1962 sob coordenação de Celso Furtado durante o governo de João Goulart (Jango), representou uma tentativa de estabilização gradual da economia brasileira, conciliando crescimento com controle inflacionário e reformas sociais.

\begin{itemize}
    \item \textbf{Diagnóstico da inflação:}
    \begin{itemize}
        \item A inflação era vista como resultado da forte expansão dos gastos públicos nos anos anteriores, especialmente durante o governo JK.
        \item O diagnóstico combinava elementos ortodoxos (déficit fiscal como causa da inflação) e heterodoxos (necessidade de reformas estruturais para lidar com os gargalos de oferta).
        \item Reconhecia também a existência de expectativas inflacionárias e mecanismos de indexação que perpetuavam o processo inflacionário.
    \end{itemize}

    \item \textbf{Medidas propostas:}
    \begin{enumerate}
        \item \textit{Controle fiscal:} Contenção de gastos públicos e aumento de receitas, buscando o equilíbrio orçamentário.
        \item \textit{Política monetária restritiva:} Limitação à expansão do crédito, com medidas específicas como as Instruções 234 (para o BB) e 235 (para bancos privados), que restringiam a concessão de crédito.
        \item \textit{Política cambial realista:} Redução dos subsídios às importações e desvalorização cambial gradual.
        \item \textit{Política salarial racionalizada:} Tentativas de compatibilizar reajustes salariais com ganhos de produtividade.
        \item \textit{Congelamento de preços seletivo:} Acordos pontuais com setores empresariais para conter aumentos de preços (medida heterodoxa).
    \end{enumerate}

    \item \textbf{Consequências do plano:}
    \begin{itemize}
        \item \textit{Resistência política:} Medidas ortodoxas enfrentaram oposição tanto de sindicatos (pela perda salarial) quanto de setores empresariais (pela retirada de subsídios).
        \item \textit{Reação dos agentes:} Expectativa de congelamento levou a aumentos preventivos de preços, agravando a inflação no curto prazo.
        \item \textit{Inconsistência interna:} O plano combinava metas conflitantes — combate à inflação e crescimento de 7\% ao ano — o que gerou descrença na sua viabilidade.
        \item \textit{Fracasso e abandono:} Em maio de 1963, o plano começou a ser abandonado. Jango passou a apoiar aumentos salariais, rompendo com o espírito de estabilização.
        \item \textit{Recessão de 1963:} A economia entrou em retração, com desaceleração do PIB e alta inflação. Explicações variam entre esgotamento da PSI (Furtado e Tavares), turbulência política (Simonsen) e sequência de medidas sem resultados (Mesquita).
    \end{itemize}

    \item \textbf{Síntese:}
    O Plano Trienal foi uma tentativa sofisticada de compatibilizar estabilização e desenvolvimento, porém esbarrou nas limitações políticas do governo Jango, na reação adversa dos agentes econômicos e na fragilidade institucional do período. A recessão de 1963 e a aceleração inflacionária subsequente marcaram o fracasso da estratégia, reforçando a leitura de que as instituições brasileiras mantinham traços extrativistas, conforme o referencial de Acemoglu e Robinson.
\end{itemize}

\section{\textbf{Quais são as principais explicações para entendermos a recessão de 1963/64? Como podemos criticar essas explicações?}}

A recessão de 1963/64 — a primeira desde 1943 — é objeto de múltiplas interpretações que buscam explicar suas causas a partir de diferentes abordagens. As principais explicações podem ser agrupadas em três vertentes:

\begin{itemize}
    \item \textbf{ Esgotamento da Política de Substituição de Importações (PSI):}
    \begin{itemize}
        \item Defendida por autores como Celso Furtado e Maria da Conceição Tavares.
        \item Argumenta que o modelo de industrialização baseado na substituição de importações havia alcançado seus limites estruturais.
        \item Destaca gargalos no setor externo e dificuldades em ampliar o mercado interno sem gerar pressões inflacionárias e desequilíbrios cambiais.
    \end{itemize}
    \textit{Crítica:} Apesar de coerente, a hipótese do esgotamento estrutural não explica a intensidade da recessão no curto prazo. Além disso, desconsidera o papel das escolhas de política econômica e do ambiente político conturbado.

    \item \textbf{ Aceleração inflacionária e incerteza (Simonsen):}
    \begin{itemize}
        \item Inspirado na lógica monetarista, argumenta que a inflação crescente gerou incerteza nos agentes econômicos, inibindo investimentos e consumo.
        \item A instabilidade política desde 1961, com rutura institucional e mudanças no regime de governo, ampliou a incerteza.
    \end{itemize}
    \textit{Crítica:} Embora a incerteza inflacionária e política seja relevante, a tese peca por reduzir o problema a um choque de expectativas, sem considerar os fatores estruturais e a complexidade institucional do período.

    \item \textbf{ Sequência de planos de estabilização com custos concentrados (Mesquita):}
    \begin{itemize}
        \item Argumenta que o Brasil testou várias estratégias de estabilização (Quadros, Jango), mas sem resultados claros no curto prazo.
        \item Os custos (redução da demanda, queda salarial, escassez de crédito) foram sentidos de imediato, enquanto os ganhos não se concretizaram.
        \item Isso reduziu a confiança dos empresários e afastou o investimento estrangeiro.
    \end{itemize}
    \textit{Crítica:} Essa abordagem é eficaz em descrever o ambiente econômico imediato, mas não explora suficientemente os condicionantes políticos e institucionais que inviabilizaram a coordenação de políticas.

    \item \textbf{ Interpretação institucional:}
    \begin{itemize}
        \item Segundo a perspectiva de Acemoglu e Robinson, a recessão pode ser explicada pelo funcionamento de \textit{instituições extrativistas}, que favoreciam grupos específicos em detrimento da estabilidade macroeconômica.
        \item A ausência de instituições inclusivas e de uma coalizão política estável inviabilizou a implementação eficaz de qualquer plano de estabilização.
    \end{itemize}
    \textit{Contribuição:} Essa leitura oferece uma crítica de fundo às demais explicações, ao ressaltar que a fragilidade institucional brasileira — marcada por desigualdade, patrimonialismo e captura do Estado — limitava sistematicamente a eficácia das políticas econômicas.
\end{itemize}

\textbf{Síntese:} As explicações tradicionais — esgotamento do modelo, aceleração inflacionária, instabilidade política — ajudam a compreender elementos específicos da recessão de 1963/6 No entanto, uma análise mais robusta exige incorporar a dimensão institucional, que evidencia os limites estruturais do Estado brasileiro para coordenar políticas de estabilização em um ambiente de intensa disputa distributiva e baixa legitimidade.

\section{\textbf{Explicar as principais reformas estruturais propostas no PAEG e quais foram suas consequências de médio e longo prazos para a economia brasileira.}}

O Programa de Ação Econômica do Governo (PAEG), implementado entre 1964 e 1966 sob liderança do ministro Roberto Campos, representou uma resposta à crise de estagflação herdada do período anterior. O plano combinou políticas de estabilização gradual com um ambicioso conjunto de reformas estruturais, que moldaram o funcionamento da economia brasileira nas décadas seguintes.

\begin{itemize}
    \item \textbf{ Reforma Tributária}
    \begin{itemize}
        \item \textit{Objetivo:} Ampliar a arrecadação e racionalizar o sistema tributário.
        \item \textit{Medidas principais:}
        \begin{itemize}
            \item Criação do ICM (Imposto sobre Circulação de Mercadorias) e do ISS (Imposto Sobre Serviços).
            \item Ampliação da base do Imposto de Renda.
            \item Extinção de tributos ineficientes e modernização da arrecadação via rede bancária.
        \end{itemize}
        \item \textit{Consequências:}
        \begin{itemize}
            \item A carga tributária aumentou de 16\% para 21\% do PIB (1963–1967).
            \item Estrutura regressiva, com ênfase em tributos indiretos, impactando negativamente a distribuição de renda.
            \item Centralização do poder fiscal na União, via FPEM, reduzindo a autonomia de estados e municípios.
        \end{itemize}
    \end{itemize}

    \item \textbf{ Reforma Financeira}
    \begin{itemize}
        \item \textit{Objetivo:} Modernizar o sistema bancário e criar condições para o financiamento de longo prazo.
        \item \textit{Medidas principais:}
        \begin{itemize}
            \item Criação do Banco Central do Brasil e do Conselho Monetário Nacional.
            \item Expansão dos bancos de investimento e do mercado de capitais.
            \item Introdução da correção monetária (ORTN) para títulos públicos e contratos.
        \end{itemize}
        \item \textit{Consequências:}
        \begin{itemize}
            \item Consolidação de um sistema financeiro mais estruturado, com maior capacidade de intermediação.
            \item Viabilização do financiamento habitacional (via SFH) e de grandes investimentos.
            \item Institucionalização da indexação, com efeitos ambíguos: inicialmente reduziu a incerteza, mas consolidou a inflação inercial.
        \end{itemize}
    \end{itemize}

    \item \textbf{ Reforma do Mercado de Trabalho}
    \begin{itemize}
        \item \textit{Objetivo:} Aumentar a flexibilidade do mercado de trabalho e reduzir os custos para os empregadores.
        \item \textit{Medida principal:} Criação do Fundo de Garantia por Tempo de Serviço (FGTS), em substituição à estabilidade decenal.
        \item \textit{Consequências:}
        \begin{itemize}
            \item Maior dinamismo no mercado de trabalho, com estímulo à contratação.
            \item Redução da proteção ao trabalhador e enfraquecimento do poder sindical.
            \item Utilização do FGTS como instrumento de financiamento habitacional e de infraestrutura.
        \end{itemize}
    \end{itemize}

    \item \textbf{ Reforma Salarial}
    \begin{itemize}
        \item \textit{Objetivo:} Controlar a inflação de custos por meio de regras rígidas de reajuste salarial.
        \item \textit{Medidas principais:}
        \begin{itemize}
            \item Reajuste anual com base em inflação passada e crescimento da produtividade.
            \item Eliminação da negociação coletiva como principal forma de reajuste.
        \end{itemize}
        \item \textit{Consequências:}
        \begin{itemize}
            \item Queda real dos salários ao longo da década de 1960.
            \item Redução da parcela do trabalho na renda nacional.
            \item Contribuição para o sucesso da estabilização, mas com aumento da desigualdade.
        \end{itemize}
    \end{itemize}
\end{itemize}

\textbf{Síntese:} O PAEG promoveu um redesenho profundo das instituições econômicas brasileiras. No médio prazo, contribuiu para a estabilização monetária e para a retomada do crescimento — viabilizando, inclusive, o chamado “milagre econômico” a partir de 1968. No entanto, os custos sociais foram significativos, com aumento da desigualdade, enfraquecimento da negociação trabalhista e institucionalização de mecanismos que dificultariam o combate à inflação no longo prazo, como a indexação generalizada.

\section{\textbf{Como podemos estabelecer relações entre o Milagre econômico e: a taxa de inflação, o setor externo e a industrialização?}}

O período do “Milagre Econômico” brasileiro (1968–1973) foi marcado por altas taxas de crescimento do PIB (média de 11,2\% ao ano), combinadas com inflação relativamente controlada e fortalecimento da industrialização. Esse desempenho excepcional foi sustentado por um conjunto coordenado de políticas econômicas e institucionais, articulando política monetária expansionista, estímulo ao investimento e um ambiente externo favorável. As relações com os três pilares mencionados podem ser descritas da seguinte forma:

\begin{itemize}
    \item \textbf{ Inflação}
    \begin{itemize}
        \item \textit{Diagnóstico dominante:} A inflação era vista como “de custos”, especialmente atribuída ao alto custo do crédito.
        \item \textit{Política de contenção:}
        \begin{itemize}
            \item Concessão de crédito subsidiado ao setor agrícola e industrial.
            \item Criação do Orçamento Monetário, que segregava despesas financeiras e mascarava a verdadeira expansão fiscal.
            \item Políticas de controle direto de preços, via o CIP (Conselho Interministerial de Preços), que congelava reajustes de preços públicos e privados com base em planilhas de custos.
            \item Política salarial de arrocho (derretimento real dos salários), persistente desde o PAEG.
        \end{itemize}
        \item \textit{Resultado:} Apesar da forte expansão da demanda e do crédito, a inflação permaneceu em trajetória declinante, graças ao controle de preços, capacidade ociosa da economia e entrada de capitais externos.
    \end{itemize}

    \item \textbf{ Setor Externo}
    \begin{itemize}
        \item \textit{Balança comercial:} Houve expansão simultânea das exportações (favorecidas por incentivos fiscais e cambiais) e das importações (insumos para a industrialização), mantendo a balança comercial relativamente equilibrada.
        \item \textit{Conta de capitais (CK):} Superavitária, sustentada pela entrada de Investimento Direto Estrangeiro (IDE) e pela formação do mercado de eurodólares, com juros internacionais baixos.
        \item \textit{Política cambial:} Adoção do regime de \textit{crawling peg} (minidesvalorizações cambiais periódicas), para preservar a competitividade externa e compensar a inflação interna.
        \item \textit{Resultado:} Forte acúmulo de reservas internacionais e equilíbrio do balanço de pagamentos, o que sustentou a estabilidade macroeconômica do período.
    \end{itemize}

    \item \textbf{ Industrialização}
    \begin{itemize}
        \item \textit{Políticas setoriais ativas:}
        \begin{itemize}
            \item Fortes subsídios para setores estratégicos (autopeças, eletrodomésticos, máquinas e equipamentos).
            \item Incentivos ao setor exportador (têxteis, calçados, alimentos industrializados).
            \item Ampliação do crédito via sociedades de crédito e investimentos, promovendo o consumo de bens duráveis.
        \end{itemize}
        \item \textit{PED – Plano Estratégico de Desenvolvimento (1968):}
        \begin{itemize}
            \item Estímulo à indústria de bens de capital e bens intermediários (siderurgia, metalurgia).
            \item Expansão da linha branca (geladeira, fogão, TV), ligada ao processo de urbanização.
        \end{itemize}
        \item \textit{Resultado:} Fortalecimento da estrutura produtiva nacional, elevação da produtividade e diversificação do parque industrial.
    \end{itemize}
\end{itemize}

\textbf{Síntese:} O Milagre Econômico resultou de uma convergência entre ambiente externo favorável, políticas industriais seletivas, repressão salarial e controle de preços. Essa articulação permitiu combinar crescimento acelerado com inflação sob controle, saldo externo positivo e avanços significativos na industrialização. No entanto, o modelo dependia de fatores conjunturais não sustentáveis (juros baixos, abundância de capital externo, controle autoritário), e seu esgotamento tornou-se evidente após o primeiro choque do petróleo (1973).

\section{\textbf{Como o Presidente Geisel reagiu diante do 1º choque do Petróleo em 1973? Quais foram as consequências para a economia brasileira a partir dessa reação?}}

O presidente Ernesto Geisel assumiu o governo em 1974, em meio aos efeitos do 1º choque do petróleo (1973), que elevou bruscamente o preço do barril e agravou a vulnerabilidade externa de economias dependentes da importação de energia, como o Brasil. Diante desse cenário, Geisel optou por uma estratégia não recessiva, priorizando a manutenção do crescimento via aumento da capacidade produtiva e substituição de importações. Essa decisão resultou na formulação e execução do \textbf{II Plano Nacional de Desenvolvimento (II PND)}.

\begin{itemize}
    \item \textbf{Reação ao choque: principais decisões}
    \begin{itemize}
        \item \textit{Rejeição do ajuste recessivo:} Apesar das pressões inflacionárias e do desequilíbrio externo, Geisel evitou políticas ortodoxas de retração da demanda, por temer os custos políticos e sociais.
        \item \textit{Adoção de uma estratégia estruturalista de longo prazo:} Lançamento do II PND (1974--1979), que visava transformar a estrutura produtiva do país para reduzir a dependência de bens importados e de petróleo.
        \item \textit{Financiamento externo:} A estratégia baseou-se em endividamento externo em um contexto de abundância de crédito internacional (petrodólares), com juros ainda baixos.
    \end{itemize}

    \item \textbf{Instrumentos e áreas prioritárias do II PND}
    \begin{itemize}
        \item \textit{Infraestrutura:} Expansão da malha ferroviária, hidroelétrica e do setor de telecomunicações.
        \item \textit{Indústria pesada:} Fortes investimentos em bens de capital, aço, alumínio, petroquímica, fertilizantes e papel e celulose.
        \item \textit{Setor energético:} Ampliação da produção nacional de petróleo (Bacia de Campos), desenvolvimento da energia hidroelétrica (Itaipu, Tucuruí) e incentivo à energia alternativa (etanol, nuclear).
        \item \textit{Política industrial e tecnológica:} Incentivos fiscais, crédito subsidiado, proteção via tarifas e reservas de mercado.
    \end{itemize}

    \item \textbf{Consequências econômicas}
    \begin{itemize}
        \item \textit{Crescimento econômico:} O PIB manteve taxas elevadas (média de 7\% ao ano), com expansão da capacidade produtiva e da base industrial.
        \item \textit{Redução da dependência de insumos estratégicos:} Queda nominal nas importações de bens básicos e aumento da produção nacional de insumos industriais.
        \item \textit{Aumento do endividamento externo:} A dívida externa brasileira passou de US\$ 12,5 bilhões (1974) para mais de US\$ 21 bilhões (1978), e os encargos com juros aumentaram rapidamente.
        \item \textit{Desequilíbrio fiscal:} A expansão dos gastos públicos e dos subsídios comprometeu a posição fiscal do Estado.
        \item \textit{Pressão inflacionária:} Apesar das tentativas de controle, a inflação voltou a crescer na segunda metade da década, pressionada pelo excesso de demanda e pelos efeitos dos choques externos subsequentes.
    \end{itemize}

    \item \textbf{Avaliação crítica}
    \begin{itemize}
        \item \textit{Avanços:} O plano gerou ganhos produtivos e estruturais, como a diversificação da base industrial, redução das importações estratégicas e fortalecimento do setor energético.
        \item \textit{Limites:} O modelo baseava-se em protecionismo, subsídios e dívida externa crescente — pilares insustentáveis no longo prazo.
        \item \textit{Legado:} O sucesso de curto prazo gerou passivos que se materializaram na crise da dívida, inflação elevada e colapso do modelo na década de 1980.
    \end{itemize}
\end{itemize}

\textbf{Síntese:} A reação de Geisel ao 1º choque do petróleo privilegiou o crescimento com endividamento, evitando medidas recessivas e apostando na substituição de importações. Embora tenha gerado avanços industriais e energéticos, o custo foi uma deterioração fiscal e externa, que culminou na chamada “década perdida” nos anos seguintes.

\section{\textbf{Como podemos caracterizar o II PND? Quais eram seus objetivos? E quais foram as consequências?}}

O II Plano Nacional de Desenvolvimento (II PND), implementado entre 1974 e 1979 no governo Ernesto Geisel, foi a principal resposta da política econômica brasileira ao 1º choque do petróleo (1973). Inserido em um contexto de crescimento com restrições externas, o plano buscava preservar o dinamismo econômico sem recorrer a um ajuste recessivo. Foi uma estratégia estruturalista, fortemente intervencionista, baseada em planejamento estatal, endividamento externo e políticas industriais seletivas.

\begin{itemize}
    \item \textbf{Caracterização do plano:}
    \begin{itemize}
        \item Plano de longo prazo, com foco na transformação estrutural da economia.
        \item Priorizava setores estratégicos: bens de capital, insumos básicos e energia.
        \item Manteve a lógica da substituição de importações, mas agora em setores mais sofisticados.
        \item Fortemente financiado por crédito externo, com participação ativa do Estado e das estatais.
    \end{itemize}

    \item \textbf{Objetivos principais:}
    \begin{enumerate}
        \item \textit{Reduzir a vulnerabilidade externa} ao petróleo e a insumos industriais importados.
        \item \textit{Expandir a infraestrutura energética e logística}, garantindo base para o crescimento.
        \item \textit{Fortalecer a capacidade industrial nacional} em áreas de alta intensidade de capital e tecnologia.
        \item \textit{Manter taxas elevadas de crescimento econômico}, mesmo diante do cenário internacional adverso.
    \end{enumerate}

    \item \textbf{Instrumentos utilizados:}
    \begin{itemize}
        \item Incentivos fiscais e crédito subsidiado (via BNDE).
        \item Reserva de mercado e proteção tarifária.
        \item Participação ativa das estatais nos setores estratégicos.
        \item Captação de financiamento externo (petrodólares).
    \end{itemize}

    \item \textbf{Consequências:}
    \begin{itemize}
        \item \textit{Produtivas e estruturais:}
        \begin{itemize}
            \item Expansão da infraestrutura energética (hidrelétricas, Itaipu, Bacia de Campos, Nuclebrás).
            \item Desenvolvimento da indústria pesada e de bens de capital.
            \item Redução da participação dos bens de capital importados nos investimentos nacionais.
            \item Fortalecimento da base exportadora industrial.
        \end{itemize}
        
        \item \textit{Macroeconômicas:}
        \begin{itemize}
            \item Forte aumento do endividamento externo (US\$ 12,5 bi em 1974 → US\$ 21 bi em 1978).
            \item Pressão inflacionária crescente e deterioração fiscal.
            \item Fragilidade estrutural do setor público devido aos subsídios e desonerações.
            \item Sustentação do crescimento a curto prazo, mas com altos custos de médio e longo prazo.
        \end{itemize}

        \item \textit{Crítica e legado:}
        \begin{itemize}
            \item Embora tenha gerado ganhos estruturais, o modelo foi financeiramente insustentável.
            \item Criou as condições para a crise da dívida e a “década perdida” nos anos 1980.
            \item Reforçou práticas extrativistas, com benefícios concentrados e sem contrapartidas produtivas universais.
        \end{itemize}
    \end{itemize}
\end{itemize}

\textbf{Síntese:} O II PND foi um ambicioso plano de industrialização via endividamento externo, que garantiu crescimento e avanços estruturais no curto prazo, mas comprometeu a sustentabilidade macroeconômica do país no longo prazo, deixando como herança uma economia altamente endividada, indexada e desigual.

\section{\textbf{Como podemos explicar o início da recessão em 1981?}}

O início da recessão em 1981 marcou uma inflexão no ciclo de crescimento iniciado nos anos 1970 e é considerado o ponto de partida da chamada “década perdida” da economia brasileira. A recessão resultou da convergência de choques externos adversos, esgotamento do modelo de crescimento com endividamento e da mudança de orientação da política econômica no governo Figueiredo.

\begin{itemize}
    \item \textbf{ Contexto internacional adverso:}
    \begin{itemize}
        \item \textit{2º choque do petróleo (1979):} Elevou novamente os preços do petróleo, pressionando a balança comercial brasileira.
        \item \textit{Aumento dos juros internacionais:} A política monetária restritiva dos EUA (Volcker shock) elevou os juros, encarecendo o serviço da dívida externa.
        \item \textit{Desaceleração do comércio global:} Reduziu a demanda por exportações brasileiras, agravando a restrição externa.
    \end{itemize}

    \item \textbf{ Esgotamento do modelo de crescimento com endividamento:}
    \begin{itemize}
        \item A estratégia adotada desde o II PND baseava-se na expansão do investimento público e no financiamento externo.
        \item O crescimento do serviço da dívida tornou esse modelo insustentável: os pagamentos de juros passaram de US\$ 600 milhões (1974) para mais de US\$ 4 bilhões em
    \end{itemize}
\end{itemize}

\section{\textbf{Explicar o processo de ajustamento do setor externo brasileiro ao longo da primeira metade da década de 1980.}}

O ajustamento do setor externo brasileiro entre 1981 e 1985 ocorreu em resposta à grave crise da dívida externa e à deterioração do balanço de pagamentos. O objetivo central era gerar superávits comerciais suficientes para garantir o pagamento do serviço da dívida externa e restaurar a credibilidade internacional do país. Esse processo foi marcado por políticas recessivas, desvalorização cambial e controle do consumo interno.

\begin{itemize}
    \item \textbf{ Contexto da crise}
    \begin{itemize}
        \item O Brasil entrou nos anos 1980 com alto nível de endividamento externo (US\$ 80 bilhões em 1982) e crescente dificuldade de rolagem.
        \item A elevação abrupta das taxas de juros internacionais, provocada pela política do Fed (Volcker), aumentou drasticamente o custo do serviço da dívida.
        \item O segundo choque do petróleo (1979) e a recessão global agravaram o desequilíbrio da balança de pagamentos.
        \item Em 1982, após a moratória do México, o Brasil perdeu acesso ao crédito voluntário internacional.
    \end{itemize}

    \item \textbf{ Estratégia de ajustamento}
    \begin{itemize}
        \item \textit{Meta:} Gerar superávit comercial crescente para garantir os pagamentos externos e restaurar a solvência.
        \item \textit{Política cambial:} 
        \begin{itemize}
            \item Adoção de forte desvalorização real do câmbio (maxidesvalorizações e \textit{crawling peg}) para incentivar exportações e desestimular importações.
        \end{itemize}
        \item \textit{Política fiscal e monetária:}
        \begin{itemize}
            \item Cortes no gasto público e aumento da taxa de juros real para conter a demanda agregada e reduzir importações.
            \item Redução do investimento público e do crédito direcionado.
        \end{itemize}
        \item \textit{Controle das importações:}
        \begin{itemize}
            \item Reforço de barreiras tarifárias e não-tarifárias para conter o déficit em conta corrente.
        \end{itemize}
    \end{itemize}

    \item \textbf{ Resultados do ajustamento}
    \begin{itemize}
        \item \textit{Superávit comercial expressivo:} O saldo da balança comercial passou de déficit em 1980 para superávits crescentes a partir de 198
        \item \textit{Compressão de importações:} As importações caíram fortemente, especialmente de bens de capital e insumos intermediários, afetando a produção industrial.
        \item \textit{Estagnação econômica:} O ajuste externo foi recessivo — queda do PIB em 1981 e 1983 — com forte impacto negativo sobre o investimento e o emprego.
        \item \textit{Aumento da inflação:} Apesar do controle cambial e fiscal, o processo foi acompanhado por inflação crescente, alimentada pela indexação e pelos efeitos das desvalorizações.
        \item \textit{Negociações com o FMI:} O Brasil recorreu ao Fundo Monetário Internacional (FMI) e aceitou condicionalidades de ajuste estrutural para obter financiamentos emergenciais.
    \end{itemize}

    \item \textbf{ Avaliação crítica}
    \begin{itemize}
        \item O ajustamento foi bem-sucedido em gerar superávit comercial e evitar default imediato.
        \item No entanto, teve alto custo social e econômico, aprofundando a recessão e a crise distributiva.
        \item A ausência de reformas estruturais e a permanência da indexação comprometeram a eficácia do ajuste no controle da inflação e na retomada do crescimento.
    \end{itemize}
\end{itemize}

\textbf{Síntese:} O processo de ajustamento externo do Brasil nos anos 1980 priorizou a geração de superávits comerciais via recessão e desvalorização cambial. Embora tenha restaurado temporariamente o equilíbrio externo, teve como contrapartida estagnação econômica, aumento da inflação e aprofundamento da crise social e fiscal.

\section{\textbf{Como podemos entender a formação do processo inflacionário brasileiro em perspectiva histórica?}}

A formação do processo inflacionário brasileiro deve ser compreendida como resultado de uma combinação de fatores estruturais, institucionais e conjunturais ao longo do século XX. A inflação no Brasil não seguiu um único padrão causal, mas evoluiu a partir de diferentes mecanismos ao longo do tempo, refletindo mudanças na política econômica, nas instituições e nas condições externas.

\begin{itemize}
    \item \textbf{ Origem e estrutura institucional (décadas de 1940–1960):}
    \begin{itemize}
        \item A inflação ganhou força com a consolidação do modelo de industrialização por substituição de importações (ISI).
        \item O Estado utilizava a emissão monetária, via Banco do Brasil, como principal forma de financiar investimentos e déficits — um arranjo institucional que criou “vazamentos monetários”.
        \item A ausência de uma autoridade monetária independente favoreceu a prática de financiamento inflacionário do gasto público.
        \item A inflação passou a ser funcional ao modelo, viabilizando uma “poupança forçada” para investimentos estatais.
    \end{itemize}

    \item \textbf{ Aceleração e indexação (anos 1960–1970):}
    \begin{itemize}
        \item O PAEG reconheceu a natureza multifatorial da inflação (pressões distributivas, estrutura produtiva e monetária).
        \item A introdução da correção monetária (ORTN) no período visava proteger contratos da corrosão inflacionária, mas institucionalizou a \textit{indexação}, tornando a inflação mais resiliente.
        \item Apesar da inflação moderada durante o “Milagre Econômico”, o arrocho salarial e o controle de preços foram utilizados como âncoras não convencionais.
    \end{itemize}

    \item \textbf{ Inflação inercial e choques externos (anos 1980):}
    \begin{itemize}
        \item O segundo choque do petróleo e a crise da dívida externa tornaram insustentável o financiamento do crescimento.
        \item O ajuste externo recessivo elevou os custos e desorganizou a estrutura produtiva.
        \item A inflação tornou-se \textit{inercial}, ou seja, autossustentada por mecanismos de indexação e expectativa de reajustes contínuos.
        \item Tentativas de controle via políticas ortodoxas fracassaram, e os planos heterodoxos (como o Cruzado) não atacaram adequadamente os fundamentos fiscais e monetários.
    \end{itemize}

    \item \textbf{ Papel das instituições extrativistas:}
    \begin{itemize}
        \item Segundo a perspectiva de Acemoglu e Robinson, o padrão inflacionário brasileiro também refletia um arranjo institucional extrativista, que permitia a apropriação do Estado por grupos específicos.
        \item A inflação funcionava como mecanismo de redistribuição regressiva da renda (via imposto inflacionário), afetando desproporcionalmente os mais pobres.
        \item As tentativas de estabilização fracassavam em grande parte porque os custos da mudança eram concentrados, enquanto os benefícios eram difusos.
    \end{itemize}

    \item \textbf{ Legado:}
    \begin{itemize}
        \item Ao longo de décadas, a inflação brasileira passou por múltiplas fases: demanda excessiva, custo, indexação e inércia.
        \item Somente nos anos 1990, com o Plano Real, a combinação de âncora nominal, reforma monetária e responsabilidade fiscal conseguiu desarmar o núcleo inercial da inflação.
    \end{itemize}
\end{itemize}

\textbf{Síntese:} O processo inflacionário brasileiro foi historicamente condicionado por instituições permissivas à emissão monetária, práticas de financiamento inflacionário, choques externos e mecanismos de indexação que perpetuavam os aumentos de preços. Entender essa trajetória exige uma leitura multidimensional que articule economia política, estrutura produtiva e restrições institucionais.

\section{\textbf{Quais foram as principais propostas para promover a desindexação da economia brasileira na década de 1980? Explique.}}

Na década de 1980, a inflação brasileira atingiu níveis crônicos e autossustentados, caracterizando-se como \textit{inflação inercial}. Nesse contexto, a desindexação — ou seja, o rompimento dos mecanismos automáticos de reajuste de preços, salários e contratos financeiros — tornou-se central nas propostas de estabilização. Diversos planos foram elaborados com esse objetivo, ainda que com diferentes enfoques e graus de sucesso.

\begin{itemize}
    \item \textbf{ Diagnóstico da inflação inercial:}
    \begin{itemize}
        \item A inflação já não era explicada apenas por desequilíbrios fiscais ou monetários, mas pela persistência de mecanismos de correção automática.
        \item A indexação generalizada (salários, contratos de aluguel, tarifas públicas, preços industriais e instrumentos financeiros) tornava a inflação resistente a choques de demanda ou medidas ortodoxas.
        \item Desse modo, a estabilização exigia um “choque de desindexação” coordenado e abrangente.
    \end{itemize}

    \item \textbf{ Propostas e experiências de desindexação:}

    \begin{itemize}
        \item \textbf{Plano Cruzado (1986):}
        \begin{itemize}
            \item \textit{Medidas:} Congelamento de preços, salários e câmbio; abolição da correção monetária em contratos de curto prazo; substituição da ORTN pela OTN com valor fixo por um ano.
            \item \textit{Objetivo:} Realizar um “choque heterodoxo” de desindexação total, rompendo com a inércia inflacionária.
            \item \textit{Limites:} O plano não atacou os desequilíbrios fiscais, houve aumento da demanda agregada e escassez generalizada. O congelamento foi rompido e a inflação voltou a acelerar.
        \end{itemize}

        \item \textbf{Plano Bresser (1987):}
        \begin{itemize}
            \item \textit{Medidas:} Novo congelamento parcial de preços e salários, tentativa de ajuste fiscal via corte de gastos e aumento de tributos.
            \item \textit{Objetivo:} Estabilizar sem recorrer à emissão monetária, mas com menor rigidez do que no Cruzado.
            \item \textit{Limites:} A desconfiança dos agentes econômicos e a persistência da indexação em contratos financeiros limitaram os efeitos do plano.
        \end{itemize}

        \item \textbf{Plano Verão (1989):}
        \begin{itemize}
            \item \textit{Medidas:} Substituição da moeda (cruzado novo), congelamento de preços e salários, e tentativa de ancorar as expectativas com a nova unidade monetária.
            \item \textit{Objetivo:} Romper a indexação, associando à política monetária mais restritiva.
            \item \textit{Limites:} A inflação já estava acima de 30\% ao mês e os mecanismos de indexação informal continuavam ativos; o plano fracassou em poucos meses.
        \end{itemize}
    \end{itemize}

    \item \textbf{ Dificuldades enfrentadas:}
    \begin{itemize}
        \item Forte resistência de grupos com contratos indexados (empresas, setor financeiro).
        \item Expectativas inflacionárias arraigadas e ausência de um plano fiscal robusto.
        \item Baixa credibilidade dos planos e do governo junto aos agentes econômicos.
        \item Manutenção de mecanismos de indexação informal mesmo após as medidas formais de congelamento.
    \end{itemize}
\end{itemize}

\textbf{Síntese:} As propostas de desindexação nos anos 1980 basearam-se em choques heterodoxos com congelamentos generalizados e tentativas de ancoragem nominal. No entanto, a ausência de disciplina fiscal, a persistência da indexação informal e a baixa credibilidade dos planos comprometeram sua eficácia. A desindexação só seria bem-sucedida anos depois, com o Plano Real (1994), que combinou ancoragem cambial, ajuste fiscal e reforma monetária gradual.

\section{\textbf{Qual foi o maior erro e, portanto, o principal causador para entendermos o fracasso do Plano Cruzado? Explique sua resposta.}}

O maior erro do Plano Cruzado (1986) — e, portanto, o principal fator responsável por seu fracasso — foi a ausência de um ajuste fiscal efetivo que acompanhasse o congelamento de preços e salários. Embora o plano tenha obtido sucesso inicial ao conter a inflação inercial por meio de um congelamento amplo, ele falhou ao não enfrentar os desequilíbrios estruturais das contas públicas, que sustentavam o processo inflacionário.

\begin{itemize}
    \item \textbf{ Falta de ajuste fiscal consistente:}
    \begin{itemize}
        \item O congelamento dos preços foi realizado sem que houvesse uma política clara de contenção do gasto público ou aumento de receitas permanentes.
        \item O pacote fiscal anunciado em dezembro de 1985 não foi implementado plenamente após a queda da inflação, resultando em um déficit crescente (2,5\% do PIB em 1986).
        \item Com a queda do imposto inflacionário e o aumento dos subsídios e salários, o desequilíbrio fiscal se agravou.
    \end{itemize}

    \item \textbf{ Expansão da demanda agregada:}
    \begin{itemize}
        \item O plano elevou o poder de compra da população com abono salarial e congelamento de preços, sem contrapartida na capacidade produtiva.
        \item Houve explosão do consumo, esgotamento dos estoques e surgimento de desabastecimento em diversos setores (alimentos, automóveis, vestuário).
        \item A política monetária foi passiva: houve aumento da base monetária para atender à demanda por moeda estável, aprofundando o excesso de liquidez.
    \end{itemize}

    \item \textbf{ Ilusão de estabilidade e ausência de mecanismos de correção:}
    \begin{itemize}
        \item O sucesso inicial do congelamento gerou euforia política e eleitoral, inibindo a adoção de medidas corretivas durante a campanha de 1986.
        \item O governo não reconheceu os sinais de desequilíbrio e preferiu manter a paralisia decisória até as eleições.
        \item O “Cruzadinho” e o “Cruzado II” vieram tarde demais e de forma descoordenada, agravando a instabilidade.
    \end{itemize}

    \item \textbf{ Retorno da indexação e perda de credibilidade:}
    \begin{itemize}
        \item Com o fracasso do congelamento e o retorno dos reajustes desordenados, a inflação voltou de forma acelerada e desorganizada.
        \item A reintrodução de minidesvalorizações cambiais e da correção monetária financeira a partir do Cruzado II restabeleceu o núcleo inercial da inflação.
        \item A economia passou a operar sob expectativas de que nenhum congelamento teria efeito duradouro, tornando futuras tentativas menos críveis.
    \end{itemize}
\end{itemize}

\textbf{Síntese:} O principal erro do Plano Cruzado foi tratar a inflação como um problema exclusivamente inercial, negligenciando sua base fiscal. Sem o apoio de um ajuste estrutural nas contas públicas, o congelamento de preços teve vida curta e gerou distorções que comprometeram sua sustentabilidade. O plano perdeu credibilidade, aprofundou desequilíbrios e contribuiu para a aceleração inflacionária dos anos seguintes.

\section{\textbf{Em quais pontos os Planos Bresser e Verão são semelhantes e em quais divergiram do Plano Cruzado?}}

Os Planos Bresser (1987) e Verão (1989) surgiram após o fracasso do Plano Cruzado (1986), em um contexto de inflação crônica e desorganização macroeconômica. Embora compartilhassem com o Cruzado o objetivo de romper a inflação inercial por meio de mecanismos de desindexação e controle de preços, ambos buscaram corrigir erros anteriores com novas estratégias. A seguir, são destacados os pontos de semelhança e de divergência.

\begin{itemize}
    \item \textbf{ Semelhanças entre Bresser, Verão e Cruzado:}
    \begin{itemize}
        \item \textit{Diagnóstico inercialista:} Todos os planos partiram da premissa de que a inflação era autossustentada por mecanismos de indexação e expectativas de reajuste.
        \item \textit{Uso do congelamento de preços e salários:} As três experiências adotaram, em maior ou menor grau, o congelamento como instrumento central para interromper o ciclo inercial.
        \item \textit{Tentativa de desindexação:} Houve esforços para eliminar ou restringir a correção monetária nos contratos, especialmente os de curto prazo.
        \item \textit{Mudança de moeda:} Tanto o Cruzado quanto o Verão envolveram uma reforma monetária — o Cruzado substituiu o Cruzeiro, e o Verão instituiu o Cruzado Novo.
    \end{itemize}

    \item \textbf{ Divergências em relação ao Plano Cruzado:}

    \textbf{Plano Bresser (1987):}
    \begin{itemize}
        \item \textit{Congelamento parcial e temporário:} Ao contrário do congelamento amplo e indefinido do Cruzado, o Bresser congelou preços e salários por 90 dias, com expectativa de liberalização gradual.
        \item \textit{Ajuste fiscal explícito:} Incluiu medidas de contenção de gastos e aumento de receitas, reconhecendo a necessidade de correção estrutural do déficit público.
        \item \textit{Negociação com setores organizados:} Procurou pactuar aumentos salariais e preços futuros com base em parâmetros oficiais.
    \end{itemize}

    \textbf{Plano Verão (1989):}
    \begin{itemize}
        \item \textit{Maior ortodoxia:} O plano combinou o congelamento com política monetária restritiva e elevação de juros reais.
        \item \textit{Fim da OTN:} Houve tentativa de remover instrumentos de indexação financeira, rompendo com a lógica da ORTN/OTN.
        \item \textit{Tentativa de ancoragem nominal:} A nova moeda, o Cruzado Novo, foi usada como âncora psicológica e tentativa de redefinir expectativas.
        \item \textit{Contexto de hiperinflação iminente:} O plano foi lançado em cenário muito mais crítico do que o Cruzado, com inflação acima de 30\% ao mês.
    \end{itemize}

    \item \textbf{ Avaliação comparativa:}
    \begin{itemize}
        \item O Plano Cruzado apostou em uma solução abrupta, baseada na mobilização popular e no congelamento total, sem correção fiscal.
        \item O Bresser tentou combinar gradualismo com ajuste fiscal, mas foi politicamente fragilizado.
        \item O Verão adotou medidas mais ortodoxas e duras, mas enfrentou um cenário de colapso inflacionário e descrença generalizada.
    \end{itemize}
\end{itemize}

\textbf{Síntese:} Embora Bresser, Verão e Cruzado compartilhassem a lógica inercialista e o uso do congelamento, os dois primeiros buscaram corrigir os erros do Cruzado com maior realismo fiscal e moderação na política de preços. Ainda assim, nenhum dos planos conseguiu romper de forma definitiva com a inflação crônica, devido à persistência da indexação informal, fragilidade fiscal e falta de credibilidade institucional.

\section{\textbf{Em quais pontos os Planos Bresser e Verão são semelhantes e em quais divergiram do Plano Cruzado?}}

Os Planos Bresser (1987) e Verão (1989) surgiram após o fracasso do Plano Cruzado (1986), em um contexto de inflação crônica e desorganização macroeconômica. Embora compartilhassem com o Cruzado o objetivo de romper a inflação inercial por meio de mecanismos de desindexação e controle de preços, ambos buscaram corrigir erros anteriores com novas estratégias. A seguir, são destacados os pontos de semelhança e de divergência.

\begin{itemize}
    \item \textbf{ Semelhanças entre Bresser, Verão e Cruzado:}
    \begin{itemize}
        \item \textit{Diagnóstico inercialista:} Todos os planos partiram da premissa de que a inflação era autossustentada por mecanismos de indexação e expectativas de reajuste.
        \item \textit{Uso do congelamento de preços e salários:} As três experiências adotaram, em maior ou menor grau, o congelamento como instrumento central para interromper o ciclo inercial.
        \item \textit{Tentativa de desindexação:} Houve esforços para eliminar ou restringir a correção monetária nos contratos, especialmente os de curto prazo.
        \item \textit{Mudança de moeda:} Tanto o Cruzado quanto o Verão envolveram uma reforma monetária — o Cruzado substituiu o Cruzeiro, e o Verão instituiu o Cruzado Novo.
    \end{itemize}

    \item \textbf{ Divergências em relação ao Plano Cruzado:}

    \textbf{Plano Bresser (1987):}
    \begin{itemize}
        \item \textit{Congelamento parcial e temporário:} Ao contrário do congelamento amplo e indefinido do Cruzado, o Bresser congelou preços e salários por 90 dias, com expectativa de liberalização gradual.
        \item \textit{Ajuste fiscal explícito:} Incluiu medidas de contenção de gastos e aumento de receitas, reconhecendo a necessidade de correção estrutural do déficit público.
        \item \textit{Negociação com setores organizados:} Procurou pactuar aumentos salariais e preços futuros com base em parâmetros oficiais.
    \end{itemize}

    \textbf{Plano Verão (1989):}
    \begin{itemize}
        \item \textit{Maior ortodoxia:} O plano combinou o congelamento com política monetária restritiva e elevação de juros reais.
        \item \textit{Fim da OTN:} Houve tentativa de remover instrumentos de indexação financeira, rompendo com a lógica da ORTN/OTN.
        \item \textit{Tentativa de ancoragem nominal:} A nova moeda, o Cruzado Novo, foi usada como âncora psicológica e tentativa de redefinir expectativas.
        \item \textit{Contexto de hiperinflação iminente:} O plano foi lançado em cenário muito mais crítico do que o Cruzado, com inflação acima de 30\% ao mês.
    \end{itemize}

    \item \textbf{ Avaliação comparativa:}
    \begin{itemize}
        \item O Plano Cruzado apostou em uma solução abrupta, baseada na mobilização popular e no congelamento total, sem correção fiscal.
        \item O Bresser tentou combinar gradualismo com ajuste fiscal, mas foi politicamente fragilizado.
        \item O Verão adotou medidas mais ortodoxas e duras, mas enfrentou um cenário de colapso inflacionário e descrença generalizada.
    \end{itemize}
\end{itemize}

\textbf{Síntese:} Embora Bresser, Verão e Cruzado compartilhassem a lógica inercialista e o uso do congelamento, os dois primeiros buscaram corrigir os erros do Cruzado com maior realismo fiscal e moderação na política de preços. Ainda assim, nenhum dos planos conseguiu romper de forma definitiva com a inflação crônica, devido à persistência da indexação informal, fragilidade fiscal e falta de credibilidade institucional.

\section{\textbf{O Plano Collor foi um plano de estabilização, sobretudo, de perfil heterodoxo. Você concorda? Explique sua resposta.}}

Sim, o Plano Collor I (março de 1990) foi, essencialmente, um plano de estabilização de perfil heterodoxo. Embora incluísse elementos ortodoxos, como metas fiscais e monetárias, seu núcleo estratégico baseava-se em uma intervenção drástica e não convencional para romper a hiperinflação — característica típica de planos heterodoxos.

\begin{itemize}
    \item \textbf{ Justificativa para o perfil heterodoxo:}
    \begin{itemize}
        \item \textit{Diagnóstico inercialista:} Assim como os planos Cruzado, Bresser e Verão, o Plano Collor considerava que a inflação brasileira era alimentada por mecanismos de indexação e expectativas de reajuste contínuo.
        \item \textit{Choque de liquidez:} A medida mais radical do plano foi o confisco de parte da base monetária e dos ativos financeiros da população (bloqueio de 80\% dos depósitos bancários acima de Cr\$ 50 mil por 18 meses), com o objetivo de interromper a circulação monetária e quebrar a espiral inflacionária.
        \item \textit{Congelamento de preços e salários:} Implementou congelamento generalizado por 30 dias, como tentativa de desindexação abrupta da economia.
        \item \textit{Reforma monetária:} Substituição do Cruzado Novo pelo Cruzeiro, a uma taxa de conversão de 1:
    \end{itemize}

    \item \textbf{ Medidas de caráter ortodoxo (secundárias):}
    \begin{itemize}
        \item Compromisso com o equilíbrio fiscal e controle da emissão monetária.
        \item Tentativas de ancoragem monetária com apoio em reformas estruturais (privatizações e abertura comercial).
        \item Envio de reformas ao Congresso (como o programa de desregulamentação e liberalização financeira), ainda que com baixa efetividade imediata.
    \end{itemize}

    \item \textbf{ Avaliação crítica:}
    \begin{itemize}
        \item \textit{Efeito imediato:} Houve forte queda da inflação nos primeiros meses, mas o efeito foi temporário — a inflação voltou a acelerar ainda em 1990.
        \item \textit{Problemas:}
        \begin{itemize}
            \item O congelamento foi rompido rapidamente.
            \item O choque de liquidez gerou colapso da atividade econômica e quebra de empresas.
            \item A confiança da população e dos agentes econômicos foi abalada pelo bloqueio de depósitos.
        \end{itemize}
        \item \textit{Persistência da indexação informal:} Muitos contratos e preços foram reajustados informalmente, minando o objetivo de desindexação plena.
    \end{itemize}
\end{itemize}

\textbf{Síntese:} O Plano Collor foi um plano de estabilização com forte conteúdo heterodoxo, centrado no congelamento, na desindexação abrupta e no bloqueio de liquidez. Apesar do impacto inicial sobre a inflação, fracassou por falta de consistência institucional, ausência de credibilidade e pela quebra de confiança provocada pelo confisco de ativos financeiros. Sua experiência evidenciou os limites das soluções heterodoxas não acompanhadas de ancoragem fiscal, monetária e institucional duradoura.



\end{multicols}

\end{document}
