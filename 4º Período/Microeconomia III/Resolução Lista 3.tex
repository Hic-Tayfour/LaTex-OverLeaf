\documentclass[12pt]{article}
\usepackage{amsmath}
\usepackage{pgfplots} % Permite criar gráficos
\pgfplotsset{compat=1.17} % Versão do pgfplots
\usepackage{graphicx} % Required for inserting images
\usepackage{newtxtext} % This package sets Times New Roman as the main font
\usepackage{ragged2e} % This package is used to justify the text
\usepackage{setspace} % This package is used to set the line spacing
\usepackage[a4paper, left=3.0cm, top=3.0cm, bottom=2.0cm, right=2.0cm]{geometry} % This package is used to set the margins
\usepackage{lipsum} % This package is used to generate filler text
\usepackage{fancyhdr} % This package is used to customize the headers and footers
\usepackage{titlesec} % This package is used to customize titles
\usepackage[portuguese]{babel} % This package is used to translate the names of the document elements
\usepackage{hyperref} % This package is used to create hyperlinks in the document
\usepackage{indentfirst}
\setlength{\parindent}{1.25cm}  % This command sets the size of the indent
\usepackage{siunitx}
\sisetup{
  output-decimal-marker = {,},
  inter-unit-product = \ensuremath{{}\cdot{}},
  per-mode = symbol
}
\DeclareSIUnit{\real}{R\$}
\newcommand{\real}[1]{R\$#1}
\usepackage{float}
\usepackage{footnotehyper}
\usepackage{hyperref}
\hypersetup{
    colorlinks=true,
    linkcolor=black,
    filecolor=magenta,      
    urlcolor=cyan,
    pdfborder={0 0 0},
}
\usepackage[normalem]{ulem} % [normalem] prevents the package from changing \emph{} command
\makeatletter
\def\@pdfborder{0 0 0} % this line is needed to change the border color to white
\def\@pdfborderstyle{/S/U/W 1} % this line is to underline the links
\makeatother
\onehalfspacing

\begin{document}

\begin{titlepage}
    \centering
    \vspace*{1cm}
    \Large\textbf{INSPER – INSTITUTO DE ENSINO E PESQUISA}\\
    \Large ECONOMIA\\
    \vspace{1.5cm}
    \Large\textbf{Lista 3}\\
    \textbf{Microeconomia}\\
    \vspace{1.5cm}
    Prof(a). Laura Karpuska \\
    Prof(a). Auxiliar  Pamela E. C. Borges
    \vfill
    \normalsize
    Hicham Munir Tayfour, \href{mailto:hichamt@al.insper.edu.br}{hichamt@al.insper.edu.br}\\
    4º Período - Economia B\\
    \vfill
    São Paulo\\
    Março/2024
\end{titlepage}

\newpage

\newpage
\tableofcontents
\thispagestyle{empty} % This command removes the page number from the table of contents page
\newpage
\setcounter{page}{1} % This command sets the page number to start from this page
\justify
\onehalfspacing

\pagestyle{fancy}
\fancyhf{}
\rhead{\thepage}



\section{\textbf{Questão 1}}

\subsection{\textbf{Letra A}}
Para encontrar o preço de monopólio \( p^m \) ótimo, usaremos a função de demanda do mercado,  \( D(p) = a - p \) e o lucro da empresa monopolista, que é a diferença entre a receita total e o custo total.
A função de lucro \(\pi(p)\) é :

\[
\pi(p) = p \cdot D(p) - c \cdot D(p)
\]
Substituindo a função  demanda na de lucro,:
\[
\pi(p) = p \cdot (a - p) - c \cdot (a - p)
\]
\[
\pi(p) = pa - p^2 - ac + pc
\]
Para maximizar o lucro, derivamos a função de lucro e igualamos a zero :
\[
\frac{d\pi(p)}{dp} = a - 2p + c = 0
\]

Isolando o  \( p^m \)obtemos o preço ótimo de monopólio:

\[
p^m = \frac{a}{2} + \frac{c}{2}
\]
\subsection{\textbf{Letra B}}
Não sei com fazer , mas sei que se trata do gráfico descontínuo dos slides de equilíbrio de Nash Parte 2

\subsection{\textbf{Letra C}}
A melhor resposta da firma \( i \), indicada por \( p_i(p_j) \), será o valor de \( p_i \) que maximiza \( u_i(p_i, p_j) \) para um dado \( p_j \). 

Para \( p_j \geq a \), a firma \( i \) não tem incentivo para usar um preço maior ou igual a \( a \), já que o lucro será zero. Então, a melhor resposta será sempre definir um preço menor que \( a \). 

Para \( p_j < a \), a firma \( i \) quer escolher um preço que seja igual ou minimamente menor que \( p_j \), para garantir que roube parte do mercado. No entanto, a firma não deve definir um preço abaixo do custo marginal \( c \).

A melhor resposta da firma \( i \) pode ser representada da seguinte forma:

\[
p_i(p_j) = 
\begin{cases}
p_j, & \text{se } p_j < a \text{ e } p_j > c, \\
a - \varepsilon, & \text{se } p_j \geq a, \\
c, & \text{se } p_j \leq c,
\end{cases}
\]

onde \( \varepsilon \) é uma quantidade bem pequena, garantindo que \( p_i < a \).

\subsection{\textbf{Letra D}}
O equilíbrio de Nash ocorre quando, dados os preços praticados pelos concorrentes, nenhuma empresa tem incentivo  para alterar sua estratégia. Se ambas as firmas estão cobrando o preço \( p^m \), e uma delas decide aumentar seu preço, ela perderá seu mercado, pois todos os consumidores comprarão da outra firma. Se  baixar o preço, a mínima vantagem obtida pela conquista da demanda não compensa o lucro que poderia ser obtido mantendo o preço \( p^m \).

Portanto, dado que o desvio de \( p^m \) não oferece vantagem suficiente para nenhuma das firmas, elas não têm incentivo para alterar suas estratégias, o que caracteriza um equilíbrio de Nash.

\subsection{\textbf{Letra E}}
Discutimos se a política de garantia do menor preço beneficia mais os consumidores ou as empresas em uma interação estratégica entre concorrentes.

Essa política pode inicialmente parecer benéfica para os consumidores, pois promete o menor preço disponível. No entanto, as empresas podem reconhecer que uma guerra de preços não é vantajosa e, consequentemente, manter os preços estáveis, evitando competição efetiva em preço. Isso pode resultar em preços mais altos e menos benefícios para os consumidores.

Portanto, embora a política de garantia do menor preço possa oferecer vantagens aos consumidores em termos de custos, ela pode levar a um equilíbrio de mercado que favoreça mais as empresas no longo prazo.


\section{\textbf{Questão 2}}
\subsection{\textbf{Letra A}}
\[
\begin{array}{cc}
\textbf{M} & \\
\begin{array}{c|cc}
 & \textbf{E} & \textbf{D} \\
\hline
\textbf{A} & \textbf{1,2,1} & 0,1,0 \\
\textbf{B} & 0,1,0 & \textbf{2,2,1} \\
\end{array}
\end{array}
\]


\[
\begin{array}{cc}
\textbf{N} & \\
\begin{array}{c|cc}
 & \textbf{E} & \textbf{D} \\
\hline
\textbf{A} & 0,1,0 & \textbf{1,3,1} \\
\textbf{B} & \textbf{1,2,2} & 0,1,0 \\
\end{array}
\end{array}
\]

Os equilíbrios de Nash para cada são : (A,E,M), (B,D,M), (A,D,N) e (B,E,N) 

\subsection{\textbf{Letra B}}
Consideramos estratégias mistas para os jogadores. As probabilidades são definidas como \( p \) para o jogador 1 escolher A, \( q \) para o jogador 2 escolher E, e \( r \) para o jogador 3 escolher M. Então, os payoffs esperados para o jogador 1 jogando A e B, o jogador 2 jogando E e D, e o jogador 3 jogando M e N.

\subsection{\textbf{Letra C}}
Usamos o princípio da indiferença para encontrar um equilíbrio de Nash em estratégias mistas onde um jogador escolhe uma estratégia pura e os outros dois escolhem suas estratégias puras com probabilidades positivas. No equilíbrio, os jogadores devem ser indiferentes entre suas estratégias puras.

\subsection{\textbf{Letra D}}
Os payoffs esperados para as estratégias mistas no jogo são calculados com base nas probabilidades de escolha de cada jogador. Para a matriz M:

\[
\textbf{Matriz M:} \quad
\begin{array}{c|cc}
 & \textbf{E} & \textbf{D} \\
\hline
\textbf{A} & \textbf{1,2,1} & 0,1,0 \\
\textbf{B} & 0,1,0 & \textbf{2,2,1} \\
\end{array}
\]

Para a matriz N:

\[
\textbf{Matriz N:} \quad
\begin{array}{c|cc}
 & \textbf{E} & \textbf{D} \\
\hline
\textbf{A} & 0,1,0 & \textbf{1,3,1} \\
\textbf{B} & \textbf{1,2,2} & 0,1,0 \\
\end{array}
\]

Seja \( p \) a probabilidade de o jogador 1 escolher A, \( q \) a probabilidade de o jogador 2 escolher E e \( r \) a probabilidade de o jogador 3 escolher M. 

Em um equilíbrio de Nash misto, os jogadores são indiferentes entre suas estratégias puras, portanto os payoffs esperados são iguais para todas as estratégias de um jogador.

Os equilíbrios de Nash destacados em negrito são aqueles nos quais cada jogador escolhe a estratégia que maximiza seu payoff esperado, dado as estratégias dos outros jogadores.

\section{\textbf{Questão 3}}
As matrizes de payoff para os ciclistas e o comitê são as seguintes:

\[
\begin{array}{cc}
\textbf{Matriz para Comitê testa Jan:} & \textbf{Matriz para Comitê testa Lance:} \\
\begin{array}{c|cc}
 & \textbf{(D)} & \textbf{(L)} \\
\hline
\textbf{(D)} & -(1+c), 1, 1 & -(1+c), 1, 1 \\
\textbf{(L)} & -1, 1, 0 & 0, 0, 0 \\
\end{array}
&
\begin{array}{c|cc}
 & \textbf{(D)} & \textbf{(L)} \\
\hline
\textbf{(D)} & 1, -(1+c), 1 & 1, -1, 0 \\
\textbf{(L)} & 1, 1, -(1+c) & 0, 0, 0 \\
\end{array}
\end{array}
\]

Vemos que o perfil de estratégias onde o comitê testa ambos os ciclistas com igual probabilidade e ambos os ciclistas competem limpos é um equilíbrio de Nash:

Se os ciclistas competem limpos, o comitê é indiferente quanto a testar, pois não encontrará doping.
Se o comitê testa os ciclistas, ambos preferem competir limpos para evitar a desqualificação.

Sendo assim, não há incentivo para nenhuma das partes desviar de suas estratégias, estabelecendo assim um equilíbrio de Nash.


\section{\textbf{Questão 4}}
\subsection{\textbf{Letra A}}
Consideramos um problema social onde uma pessoa idosa precisa de ajuda para cruzar a rua e há \( n \) pedestres que podem escolher ajudar ou não.

Os elementos do jogo são:

Jogadores: Os \( n \) pedestres.Conjunto de estratégias: Cada pedestre pode escolher entre Ajudar 
 (A) ou Não Ajudar (N). Funções de payoff: O payoff de cada pedestre é dado por:
    Se um pedestre decide ajudar, seu payoff é \( v - c \).
    Se um pedestre não ajuda mas outro pedestre ajuda, seu payoff é \( v \).
    Se nenhum pedestre ajudar, todos recebem um payoff de 0.


Onde:

\( v \) é o benefício recebido por ajudar.
 \( c \) é o custo de ajudar, com \( c < v \).

A representação desse jogo na forma estratégica pode ser usado para encontrar os equilíbrios de estratégias puras e mistas


\subsection{\textbf{Letra B}}
Analisamos a existência de equilíbrios de Nash em estratégias puras no contexto do problema social dado. O jogo é estruturado da seguinte forma:

Se um pedestre decide ajudar e os outros não, o payoff é \( v - c \).
Se um pedestre decide não ajudar mas alguém ajuda, o payoff é \( v \).
Se ninguém ajuda, todos recebem um payoff de 0.

Para um equilíbrio de Nash em estratégias puras, não deve haver incentivo para que nenhum jogador desvie unilateralmente da sua estratégia, assumindo que as estratégias dos outros jogadores permaneçam inalteradas. No entanto, neste jogo:

Se qualquer pedestre escolher ajudar, os outros terão um incentivo para escolher não ajudar, pois isso aumenta seu payoff para \( v \) em vez de \( v - c \).
Se ninguém ajudar, qualquer pedestre pode aumentar seu payoff de 0 para \( v - c \) ao decidir ajudar.

Dessa forma, sempre há um incentivo para desviar da estratégia comum, indicando que não existe um equilíbrio de Nash em estratégias puras para este jogo.

\subsection{\textbf{Letra C}}
Para encontrar um equilíbrio simétrico em estratégias mistas no jogo social em questão, consideramos que cada pedestre adota uma estratégia mista onde ajudar ocorre com probabilidade \( p \) e não ajudar com probabilidade \( 1 - p \). No equilíbrio, cada pedestre deve ser indiferente entre as duas ações, o que implica que os payoffs esperados de ambas as ações devem ser iguais.

O payoff esperado para ajudar, que ocorre com probabilidade \( p \), é \( v - c \), uma vez que o pedestre obtém um benefício \( v \) e incorre em um custo \( c \). O payoff esperado para não ajudar é o benefício \( v \) multiplicado pela probabilidade de pelo menos um dos \( n - 1 \) outros pedestres decidir ajudar, que é \( 1 - (1 - p)^{n-1} \).

Igualamos os payoffs esperados para obter a seguinte equação:

\[
v - c = v \cdot \left(1 - (1 - p)^{n-1}\right)
\]

Esta é a condição para o equilíbrio simétrico em estratégias mistas. Podemos resolver para \( p \) numericamente ou analiticamente, dependendo dos valores de \( n \), \( v \), e \( c \).

\subsection{\textbf{Letra D}}
Calculamos a probabilidade de que pelo menos um pedestre decida ajudar no equilíbrio de estratégias mistas. Seja \( p \) a probabilidade de um pedestre escolher ajudar. A probabilidade de que todos os \( n \) pedestres decidam não ajudar é \( (1 - p)^n \). Logo, a probabilidade de que pelo menos um pedestre ajude é dada por:

\[ P(\text{pelo menos um ajuda}) = 1 - (1 - p)^n \]

Essa probabilidade é função do número de pedestres \( n \) e da probabilidade \( p \), que é determinada pelo equilíbrio em estratégias mistas. Quando \( n \) é grande, mesmo um \( p \) pequeno pode resultar em uma alta probabilidade de pelo menos um pedestre ajudar devido ao efeito cumulativo das decisões independentes de cada pedestre.



\end{document}