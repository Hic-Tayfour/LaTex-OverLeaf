\documentclass[a4paper,12pt]{article}[abntex2]
\bibliographystyle{abntex2-alf}
\usepackage{siunitx} % Fornece suporte para a tipografia de unidades do Sistema Internacional e formatação de números
\usepackage{booktabs} % Melhora a qualidade das tabelas
\usepackage{tabularx} % Permite tabelas com larguras de colunas ajustáveis
\usepackage{graphicx} % Suporte para inclusão de imagens
\usepackage{newtxtext} % Substitui a fonte padrão pela Times Roman
\usepackage{ragged2e} % Justificação de texto melhorada
\usepackage{setspace} % Controle do espaçamento entre linhas
\usepackage[a4paper, left=3.0cm, top=3.0cm, bottom=2.0cm, right=2.0cm]{geometry} % Personalização das margens do documento
\usepackage{lipsum} % Geração de texto dummy 'Lorem Ipsum'
\usepackage{fancyhdr} % Customização de cabeçalhos e rodapés
\usepackage{titlesec} % Personalização dos títulos de seções
\usepackage[portuguese]{babel} % Adaptação para o português (nomes e hifenização
\usepackage{hyperref} % Suporte a hiperlinks
\usepackage{indentfirst} % Indentação do primeiro parágrafo das seções
\sisetup{
  output-decimal-marker = {,},
  inter-unit-product = \ensuremath{{}\cdot{}},
  per-mode = symbol
}
\DeclareSIUnit{\real}{R\$}
\newcommand{\real}[1]{R\$#1}
\setlength{\headheight}{14.49998pt}
\usepackage{float} % Melhor controle sobre o posicionamento de figuras e tabelas
\usepackage{footnotehyper} % Notas de rodapé clicáveis em combinação com hyperref
\hypersetup{
    colorlinks=true,
    linkcolor=black,
    filecolor=magenta,      
    urlcolor=cyan,
    citecolor=black,        
    pdfborder={0 0 0},
}
\usepackage[normalem]{ulem} % Permite o uso de diferentes tipos de sublinhados sem alterar o \emph{}
\makeatletter
\def\@pdfborder{0 0 0} % Remove a borda dos links
\def\@pdfborderstyle{/S/U/W 1} % Estilo da borda dos links
\makeatother
\onehalfspacing

\begin{document}

\begin{titlepage}
    \centering
    \vspace*{1cm}
    \Large\textbf{INSPER – INSTITUTO DE ENSINO E PESQUISA}\\
    \Large ECONOMIA\\
    \vspace{1.5cm}
    \Large\textbf{Tradução Tópico 8.2 - HPE}\\
    \vspace{1.5cm}
    Prof. Pedro Duarte\\
    Prof. Auxiliar Guilherme Mazer\\
    \vfill
    \normalsize
    Hicham Munir Tayfour, \href{mailto:hichamt@al.insper.edu.br}{hichamt@al.insper.edu.br}\\
    4º Período - Economia B\\
    \vfill
    São Paulo\\
    Abril/2024
\end{titlepage}

\newpage
\tableofcontents
\thispagestyle{empty} % This command removes the page number from the table of contents page
\newpage
\setcounter{page}{1} % This command sets the page number to start from this page
\justify
\onehalfspacing

\pagestyle{fancy}
\fancyhf{}
\rhead{\thepage}

\section{\textbf{Von Neumann e Morgenstern (1944)}}

\subsection{\textbf{Capítulo 1 : FORMULAÇÃO DO PROBLEMA ECONÔMICO}}

\subsubsection{\textbf{O Método Matemático na Economia}}

\textbf{Observações Iniciais}

O propósito deste livro é apresentar uma discussão sobre algumas questões fundamentais da teoria econômica que requerem um tratamento diferente daquele que encontraram até agora na literatura. A análise trata de alguns problemas básicos decorrentes do estudo do comportamento econômico, que têm sido o foco de atenção dos economistas por muito tempo. Eles têm sua origem nas tentativas de encontrar uma descrição exata do esforço do indivíduo para obter um máximo de utilidade ou, no caso do empreendedor, um máximo de lucro. É bem conhecido quais dificuldades consideráveis - e de fato insuperáveis - essa tarefa envolve, mesmo com um número limitado de situações típicas, como, por exemplo, no caso da troca de bens, direta ou indireta, entre duas ou mais pessoas, de monopólio bilateral, de duopólio, de oligopólio e de concorrência livre. Ficará claro que a estrutura desses problemas, familiar a todo estudante de economia, é em muitos aspectos bastante diferente da maneira como são concebidos atualmente. Além disso, ficará evidente que sua formulação exata e subsequente solução só pode ser alcançada com o auxílio de métodos matemáticos que divergem consideravelmente das técnicas aplicadas por economistas matemáticos mais antigos ou contemporâneos

Nossas considerações nos levarão à aplicação da teoria matemática dos “jogos de estratégia”, desenvolvida por um de nós em várias etapas sucessivas em 1928 e 1940-1941. Após a apresentação dessa teoria, sua aplicação a problemas econômicos no sentido indicado acima será empreendida. Ficará evidente que ela oferece uma nova abordagem para várias questões econômicas ainda não resolvidas.

Primeiro, precisamos entender de que maneira essa teoria dos jogos pode ser relacionada à teoria econômica e quais são seus elementos comuns. Isso pode ser feito melhor ao declarar brevemente a natureza de alguns problemas econômicos fundamentais, para que os elementos comuns fiquem claros. Tornará evidente que não há nada de artificial em estabelecer essa relação, mas, pelo contrário, essa teoria dos jogos de estratégia é o instrumento adequado para desenvolver uma teoria do comportamento econômico.

Seria um equívoco interpretar nossas discussões como meramente apontando uma analogia entre essas duas esferas. Esperamos estabelecer satisfatoriamente, após desenvolver algumas esquematizações plausíveis, que os problemas típicos de comportamento econômico se tornam estritamente idênticos às noções matemáticas de jogos de estratégia adequados.

\textbf{Dificuldades da Aplicação do Método Matemático}


Pode ser oportuno começar com algumas observações sobre a natureza da teoria econômica e discutir brevemente o papel que a matemática pode desempenhar em seu desenvolvimento.

Primeiro, estejamos cientes de que não existe atualmente um sistema universal de teoria econômica e que, se um dia for desenvolvido, provavelmente não será durante nossa vida. A razão para isso é simplesmente que a economia é uma ciência muito difícil para permitir sua construção rapidamente, especialmente diante do conhecimento muito limitado e da descrição imperfeita dos fatos com os quais os economistas lidam. Apenas aqueles que não apreciam essa condição provavelmente tentarão construir sistemas universais. Mesmo em ciências muito mais avançadas do que a economia, como a física, não há sistema universal disponível no momento.

Para continuar a analogia com a física: ocasionalmente, uma teoria física específica parece fornecer a base para um sistema universal, mas em todos os casos até o momento, essa aparência não durou mais do que uma década no máximo. O trabalho cotidiano do físico de pesquisa certamente não está envolvido com tais objetivos elevados, mas sim com problemas específicos que estão “maduros”. Provavelmente não haveria progresso algum na física se uma tentativa séria fosse feita para impor esse superpadrão. O físico trabalha em problemas individuais, alguns de grande significado prático, outros de menor importância. Unificações de campos que antes eram divididos e distantes podem alternar com esse tipo de trabalho. No entanto, tais ocorrências felizes são raras e acontecem apenas depois que cada campo foi minuciosamente explorado. Considerando o fato de que a economia é muito mais difícil, muito menos compreendida e, sem dúvida, em um estágio muito anterior de sua evolução como ciência do que a física, não se deve esperar mais do que um desenvolvimento do tipo acima na economia também.

Segundo, devemos observar que as diferenças nas questões científicas tornam necessário empregar métodos variados, que posteriormente podem ter que ser descartados se surgirem melhores. Isso tem uma dupla implicação: em alguns ramos da economia, o trabalho mais frutífero pode ser o da descrição cuidadosa e paciente; na verdade, este pode ser de longe o maior domínio no momento e por algum tempo. Em outros casos, pode ser possível desenvolver uma teoria de maneira estrita, e para isso, o uso da matemática pode ser necessário.

A matemática realmente foi usada na teoria econômica, talvez até de maneira exagerada. Em qualquer caso, seu uso não foi muito bem-sucedido. Isso é contrário ao que se observa em outras ciências: na maioria das ciências, a matemática foi aplicada com grande sucesso e dificilmente poderiam prescindir dela. No entanto, a explicação desse fenômeno é bastante simples.

Não é que exista alguma razão fundamental pela qual a matemática não deva ser usada na economia. Os argumentos frequentemente ouvidos de que, devido ao elemento humano, aos fatores psicológicos etc., ou porque supostamente não há medição de fatores importantes, a matemática não encontrará aplicação, podem ser todos considerados completamente equivocados. Quase todas essas objeções foram feitas, ou poderiam ter sido feitas, muitos séculos atrás em campos onde a matemática é agora o principal instrumento de análise. Esse “poderia ter sido” é usado no seguinte sentido: vamos tentar nos imaginar no período que precedeu a fase matemática ou quase matemática do desenvolvimento da física, ou seja, no século XVI, ou na química e biologia, ou seja, no século XVIII. Considerando a atitude cética daqueles que se opõem à economia matemática em princípio, a perspectiva nas ciências físicas e biológicas nesses períodos iniciais dificilmente poderia ter sido melhor do que a da economia - mutatis mutandis - no presente.

Quanto à falta de medição dos fatores mais importantes, o exemplo da teoria do calor é bastante instrutivo; antes do desenvolvimento da teoria matemática, as possibilidades de medições quantitativas eram menos favoráveis nesse campo do que estão agora na economia. As medições precisas da quantidade e qualidade do calor (energia e temperatura) foram o resultado e não os antecedentes da teoria matemática. Isso deve ser contrastado com o fato de que as noções quantitativas e exatas de preços, dinheiro e taxa de juros já foram desenvolvidas há séculos.

Outro grupo de objeções contra medições quantitativas na economia gira em torno da falta de divisibilidade indefinida das quantidades econômicas. Isso supostamente é incompatível com o uso do cálculo infinitesimal e, portanto, (!) da matemática. É difícil entender como tais objeções podem ser mantidas à luz das teorias atômicas na física e na química, a teoria dos quanta na eletrodinâmica, etc., e o notório e contínuo sucesso da análise matemática dentro dessas disciplinas.

Neste ponto, é apropriado mencionar outro argumento familiar da literatura econômica que pode ser revivido como uma objeção ao procedimento matemático.

Para elucidar as concepções que estamos aplicando à economia, demos e podemos dar novamente algumas ilustrações da física. Existem muitos cientistas sociais que se opõem a traçar tais paralelos por vários motivos, entre os quais geralmente se encontra a afirmação de que a teoria econômica não pode ser modelada após a física, uma vez que é uma ciência dos fenômenos sociais, humanos, que deve levar em conta a psicologia, etc. Tais declarações são, no mínimo, prematuras. É sem dúvida razoável descobrir o que levou ao progresso em outras ciências e investigar se a aplicação dos mesmos princípios não pode levar ao progresso também na economia. Caso surja a necessidade da aplicação de princípios diferentes, isso só poderia ser revelado no curso do desenvolvimento real da teoria econômica. Isso por si só constituiria uma grande revolução. Mas, com toda a certeza, ainda não chegamos a tal estado - e não é de forma alguma certo que haverá necessidade de princípios científicos totalmente diferentes - seria muito imprudente considerar qualquer coisa além da busca de nossos problemas da maneira que resultou no estabelecimento da ciência física.

A razão pela qual a matemática não teve mais sucesso na economia deve ser encontrada em outro lugar. A falta de sucesso real se deve em grande parte a uma combinação de circunstâncias desfavoráveis, algumas das quais podem ser removidas gradualmente. Para começar, os problemas econômicos não foram formulados claramente e muitas vezes são expressos de forma tão vaga que o tratamento matemático a priori parece impossível porque é bastante incerto quais são os problemas reais. Não faz sentido usar métodos exatos onde não há clareza nos conceitos e questões aos quais eles devem ser aplicados. Consequentemente, a tarefa inicial é esclarecer o conhecimento da matéria por meio de um trabalho descritivo cuidadoso adicional. Mas mesmo nas partes da economia em que o problema descritivo foi tratado de maneira mais satisfatória, as ferramentas matemáticas raramente foram usadas adequadamente. Elas foram tratadas de forma inadequada, como nas tentativas de determinar um equilíbrio econômico geral apenas contando o número de equações e incógnitas, ou levaram a meras traduções de uma forma literária de expressão em símbolos, sem qualquer análise matemática subsequente.

Em seguida, o embasamento empírico da ciência econômica é definitivamente inadequado. Nosso conhecimento dos fatos relevantes da economia é incomparavelmente menor do que o conhecimento na física na época em que a matematização desse assunto foi alcançada. Na verdade, a ruptura decisiva que ocorreu na física no século XVII, especificamente no campo da mecânica, só foi possível devido aos desenvolvimentos anteriores na astronomia. Foi respaldada por vários milênios de observação sistemática, científica e astronômica, culminando em um observador de calibre sem igual, Tycho de Brahe. Nada disso ocorreu na ciência econômica. Seria absurdo esperar na física por Kepler e Newton sem Tycho - e não há motivo para esperar um desenvolvimento mais fácil na economia.

Esses comentários óbvios não devem ser interpretados, é claro, como uma crítica à pesquisa estatístico-econômica, que oferece a verdadeira promessa de progresso na direção adequada.

A falta de sucesso real da economia matemática se deve, consequentemente, a uma combinação de circunstâncias desfavoráveis, algumas das quais podem ser gradualmente superadas. Para começar, os problemas econômicos não foram formulados de forma clara e muitas vezes são expressos de maneira tão vaga que o tratamento matemático a priori parece impossível, pois é bastante incerto quais são os problemas reais. Não faz sentido usar métodos exatos onde não há clareza nos conceitos e questões aos quais eles devem ser aplicados. Consequentemente, a tarefa inicial é esclarecer o conhecimento da matéria por meio de um trabalho descritivo cuidadoso adicional. Mas mesmo nas partes da economia em que o problema descritivo foi tratado de maneira mais satisfatória, as ferramentas matemáticas raramente foram usadas adequadamente. Elas foram tratadas de forma inadequada, como nas tentativas de determinar um equilíbrio econômico geral apenas contando o número de equações e incógnitas, ou levaram a meras traduções de uma forma literária de expressão em símbolos, sem qualquer análise matemática subsequente.

À luz dessas observações, podemos descrever nossa própria posição da seguinte forma: O objetivo deste livro não está na direção da pesquisa empírica. O avanço desse lado da ciência econômica, em qualquer escala reconhecida como necessária, é claramente uma tarefa de proporções vastas. Pode-se esperar que, como resultado das melhorias na técnica científica e da experiência adquirida em outros campos, o desenvolvimento da economia descritiva não leve tanto tempo quanto a comparação com a astronomia sugeriria. Mas, em qualquer caso, a tarefa parece transcender os limites de qualquer programa planejado individualmente.

Vamos tentar utilizar apenas algumas experiências comuns relacionadas ao comportamento humano que se prestam a tratamento matemático e que têm importância econômica.

Acreditamos que a possibilidade de um tratamento matemático desses fenômenos refuta as objeções “fundamentais” mencionadas.

No entanto, será visto que esse processo de matematização não é nada óbvio. De fato, as objeções mencionadas acima podem ter suas raízes em parte nas dificuldades bastante óbvias de qualquer abordagem matemática direta. Acharemos necessário recorrer a técnicas de matemática que ainda não foram usadas na economia matemática e é bem possível que estudos futuros resultem na criação de novas disciplinas matemáticas.

Para concluir, também podemos observar que parte do sentimento de insatisfação com o tratamento matemático da teoria econômica decorre em grande parte do fato de que frequentemente são oferecidas não provas, mas meras afirmações que realmente não são melhores do que as mesmas afirmações dadas em forma literária. Com muita frequência, as provas estão ausentes porque um tratamento matemático foi tentado em campos que são tão vastos e tão complicados que, por muito tempo, até que se adquira muito mais conhecimento empírico, dificilmente há qualquer motivo para esperar progresso mais mathematico. O fato de que esses campos foram atacados dessa maneira - como, por exemplo, a teoria das flutuações econômicas, a estrutura temporal da produção, etc. - indica o quanto as dificuldades associadas estão sendo subestimadas. Elas são enormes e não estamos de forma alguma preparados para elas.

Referimos-nos à natureza e às possibilidades dessas mudanças na técnica matemática - na verdade, na própria matemática - que uma aplicação bem-sucedida da matemática a um novo assunto pode produzir. É importante visualizá-las em sua perspectiva adequada.

Não se deve esquecer que essas mudanças podem ser muito consideráveis. A fase decisiva da aplicação da matemática à física - a criação por Newton de uma disciplina racional da mecânica - foi possível apenas por causa dos desenvolvimentos anteriores na astronomia. Foi respaldada por vários milênios de observação sistemática, científica e astronômica, culminando em um observador de calibre sem igual, Tycho de Brahe. Nada disso ocorreu na ciência econômica. Seria absurdo esperar na física por Kepler e Newton sem Tycho - e não há motivo para esperar um desenvolvimento mais fácil na economia.

Essas observações devem ser lembradas em relação à ênfase excessiva atual no uso de cálculo, equações diferenciais, etc., como as principais ferramentas da economia matemática.

\textbf{Limitações Necessárias dos Objetivos}

Precisamos retornar, portanto, à posição indicada anteriormente: É necessário começar com os problemas que são descritos claramente, mesmo que não sejam tão importantes de qualquer outro ponto de vista. Além disso, deve ser acrescentado que o tratamento desses problemas gerenciáveis pode levar a resultados que já são bastante conhecidos, mas as provas exatas podem ainda estar faltando. Antes de serem fornecidas, a respectiva teoria simplesmente não existe como uma teoria científica. Os movimentos dos planetas eram conhecidos muito antes de seus cursos terem sido calculados e explicados pela teoria de Newton, e o mesmo se aplica a muitas instâncias menores e menos dramáticas. E de forma semelhante na teoria econômica, certos resultados - digamos, a indeterminação do monopólio bilateral - podem já ser conhecidos. No entanto, é interessante derivá-los novamente de uma teoria exata. O mesmo poderia e deveria ser dito sobre praticamente todos os teoremas econômicos estabelecidos.

Por fim, pode ser acrescentado que não propomos levantar a questão da significância prática dos problemas tratados. Isso está alinhado com o que foi dito acima sobre a seleção de campos para a teoria. A situação não é diferente aqui daquela em outras ciências. Lá também as questões mais importantes do ponto de vista prático podem ter estado completamente fora de alcance durante longos e frutíferos períodos de seu desenvolvimento. Isso certamente ainda é verdade na economia, onde é de extrema importância saber como estabilizar o emprego, como aumentar a renda nacional ou como distribuí-la adequadamente. Ninguém pode realmente responder a essas perguntas, e não precisamos nos preocupar com a pretensão de que pode haver respostas científicas no momento.

O grande progresso em todas as ciências ocorreu quando, no estudo de problemas que eram modestos em comparação com os objetivos finais, foram desenvolvidos métodos que poderiam ser estendidos ainda mais. A queda livre é um fenômeno físico muito trivial, mas foi o estudo desse fato extremamente simples e sua comparação com o material astronômico que trouxe à tona a mecânica.

Acreditamos que o mesmo padrão de modéstia deve ser aplicado à economia. É inútil tentar explicar — e "sistematicamente", a propósito — tudo o que é econômico. O procedimento correto é obter primeiro a máxima precisão e domínio em um campo limitado e, em seguida, avançar para outro, um pouco mais amplo, e assim por diante. Isso também eliminaria a prática prejudicial de aplicar teorias chamadas à reforma econômica ou social, onde elas não são de forma alguma úteis.

Acreditamos que é necessário conhecer o máximo possível sobre o comportamento do indivíduo e sobre as formas mais simples de troca. Esse ponto de vista foi adotado com sucesso notável pelos fundadores da escola de utilidade marginal, mas ainda assim não é geralmente aceito. Os economistas frequentemente apontam para questões muito maiores, mais "urgentes", e ignoram tudo o que impede que façam afirmações sobre essas questões. A experiência em ciências mais avançadas, como a física, indica que essa impaciência apenas atrasa o progresso, incluindo o tratamento das questões "urgentes". Não há motivo para assumir a existência de atalhos.

\textbf{Conclusão}

É essencial perceber que os economistas não podem esperar um destino mais fácil do que o que aconteceu com cientistas em outras disciplinas. Parece razoável esperar que eles tenham que abordar primeiro problemas contidos nos fatos mais simples da vida econômica e tentar estabelecer teorias que os expliquem e que realmente estejam em conformidade com padrões científicos rigorosos. Podemos ter confiança suficiente de que, a partir desse ponto, a ciência da economia continuará a crescer, abrangendo gradualmente questões de maior importância do que aquelas com as quais se deve começar.

O campo abordado neste livro é muito limitado, e o abordamos com esse sentido de modéstia. Não nos preocupamos se os resultados de nosso estudo estão de acordo com visões recentemente adquiridas ou mantidas por muito tempo, pois o importante é o desenvolvimento gradual de uma teoria, baseada em uma análise cuidadosa da interpretação cotidiana comum dos fatos econômicos. Essa fase preliminar é necessariamente heurística, ou seja, a fase de transição de considerações de plausibilidade não matemáticas para o procedimento formal da matemática. A teoria finalmente obtida deve ser matematicamente rigorosa e conceitualmente geral. Suas primeiras aplicações são necessariamente para problemas elementares em que o resultado nunca esteve em dúvida e nenhuma teoria é realmente necessária. Nesta fase inicial, a aplicação serve para corroborar a teoria. A próxima etapa se desenvolve quando a teoria é aplicada a situações um pouco mais complicadas, nas quais ela pode ir além do óbvio e do familiar. Aqui, teoria e aplicação se corroboram mutuamente. Além disso, está o campo do sucesso real: previsão genuína pela teoria. É bem conhecido que todas as ciências matematizadas passaram por essas fases sucessivas de evolução

\subsubsection{\textbf{Discussão Qualitativa do Problema do Comportamento Racional}}
\textbf{O Problema do Comportamento Racional}

O objeto da teoria econômica é o mecanismo muito complicado de preços e produção, bem como dos ganhos e gastos de renda. No decorrer do desenvolvimento da economia, foi constatado, e agora é quase universalmente aceito, que uma abordagem para esse vasto problema é obtida pela análise do comportamento dos indivíduos que constituem a comunidade econômica. Essa análise foi levada bastante longe em muitos aspectos, e embora ainda exista muita discordância, a importância dessa abordagem não pode ser questionada, independentemente das dificuldades que possam surgir. Os obstáculos são realmente consideráveis, mesmo que a investigação deva, a princípio, estar limitada às condições da estática econômica, como deve ser. Uma das principais dificuldades reside em descrever adequadamente as suposições que devem ser feitas sobre os motivos do indivíduo. * Esse problema foi tradicionalmente formulado assumindo que o consumidor deseja obter um máximo de utilidade ou satisfação, e o empreendedor um máximo de lucros.

As dificuldades conceituais e práticas da noção de utilidade, e especialmente das tentativas de descrevê-la como um número, são bem conhecidas, e seu tratamento não está entre os objetivos principais deste trabalho. No entanto, seremos forçados a discuti-las em algumas instâncias, especialmente na seção 3.3. e 3.5. Deixe-me dizer desde já que o ponto de vista do presente livro sobre essa questão muito importante e interessante será principalmente oportunista. Desejamos concentrar-nos em um problema - que não é o da medição de utilidades e preferências - e, portanto, tentaremos simplificar todas as outras características na medida do razoavelmente possível. Portanto, assumiremos que o objetivo de todos os participantes do sistema econômico, consumidores e empreendedores, é o dinheiro, ou equivalente a uma única mercadoria monetária. Isso deve ser livremente divisível e substituível, livremente transferível e idêntico, mesmo no sentido quantitativo, a qualquer “satisfação” ou “utilidade” desejada por cada participante. (Para o caráter quantitativo da utilidade, cf. 3.3. citado acima.)

Às vezes, alega-se na literatura econômica que discussões sobre as noções de utilidade e preferência são completamente desnecessárias, uma vez que são definições puramente verbais sem consequências empiricamente observáveis, ou seja, inteiramente tautológicas. Não nos parece que essas noções sejam qualitativamente inferiores a certas noções bem estabelecidas e indispensáveis na física, como força, massa, carga, etc. Ou seja, enquanto estão em sua forma imediata meras definições, elas se tornam sujeitas a controle empírico por meio das teorias que são construídas sobre elas - e de nenhuma outra maneira. Assim, a noção de utilidade é elevada acima do status de tautologia por teorias econômicas que a utilizam e cujos resultados podem ser comparados com a experiência ou, pelo menos, com o senso comum.

O indivíduo que tenta obter esses máximos respectivos também é considerado agir “racionalmente”. No entanto, pode-se afirmar com segurança que, atualmente, não existe um tratamento satisfatório da questão do comportamento racional. Pode haver, por exemplo, várias maneiras de atingir a posição ótima; elas podem depender do conhecimento e compreensão que o indivíduo possui e das opções de ação disponíveis para ele. Um estudo de todas essas questões em termos qualitativos não as esgotará, porque elas implicam, como deve ser evidente, relações quantitativas. Portanto, seria necessário formulá-las em termos quantitativos para que todos os elementos da descrição qualitativa sejam levados em consideração. Isso é uma tarefa extremamente difícil, e podemos dizer com segurança que ela não foi realizada na extensa literatura sobre o assunto. A principal razão para isso reside, sem dúvida, na falta de desenvolvimento e aplicação de métodos matemáticos adequados ao problema; isso teria revelado que o problema do máximo, que se supõe corresponder à noção de racionalidade, não está formulado de forma inequívoca. Na verdade, uma análise mais exaustiva (a ser apresentada em 4.3.-4.5.) revela que as relações significativas são muito mais complicadas do que o uso popular e “filosófico” da palavra “racional” indica.

Uma valiosa descrição preliminar qualitativa do comportamento do indivíduo é oferecida pela Escola Austríaca, especialmente na análise da economia do isolado “Robinson Crusoe”. Também podemos observar algumas considerações de Bohm-Bawerk sobre a troca entre duas ou mais pessoas. A exposição mais recente da teoria das escolhas individuais na forma de análise de curvas de indiferença baseia-se nos mesmos fatos ou supostos fatos, mas usa um método que muitas vezes é considerado superior em muitos aspectos. Sobre isso, consultamos as discussões em 2.1.1. e 3.

No entanto, esperamos obter uma compreensão real do problema da troca estudando-o de um ângulo completamente diferente; isto é, da perspectiva de um “jogo de estratégia”. Nossa abordagem ficará clara em breve, especialmente após algumas ideias que foram avançadas, digamos por Bohm-Bawerk - cujas opiniões podem ser consideradas apenas como um protótipo desta teoria - serem dadas uma formulação quantitativa correta.

\textbf{Economia de “Robinson Crusoe” e Economia de Troca Social}

Vamos examinar mais de perto o tipo de economia representado pelo modelo de “Robinson Crusoe”, ou seja, uma economia de uma única pessoa isolada ou organizada sob uma única vontade. Essa economia é confrontada com certas quantidades de mercadorias e uma série de necessidades que podem ser satisfeitas. O problema é obter uma satisfação máxima. Isso é, considerando em particular nossa suposição acima do caráter numérico da utilidade, de fato um problema máximo comum, cuja dificuldade aparentemente depende do número de variáveis e da natureza da função a ser maximizada; mas isso é mais uma dificuldade prática do que teórica. Se abstrairmos da produção contínua e do fato de que o consumo também se estende ao longo do tempo (e muitas vezes utiliza bens duráveis), obtemos o modelo mais simples possível. Pensou-se ser possível usá-lo como base para a teoria econômica, mas essa tentativa - notavelmente uma característica da versão austríaca - foi frequentemente contestada. A principal objeção contra o uso desse modelo muito simplificado de um indivíduo isolado para a teoria de uma economia de troca social é que ele não representa um indivíduo exposto às múltiplas influências sociais. Portanto, diz-se que analisa um indivíduo que pode se comportar de maneira bastante diferente se suas escolhas forem feitas em um mundo social onde ele estaria exposto a fatores de imitação, publicidade, costumes, etc. Esses fatores certamente fazem uma grande diferença, mas é questionável se eles alteram as propriedades formais do processo de maximização. De fato, este último nunca foi implicado, e como estamos preocupados apenas com esse problema, podemos deixar de fora as considerações sociais acima.

Outras diferenças entre “Crusoe” e um participante de uma economia de troca social também não nos preocuparão. Tal é a inexistência de dinheiro como meio de troca no primeiro caso, onde há apenas um padrão de cálculo, para o qual qualquer mercadoria pode servir. Essa dificuldade, de fato, foi superada por nossa suposição em 2.1.2. de uma noção quantitativa e até monetária de utilidade. Enfatizamos novamente: Nosso interesse reside no fato de que, mesmo após todas essas simplificações drásticas, Crusoe enfrenta um problema formal bastante diferente daquele enfrentado por um participante em uma economia social.

Esse tipo de problema não é tratado em lugar algum na matemática clássica. Ressaltamos, correndo o risco de sermos pedantes, que isso não é um problema máximo condicional, nem um problema de cálculo de variações, de análise funcional, etc. Ele surge com total clareza, mesmo nas situações mais “elementares”, por exemplo, quando todas as variáveis podem assumir apenas um número finito de valores.

Uma expressão particularmente marcante do equívoco popular sobre esse pseudo-problema máximo é a famosa afirmação segundo a qual o objetivo do esforço social é o “maior bem possível para o maior número possível”. Um princípio orientador não pode ser formulado pela exigência de maximizar duas (ou mais) funções ao mesmo tempo.

Esse princípio, tomado literalmente, é contraditório (em geral, uma função não terá um máximo onde a outra função tem um). É tão inadequado quanto dizer, por exemplo, que uma empresa deve obter preços máximos com volume de negócios máximo ou uma receita máxima com despesas mínimas. Se algum grau de importância desses princípios ou uma média ponderada for pretendido, isso deve ser declarado. No entanto, na situação dos participantes de uma economia social, nada disso é pretendido, mas todos os máximos são desejados ao mesmo tempo - por vários participantes.

Seria um erro acreditar que isso pode ser evitado, como a dificuldade no caso de Crusoe mencionada na nota de rodapé 2 na página 10, por um mero recurso aos dispositivos da teoria da probabilidade. Cada participante pode determinar as variáveis que descrevem suas próprias ações, mas não as dos outros. No entanto, essas variáveis “estranhas” não podem, do ponto de vista dele, ser descritas por suposições estatísticas. Isso ocorre porque os outros são orientados, assim como ele próprio, por princípios racionais - o que quer que isso signifique - e nenhum modus procedendi pode estar correto sem tentar entender esses princípios e as interações dos interesses conflitantes de todos os participantes.

Às vezes, alguns desses interesses correm mais ou menos paralelos - então estamos mais próximos de um problema máximo simples. Mas eles também podem ser opostos. A teoria geral deve abranger todas essas possibilidades, todos os estágios intermediários e todas as suas combinações.

A diferença entre a perspectiva de Crusoe e a de um participante em uma economia social também pode ser ilustrada da seguinte maneira: Além das variáveis que sua vontade controla, Crusoe recebe uma série de dados que são “mortos”; eles são o pano de fundo físico inalterável da situação. (Mesmo quando aparentemente variáveis, cf. nota de rodapé 2 na página 10, eles são realmente regidos por leis estatísticas fixas.) Nenhum dado com o qual ele precisa lidar reflete a vontade ou intenção de outra pessoa de natureza econômica - com base em motivos da mesma natureza que os seus próprios. Por outro lado, um participante em uma economia de troca social enfrenta dados desse último tipo também: eles são o produto das ações e volições de outros participantes (como preços). Suas ações serão influenciadas por sua expectativa em relação a esses dados, e eles, por sua vez, refletem a expectativa dos outros participantes em relação às ações dele.

Assim, o estudo da economia de Crusoe e o uso dos métodos aplicáveis a ela têm um valor muito mais limitado para a teoria econômica do que se supunha até agora, mesmo pelos críticos mais radicais. Os fundamentos dessa limitação não estão no campo das relações sociais que mencionamos antes - embora não questionemos sua importância - mas sim surgem das diferenças conceituais entre o problema máximo original (de Crusoe) e o problema mais complexo esboçado acima.

Esperamos que o leitor seja convencido pelo exposto de que enfrentamos aqui e agora uma dificuldade realmente conceitual - e não meramente técnica. E é esse problema que a teoria dos “jogos de estratégia” visa principalmente resolver.

\textbf{O Número de Variáveis e o Número de Participantes}

O arranjo formal que usamos nos parágrafos anteriores para indicar os eventos em uma economia de troca social fez uso de um número de "variáveis" que descreviam as ações dos participantes nessa economia. Assim, cada participante recebe um conjunto de variáveis, "suas" variáveis, que juntas descrevem completamente suas ações, ou seja, expressam precisamente as manifestações de sua vontade. Chamamos esses conjuntos de conjuntos parciais de variáveis. Os conjuntos parciais de todos os participantes constituem juntos o conjunto de todas as variáveis, a ser chamado de conjunto total. Portanto, o número total de variáveis é determinado primeiro pelo número de participantes, ou seja, de conjuntos parciais, e segundo pelo número de variáveis em cada conjunto parcial.

Do ponto de vista puramente matemático, não haveria nada de objetável em tratar todas as variáveis de qualquer conjunto parcial como uma única variável, "a" variável do participante correspondente a esse conjunto parcial. De fato, este é um procedimento que vamos usar com frequência em nossas discussões matemáticas; ele não faz absolutamente nenhuma diferença conceitual e simplifica bastante as notações.

Por enquanto, no entanto, propomos distinguir entre si as variáveis dentro de cada conjunto parcial. Os modelos econômicos aos quais naturalmente somos levados sugerem esse procedimento; assim, é desejável descrever para cada participante a quantidade de cada bem específico que ele deseja adquirir por uma variável separada, etc.

Agora devemos enfatizar que qualquer aumento no número de variáveis dentro do conjunto parcial de um participante pode complicar nosso problema tecnicamente, mas apenas tecnicamente. Assim, em uma economia de Crusoe - onde existe apenas um participante e apenas um conjunto parcial que então coincide com o conjunto total - isso pode tornar a determinação necessária de um máximo tecnicamente mais difícil, mas não alterará o caráter “máximo puro” do problema. No entanto, por outro lado, o aumento do número de participantes - ou seja, dos conjuntos parciais de variáveis - resulta em algo de natureza muito diferente. Para usar uma terminologia que se mostrará significativa, a dos jogos, isso equivale a um aumento no número de jogadores no jogo. No entanto, para os casos mais simples, um jogo de três pessoas é fundamentalmente diferente de um jogo de duas pessoas, um jogo de quatro pessoas de um jogo de três pessoas, etc. As complicações combinatoriais do problema - que, como vimos, não é um problema máximo de forma alguma - aumentam tremendamente com cada aumento no número de jogadores, como mostraremos posteriormente.

Nós exploramos esse assunto em detalhes, especialmente porque na maioria dos modelos econômicos ocorre uma mistura peculiar desses dois fenômenos. Sempre que o número de jogadores, ou seja, de participantes em uma economia social, aumenta, a complexidade do sistema econômico geralmente também aumenta; por exemplo, o número de mercadorias e serviços trocados, os processos de produção utilizados, etc. Assim, é provável que o número de variáveis em cada conjunto parcial de um participante aumente. No entanto, o número de participantes, ou seja, de conjuntos parciais, também aumentou. Portanto, ambas as fontes que discutimos contribuem \textit{pari passu} para o aumento total no número de variáveis. É essencial visualizar cada fonte em seu papel adequado.

\textbf{O Caso de Muitos Participantes: Concorrência Livre}

Ao elaborar o contraste entre uma economia de Crusoe e uma economia de troca social nas seções 2.2.2 a 2.2.4, enfatizamos os recursos desta última que se tornam mais proeminentes quando o número de participantes - sendo maior que 1 - é de tamanho moderado. O fato de que cada participante é influenciado pelas reações antecipadas dos outros às suas próprias ações, e que isso é verdade para cada um dos participantes, é o cerne da questão (no que diz respeito aos vendedores) nos problemas clássicos de duopólio, oligopólio, etc. Quando o número de participantes se torna realmente grande, surge alguma esperança de que a influência de cada participante específico se torne negligível, e que as dificuldades mencionadas possam diminuir e uma teoria mais convencional se torne possível. Essas são, naturalmente, as condições clássicas da “concorrência livre”. De fato, esse foi o ponto de partida de grande parte do que há de melhor na teoria econômica. Comparados com esse caso de grande número - a concorrência livre - os casos de pequeno número do lado dos vendedores - monopólio, duopólio, oligopólio - eram até considerados exceções e anomalias. (Mesmo nesses casos, o número de participantes ainda é muito grande em vista da concorrência entre os compradores. Os casos envolvendo números realmente pequenos são os de monopólio bilateral, de troca entre um monopólio e um oligopólio, ou dois oligopólios, etc.)

Em justiça ao ponto de vista tradicional, isso deve ser dito: é um fenômeno bem conhecido em muitos ramos das ciências exatas e físicas que números muito grandes muitas vezes são mais fáceis de lidar do que os de tamanho médio. Uma teoria quase exata de um gás, contendo cerca de 1026 partículas em movimento livre, é incomparavelmente mais fácil do que a do sistema solar, composto por 9 corpos principais; e ainda mais do que a de uma estrela múltipla de três ou quatro objetos de tamanho semelhante. Isso se deve, é claro, à excelente possibilidade de aplicar as leis da estatística e das probabilidades no primeiro caso.

Essa analogia, no entanto, está longe de ser perfeita para o nosso problema. A teoria da mecânica para 2, 3, 4, * • • corpos é bem conhecida e, em sua forma teórica geral (distinguida de sua forma especial e computacional), é a base da teoria estatística para grandes números. Para a economia de troca social - ou seja, para os “jogos de estratégia” equivalentes - a teoria de 2, 3, 4, • • participantes estava até agora ausente. É essa necessidade que nossas discussões anteriores visavam estabelecer e que nossas investigações subsequentes se esforçarão para satisfazer. Em outras palavras, somente após o desenvolvimento satisfatório da teoria para números moderados de participantes será possível decidir se números extremamente grandes de participantes simplificam a situação. Vamos dizer novamente: compartilhamos a esperança - principalmente por causa da analogia mencionada acima em outros campos! - que tais simplificações de fato ocorram. As afirmações atuais sobre a concorrência livre parecem ser suposições valiosas e antecipações inspiradoras de resultados. Mas elas não são resultados e é cientificamente incorreto tratá-las como tal enquanto as condições que mencionamos acima não forem satisfeitas.

Existe na literatura uma quantidade considerável de discussão teórica que pretende mostrar que as zonas de indeterminação (de taxas de câmbio) - que indubitavelmente existem quando o número de participantes é pequeno - estreitam e desaparecem à medida que o número aumenta. Isso então forneceria uma transição contínua para o caso ideal de concorrência livre - para um número muito grande de participantes - onde todas as soluções seriam nitidamente e unicamente determinadas. Embora se deva esperar que isso de fato se revele o caso em suficiente generalidade, não se pode conceder que algo assim tenha sido estabelecido de forma conclusiva até agora. Não há como escapar disso: o problema deve ser formulado, resolvido e compreendido para pequenos números de participantes antes que algo possa ser comprovado sobre as mudanças de seu caráter em qualquer caso limite de grandes números, como a concorrência livre.

Uma reabertura realmente fundamental deste assunto é mais desejável porque não é certo nem provável que um mero aumento no número de participantes sempre leve, em última instância, às condições de concorrência livre. As definições clássicas de concorrência livre envolvem postulados adicionais além da grandeza desse número. Por exemplo, está claro que se certos grandes grupos de participantes agirem juntos por qualquer motivo que seja, então o grande número de participantes pode não se tornar efetivo; as trocas decisivas podem ocorrer diretamente entre grandes “coalizões”, poucas em número, e não entre indivíduos, muitos em número, agindo independentemente. Nossa discussão subsequente sobre “jogos de estratégia” mostrará que o papel e o tamanho das “coalizões” são decisivos em todo o assunto. Consequentemente, a dificuldade acima - embora não seja nova - ainda permanece como o problema crucial. Qualquer teoria satisfatória da “transição limite” de pequenos números de participantes para grandes números terá que explicar em que circunstâncias essas grandes coalizões serão ou não formadas - ou seja, quando os grandes números de participantes se tornarão efetivos e levarão a uma concorrência mais ou menos livre. Qual dessas alternativas é provável de surgir dependerá dos dados físicos da situação. Responder a essa pergunta é, pensamos, o verdadeiro desafio para qualquer teoria de concorrência livre.

\textbf{A Teoria de Lausanne}

Esta seção não deve ser concluída sem uma referência à teoria do equilíbrio da Escola de Lausanne e também de vários outros sistemas que levam em consideração o “planejamento individual” e os planos individuais interligados. Todos esses sistemas prestam atenção à interdependência dos participantes em uma economia social. No entanto, isso é invariavelmente feito sob restrições abrangentes. Às vezes, a concorrência livre é assumida, após a introdução da qual os participantes enfrentam condições fixas e agem como um número de Robinson Crusoes - exclusivamente focados em maximizar suas satisfações individuais, que sob essas condições são novamente independentes. Em outros casos, outros dispositivos restritivos são usados, todos os quais excluem o jogo livre de “coalizões” formadas por qualquer ou todos os tipos de participantes. Frequentemente, existem suposições definitivas, mas às vezes ocultas, sobre as maneiras pelas quais seus interesses em parte paralelos e em parte opostos influenciarão os participantes e os farão cooperar ou não, conforme o caso. Esperamos ter mostrado que tal procedimento equivale a uma petitio principii - pelo menos no plano em que gostaríamos de colocar a discussão. Ele evita a dificuldade real e lida com um problema verbal, que não é o empiricamente dado. Claro que não desejamos questionar a importância dessas investigações - mas elas não respondem às nossas perguntas.

\subsubsection{\textbf{A Noção de Utilidade}}
\textbf{Preferências e Utilidades}

Já afirmamos em 2.1.1. de que maneira desejamos descrever o conceito fundamental de preferências individuais por meio de uma noção de utilidade bastante abrangente. Muitos economistas podem considerar que estamos assumindo muito (cf. a enumeração das propriedades que postulamos em 2.1.1.) e que nosso ponto de vista é um retrocesso em relação à técnica mais cautelosa das “curvas de indiferença” modernas.

Antes de tentar qualquer discussão específica, vamos declarar como uma desculpa geral que nosso procedimento, na pior das hipóteses, é apenas a aplicação de um dispositivo preliminar clássico de análise científica: dividir as dificuldades, ou seja, concentrar-se em uma (o assunto próprio da investigação em questão) e reduzir todas as outras o máximo possível, por meio de suposições simplificadoras e esquematizadoras. Devemos também acrescentar que esse tratamento autoritário de preferências e utilidades é empregado no corpo principal de nossa discussão, mas investigaremos incidentalmente em certa medida as mudanças que a evitação das suposições em questão causaria em nossa teoria (cf. 66., 67.).

No entanto, sentimos que pelo menos uma parte de nossas suposições - a de tratar utilidades como quantidades numericamente mensuráveis - não é tão radical como muitas vezes se assume na literatura. Tentaremos provar este ponto específico nos parágrafos a seguir. Esperamos que o leitor nos perdoe por discutir apenas incidentalmente, de forma condensada, um assunto de tamanha importância conceitual como o da utilidade. Parece, no entanto, que mesmo algumas observações podem ser úteis, porque a questão da mensurabilidade das utilidades é semelhante em caráter a perguntas correspondentes nas ciências físicas.

Historicamente, a utilidade foi inicialmente concebida como quantitativamente mensurável - essencialmente um valor numérico. No entanto, objeções válidas foram levantadas contra essa visão em sua forma original e ingênua. É claro que toda medição - ou melhor, toda alegação de mensurabilidade - deve ser baseada em alguma sensação imediata, que possivelmente não pode e certamente não precisa ser analisada mais a fundo. 1 No caso da utilidade, a sensação imediata de preferência - de um objeto ou agregado de objetos em relação a outro - fornece essa base. Mas isso nos permite apenas dizer quando, para uma pessoa, uma utilidade é maior do que outra. Não é em si uma base para comparação numérica de utilidades para uma pessoa nem para qualquer comparação entre pessoas diferentes. Uma vez que não há maneira intuitivamente significativa de adicionar duas utilidades para a mesma pessoa, a suposição de que as utilidades têm caráter não numérico parece até plausível. O método moderno de análise de curvas de indiferença é um procedimento matemático para descrever essa situação.

\textbf{Princípios de Medição: Preliminares}

Tudo isso lembra fortemente as condições existentes no início da teoria do calor: também foi baseada no conceito intuitivamente claro de um corpo se sentindo mais quente do que outro, mas não havia uma maneira imediata de expressar significativamente em quanto, ou quantas vezes, ou em que sentido.

Essa comparação com o calor também mostra como é difícil prever a priori qual será a forma final de tal teoria. As indicações acima não revelam de forma alguma o que, como sabemos agora, aconteceu posteriormente. Descobriu-se que o calor permite uma descrição quantitativa não por um número, mas por dois: a quantidade de calor e a temperatura. O primeiro é diretamente numérico porque acabou sendo aditivo e também de uma maneira inesperada conectado à energia mecânica, que já era numérica. O último também é numérico, mas de maneira muito mais sutil; não é aditivo em nenhum sentido imediato, mas uma escala numérica rígida para ele surgiu do estudo do comportamento concordante dos gases ideais e do papel da temperatura absoluta em conexão com o teorema da entropia.

O desenvolvimento histórico da teoria do calor indica que devemos ter extremo cuidado ao fazer afirmações negativas sobre qualquer conceito com a pretensão de finalidade. Mesmo que as utilidades pareçam muito não numéricas hoje, a história da experiência na teoria do calor pode se repetir, e ninguém pode prever com que ramificações e variações. 1 E isso certamente não deve desencorajar explicações teóricas das possibilidades formais de uma utilidade numérica.

\textbf{Probabilidade e Utilidades Numéricas}

Podemos ir ainda um passo além das duplas negações acima, que eram apenas precauções contra afirmações prematuras da impossibilidade de uma utilidade numérica. Pode ser demonstrado que, nas condições em que se baseia a análise das curvas de indiferença, é necessário muito pouco esforço adicional para alcançar uma utilidade numérica.

Já foi apontado repetidamente que uma utilidade numérica depende da possibilidade de comparar diferenças nas utilidades. Isso pode parecer - e de fato é - uma suposição mais abrangente do que a mera capacidade de expressar preferências. No entanto, parece que as alternativas às quais as preferências econômicas devem ser aplicadas são tais que eliminam essa distinção.

Vamos, por enquanto, aceitar a imagem de um indivíduo cujo sistema de preferências é abrangente e completo, ou seja, que, para quaisquer dois objetos ou, melhor dizendo, para quaisquer dois eventos imaginados, possui uma clara intuição de preferência.

Mais precisamente, esperamos que ele, para quaisquer dois eventos alternativos que lhe sejam apresentados como possibilidades, seja capaz de dizer qual dos dois ele prefere.

É uma extensão muito natural dessa imagem permitir que tal indivíduo compare não apenas eventos, mas até mesmo combinações de eventos com probabilidades declaradas.

Por uma combinação de dois eventos, queremos dizer o seguinte: Vamos denotar os dois eventos como B e C e usar, para simplificar, a probabilidade de 50-50. Então, a “combinação” é a perspectiva de ver B ocorrer com uma probabilidade de 50 e (se B não ocorrer) C com a probabilidade (restante) de 50. Ressaltamos que as duas alternativas são mutuamente exclusivas, de modo que nenhuma possibilidade de complementaridade ou similar existe. Além disso, existe uma certeza absoluta da ocorrência de B ou C.

Para reafirmar nossa posição, esperamos que o indivíduo em questão possua uma intuição clara sobre se prefere o evento A à combinação 50-50 de B ou C, ou vice-versa. É claro que, se ele prefere A a B e também a C, então ele também preferirá essa combinação; da mesma forma, se ele prefere B e C a A, então ele também preferirá a combinação. No entanto, se ele preferir A a B, mas ao mesmo tempo C a A, qualquer afirmação sobre sua preferência de A em relação à combinação contém informações fundamentalmente novas. Especificamente: se ele agora prefere A à combinação 50-50 de B e C, isso fornece uma base plausível para a estimativa numérica de que sua preferência por A em relação a B é maior do que sua preferência por C em relação a A.

Se esse ponto de vista for aceito, então existe um critério para comparar a preferência de C por A com a preferência de A por B. É bem conhecido que, assim, as utilidades - ou melhor, as diferenças de utilidades - se tornam numericamente mensuráveis.

O fato de que a possibilidade de comparação entre A, B e C apenas nesse grau já é suficiente para uma medição numérica de “distâncias” foi observado pela primeira vez na economia por Pareto. Exatamente o mesmo argumento, no entanto, foi feito por Euclides para a posição de pontos em uma linha - na verdade, é a base de sua clássica derivação de distâncias numéricas

A introdução de medidas numéricas pode ser alcançada de forma ainda mais direta se utilizarmos todas as probabilidades possíveis. De fato: considere três eventos, C, A e B, para os quais a ordem das preferências do indivíduo é a que foi declarada. Seja a um número real entre 0 e 1, de modo que A seja exatamente igualmente desejável à combinação de eventos consistindo de uma chance de probabilidade 1 - a para B e a chance restante de probabilidade a para C. Sugerimos, então, o uso de a como uma estimativa numérica para a razão da preferência de A sobre B em relação à preferência de C sobre B.8 Uma elaboração exata e exaustiva dessas ideias requer o uso do método axiomático. Um tratamento simples com base nesse método é, de fato, possível. Discutiremos isso nas seções 3.5-3.7.

Para evitar mal-entendidos, afirmamos que os “eventos” que foram usados acima como substrato de preferências são concebidos como eventos futuros, de modo a tornar todas as alternativas logicamente possíveis igualmente admissíveis. No entanto, seria uma complicação desnecessária, no que diz respeito aos nossos objetivos atuais, nos envolvermos com os problemas das preferências entre eventos em diferentes períodos do futuro. 1 No entanto, parece que essas dificuldades podem ser evitadas localizando todos os “eventos” nos quais estamos interessados em um mesmo momento padronizado, de preferência no futuro imediato.

As considerações acima dependem vitalmente do conceito numérico de probabilidade, de modo que algumas palavras sobre este último podem ser apropriadas.

A probabilidade frequentemente foi visualizada como um conceito subjetivo, mais ou menos na natureza de uma estimativa. Como propomos usá-la na construção de uma estimativa numérica individual de utilidade, a visão acima da probabilidade não serviria ao nosso propósito. O procedimento mais simples é, portanto, insistir na interpretação alternativa, perfeitamente bem fundamentada, da probabilidade como frequência em longos períodos. Isso fornece diretamente a base numérica necessária.

Esse procedimento para uma medição numérica das utilidades do indivíduo depende, é claro, da hipótese de completude no sistema de preferências individuais. 8 É concebível - e até de certa forma mais realista - permitir casos em que o indivíduo não é capaz de afirmar qual das duas alternativas ele prefere nem que elas são igualmente desejáveis. Nesse caso, o tratamento por meio das curvas de indiferença também se torna impraticável.
Quão real essa possibilidade é, tanto para indivíduos quanto para organizações, parece ser uma questão extremamente interessante, mas é uma questão de fato. Certamente merece estudo adicional. Vamos reconsiderá-lo brevemente na seção 3.7.2.

De qualquer forma, esperamos ter mostrado que o tratamento por meio das curvas de indiferença implica ou em excesso ou em insuficiência: se as preferências do indivíduo não são todas comparáveis, então as curvas de indiferença não existem. 1 Se as preferências do indivíduo são todas comparáveis, então podemos até obter uma utilidade numérica (única) que torna as curvas de indiferença supérfluas.

Tudo isso se torna, é claro, sem sentido para o empreendedor que pode calcular em termos de custos e lucros (monetários).

Poderia ser levantada a objeção de que não é necessário entrar em todos esses detalhes intrincados sobre a mensurabilidade da utilidade, uma vez que, evidentemente, o indivíduo comum, cujo comportamento se deseja descrever, não mede suas utilidades exatamente, mas conduz suas atividades econômicas em uma esfera de considerável incerteza. O mesmo vale, é claro, para grande parte de seu comportamento em relação à luz, calor, esforço muscular, etc. No entanto, para construir uma ciência da física, esses fenômenos tiveram que ser medidos. E posteriormente, o indivíduo passou a usar os resultados de tais medições - diretamente ou indiretamente - até mesmo em sua vida cotidiana. O mesmo pode ocorrer na economia em algum momento futuro. Uma vez que uma compreensão mais completa do comportamento econômico tenha sido alcançada com a ajuda de uma teoria que utiliza esse instrumento, a vida do indivíduo pode ser materialmente afetada. Portanto, não é uma digressão desnecessária estudar esses problemas.

\textbf{Princípios de Medição: Discussão Detalhada}

O leitor pode sentir, com base no exposto, que obtivemos uma escala numérica de utilidade apenas supondo o princípio, ou seja, postulando realmente a existência de tal escala. Argumentamos na seção 3.3.2 que, se um indivíduo prefere A à combinação 50-50 de B e C (enquanto prefere C a A e A a B), isso fornece uma base plausível para a estimativa numérica de que essa preferência de A por B excede a de C por A. Não estamos postulando aqui - ou considerando como certo - que uma preferência pode superar outra, ou seja, que tais afirmações têm significado? Essa visão seria um completo equívoco de nosso procedimento.

Por outro lado, a quantidade de “posição” definida fisicamente-geometricamente não permite essa operação, 1 mas permite a operação de formar o “centro de gravidade” de duas posições. 2 Outros conceitos físico-geométricos, geralmente chamados de “vetoriais” - como velocidade e aceleração - permitem a operação de “adição”.

Em todos esses casos em que uma operação “natural” recebe um nome que lembra uma operação matemática - como os exemplos de “adição” acima - deve-se evitar cuidadosamente mal-entendidos. Essa nomenclatura não pretende afirmar que as duas operações com o mesmo nome são idênticas - isso manifestamente não é o caso; ela apenas expressa a opinião de que elas possuem características semelhantes e a esperança de que alguma correspondência entre elas seja estabelecida no final. Isso, é claro, quando viável, é feito encontrando um modelo matemático para o domínio físico em questão, dentro do qual ess quantidades são definidas por números, de modo que no modelo a operação matemática descreve a operação “natural” sinônima.

Para retornar aos nossos exemplos: "energia" e "massa" tornaram-se números nos modelos matemáticos pertinentes, com a "adição" natural tornando-se adição ordinária. "Posição", bem como as quantidades vetoriais, tornaram-se triplas de números chamadas coordenadas ou componentes, respectivamente. O "conceito natural" de "centro de gravidade" de duas posições \( [x_1, x_2, x_3] \) e \( [x'_1, x'_2, x'_3] \), com as "massas" \( \alpha_1, 1 - \alpha_1 \) (cf. nota de rodapé 2 acima), torna-se
\[ [ \alpha_{x} + (1 - \alpha)_{x'}, \alpha_{x} + (1 - \alpha)_{x'}, \alpha_{z} + (1 - \alpha)_{z'} ] \]
A "operação natural" de "adição" de vetores \( [x_i, x_i, x_i] \) e \( [x'_i, x'_i, x'_i] \) também se torna
\[ [x_i + x'_i, x_i + x'_i, x_i + x'_i] \]
O que foi dito acima sobre as operações "naturais" e matemáticas se aplica igualmente a relações naturais e matemáticas. Os vários conceitos de "maior" que ocorrem na física - maior energia, força, calor, velocidade, etc. - são bons exemplos.
Essas relações "naturais" são a melhor base para construir modelos matemáticos que correlacionam o domínio físico com elas. 

Aqui, uma observação adicional deve ser feita. Suponha que um modelo matemático satisfatório para um domínio físico, no sentido descrito acima, tenha sido encontrado, e que as quantidades físicas em consideração tenham sido correlacionadas com números. Nesse caso, não é necessariamente verdade que a descrição (do modelo matemático) forneça uma maneira única de correlacionar as quantidades físicas com números; ou seja, ela pode especificar uma família inteira de tais correlações - o nome matemático é mapeamentos - qualquer uma das quais pode ser usada para os propósitos da teoria. A passagem de uma dessas correlações para outra equivale a uma transformação dos dados numéricos que descrevem as quantidades físicas. Dizemos então que, nesta teoria, as quantidades físicas em questão são descritas por números até aquele sistema de transformações. O nome matemático de tais sistemas de transformação é grupos.

Exemplos de tais situações são numerosos. Assim, o conceito geométrico de distância é um número, até a multiplicação por fatores (positivos) constantes. A situação relativa à quantidade física de massa é a mesma. O conceito físico de energia é um número, até qualquer transformação linear (ou seja, adição de qualquer constante e multiplicação por qualquer constante positiva). O conceito de posição é definido até uma transformação linear ortogonal inhomogênea. Os conceitos vetoriais são definidos até transformações homogêneas do mesmo tipo.

É concebível que uma quantidade física seja um número até uma transformação monótona. Isso ocorre com quantidades para as quais existe apenas uma relação “maior” de forma “natural” - e nada mais. Por exemplo, isso era verdade para a temperatura enquanto apenas o conceito de “mais quente” era conhecido; aplica-se à escala de dureza dos minerais; também se aplica à noção de utilidade quando baseada na ideia convencional de preferência. Em tais casos, pode-se ser tentado a considerar que a quantidade em questão não é numérica de forma alguma, dada a arbitrariedade da descrição por números. No entanto, parece preferível evitar tais afirmações qualitativas e, em vez disso, declarar objetivamente até que sistema de transformações a descrição numérica está determinada. O caso em que o sistema consiste em todas as transformações monótonas é, naturalmente, um caso bastante extremo; várias graduações no outro extremo da escala são os sistemas de transformação mencionados acima: transformações lineares ortogonais homogêneas ou inhomogêneas no espaço, transformações lineares de uma variável numérica, multiplicação dessa variável por uma constante. Em alguns casos, a descrição numérica é absolutamente rigorosa, ou seja, não são necessárias quaisquer transformações.

Dada uma quantidade física, o sistema de transformações pelo qual ela é descrita por números pode variar com o tempo, ou seja, com o estágio de desenvolvimento do assunto. Assim, a temperatura originalmente era um número apenas até qualquer transformação monótona. Com o desenvolvimento da termometria, especialmente da termometria ideal de gás concordante, as transformações foram restritas às lineares, ou seja, apenas o zero absoluto e a unidade absoluta estavam ausentes. Desenvolvimentos subsequentes da termodinâmica até fixaram o zero absoluto, de modo que o sistema de transformação na termodinâmica consiste apenas na multiplicação por constantes. Poderíamos multiplicar exemplos, mas parece não haver necessidade de entrar mais a fundo nesse assunto.

Para a utilidade, a situação parece ser de natureza semelhante. Pode-se adotar a postura de que o único dado "natural" nesse domínio é a relação "maior", ou seja, o conceito de preferência. Nesse caso, as utilidades são numéricas até uma transformação monótona. Esse é, de fato, o ponto de vista geralmente aceito na literatura econômica, melhor expresso na técnica das curvas de indiferença.

Para restringir o sistema de transformações, seria necessário descobrir operações ou relações “naturais” adicionais no domínio da utilidade. Assim, Pareto apontou que uma relação de igualdade para diferenças de utilidade seria suficiente; em nossa terminologia, isso reduziria o sistema de transformação às transformações lineares. No entanto, como essa relação não parece ser verdadeiramente “natural”, ou seja, uma que possa ser interpretada por observações reproduzíveis, a sugestão não atinge o objetivo.

\textbf{3.6. Estrutura Conceitual do Tratamento Axiomático das Utilidades Numéricas}

A falha de um dispositivo em particular não precisa excluir a possibilidade de alcançar o mesmo fim por meio de outro dispositivo. Nossa alegação é que o domínio da utilidade contém uma operação ``natural'' que restringe o sistema de transformações exatamente na mesma extensão que o outro dispositivo teria feito. Isto é a combinação de duas utilidades com duas probabilidades alternativas $\alpha, 1 - \alpha, (0 < \alpha < 1)$ conforme descrito em 3.3.2. O processo é tão semelhante à formação de centros de gravidade mencionados em 3.4.3. que pode ser vantajoso usar a mesma terminologia. Assim temos para utilidades $u, v$ a relação ``natural'' $u > v$ (leia-se: $u$ é preferível a $v$), e a operação ``natural'' $\alpha u + (1 - \alpha)v, (0 < \alpha < 1)$, (leia-se: centro de gravidade de $u, v$ com os respectivos pesos $\alpha, 1 - \alpha$; ou: combinação de $u, v$ com as probabilidades alternativas $\alpha, 1 - \alpha$). Se a existência---e a observabilidade reprodutível---desses conceitos é concedida, então nosso caminho está claro: Devemos encontrar uma correspondência entre utilidades e números que carrega a relação $u > v$ e a operação $\alpha u + (1 - \alpha)v$ para utilidades nos conceitos sinônimos para números.

Denomine a correspondência por
\[ u \rightarrow p = \nu(u), \]
sendo $u$ a utilidade e $\nu(u)$ o número ao qual a correspondência se atribui. Nossos requisitos são então:

$$
\quad u > v \quad \text{implica} \quad \nu(u) > \nu(v),
$$

$$
\quad \nu(\alpha u + (1 - \alpha)v) = \alpha \nu(u) + (1 - \alpha)\nu(v).
$$

Se duas tais correspondências
$$
\quad u \rightarrow p = \nu(u)
$$
$$
\quad u \rightarrow p' = \nu'(u),
$$

deveriam existir, então elas estabelecem uma correspondência entre números
\[ \rho \leftrightarrow \rho', \]
para o qual também podemos escrever
\[ \rho' = \phi(\rho). \]

Note que em cada caso, o lado esquerdo tem os conceitos ``naturais'' para utilidades, e o lado direito os convencionais para números.\footnote{Agora estes são aplicados a números, $\rho$.}

$$
\alpha p + (1 - \alpha)q \text{ permanece inalterado (cf. nota de rodapé 1 na p. 24). Isto é,}
$$
\begin{equation}
p > q \text{ implica } \phi(p) > \phi(q),
\end{equation}

\begin{equation}
\phi(\alpha p + (1 - \alpha)q) = \alpha \phi(p) + (1 - \alpha)\phi(q).
\end{equation}

\text{Portanto, } $\phi(p)$ \text{ deve ser uma função linear, ou seja,}

\begin{equation}
p' = \phi(p) = \omega_0 p + \omega_1,
\end{equation}
onde $\omega_0, \omega_1$ são números fixos (constantes) com $\omega_0 \geq 0$.
Então, observamos o seguinte: se existe alguma valoração numérica das utilidades, então ela está determinada até uma transformação linear. Ou seja, a utilidade é um número sujeito a uma transformação linear.

Para que uma valoração numérica no sentido acima exista, é necessário postular certas propriedades da relação \(u > v\) e da operação \(au + (1 - a)v\) para utilidades. A seleção desses postulados ou axiomas e sua análise subsequente levam a problemas de certo interesse matemático. A seguir, apresentamos um esboço geral da situação para orientar o leitor; uma discussão completa pode ser encontrada no Apêndice.

A escolha dos axiomas não é uma tarefa puramente objetiva. Geralmente, espera-se alcançar algum objetivo específico - alguns teoremas específicos devem ser deriváveis a partir dos axiomas - e, até certo ponto, o problema é exato e objetivo. No entanto, além disso, existem sempre outros desideratos importantes de natureza menos exata: os axiomas não devem ser muito numerosos, seu sistema deve ser o mais simples e transparente possível, e cada axioma deve ter um significado intuitivo imediato pelo qual sua adequação possa ser julgada diretamente. Em uma situação como a nossa, esse último requisito é particularmente vital, apesar de sua ambiguidade: queremos tornar um conceito intuitivo passível de tratamento matemático e ver com clareza quais hipóteses isso requer.

A parte objetiva do nosso problema é clara: os postulados devem implicar a existência de uma correspondência (3:2:a) com as propriedades (3:1:a) e (3:1:b) conforme descrito em 3.5.1. Os desideratos adicionais, mesmo estéticos, indicados acima, não determinam uma única maneira de encontrar esse tratamento axiomático. A seguir, formularemos um conjunto de axiomas que parece ser essencialmente satisfatório.

\textbf{Observações Gerais Sobre os Axiomas}

Neste ponto, talvez seja prudente parar e reconsiderar a situação. Será que não mostramos demais? Podemos derivar dos postulados (3:A)-(3:C) o caráter numérico da utilidade no sentido de (3:2:a) e (3:1:a), (3:1:b) em 3.5.1; e (3:1:b) afirma que os valores numéricos da utilidade combinam (com probabilidades) como expectativas matemáticas! No entanto, o conceito de expectativa matemática frequentemente foi questionado, e sua legitimidade certamente depende de alguma hipótese sobre a natureza de uma "expectativa". Não estamos, então, introduzindo, de alguma forma oblíqua, as hipóteses que envolvem a expectativa matemática?

Mais especificamente: não pode existir em um indivíduo uma utilidade (positiva ou negativa) relacionada ao simples ato de "arriscar", de jogar, que a utilização da expectativa matemática oblitera?

Como nossos axiomas (3:A)-(3:C) lidam com essa possibilidade? Pelo que podemos ver, nossos postulados não tentam evitá-la. Mesmo aquele que mais se aproxima de excluir uma "utilidade de jogo" (3:C:b) (conforme sua discussão em 3.6.2.), parece ser plausível e legítimo, a menos que um sistema de psicologia muito mais refinado seja usado do que o atualmente disponível para fins econômicos. O fato de que uma utilidade numérica, com uma fórmula equivalente ao uso de expectativas matemáticas, pode ser construída com base em (3:A)-(3:C), parece indicar isso: praticamente definimos a utilidade numérica como aquilo para o qual o cálculo de expectativas matemáticas é legítimo. Desde que (3:A)-(3:C) garantem que a construção necessária pode ser realizada, conceitos como uma "utilidade específica de jogo" não podem ser formulados sem contradição nesse nível.

Como afirmamos anteriormente, nossos axiomas são baseados na relação \(u > v\) e na operação \(au + (1 - a)v\) para utilidades. Parece notável que esta última pode ser considerada como mais imediatamente dada do que a primeira: dificilmente se pode duvidar que alguém que possa imaginar duas situações alternativas com as utilidades respectivas \(u, v\) também possa conceber a perspectiva de ter ambas com as probabilidades respectivas dadas \(a, 1 - a\). Por outro lado, podemos questionar o postulado do axioma (3:A:a) para \(u > v\), ou seja, a completude dessa ordenação.

Considere este ponto por um momento. Concedemos que alguém pode duvidar se uma pessoa pode sempre decidir qual de duas alternativas com as utilidades \( u, v \) ele prefere. Mas, seja qual for o mérito desta dúvida, essa possibilidade - ou seja, a completude do sistema de preferências (individuais) - deve ser assumida até mesmo para os propósitos do "método da curva de indiferença" (cf. nossas observações em (3:A:a.) em 3.6.2.). Mas se essa propriedade de \( u > v \) é assumida, então o nosso uso do muito menos questionável \( au + (1 - \alpha)v \) também produz as utilidades numéricas.

Se a suposição de comparabilidade geral não é feita, uma teoria matemática baseada em \( au + (1 - \alpha)v \) juntamente com o que resta de \( u > v \) ainda é possível. Isso leva ao que pode ser descrito como um conceito vetorial muitas vezes dimensional da utilidade, que é um conjunto mais complicado e menos satisfatório, mas não pretendemos tratá-lo sistematicamente neste momento.

Esta breve exposição não pretende esgotar o assunto, mas esperamos ter transmitido os pontos essenciais. Para evitar mal-entendidos, as seguintes observações adicionais podem ser úteis:

Reenfatizamos que estamos considerando apenas utilidades experimentadas por uma pessoa. Essas considerações não implicam nada a respeito de comparações das utilidades pertencentes a diferentes indivíduos.

Não se pode negar que a análise dos métodos que fazem uso da expectativa matemática (cf. nota de rodapé 1 na p. 28 para a literatura) está longe de ser concluída no momento presente. Nossas observações em 3.7.1. vão nesta direção, mas muito mais deve ser dito a este respeito. Existem muitas questões interessantes envolvidas, que no entanto estão além do escopo deste trabalho.

Para os nossos propósitos, basta observar que a validade dos axiomas simples e plausíveis (3:A)-(3:C) em 3.6.1. para a relação \( u > v \) e a operação \( au + (1 - \alpha)v \) faz as utilidades serem números até uma transformação linear no sentido discutido nestas seções.

\texbf{O Papel do Conceito de Utilidade Marginal}

A análise precedente deixou claro que nos sentimos livres para fazer uso de um conceito numérico de utilidade. Por outro lado, discussões subsequentes mostrarão que não podemos evitar a suposição de que todos os sujeitos da economia em consideração estão completamente informados sobre as características físicas da situação na qual operam e são capazes de realizar todas as operações estatísticas, matemáticas, etc., que este conhecimento torna possível. A natureza e importância dessa suposição receberam atenção extensiva na literatura e o assunto está provavelmente muito longe de ser esgotado. Propomos não entrar nisso. A questão é demasiadamente vasta e difícil e acreditamos que é melhor ``dividir as dificuldades''. Ou seja, desejamos evitar essa complicação que, embora interessante por si só, deve ser considerada separadamente do nosso problema atual.

Na verdade, pensamos que nossas investigações --- embora assumam ``informação completa'' sem nenhuma discussão adicional --- fazem uma contribuição ao estudo deste assunto. Verá-se que muitos fenômenos econômicos e sociais que são usualmente atribuídos ao estado de ``informação incompleta'' do indivíduo fazem sua aparição em nossa teoria e podem ser satisfatoriamente interpretados com sua ajuda. Já que nossa teoria assume ``informação completa'', concluímos que esses fenômenos não têm nada a ver com a ``informação incompleta'' do indivíduo. Alguns exemplos particularmente marcantes disso serão encontrados nos conceitos de ``discriminação'' em 33.1., de ``exploração incompleta'' em 38.3., e do ``transfer'' ou ``tributo'' em 46.11., 46.12.

Com base no exposto, até nos aventuraríamos a questionar a importância geralmente atribuída à informação incompleta em seu sentido convencional na teoria econômica e social. Aparecerá que alguns fenômenos que a primeira vista teriam que ser atribuídos a este fator, não têm nada a ver com ele.

Vamos agora considerar um indivíduo isolado com características físicas definidas e com quantidades definidas de bens à sua disposição. Em vista do que foi dito acima, ele está em posição de determinar a utilidade máxima que pode ser obtida nesta situação. Já que o máximo é uma quantidade bem definida, o mesmo é verdadeiro para o aumento que ocorre quando uma unidade de qualquer bem definido é adicionada ao estoque de todos os bens em posse do indivíduo. Isso é, claro, a noção clássica da utilidade marginal de uma unidade da mercadoria em questão.

Estas quantidades são claramente de importância decisiva na economia de "Robinson Crusoé". A utilidade marginal acima obviamente corresponde ao máximo esforço que ele estará disposto a fazer - se ele se comportar de acordo com os critérios habituais de racionalidade - para obter uma unidade adicional dessa mercadoria.

No entanto, não é nada claro qual significado isso tem na determinação do comportamento de um participante em uma economia de troca social. Vimos que os princípios de comportamento racional neste caso ainda aguardam formulação, e que certamente não são expressos por um requisito máximo do tipo Crusoé. Assim, deve ser incerto se a utilidade marginal tem algum significado neste caso.

Declarações positivas sobre este assunto serão possíveis apenas depois que tivermos sucesso no desenvolvimento de uma teoria do comportamento racional em uma economia de troca social, - ou seja, como foi dito antes, com a ajuda da teoria dos "jogos de estratégia". Verá-se que a utilidade marginal, de fato, desempenha um papel importante neste caso também, mas de uma maneira mais sutil do que geralmente se supõe.


\section{\textbf{Leonard (2010)}}

\subsection{\textbf{Introdução}}
A teoria dos jogos, pode-se afirmar com razão, provou ser uma das contribuições científicas mais significativas do século XX. Ainda que de forma hesitante e desigual, e de maneira completamente imprevisível em 1944, quando a “Teoria dos Jogos e Comportamento Econômico” foi publicada, ela afetou não apenas a economia e a ciência política, mas também a biologia evolutiva, a ética e a filosofia propriamente dita. Dentro da economia, áreas específicas como a teoria microeconômica, organização industrial, comércio internacional e economia experimental foram todas remodeladas sob a influência da teoria. Embora a teoria dos jogos tenha surgido inicialmente como uma contribuição crítica de fora, ela agora foi completamente abraçada pela disciplina econômica, como indicado pela concessão do Prêmio Nobel Memorial de Economia a John Nash, John Harsanyi e Reinhard Selten em 1994, e a Robert Aumann e Thomas Schelling em 2005.

Vários aspectos desse desenvolvimento têm recebido a atenção de historiadores da economia e outros estudiosos. Em 1992, sob a editoria de Roy Weintraub, um conjunto exploratório de ensaios intitulado “Rumo a uma História da Teoria dos Jogos” apresentou tanto relatos históricos quanto reminiscências. A partir dessa contribuição, um livro de 1996 escrito por Robert Dimand e MaryAnn Dimand forneceu um levantamento histórico das várias contribuições teóricas de jogos na primeira metade do século. Em sua monografia de 2003 sobre a evolução da racionalidade econômica, Nicola Giocoli dedica considerável atenção à teoria dos jogos, especialmente à forma como afetou a concepção neoclássica do agente econômico. Um tema semelhante, tratado de maneira diferente, é central no livro “Machine Dreams” de Philip Mirowski, de 2002, que retrata a história da teoria dos jogos como parte do surgimento do pensamento “ciborgue”, vinculado de maneira essencial ao trabalho de von Neumann em computação e autômatos.

Com o lançamento em 1998 da biografia de John Nash escrita por Sylvia Nasar, “Uma Mente Brilhante”, e sua subsequente adaptação para um filme de Hollywood, o público interessado na história da teoria dos jogos cresceu para incluir o público em geral. O livro “A Beautiful Math” de Tom Siegfried (2006) busca explicar a um amplo público o sucesso da teoria dos jogos de Nash em todo o espectro científico, e um documentário da BBC de 2007, “The Trap”, tenta mostrar como a vida social e política contemporânea foi moldada pela adoção de concepções de racionalidade baseadas na teoria dos jogos na esfera de políticas. O filme traça uma linha direta desde os primeiros dias da teoria dos jogos até a Segunda Guerra Mundial, o RAND, o Pentágono de McNamara e, finalmente, a reforma do estado de bem-estar social britânico por Margaret Thatcher.

Embora cada uma dessas contas tenha seus pontos fortes, nenhuma delas, a meu ver, faz justiça ao rico processo histórico pelo qual von Neumann e Morgenstern foram levados à criação da “Teoria dos Jogos e Comportamento Econômico”. Poucas delas tratam os autores como figuras de carne e osso, e nenhuma delas considera o contexto cultural e político da Europa do fim do século XIX e entre guerras, sem o qual, na minha opinião, a origem da teoria dos jogos não pode ser compreendida. Em algumas delas, von Neumann é tratado como um degrau necessário em direção a John Nash; em outras, Morgenstern é completamente omitido e a teoria dos jogos de von Neumann é forçada a prenunciar um futuro de máquinas e computação.

É um lugar-comum que todo relato histórico deve equilibrar o fato de que o narrador é “onisciente” com o fato de que, em qualquer momento histórico dado, o futuro não estava totalmente determinado nem conhecido. O relato atual é uma tentativa de contar a história de von Neumann, Morgenstern e a criação da teoria dos jogos de uma maneira que explore a onisciência do autor apenas minimamente. Cada esforço é feito para restaurar a especificidade histórica do assunto, manter o futuro em suspenso e permitir que nossos personagens se desenvolvam ao longo do tempo em resposta às circunstâncias em constante mudança. A história nos leva de volta ao mundo de Budapeste e Viena na primeira metade do século XX, com seus cafés de xadrez, debates sobre a natureza e o propósito da economia e intensa preocupação com a política. Em seguida, nos transporta para o mundo muito diferente de Princeton nos anos 1930 e os Estados Unidos do pós-guerra. Quando tratamos de von Neumann nos anos 1920, não é com a consciência de que ele viria mais tarde para os Estados Unidos ou trabalharia na bomba atômica ou no computador, mas sim suspendendo deliberadamente nosso conhecimento desses desenvolvimentos. Ao considerarmos Morgenstern na Viena do período entre guerras, o tratamos não como o futuro coautor de um livro em Princeton durante a guerra, mas como um economista austríaco dissidente em um ambiente cultural específico.

Em um ensaio de 1992, publicado no volume mencionado editado por Weintraub, eu retratei o artigo de 1928 de von Neumann sobre jogos como uma contribuição isolada, com pouca relação com o trabalho ou interesses de seus contemporâneos. Desde então, minha opinião mudou consideravelmente, após ser levado por um comentário passageiro do matemático Ernest Zermelo, a ver o artigo como conectado à rica discussão sobre a psicologia e a matemática do xadrez nas primeiras décadas do século XX. Nossa história, portanto, começa no tabuleiro de xadrez. O Capítulo 1 discute a importância cultural do xadrez na Europa Central e o interesse emergente na psicologia do jogo, bem como a ideia, talvez melhor ilustrada nos escritos do campeão de xadrez alemão e matemático Emanuel Lasker, de que o xadrez poderia fornecer algumas ideias sobre interação econômica e social. O xadrez emerge como uma fonte frutífera de reflexão matemática, psicológica e sociológica, e a origem de um discurso emergente de “equilíbrio e luta”.

Nos capítulos que tratam do contexto húngaro de von Neumann, dedicamos nossa atenção a dois grupos entrelaçados: a comunidade judaica assimilada do país e seus matemáticos. A relevância destes últimos é óbvia: para um país de pequeno porte e desenvolvimento limitado, a Hungria produziu uma notável cultura matemática, e o prodigioso von Neumann permaneceu orgulhoso de suas origens até o fim. Quanto ao tema da comunidade judaica húngara, o assunto é abordado aqui tanto por sua relevância cultural na época quanto por sua importância, especialmente mais tarde, na vida de von Neumann. O Capítulo 4 considera sua jornada nos anos 1920 de Budapeste para Göttingen, onde ele fez sua primeira incursão na matemática dos jogos. Como uma tentativa de fornecer um tratamento matemático de um campo incomum, o artigo traz a influência de Hilbert de Göttingen, que permaneceria importante no trabalho de von Neumann nessa área.

Mudando para Viena nos Capítulos 5 e 6, consideramos o início da carreira do muito diferente Oskar Morgenstern. Ele começou como um estudante convencional da Escola Austríaca de economia não matemática, e as influências a que ele foi exposto refletem a riqueza do debate no período: a crítica epistemológica agora esquecida de Hans Mayer; as fantasias românticas universalistas de Othmar Spann; e a mistura de crítica teórica e didática política de Ludwig von Mises. Com o tempo, Morgenstern rompeu com a maioria desses mentores economistas, estimulado, em parte, pelo matemático vienense Karl Menger, que também acontecia a ser - e a ironia não foi perdida por ele - o filho do fundador da Escola Austríaca.

Se ganhar acesso aos papéis de Morgenstern foi fácil, a situação foi diferente para os de Karl Menger. De fato, levou meses de negociação diplomática antes que eu me encontrasse numa sexta-feira à noite em Chicago, numa sala empoeirada do Instituto de Tecnologia de Illinois, com nada diante de mim além do fim de semana e vinte caixas de papéis de Menger, praticamente intocadas desde sua morte em 1985. Essa exploração não apenas revelou tesouros encadernados com fita relacionados à vida de seu pai economista, mas também me permitiu abordar Menger de um ângulo diferente. Ficou claro que havia conexões sutis entre as diversas esferas de atividade de Menger, incluindo os debates sobre os fundamentos da matemática, sua teoria formal da ética e seu envolvimento político em Viena nos anos 1930 (veja Leonard 1998). Como mostramos no Capítulo 7, a política foi um fator importante ao longo do tempo, seja moldando a dissociação de Menger de L. E. J. Brouwer ou provocando seu livro de 1934, “Moralidade, Decisão e Organização Social”. A visão de Morgenstern neste último, de uma solução matemática para o que ele considerava fraquezas no tratamento ortodoxo do agente econômico racional, mostrou ainda que apenas dois passos curtos separavam a política vienense e o debate sobre a teoria econômica.

Esse complexo entrelaçamento de temas econômicos, matemáticos e políticos é explorado no Capítulo 8, onde investigamos a aliança formada entre Morgenstern e Menger. Isso se deveu em grande parte à busca comum pela “pureza” e não estava desconectado das lutas locais pelo poder na pesquisa econômica vienense. Rejeitando o que percebiam como a infiltração da política na economia, personificada no trabalho de von Mises à direita e Neurath à esquerda, eles sentiram que o uso da matemática, na medida em que exigia pensamento claro e demonstração lógica, poderia tornar a análise econômica mais “isenta de valores”

Em sua conta de como sua construção dessa “geometria” da sociedade estava enraizada na agitação política de 1933-34, Menger também despertou meu interesse no que podemos chamar de criatividade individual: neste caso, a do matemático criativo em um contexto específico, movendo-se no tempo psicológico, por assim dizer, da “página em branco” à construção ou prova final. Por qual caminho misterioso, especialmente ao escrever sobre questões “reflexivas”, como os campos da racionalidade ou interação social, o matemático avançava dos primeiros passos a um resultado digno do nome? O conhecimento de que a página, é claro, nunca estava completamente em branco não diminuiu o poder dessa imagem para mim. Nem fui desencorajado pela realização de que o processo criativo interno nunca poderia ser completamente recuperado: vislumbrá-lo seria suficiente.

Foi à luz dessa experiência com Menger que renovei minha investigação sobre von Neumann. Se, como resultado do exame do mundo do xadrez, seu interesse inicial por jogos agora fazia sentido, ainda restava o enigma de por que ele retornou à teoria dos jogos no final dos anos 1930, após um hiato de mais de uma década. A sabedoria convencional entre os historiadores da economia parecia se basear em duas ideias: que ele sempre se interessou pelo assunto, como evidenciado pelo aparecimento da técnica minimax em seu artigo de 1937 sobre crescimento econômico em equilíbrio, e que foi Morgenstern quem o trouxe de volta ao assunto quando se encontraram no final dos anos 1930. Um valioso artigo da historiadora dinamarquesa de matemática, Tinne Kjeldsen (2002), refutou a primeira ideia, destrinchando as conexões entre os papéis de von Neumann de 1928 e 1937: o primeiro não envolvia um teorema de ponto fixo, e o segundo, que envolvia, foi observado por von Neumann como tendo apenas uma relação acidental com o primeiro. A história orientada pela técnica estava errada, mostrou Kjeldsen. A ideia de que von Neumann foi estimulado pelos quebra-cabeças teóricos de Morgenstern, apresentados nos chás da tarde em Fine Hall, embora certamente verdadeira, de alguma forma não parecia adequada para explicar o esforço hercúleo do matemático, em 1940-1941, no meio da Segunda Guerra Mundial.

Para superar esse impasse, li e reli “A Teoria dos Jogos e Comportamento Econômico” e, ao mesmo tempo, aprofundei-me na vida de von Neumann nos anos 1930. A exploração de novas correspondências - algumas fornecidas por Marina von Neumann Whitman, mais recentemente traduzidas - trouxe uma dimensão humana muito necessária às suas atividades, e leituras adicionais sobre a história social e política húngara lançaram luz sobre suas preocupações como um expatriado judeu na época. Gradualmente, ficou claro que o reavivamento do interesse de von Neumann pelos jogos devia-se em parte à sua experiência com a agitação política do final dos anos 1930, um período verdadeiramente dramático durante o qual questões sociais atraíram grande parte de sua atenção. Nossa análise desse interlúdio no Capítulo 9 nos permite entender o retorno de von Neumann à teoria dos jogos, bem como certos destaques em sua nova matemática de coalizões, como a ideia de múltiplas ordens sociais e o papel estabilizador desempenhado pelas convenções relacionadas à discriminação social. Essa fase da vida criativa de von Neumann agora faz sentido tanto do ponto de vista pessoal quanto histórico, e ficou claro que a teoria dos jogos foi moldada apenas parcialmente pela discussão da teoria econômica. A colaboração com Morgenstern e o efeito dessa experiência sobre este último são discutidos nos Capítulos 10 e 11.

O “choque emocional” que afetou von Neumann durante esse período foi a destruição nazista do mundo que ele conhecia, aquela cultura mitteleuropeia socialmente heterogênea de matemática e ciência. Se seu desenvolvimento em 1940 de uma nova matemática da sociedade foi em parte uma reação a isso - interna e simbólica em sua natureza - também o foi sua decisão simultânea de “entrar em guerra”, discutida no Capítulo 12. Embora o envolvimento de von Neumann na guerra tenha expressado uma atitude “apocalíptica” frequentemente evocada em conexão com os expatriados húngaros, também teve o efeito mais mundano de ver uma pequena parte da teoria dos jogos sendo usada na pesquisa operacional, em problemas de bombardeio estratégico e busca de submarinos, mundos distantes da matemática abstrata de conjuntos estáveis e ordens sociais discriminatórias.

Foi nessa capacidade instrumental que a teoria dos jogos recebeu a aprovação pós-guerra de uma comunidade científica que se estendia além de seus autores originais. Após a guerra, ela se tornou um elemento definidor na visão de mundo da RAND Corporation. Adotada inicialmente como uma ferramenta analítica para ajudar em problemas de bombardeio estratégico ou perseguição de caças, à medida que a RAND evoluía, a teoria dos jogos se tornava um elemento integral em uma complexa rede de atividades socio-científicas naquela instituição, muitas delas reflexo da cultura da Guerra Fria predominante. Isso é discutido em nosso último capítulo, o 13.

Este livro é o resultado de um longo e absorvente período de “trabalho de detetive”, uma investigação que, para o compreensível desagrado de meu editor e muitos colegas, foi tão agradável quanto a publicação dos resultados. Uma dica revelada em uma nota de rodapé de artigo ou carta arquivada abriria uma nova perspectiva, dando origem a meses de novas leituras. O livro foi moldado pelos resultados, muitas vezes serendipitosos, da pesquisa e da leitura, com novas incursões nos arquivos sendo realizadas até um ano antes da entrega do manuscrito final. Levei um tempo para perceber que nem todas as perspectivas, nem todos os resultados, poderiam necessariamente ser apresentados no mesmo livro.

Na narrativa que se desenrola, eu intencionalmente me esforcei para imergir o leitor nos processos de pensamento de von Neumann, Morgenstern e seus contemporâneos. Simultaneamente, retratei os contextos intelectuais e sociais mais amplos nos quais eles viveram e trabalharam. Essa tentativa de entrelaçar biografia, criatividade científica e os eventos externos em evolução tem sido um dos aspectos mais gratificantes de escrever este livro. Espero que isso fique evidente para o leitor à medida que ele ou ela se dedica à leitura. Durante o início do século XX, Viena era um caldeirão de fermentação intelectual e cultural. Muitos dos intelectuais por trás do movimento Social Democrata eram judeus vienenses, e o partido, como herdeiro da tradição liberal austríaca, contava com o apoio de grande parte dos círculos profissionais e comerciais judeus da capital. Essa dinâmica acentuou ainda mais a percepção de uma divisão entre os socialistas urbanos e “ateus” de Viena e os interesses conservadores e católicos da Áustria provincial. Ao longo dos anos 1920, as tensões aumentaram. Um momento crucial ocorreu em 14 de julho de 1927, quando um júri absolveu três jovens do Frontkämpfen acusados de assassinar um membro do Schutzbund. O veredicto provocou uma reação popular imediata e massiva. Em 15 de julho, milhares de trabalhadores protestantes desceram sobre Viena ao longo das principais artérias que levavam à Ringstrasse. O resultado foi um tumulto no qual vários policiais foram espancados até a morte, o Justizpallas foi incendiado e cerca de oitenta e cinco manifestantes foram mortos pelas forças de segurança. Esses eventos tiveram um impacto duradouro nas relações políticas da Primeira República Austríaca. Assim, quando Menger retornou de Amsterdã alguns meses depois, encontrou uma Viena tensa e instável.

Particularmente abalados por esses eventos estavam os intelectuais social-democratas, incluindo seus porta-vozes no Círculo de Viena, Neurath e Hahn. As raízes do Círculo de Viena remontam ao Verein Ernst Mach, um grupo de discussão filosófica iniciado no início dos anos 1920 por Hahn e Neurath, juntamente com Moritz Schlick e o físico Philipp Frank, todos dedicados a perpetuar o empirismo machiano diante de uma mudança geral em direção à metafísica nos círculos filosóficos de língua alemã. Em 1923, por meio da intervenção de Hahn, Schlick mudou-se de Praga para Viena para assumir a cátedra Mach-Boltzmann na filosofia das ciências indutivas, e com o tempo o grupo expandiu-se para incluir Victor Kraft, Kurt Reidemeister, Rudolf Carnap, Herbert Feigl e Friedrich Waismann. Em 1929, com seu manifesto audacioso, o grupo Mach tornou-se mais claramente identificado como o Schlick Kreis, ou Círculo de Viena. A maioria das histórias do grupo foca de forma um tanto estreita em sua filosofia, mas havia uma importante dimensão política na virada em direção ao empirismo lógico. Isso é um refrão constitutivo nos escritos de Neurath e Hahn e é fundamental para entender Menger.

\subsection{\textbf{Capítulo 11}}
\subsubsection{\textbf{A 'Introdução'}}

Morgenstern estava energizado durante esse período ao escrever a parte da Teoria dos Jogos que seria mais amplamente lida: sua introdução acessível, a "Formulação do Problema Econômico". Um manifesto para uma nova economia matemática, este capítulo representa a confluência das duas diferentes tradições intelectuais dos dois autores, com traços da crítica austríaca e, ao mesmo tempo, impregnado pela influência de von Neumann.

O ensaio defende o uso de um certo tipo de matemática na economia e discute o comportamento racional, utilidade e os novos conceitos de solução e padrão de comportamento. O estado primitivo da economia, enfatiza-se repetidamente, resultou de uma compreensão inadequada das categorias e fatos básicos, sobre os quais uma matemática não sofisticada e imitativa foi enxertada. A aplicação simples das metáforas físicas da mecânica racional ou sistemas conservativos à interação social é rejeitada.

Da mesma forma, a ideia de que os mundos natural e social constituem dois domínios diferentes, a serem tratados de forma distinta na ciência, é rejeitada. Essa rejeição da distinção entre Geisteswissenschaft e Naturwissenschaft, uma postura rebelde já adotada por Karl Menger na década de 1930, continuou a ruptura de Morgenstern com a tradição econômica austríaca. Em vez disso, seguindo von Neumann, o mundo empírico, como uma coleção heterogênea de fenômenos, sejam eles naturais ou sociais, é visto como um repositório de corpos potenciais de matemática, e a proeminência dos fenômenos é interpretada como um sinal de sua promessa nesse sentido. Sugere-se que uma nova matemática será necessária no domínio social devido à "importância dos fenômenos sociais, à riqueza e multiplicidade de suas manifestações e à complexidade de sua estrutura".

Qualquer ambiguidade anteriormente demonstrada por Morgenstern em relação ao uso da matemática agora desapareceu. Uma seção é dedicada a refutar os argumentos normalmente levantados contra a análise matemática do comportamento humano - fatores psicológicos, dificuldades de medição e assim por diante. Argumentos semelhantes foram levantados contra o uso da matemática na física do século XVI e na química e biologia do século XVIII, todas as quais agora dependem completamente de técnicas matemáticas. Além disso, o desenvolvimento matemático facilitou o desenvolvimento conceitual: na teoria do calor, por exemplo, a energia e a temperatura surgiram da matemática; elas não a precederam.

A razão pela qual a racionalidade individual nunca foi tratada adequadamente reside na falta de métodos matemáticos capazes de quantificar todos os elementos na descrição qualitativa usual. As "relações significativas são muito mais complicadas do que o uso popular e 'filosófico' da palavra 'racional' indica" (p. 9). No cerne dessa complexidade está a transição do mundo de Robinson Crusoé para a economia social, onde a otimização do indivíduo depende de "variáveis vivas" - ou seja, das decisões dos outros. Isso é um problema conceitualmente novo, "em nenhum lugar tratado na matemática clássica... nenhum problema de máximo condicional, nenhum problema de cálculo de variações, de análise funcional, etc." (p. 11). Na economia social, à medida que se passa de dois para três ou mais participantes, a complexidade aumenta, na medida em que cada indivíduo precisa levar em conta um número crescente de outros jogadores e o número de variáveis de escolha para cada jogador aumenta. As "complicações combinatoriais do problema... aumentam tremendamente a cada aumento no número de jogadores" (ibid).

Morgenstern reflete sobre a relação entre a nova teoria e a teoria atual da competição. Ao contrário da mecânica, na qual a teoria para dois, três, quatro e mais corpos é bem conhecida e fornece uma base para a teoria estatística do caso de grandes números, o mesmo não pode ser dito da economia de troca social. Até agora, não havia uma teoria para pequenos números de participantes. Diferentemente da mecânica, na economia, não se pode afirmar se a transição de pequenos para grandes números de agentes representa ou não uma simplificação. As “afirmações atuais sobre a livre concorrência parecem ser suposições valiosas e antecipações inspiradoras de resultados”, mas, sem uma teoria para pequenos números, elas não são resultados científicos. “O problema deve ser formulado, resolvido e compreendido para pequenos números de participantes antes que algo possa ser comprovado sobre as mudanças de seu caráter em qualquer caso limite de grandes números, como a livre concorrência” (p. 14). Isso é especialmente importante devido à crença generalizada de que o aumento do número de participantes inevitavelmente leva à livre concorrência. As teorias atuais de planos individuais interligados, que faziam referência velada a Hayek, à Escola de Lausanne e a outros, impõem restrições abrangentes: às vezes, a livre concorrência é assumida; outras vezes, a formação de coalizões é simplesmente descartada. Isso equivale a um petitio principii, evitando a verdadeira dificuldade. No entanto, se houver formação de coalizões de participantes atuando juntos, a situação muda drasticamente. A nova teoria enfatiza a importância da formação de coalizões, e uma nota de rodapé aqui faz referência a sindicatos, cooperativas de consumidores, cartéis e “possivelmente algumas organizações mais no âmbito político” (p. 15, n. 1).

Várias páginas são dedicadas a defender o uso de utilidades numericamente mensuráveis e cardinais, especialmente dada a percepção de que isso representa um retrocesso em relação à teoria das curvas de indiferença e à suposta impossibilidade de comparação interpessoal de utilidade. A situação atual, argumenta Morgenstern, é muito semelhante ao estágio inicial da teoria do calor, quando um objeto era visto como mais quente ou mais frio do que outro, mas nenhuma comparação quantitativa era possível. Descobriu-se que a comparação era possível em termos tanto da quantidade de calor quanto da temperatura. O primeiro é diretamente numérico e aditivo, relacionado a uma energia mecânica numérica. O último é mais sutilmente numérico, não sendo aditivo, mas ainda assim baseado em uma “escala rigidamente numérica” que emergiu do estudo do comportamento dos gases ideais e do papel da temperatura em relação ao teorema da entropia. A experiência histórica da teoria do calor, portanto, deve fornecer algum encorajamento em relação a uma teoria da utilidade.

O passo da teoria da indiferença para a utilidade cardinal é, na verdade, relativamente pequeno. Se for plausível acreditar que um indivíduo pode escolher entre dois itens, A e B, então também é plausível acreditar que ele pode escolher entre um item, A, e uma combinação (B e C, cada um com uma probabilidade de cinquenta por cento). Se ele prefere A tanto a B quanto a C, então escolherá A em vez do pacote. No entanto, se ele prefere A a B e C a A, então, se ele escolher A em vez do pacote, podemos assumir que ele prefere A a B mais do que prefere C a A. Especificamente, se o indivíduo é indiferente entre A e a combinação (B com probabilidade a, C com probabilidade 1-a), então a pode ser usado como uma estimativa numérica da razão de preferência de A sobre B em relação à de C sobre B. A teoria da curva de indiferença, portanto, assume ou muito ou pouco: o primeiro se as preferências não são comparáveis, o segundo se de fato são, em que caso uma utilidade numérica pode ser obtida, tornando a curva de indiferença supérflua.

À objeção de que o indivíduo geralmente opera em um ambiente vago e nebuloso, Morgenstern responde que o mesmo poderia ser dito sobre a conduta do indivíduo "em relação à luz, ao calor, ao esforço muscular, etc." (p. 20). "Mas, para construir uma ciência da física, esses fenômenos tiveram que ser medidos. E posteriormente, o indivíduo passou a usar os resultados dessas medições - direta ou indiretamente - até mesmo em sua vida cotidiana" (ibid). Talvez a vida econômica material do indivíduo um dia possa ser afetada de maneira semelhante, graças a tais medições, e segue-se uma discussão detalhada dos princípios de medição e das operações sobre medidas como massa, distância e posição. No caso da utilidade, a teoria da curva de indiferença assume que só se pode falar da relação "maior" - ou seja, que as utilidades são numéricas até uma transformação monotônica. Segue-se a demonstração de von Neumann de que o conjunto de transformações pode ser reduzido ao conjunto de todas as transformações lineares. Na nova teoria, Morgenstern conclui que o conceito de utilidade marginal pode não ter mais nenhum significado.

O final da introdução deixa de lado considerações filosóficas amplas para uma descrição encapsulada da teoria à qual o restante do livro é dedicado. A ideia teórica central, além, obviamente, da crença de que as situações sociais devem ser descritas em termos de jogos de coalizão, é a identificação do conjunto de possíveis resultados em um jogo. Esses resultados são soluções, no sentido de que constituem equilíbrios dos quais não há tendência de afastamento. Uma solução é plausivelmente um "conjunto de regras para cada participante, que lhe diz como se comportar em cada situação que possa concebivelmente surgir" (ibid). A solução deve levar em conta o fato de que alguns eventos enfrentados pelos participantes podem ser determinados apenas estatisticamente e que alguns indivíduos podem agir de forma irracional. Assim, o conceito de jogo desempenha a mesma função para problemas econômicos e sociais que os "modelos geométrico-matemáticos" desempenham para as ciências físicas: precisos, exaustivos, não excessivamente complicados e semelhantes à realidade, assim como a interpretação simplificadora de Newton do sistema solar como um pequeno número de 'masspoints'. A solução é uma "enumeração combinatória de enorme complexidade", indicando o mínimo que o participante em consideração pode obter se se comportar de forma racional. Ele pode obter mais se os outros se comportarem de forma "irracional" - ou seja, cometerem erros. Assim como no desenvolvimento da relatividade geral e da mecânica quântica, esses são "desiderata" pré-teóricos, características que a teoria pode ser esperada para mostrar se for considerada satisfatória.

Para a sociedade como um todo, a "solução" ideal seria aquela que descreve a quantidade obtida por cada jogador, jogando de forma racional. Se tal imputação única estivesse disponível, então a "estrutura da sociedade em consideração seria extremamente simples: existiria um estado absoluto de equilíbrio no qual a parcela quantitativa de cada participante seria precisamente determinada" (p. 34). No entanto, tal solução simples geralmente não existe, portanto, a noção precisa ser ampliada - algo que "está intimamente ligado a certas características inerentes da organização social que são bem conhecidas do ponto de vista do 'senso comum', mas até agora não foram vistas de maneira adequada" (ibid).

O conjunto de jogos para os quais essa solução determinada simples se aplica são aqueles do tipo dois jogadores, soma zero, que, embora não sejam típicos de "grandes processos econômicos", contêm características universalmente importantes de todos os jogos e formam a base para a teoria geral. As divergências mais simples desse caso básico são aquelas em que o requisito de soma zero não se aplica mais, o problema do monopólio bilateral ou, quando outro jogador entra em cena - o jogo de três pessoas, soma zero. No segundo caso, porque existem três possibilidades de formação de coalizões, de dois jogadores contra o restante, encontra-se três imputações possíveis, dependendo de qual coalizão é formada. A solução se torna um conjunto de imputações possíveis, em vez de uma única, e isso é algo que caracteriza os jogos em geral.

Para a sociedade como um todo, a "solução" ideal seria aquela que descreve a quantidade obtida por cada jogador, jogando de forma racional. Se tal imputação única estivesse disponível, então a "estrutura da sociedade em consideração seria extremamente simples: existiria um estado absoluto de equilíbrio no qual a parcela quantitativa de cada participante seria precisamente determinada" (p. 34). No entanto, tal solução simples geralmente não existe, portanto, a noção precisa ser ampliada - algo que "está intimamente ligado a certas características inerentes da organização social que são bem conhecidas do ponto de vista do 'senso comum', mas até agora não foram vistas de maneira adequada" (ibid).

O conjunto de jogos para os quais essa solução determinada simples se aplica são aqueles do tipo dois jogadores, soma zero, que, embora não sejam típicos de "grandes processos econômicos", contêm características universalmente importantes de todos os jogos e formam a base para a teoria geral. As divergências mais simples desse caso básico são aquelas em que o requisito de soma zero não se aplica mais, o problema do monopólio bilateral ou, quando outro jogador entra em cena - o jogo de três pessoas, soma zero. No segundo caso, porque existem três possibilidades de formação de coalizões, de dois jogadores contra o restante, encontra-se três imputações possíveis, dependendo de qual coalizão é formada. A solução se torna um conjunto de imputações possíveis, em vez de uma única, e isso é algo que caracteriza os jogos em geral. Na verdade, enquanto algumas dessas imputações são realmente formadas, as outras estão presentes em uma existência "virtual": embora não tenham se materializado, contribuíram essencialmente para moldar e determinar a realidade atual (p. 36).

Em um jogo de n pessoas, uma "solução deve ser um sistema de imputações que possua como um todo algum tipo de equilíbrio e estabilidade, cuja natureza tentaremos determinar. Ressaltamos que essa estabilidade - seja qual for - será uma propriedade do sistema como um todo e não das imputações individuais das quais ele é composto" (ibid).

Segue uma apresentação sistemática dos elementos esboçados no manuscrito anterior de von Neumann, de 1940. Uma imputação ou resultado, x, é dito dominar outro, y, "quando existe um grupo de participantes, cada um dos quais prefere sua situação individual em x à de y, e que estão convencidos de que são capazes, como grupo - ou seja, como uma aliança - de impor suas preferências" (p. 38). Se a dominação fosse uma ordenação transitiva, então um único melhor resultado seria a imputação que dominava todas as outras e não era dominada por nenhuma. No entanto, não é transitiva: x pode dominar y; y pode dominar z, mas z pode dominar x. Isso ocorre porque o "conjunto efetivo", a coalizão preferida, pode não ser o mesmo em cada caso, algo que é "um fenômeno muito típico em todas as organizações sociais. As relações de dominação entre várias imputações... ou seja, entre vários estados da sociedade - correspondem às várias maneiras pelas quais essas podem desestabilizar - ou seja, perturbar - umas às outras" (p. 39).

Uma "solução", baseada, como já vimos, no conceito de dominação, está relacionada aos "padrões de comportamento" - ou seja, o conjunto específico de regras, costumes ou instituições que regem a organização social em um determinado momento. Para entender a analogia, o leitor é aconselhado a "esquecer temporariamente a analogia com os jogos e pensar inteiramente em termos de organização social" (p. 41, n. 1):

O cenário de uma economia social ou, para ampliar a perspectiva, de uma sociedade, é dado. De acordo com toda a tradição e experiência, os seres humanos têm uma maneira característica de se ajustar a esse contexto. Isso consiste em não estabelecer um único sistema rígido de distribuição, ou seja, de imputação, mas sim uma variedade de alternativas, que provavelmente expressarão alguns princípios gerais, mas ainda assim diferirão entre si em muitos aspectos particulares. Esse sistema de imputações descreve a "ordem estabelecida da sociedade" ou o "padrão de comportamento aceito" (p. 41).

A natureza circular da definição, baseada nas características de dominação das imputações dos membros, confere a cada solução uma espécie de estabilidade interna. No entanto, uma solução será adotada apenas na medida em que o padrão subjacente de comportamento comande aceitação geral: a teoria não prevê qual solução será observada nem qual imputação dentro de qualquer solução prevalecerá. Essa abordagem da teoria social contrasta com aquelas que normalmente envolvem algum conceito de "propósito social". Este último geralmente se baseia em princípios relacionados à distribuição ou a outros objetivos gerais, que são inevitavelmente arbitrários e geralmente são apoiados por argumentos sobre estabilidade interna ou a desejabilidade da distribuição resultante: "Pouco pode ser dito sobre o último tipo de motivação. Nosso problema não é determinar o que deve acontecer em conformidade com qualquer conjunto de princípios a priori - necessariamente arbitrários -, mas investigar onde está o equilíbrio das forças. No que diz respeito à primeira motivação, nosso objetivo foi dar a esses argumentos uma forma precisa e satisfatória, abordando tanto os objetivos globais quanto as distribuições individuais... Uma teoria que seja consistente nesse ponto não pode deixar de fornecer uma descrição precisa de toda a interação de interesses econômicos, influência e poder" (p. 43).

\subsubsection{\textbf{Catharsis}}
A colaboração com von Neumann foi um evento enorme na vida de Morgenstern, levando-o a terreno desconhecido. Embora ele tenha percebido como o teorema minimax resolveu o problema de Holmes e Moriarty, é justo dizer que as intricadas questões matemáticas do conjunto estável permaneceram além de sua compreensão, assim como para muitos leitores. Se antes ele buscava uma lógica de previsão e interação, agora tinha mais do que podia lidar. Questões que, alguns anos antes, o teriam preocupado como economista austríaco - que o agente da teoria dos jogos era onisciente, uma espécie de "demi-deus", que a teoria esvaziava a passagem do tempo - agora empalideciam diante das 600 páginas de análise com as quais ele estava publicamente associado. Era uma posição estranha para um coautor: "Um dia desses, tenho que escrever algumas coisas sobre a história do livro (e minha participação mínima; mas parece que atuei como um tipo de fator catalítico)".

Apesar de toda a sua postura pública, Morgenstern revela em particular uma sensibilidade refinada para as dimensões humanas da colaboração intelectual. Quanto mais ele conhecia von Neumann, mais percebia que nunca o conheceria completamente: "Ele é um homem misterioso. No momento em que ele toca algo científico, fica totalmente entusiasmado, claro, vivo, depois afunda, sonha, fala superficialmente em uma mistura estranha". Ao contrário do marido, Morgenstern compartilhava o amor de Klari pela música, algo que eles desfrutavam enquanto von Neumann ficava à parte. "Ontem para o jantar na casa dos Neumann, ... Depois fomos para o meu lugar, e Clari e eu ouvimos Brahms, enquanto Johnny folheava Sorokin com surpresa". Com o tempo, ele se viu se tornando uma espécie de intermediário entre o casal. Von Neumann falava com ele sobre as dificuldades da vida conjugal, as tensões da paternidade. E Morgenstern confidenciava em Klari como nunca antes.

Vez após vez, ele voltava à mudança que estava passando: "Tenho uma clara sensação de liberdade de preconceitos e laços com teorias e visões gerais, como se estivesse trocando de pele. Espero que algumas coisas permaneçam. No nível emocional, estou mais aberto a pequenas alegrias e vejo o quanto eu era um completo idiota. Levava tudo muito tragicamente. Isso deve vir da minha educação na Primeira Guerra Mundial". Ele sentia como se seus hábitos originais de pensamento e expressão estivessem murchando. Mas ele acolheu isso: "por trás da ruptura, vejo a luz de uma nova ciência".

Quanto mais ele se aproximava de von Neumann e da matemática, menos confortável se sentia consigo mesmo. "Não posso nem quero abrir mão da teoria dos conjuntos... Fui um idiota por não ter estudado matemática, mesmo como um complemento na Universidade de Viena, em vez dessa filosofia boba, que consumiu tanto do meu tempo e do qual resta tão pouco. Depois de Fraenkel, vou ler Hausdorff". Isso, por sua vez, o lembrou de onde suas prioridades deveriam estar. Era agradável ter companhia feminina, mas não o tocava tão profundamente quanto seu trabalho, que agora o interessava mais do que qualquer outra coisa: "Isso acabou" (ibid).

Ele almoçou com o matemático Shizuo Kakutani, que disse que considerava von Neumann no mesmo patamar de Hilbert ou Poincaré. Kakutani lembrou que havia apresentado recentemente a von Neumann um problema sobre o qual havia pensado por anos, apenas para receber um telefonema do último uma hora e meia depois com uma solução - aparentemente descoberta durante o almoço. Na mesma tarde, Kakutani ficou maravilhado quando von Neumann apresentou uma prova em Teoria dos Operadores ao longo de várias horas, sem anotações e sem ter escrito nada anteriormente no papel.

Von Neumann estava cada vez mais imerso no trabalho de guerra, escapando frequentemente para Nova York, Washington e outros lugares para fornecer conselhos matemáticos sobre problemas militares. Morgenstern era um estrangeiro inimigo, mas esse fato e, se seu diário for alguma indicação, a própria guerra, empalideciam diante da catarse pessoal que ele estava passando. Em certo momento, em janeiro de 1942, von Neumann correu para o Aberdeen Proving Ground, deixando Morgenstern com uma "grande dor de cabeça" - ou seja, dez páginas sobre a análise do pôquer. Ele concordou com von Neumann que era improvável que os economistas construíssem sobre o livro antes de muito tempo, pois estavam mal equipados para absorver seu conteúdo.

A exposição ao trabalho matemático lançou uma nova luz sobre as ciências sociais: "Ontem vi Weyl enquanto ele pensava em um novo trabalho matemático. Essa concentração total evidente. Recentemente, perguntei a Johnny por que, em sua opinião, o poder criativo dos matemáticos diminui com tanta frequência tão cedo. Ele achava que devia ser o aumento das dificuldades de concentração. Que diferença no trabalho das ciências sociais. Quanto mais soltos, mais fáceis os processos mentais. Às vezes, penso que seria valioso escrever uma autobiografia contendo o pano de fundo científico de tal análise, especialmente se eu pudesse conectá-la às ciências reais. Realmente não tenho ilusões sobre as ciências sociais". Assim, quando se tratava de seus pares disciplinares, havia pouco espaço para indulgência privada. Fortalecido pela afirmação de von Neumann de que, no futuro, consideraríamos a economia matemática do século XX contemporânea a Newton, Morgenstern ficou "muito irritado com esses terríveis autores de livros didáticos. Não há nada sólido. Tudo é de terceira classe, cheio de erros, falta de conhecimento da literatura etc.". Comentários mordazes foram dirigidos a Hayek, Keynes e outros. Uma visita de Eve Burns, então envolvida em seu trabalho no sistema de seguridade social, convenceu Morgenstern de que ele agora estava "vivendo fora do mundo da maioria dos economistas. Parece que entrei em contato muito mais com o verdadeiro espírito matemático, e uma vez que os olhos se abrem, permanece assim".

Em maio de 1943, ele ministrou uma palestra para seu próprio departamento sobre a teoria dos jogos, recebendo uma reação mista. Seu maior crítico foi F. A. Lutz, cujo próprio trabalho sobre Bentham Morgenstern admirava muito, mas cujo interesse no cenário atual aparentemente o deixou “decididamente hostil e irônico” em relação à teoria dos jogos. Quando se descobriu que o perturbado Lutz se aproximou mais tarde de Weyl, perguntando o que estava errado com o uso do cálculo na economia, Morgenstern riu e relatou tudo a um von Neumann divertido. De um encontro com o reformador britânico William Beveridge, ele saiu furioso. As alegações dos economistas de que poderiam definir e alcançar o pleno emprego eram uma farsa, semelhante às alegações dos alquimistas na Idade Média, por que alguns deles foram decapitados. “Ontem, quando vi Johnny novamente, me ocorreu novamente que nas ciências sociais há poucas pessoas de tamanha eminência intelectual como nas matemáticas e nas ciências naturais. (Talvez Bohm-Bawerk, Menger, com certeza não Marshall!). É por isso que os jovens talentosos economistas em potencial não podem realmente olhar para ninguém, ou ter uma sensação para o espírito da ciência. Há uma atmosfera completamente diferente na economia em relação a esses outros círculos. Os economistas não parecem saber disso; caso contrário, eles mostrariam muito mais humildade, o que está completamente ausente em seu caráter”.

As reflexões de Morgenstern sobre Keynes são implacavelmente negativas e contínuas com suas críticas anteriores sobre a falta de precisão de Keynes no tratamento das expectativas. Ele sentia que uma das razões pela qual os alunos eram atraídos por Keynes, assim como por John Hicks, era porque a crítica dos Institucionalistas ou de Clark não estava no mesmo nível intelectual. Uma carta de Haberler provocou a reflexão de que Keynes era "um dos maiores charlatães que já apareceu no cenário econômico. E todos estão prostrados diante dele... Alguém deveria realmente enfrentar Keynes um dia desses, ele é brilhante, muito inteligente, e seus oponentes não são páreo para ele". Vários meses depois, ele estava lendo mais sobre Keynes: "uma estranha mistura de bom conhecimento e pensamento pseudocientífico, mas inteligente em seu modo de expressão e, portanto (por essas três razões), tão perigoso". Em setembro de 1943, ele recebeu uma carta de Hayek, que "odiava a ciência como sempre", e alegou ter ouvido rumores curiosos sobre o livro. "Ele vai achar ainda mais 'curioso' quando o ver... Ele está em um beco sem saída. A Teoria Pura do Capital não vale a pena ler". Um ano depois, ele leu O Caminho para a Servidão, para o qual ele conseguiu dar um elogio relutante: "Não vale muito. Só é maravilhoso em seu amor pela liberdade. Ele deveria ver o que é dito em nosso livro sobre simetria e justiça! Não há nada profundo nisso. Naturalmente, eu também não gosto de planejamento, mas a situação intelectual é muito mais complicada".

Ele agora sentia que o departamento de economia de Princeton era ainda mais provinciano e insatisfatório: "Há apenas 4 ou 5 cursos de pós-graduação, ainda sem seminários, sem discussões. Os alunos também não gostam. [X] é totalmente inadequado: os matemáticos têm colóquios semanais, assim como os físicos, psicólogos e químicos. Precisamos de alguém com um verdadeiro espírito científico... algo precisa acontecer". No Instituto, ele assistiu a palestras e seminários em uma variedade de assuntos, sendo o mais recente neste momento sobre probabilidade pelo estatístico de Princeton, Sam Wilks, que disse ter pouco estima pelo que encontrou em "estatísticas e economia matemática à la Econometrica".

Em março de 1943, pouco antes de o livro ir para a gráfica, Bertrand Russell visitou Princeton, para alegria de Morgenstern, mas também para sua decepção. Russell fez comentários sobre política e cultura, que Morgenstern considerou imprecisos e mal defendidos. Por exemplo, ele falou sobre o "ódio inalterável e profundamente enraizado à Inglaterra entre o povo comum" dos Estados Unidos, o que Morgenstern considerou completamente exagerado. Também ficou claro que Russell havia perdido um pouco o contato com a matemática, e Morgenstern sentiu que ele foi superado, como personalidade, por Weyl, Johnny e Siegel, e, como lógico, por Gödel. Em outubro, Morgenstern recusou um convite da Sociedade Filosófica Americana para falar sobre a influência que a filosofia teve sobre ele: "Tenho medo de que seria muito trabalho, e não seria divertido para eles se eu dissesse abertamente o que penso agora...". O ano de 1943 terminou com um jantar com Russell, os von Neumanns e Siegel na Taverna Nassau, onde a conversa foi "interessante e difícil", e Russell estimulante, como sempre, mas desatualizado. "Ele disse que a mecânica quântica é inexata, ao que Johnny respondeu imediatamente, e viu-se que R. não estava informado. Ele também não está familiarizado com a matemática mais recente... [No entanto] Tenho muito a agradecê-lo, e eu lhe disse isso. Só gostaria de ter lido mais dele antes. E isso me traz de volta ao velho tema; tempo desperdiçado com 'filosofia'". Uma conversa posterior com Herman Weyl voltou-se para Kant, pelo qual o matemático, em sua juventude, havia sido impressionado, até descobrir a geometria de Hilbert. "Isso não só o trouxe para a matemática, mas também limpou toda a filosofia como uma pseudociência de seu caminho".

\subsubsection{\textbf{Recepção}}
A teoria dos jogos de Von Neumann reuniu considerações empíricas, matemáticas e estéticas de uma maneira bastante diferente de tudo que se viu na teoria social antes. Isso deixou alguns perplexos e irritou outros. Em uma resenha inicial do livro para o American Journal of Sociology, e depois em uma carta para Morgenstern, Herbert Simon expressou dúvidas sobre o conceito de estabilidade. O conjunto de soluções poderia ser reduzido, ele sugeriu, aceitando que "um jogador não ajudará a manter uma imputação se ele receber a menor parte dessa imputação". Se essa restrição era aceitável ou não, ele disse, era, em última análise, uma questão empírica: "Não tenho certeza de que entendi a posição do Professor von Neumann sobre este ponto, mas fiquei com a impressão clara da discussão com ele que sua preferência pela definição que você usou foi baseada em grande parte em fundamentos estéticos e formais. Sendo um cientista social em vez de um matemático, não estou disposto a permitir que a teoria formal conduza os fatos a esse ponto – embora reconheça que métodos semelhantes tenham sido muito frutíferos tanto na teoria da relatividade quanto na mecânica quântica" (ibid).

Em 1951, em Nova York, em um encontro interdisciplinar para discutir mecanismos de causa circular e feedback em sistemas biológicos e sociais, na Oitava Conferência de "Cibernética", o estatístico Leonard "Jimmie" Savage enfatizou essa característica da teoria dos jogos. Ele havia sido assistente de Von Neumann no Instituto em 1941, enquanto o livro estava sendo escrito.

"Até onde pude entender, a teoria de Von Neumann simplesmente não faz previsões testáveis sobre jogos envolvendo muitas pessoas. Embora uma grande quantidade de maquinário matemático seja construída nesse contexto, Von Neumann, nem por escrito nem em conversa, parece-me deixar claro quais consequências empíricas esse maquinário pode sugerir. A situação é totalmente diferente da dos jogos de duas pessoas, sobre os quais os escritos de Von Neumann sugerem consequências bastante definidas. Assim, embora um experimentador como Bavelas possa e realmente teste as consequências da teoria de soma zero para duas pessoas, acho que ele nem saberia o que procurar no caso de jogos envolvendo muitas pessoas."

Outras opiniões foram mistas. E. J. Gumbel, da New School for Social Research, sentiu que havia um longo caminho a percorrer antes que a teoria pudesse fornecer "resultados tangíveis para a solução racional de problemas econômicos" (1945, p. 210). Ele duvidava se o método, que se baseava em uma forma capitalista de produção, poderia cobrir "toda a economia racional", e, como muitos leitores das ciências sociais, ficou desencorajado por "páginas e páginas cheias de fórmulas". Escrevendo em Science and Society, Louis Weisner, do Hunter College, curiosamente encontrou a visão dos autores "limitada por todos os lados pela doutrina da utilidade marginal" (1945, p. 368). Ele disse que "eles excluem explicitamente a economia de uma sociedade comunista de seu programa porque sua teoria exige que o combate e a competição prevaleçam na distribuição do produto social". Por causa disso, eles ficaram em silêncio sobre os problemas da produção econômica, uma reticência que "não era peculiar aos autores, mas à escola de economia da utilidade marginal à qual eles pertenciam". Sua teoria não tinha nada a dizer sobre por que o mercantilismo deu lugar ao capitalismo industrial e por que este último deu lugar ao imperialismo. Para tais questões, sua teoria, "sendo estática e não histórica, não pode fornecer respostas de forma alguma".

Escrevendo no View, the Modern Magazine, ao lado de Andre Breton, Man Ray e Meyer Schapiro, o algebrista de Princeton, Claude Chevalley, celebrou o aparecimento de teorias de conjuntos e grupos na ciência social. Ele disse que as aplicações de matemática até então tinham sido mal sucedidas porque, com muita frequência, tentou-se seguir "o padrão indicado pelas teorias mecânicas ou físicas, onde o foco são as equações diferenciais que expressam o futuro imediato de um sistema em termos de sua condição atual" (1945, p. 43). Chevalley deu uma apresentação verbal sucinta da teoria matemática, notando que a questão da existência de uma solução para todos os jogos com n jogadores permanecia "infelizmente aberta". Ele elogiou o livro, esperando que ele ajudasse a economia a "emergir de sua condição atual de vaguidão e confusão para o patamar de um conjunto de declarações precisas sobre situações precisamente definidas".

No Journal of Political Economy, Jacob Marschak, da Universidade de Chicago, escreveu uma extensa resenha, que também apareceu como um artigo da Cowles Commission. "Nem tudo está bem com a economia estática", ele começou. A primeira parte da resenha é dedicada a uma exposição magistral do jogo geral de três pessoas, com dois compradores e um vendedor. A solução que emergirá, e a imputação dentro dela, dependerá de padrões de comportamento, "códigos legais ou morais" (1946, p. 104). No jogo de soma zero de duas pessoas, a simetria assumida de inteligência dos jogadores é contrastada com a análise de Keynes no Capítulo 10 da Teoria Geral, que descreve a importância para o mercado de ações daquela "terceira etapa onde dedicamos nossa inteligência para antecipar o que a opinião média espera que a opinião média seja" (p. 106). A principal conquista da teoria de von Neumann, Marschak sentiu, não estava tanto nos seus resultados concretos, mas em ter "introduzido na economia as ferramentas da lógica moderna e em usá-las com uma capacidade de generalização surpreendente" (p. 114). Ele sublinhou o procedimento pelo qual conceitos empíricos gerados pela experiência foram formulados em conceitos teóricos que depois foram desvinculados da experiência até que as conclusões finais fossem alcançadas. Neste ponto, eles foram "materializados", transformados novamente na linguagem do campo concreto, preparados para teste empírico. Ele concordou com von Neumann que a economia matemática até então havia emprestado ilegitimamente suas ferramentas da física antiga, "sem mais escrutínio das bases lógicas do que era então usual na própria física" (p. 115). Não surpreendentemente, no entanto, Marschak sentiu que os autores subestimaram o valor dos trabalhos existentes em economia matemática, especialmente quando comparados com as "proposições vagas da economia pré-matemática" (ibid). Ele encerrou com altos elogios ao livro pela "simplicidade, clareza e paciência", um "espetáculo de pensamento vigoroso". Ele concluiu: "Mais dez livros desses, e o progresso da economia está assegurado" (ibid).


Outro russo, bem treinado em matemática e estatística, estava então de licença da Iowa State com uma Bolsa Guggenheim na Cowles Commission, onde escreveu uma resenha do livro. Após uma análise sistemática, Leonid Hurwicz também expressou consternação, ainda maior que a de Marschak, pelos ataques indiscriminados de von Neumann e Morgenstern às técnicas atuais usadas em economia. A disciplina, disse ele, não podia se dar ao luxo de se desenvolver da maneira mais "lógica" quando resultados eram necessários para lidar com flutuações de emprego e similares. Nem sempre era necessário uma prova matemática para saber que um cartel era provável, nem estava claro que os resultados finais da teoria dos jogos seriam tão diferentes dos obtidos pelos métodos atuais a ponto de justificar a dureza do capítulo inicial. Essas "críticas vagas", ele sentiu, dificilmente eram "dignas das realizações construtivas do resto do livro" – o que deve ter magoado Morgenstern. Ele também notou a curiosa ausência de referência a escritores anteriores: poderia-se concluir que apenas Böhm-Bawerk e Pareto eram conhecidos pelos autores. Não obstante, o trabalho foi uma conquista espetacular, "notavelmente lúcido e fascinante", um "evento raro" (1945, p. 925).

Escrevendo no *Economic Journal*, Richard Stone, de Cambridge, demonstrou sensibilidade à amplitude da obra, observando que ela interessaria não apenas a matemáticos, economistas e estudantes de teoria dos jogos em si, mas também a sociólogos, teóricos políticos e estudantes de estratégia diplomática e militar. Carl Kaysen, de Harvard, escreveu uma resenha detalhada e impressionante na *Review of Economic Studies*. Em suas palavras, é um "ato de alguma temeridade para um mero economista julgar uma grande obra na qual um escritor que é um brilhante lógico, matemático e físico, com a ajuda de um economista, oferece a todos os economistas um novo fundamento para seu pensamento, um fundamento que é ao mesmo tempo vasto e finamente elaborado, feito com trabalho infinito" (1946-47, p. 12). No entanto, ele considerou os axiomas de utilidade como um ponto fraco. Enquanto von Neumann estava satisfeito em ignorar essa questão com as observações de que era impossível construir um sistema consistente de axiomas que permitisse uma utilidade para o jogo, e que pelo menos uma axiomatização de um domínio psicológico havia sido alcançada, Kaysen enfatizou o fato de que as pessoas compravam bilhetes de loteria, adquiriam apólices de seguro e atuavam como empreendedores. Kaysen observou que, se von Neumann e Morgenstern acharam "impossível formular o conceito de utilidade do jogo de forma consistente, tanto pior para seus sistemas de postulados" (p. 13). A principal dificuldade, no entanto, não estava na utilidade, mas na indeterminação do conceito de solução, que muitas vezes permitia uma gama infinita de valores a serem alcançados por um único jogador. Para ter "valor prático em casos realmente complexos", a teoria teria que dizer algo sobre quais forças estariam operacionais em uma determinada situação, quais coalizões seriam formadas, quais pagamentos de compensação seriam feitos. Por enquanto, Kaysen concluiu que a teoria dos jogos não prometia uma revolução na economia.

\subsubsection{\textbf{Conclusão}}
No curto prazo, Kaysen provou estar bastante certo. Embora o livro tenha encontrado leitores isolados e minuciosos como Gerard Debreu, Herbert Simon e John Harsanyi, e embora o tratamento de von Neumann da utilidade esperada tenha despertado o interesse de economistas-estatísticos como Friedman e Savage, a obra como um todo não foi amplamente aceita pela comunidade de economistas. Como previsto por von Neumann e anunciado em algumas das resenhas, o aparato matemático do livro, a dificuldade de acesso e a pura estranheza do assunto combinaram-se para torná-lo árido aos olhos de muitos economistas leitores. O que, portanto, aconteceu com ele? Se os economistas acadêmicos não se importavam com a teoria dos jogos no período imediatamente após a guerra, quem se importava?

Para responder a essa pergunta, precisamos nos afastar novamente da economia e considerar o que estava acontecendo durante a Segunda Guerra Mundial, quando von Neumann e Morgenstern estavam, embora cada vez mais esporadicamente, escrevendo juntos. Em resumo, a guerra viu o desenvolvimento da pesquisa operacional e, com isso, a aplicação de uma pequena e relativamente insignificante parte da teoria dos jogos à análise de engajamentos militares. Foi aí que a teoria dos jogos fez sua entrada no mundo, por assim dizer. Essa experiência de guerra, por sua vez, moldou o período pós-guerra, com a promoção contínua da pesquisa operacional e, com o tempo, o desenvolvimento da ciência social da Guerra Fria. Nesse mundo isolado, centrado na RAND Corporation, a teoria dos jogos se tornou um elemento constitutivo, naturalmente saindo das mãos de von Neumann no processo. Ao longo da guerra e depois dela, seja em aplicações de campo de batalha ou na autoanálise social do laboratório da Guerra Fria, as demandas feitas à teoria dos jogos eram novas. Em sua aplicação, seja em duelos de bombardeiros e caças ou em jogos experimentais, houve um colapso distinto de escopo, deixando a teoria muito distante do amplo tema de configuração e reconfiguração social que originalmente inspirou von Neumann. No meio dos acidentes da história, seu objetivo de criar uma matemática abstrata de explicação social ambiciosa foi silenciosamente esquecido.
\end{document}