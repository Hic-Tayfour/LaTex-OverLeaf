\documentclass[a4paper,12pt]{article}[abntex2]
\bibliographystyle{abntex2-alf}
\usepackage{siunitx} % Fornece suporte para a tipografia de unidades do Sistema Internacional e formatação de números
\usepackage{booktabs} % Melhora a qualidade das tabelas
\usepackage{tabularx} % Permite tabelas com larguras de colunas ajustáveis
\usepackage{graphicx} % Suporte para inclusão de imagens
\usepackage{newtxtext} % Substitui a fonte padrão pela Times Roman
\usepackage{ragged2e} % Justificação de texto melhorada
\usepackage{setspace} % Controle do espaçamento entre linhas
\usepackage[a4paper, left=3.0cm, top=3.0cm, bottom=2.0cm, right=2.0cm]{geometry} % Personalização das margens do documento
\usepackage{lipsum} % Geração de texto dummy 'Lorem Ipsum'
\usepackage{fancyhdr} % Customização de cabeçalhos e rodapés
\usepackage{titlesec} % Personalização dos títulos de seções
\usepackage[portuguese]{babel} % Adaptação para o português (nomes e hifenização
\usepackage{hyperref} % Suporte a hiperlinks
\usepackage{indentfirst} % Indentação do primeiro parágrafo das seções
\sisetup{
  output-decimal-marker = {,},
  inter-unit-product = \ensuremath{{}\cdot{}},
  per-mode = symbol
}
\DeclareSIUnit{\real}{R\$}
\newcommand{\real}[1]{R\$#1}
\setlength{\headheight}{14.49998pt}
\usepackage{float} % Melhor controle sobre o posicionamento de figuras e tabelas
\usepackage{footnotehyper} % Notas de rodapé clicáveis em combinação com hyperref
\hypersetup{
    colorlinks=true,
    linkcolor=black,
    filecolor=magenta,      
    urlcolor=cyan,
    citecolor=black,        
    pdfborder={0 0 0},
}
\usepackage[normalem]{ulem} % Permite o uso de diferentes tipos de sublinhados sem alterar o \emph{}
\makeatletter
\def\@pdfborder{0 0 0} % Remove a borda dos links
\def\@pdfborderstyle{/S/U/W 1} % Estilo da borda dos links
\makeatother
\onehalfspacing

\begin{document}

\begin{titlepage}
    \centering
    \vspace*{1cm}
    \Large\textbf{INSPER – INSTITUTO DE ENSINO E PESQUISA}\\
    \Large ECONOMIA\\
    \vspace{1.5cm}
    \Large\textbf{Tradução Tópico 9.2 - HPE}\\
    \vspace{1.5cm}
    Prof. Pedro Duarte\\
    Prof. Auxiliar Guilherme Mazer\\
    \vfill
    \normalsize
    Hicham Munir Tayfour, \href{mailto:hichamt@al.insper.edu.br}{hichamt@al.insper.edu.br}\\
    4º Período - Economia B\\
    \vfill
    São Paulo\\
    Maio/2024
\end{titlepage}

\newpage
\tableofcontents
\thispagestyle{empty} % This command removes the page number from the table of contents page
\newpage
\setcounter{page}{1} % This command sets the page number to start from this page
\justify
\onehalfspacing

\pagestyle{fancy}
\fancyhf{}
\rhead{\thepage}

\section{\textbf{Lucas e Sargent (1979)}}
\subsection{\texbf{Após a Macroeconomia Keynesiana}}

Para o economista aplicado, a aplicação confiante e aparentemente bem-sucedida dos princípios keynesianos à política econômica que ocorreu nos Estados Unidos na década de 1960 foi um evento de incomparável significado e satisfação. Esses princípios levaram a um conjunto de relações quantitativas simples entre a política fiscal e a atividade econômica em geral, cuja lógica básica poderia ser (e foi) explicada ao público em geral e que poderia ser aplicada para gerar melhorias no desempenho econômico beneficiando a todos. Parecia uma economia tão livre de dificuldades ideológicas quanto, digamos, a química ou a física aplicada, prometendo uma expansão direta nas possibilidades econômicas. Poder-se-ia argumentar sobre como essa bonança deveria ser distribuída, mas parecia uma simples falha de lógica se opor à própria bonança. Compreensivelmente e corretamente, os não economistas receberam essa promessa com ceticismo no início; a prosperidade crescente e suave dos anos Kennedy-Johnson fez muito para diminuir essas dúvidas.

Nós nos demoramos nesses dias idílicos da economia keynesiana porque, sem esforço consciente, eles são difíceis de lembrar hoje. Na década atual, a economia dos EUA passou por sua primeira grande depressão desde a década de 1930, acompanhada de taxas de inflação superiores a 10\% ao ano. Esses eventos foram transmitidos (com o consentimento dos governos envolvidos) para outros países avançados e, em muitos casos, foram amplificados. Esses eventos não surgiram de uma reversão reacionária para princípios "clássicos" ultrapassados, de dinheiro apertado e orçamentos equilibrados. Pelo contrário, eles foram acompanhados por enormes déficits orçamentários do governo e altas taxas de expansão monetária, políticas que, embora admitindo um risco admitido de inflação, prometiam de acordo com a doutrina keynesiana moderna um rápido crescimento real e baixas taxas de desemprego.

Que essas previsões estavam terrivelmente incorretas e que a doutrina na qual se baseavam é fundamentalmente falha são agora simples questões de fato que não envolvem novidades na teoria econômica. A tarefa agora enfrentada pelos estudantes contemporâneos do ciclo de negócios é vasculhar os destroços, determinando quais características desse notável evento intelectual chamado Revolução Keynesiana podem ser resgatadas e bem utilizadas e quais outras devem ser descartadas. Embora esteja longe de ser claro qual será o resultado desse processo, já é evidente que necessariamente envolverá a reabertura de questões básicas em economia monetária que têm sido vistas desde os anos trinta como "fechadas" e a reavaliação de todos os aspectos do arcabouço institucional dentro do qual a política monetária e fiscal é formulada nos países avançados.

Este artigo é um relatório de progresso inicial sobre esse processo de reavaliação e reconstrução. Começamos revisando a estrutura econométrica por meio da qual a teoria keynesiana evoluiu de uma conversa desconectada e qualitativa sobre atividade econômica para um sistema de equações que pode ser comparado aos dados de maneira sistemática e que fornece um guia operacional na tarefa necessariamente quantitativa de formular política monetária e fiscal. Em seguida, identificamos aqueles aspectos dessa estrutura que foram centrais para seu fracasso nos anos setenta. Ao fazer isso, nossa intenção é estabelecer que as dificuldades são fatais: que os modelos macroeconômicos modernos não têm valor em orientar a política e que essa condição não será remediada por modificações ao longo de qualquer linha que esteja sendo atualmente perseguida. Este diagnóstico sugere certos princípios que uma teoria útil dos ciclos de negócios deve ter. Concluímos revisando algumas pesquisas recentes consistentes com esses princípios.

\subsubsection{\textbf{Modelos Macroeconométricos}}

A Revolução Keynesiana foi, na forma em que teve sucesso nos Estados Unidos, uma revolução no método. Esse não era o objetivo de Keynes (1936)¹, nem é a visão de todos os seus seguidores mais eminentes. No entanto, se não se vê a revolução dessa maneira, é impossível explicar algumas de suas características mais importantes: a evolução da macroeconomia em uma disciplina científica quantitativa, o desenvolvimento de descrições estatísticas explícitas do comportamento econômico, a crescente dependência dos funcionários do governo na expertise econômica técnica, e a introdução do uso da teoria de controle matemático para gerenciar uma economia. É o fato de que a teoria keynesiana se prestou tão prontamente à formulação de modelos econométricos explícitos que explica a posição científica dominante que ela alcançou na década de 1960.

Por causa disso, nem o sucesso da Revolução Keynesiana nem seu eventual fracasso podem ser entendidos no nível puramente verbal em que o próprio Keynes escreveu. É necessário saber algo sobre a maneira como os modelos macroeconométricos são construídos e as características que eles devem ter para "funcionar" como auxílios na previsão e avaliação de políticas. Para discutir essas questões, introduzimos alguma notação.

Um modelo econométrico é um sistema de equações envolvendo um número de variáveis endógenas (variáveis determinadas pelo modelo), variáveis exógenas (variáveis que afetam o sistema, mas não são afetadas por ele) e choques estocásticos ou aleatórios. A ideia é usar dados históricos para estimar o modelo e, em seguida, utilizar a versão estimada para obter estimativas das consequências de políticas alternativas. Por razões práticas, é comum usar um modelo linear padrão, assumindo a forma estrutural.

\begin{equation}
    A_0 Y_t + A_1 Y_{t-1} + \cdots + A_m Y_{t-m} = B_0 X_t + B_1 X_{t-1} + \cdots + B_n X_{t-n} + \varepsilon_t
\end{equation}

\begin{equation}
    R_0 \varepsilon_t + R_1 \varepsilon_{t-1} + \cdots + R_r \varepsilon_{t-r} = U_t, \quad R_0 \equiv I
\end{equation}

Aqui, $yt$ é um vetor $(LX1)$ de variáveis endógenas, $xt$ é um vetor $(KX1)$ de variáveis exógenas, e $et$ e $ut$ são cada um vetores $(LX1)$ de perturbações aleatórias. As matrizes $Aj$ são cada uma $(LXL)$; os $Bj's$ são $(LXK)$, e os $Rj's$ são cada um $(LXL)$. O processo de perturbação $(LXL)$ $ut$ é suposto ser um processo não correlacionado serialmente com $Eut = 0$ e com matriz de covariância contemporânea $Eutu! — £$ e $Eutu's = 0$ para todos $t ≠ s$. As características definidoras das variáveis exógenas $xt$ é que elas são não correlacionadas com os $e's$ em todos os atrasos, de modo que $EutXs$ é uma matriz $(LXK)$ de zeros para todos $t$ e $s$.

As equações (1) são $L$ equações nos $L$ valores atuais $yt$ das variáveis endógenas. Cada uma dessas equações estruturais é uma relação comportamental, identidade ou condição de limpeza de mercado, e cada uma em princípio pode envolver um número de variáveis endógenas. As equações estruturais geralmente não são equações de regressão³ porque os $et's$ são, em geral, pela lógica do modelo, supostos estar correlacionados com mais de um componente do vetor $yt$ e muito possivelmente um ou mais componentes dos vetores $yt_ b • • • yt-nv$. O modelo estrutural (1) e (2) pode ser resolvido para $yt$ em termos de $y's$ passados e $x's$ e choques passados. Este sistema de forma reduzida é

\begin{equation}
    Y_t = -P_1 Y_{t-1} - \cdots - P_{r+m} Y_{t-r-m} + Q_0 X_t + \cdots + Q_{r+n} X_{t-n-r} + A_0^{-1} u_t
\end{equation}

no qual 

\[
P_s = A_0^{-1} \sum_{j=-\infty}^{\infty} R_j A_{s-j}
\]

\[
Q_s = A_0^{-1} \sum_{j=-\infty}^{\infty} R_j B_{s-j}
\]

As equações de forma reduzida são equações de regressão, ou seja, o vetor de perturbação $A~dut$ é ortogonal a $yt- i , ..., yt-r_m,xt , ..., Xt-n-r$. Isso segue das suposições de que os xs são exógenos e que os u's são não correlacionados serialmente. Portanto, sob condições gerais, a forma reduzida pode ser estimada de maneira consistente pelo método dos mínimos quadrados. Os parâmetros populacionais da forma reduzida (3) juntamente com os parâmetros de uma autorregressão vetorial para $xt$

\[
x_t = C_1 x_{t-1} + \cdots + C_p x_{t-p} + a_t
\]

Onde $Eat = 0$ e $Ea t' x t-j = 0$ para $j \geq l$ descrevem completamente todos os primeiros e segundos momentos do processo $(yt, xt)$. Dada uma série temporal suficientemente longa, boas estimativas dos parâmetros da forma reduzida - os $P/s$ e $Qj's$ - podem ser obtidas pelo método dos mínimos quadrados. Tudo que o exame dos dados por si só pode entregar são estimativas confiáveis desses parâmetros.

Geralmente não é possível trabalhar de trás para frente a partir das estimativas dos $F s$ e $Q's$ sozinhos para derivar estimativas únicas dos parâmetros estruturais, os $A/s$, $B/s$ e $R/s$. Em geral, números infinitos de $As$, $B's$ e $R's$ são compatíveis com um único conjunto de $Fs$ e $Q's$. Este é o problema de identificação da econometria. Para derivar um conjunto de parâmetros estruturais estimados, é necessário saber muito sobre eles com antecedência. Se informações prévias suficientes forem impostas, é possível extrair estimativas dos $A/s$, $B/s$, $R/s$ implicados pelos dados em combinação com as informações prévias.

Para fins de previsão ex ante, ou a previsão incondicional do vetor $yt+1$, $yt+2$, ... dada a observação de $ys$ e $xs$, $s \leq t$, a forma reduzida estimada (3), juntamente com (4), é suficiente. Isso é simplesmente um exercício em um tipo sofisticado de extrapolação, que não requer compreensão dos parâmetros estruturais, ou seja, a economia do modelo.

Para fins de previsão condicional, ou a previsão do comportamento futuro de alguns componentes de $yt$ e $xt$ condicionais a valores particulares de outros componentes, selecionados pela política, é necessário conhecer os parâmetros estruturais. Isso ocorre porque uma mudança na política necessariamente altera alguns dos parâmetros estruturais (por exemplo, aqueles que descrevem o comportamento passado das variáveis de política em si) e, portanto, afeta os parâmetros da forma reduzida de uma maneira altamente complexa (veja as equações que definem $Ps$ e $Os$ acima). A menos que se saiba quais parâmetros estruturais permanecem invariantes à medida que a política muda e quais mudam (e como), um modelo econométrico não tem valor na avaliação de políticas alternativas. Deve ficar claro que isso é verdade independentemente de quão bem (3) e (4) se ajustam aos dados históricos ou de quão bem eles se saem na previsão incondicional.

Nossa discussão até este ponto foi altamente geral, e as considerações formais que revisamos não são de forma alguma específicas para modelos keynesianos. O problema de identificar um modelo estrutural a partir de uma coleção de séries temporais econômicas é um que deve ser resolvido por qualquer pessoa que afirme ter a capacidade de dar conselhos econômicos quantitativos. Os modelos keynesianos mais simples são tentativas de solução para este problema, assim como as versões em grande escala atualmente em uso. Assim também são os modelos monetaristas que implicam a desejabilidade de regras de crescimento monetário fixo. Assim, para esse assunto, é o conselho de poltrona dado por economistas que afirmam estar fora da tradição econométrica, embora neste caso a estrutura subjacente implícita não esteja exposta à crítica profissional. Qualquer procedimento que leve do estudo do comportamento econômico observado à avaliação quantitativa de políticas econômicas alternativas envolve as etapas, executadas mal ou bem, explicitamente ou implicitamente, que delineamos.

\subsubsection{\textbf{Macroeconometria Keynesiana}}


Em modelos macroeconométricos keynesianos, os parâmetros estruturais são identificados pela imposição de vários tipos de restrições a priori nos $Aj's$, $B/s$ e $R/s$. Essas restrições geralmente se enquadram em uma das três categorias a seguir:

(a) Definição a priori de muitos dos elementos dos $Aj's$ e $Bj's$ para zero.

(b) Restrições sobre as ordens de correlação serial e a extensão da correlação serial cruzada do vetor de perturbação $et$, restrições que equivalem a definir a priori muitos elementos dos $Rj's$ para zero.

(c) Classificação a priori de variáveis como exógenas e endógenas. Uma abundância relativa de variáveis exógenas auxilia na identificação.

Os grandes modelos macroeconométricos keynesianos existentes estão abertos a sérios desafios pela maneira como introduziram cada tipo de restrição.

A Teoria Geral de Keynes era rica em sugestões para restrições do tipo (a). Nele, ele propôs uma teoria de determinação da renda nacional construída a partir de várias relações simples, cada uma envolvendo apenas algumas variáveis. Uma delas, por exemplo, era a "lei fundamental" que relacionava os gastos com consumo à renda. Isso sugeriu uma "linha" nas equações (1) envolvendo o consumo atual, a renda atual e nenhuma outra variável, impondo assim muitas restrições de zero nos $Aj's$ e $B-s$. Da mesma forma, a relação de preferência por liquidez expressava a demanda por dinheiro como uma função apenas da renda e de uma taxa de juros. Ao traduzir os blocos de construção do sistema teórico keynesiano em equações explícitas, foram construídos modelos da forma (1) e (2) com muitas restrições teóricas do tipo (a).

Restrições nos coeficientes $Rj$ que governam o comportamento dos termos de erro em (1) são mais difíceis de motivar teoricamente porque os erros são, por definição, movimentos nas variáveis que a teoria econômica não pode explicar. Os primeiros econométricos adotaram suposições padrão de livros didáticos de estatística, restrições que se mostraram úteis na experimentação agrícola que forneceu o principal impulso para o desenvolvimento da estatística moderna. Novamente, essas restrições, bem motivadas ou não, envolvem a definição de muitos elementos nos $R-s$ como zero, auxiliando assim na identificação da estrutura do modelo.

A classificação de variáveis em exógenas e endógenas também foi feita com base em considerações prévias. Em geral, as variáveis eram classificadas como endógenas, que eram, como uma questão de fato institucional, determinadas em grande parte pelas ações de agentes privados (como consumo ou despesas de investimento privado). Variáveis exógenas eram aquelas sob controle governamental (como taxas de impostos ou a oferta de dinheiro). Esta divisão pretendia refletir os significados comuns das palavras endógenas - "determinadas pelo sistema [econômico]" - e exógenas - "afetando o sistema [econômico] mas não afetadas por ele".

Até meados da década de 1950, modelos econométricos haviam sido construídos que se ajustavam bem aos dados de séries temporais, no sentido de que suas formas reduzidas (3) acompanhavam de perto os dados passados e se mostraram úteis na previsão de curto prazo. Além disso, por meio de restrições dos três tipos revisados acima, seus parâmetros estruturais $A-v Bj, Rk$ poderiam ser identificados. Usando essa estrutura estimada, os modelos poderiam ser simulados para obter estimativas das consequências de diferentes políticas econômicas governamentais, como taxas de impostos, despesas ou política monetária.

Esta solução keynesiana para o problema de identificar um modelo estrutural tornou-se cada vez mais suspeita como resultado de desenvolvimentos teóricos e estatísticos. Muitos desses desenvolvimentos são devidos aos esforços de pesquisadores simpáticos à tradição keynesiana, e muitos foram avançados bem antes do fracasso espetacular dos modelos keynesianos na década de 1970.

Desde o seu início, a macroeconomia tem sido criticada por sua falta de fundamentos na teoria microeconômica e de equilíbrio geral. Como foi reconhecido desde cedo por comentaristas astutos como Leontief (1965, desaprovando) e Tobin (1965, aprovando), a criação de um ramo distinto da teoria com seus próprios postulados distintos foi o objetivo consciente de Keynes. No entanto, um tema principal do trabalho teórico desde a Teoria Geral tem sido a tentativa de usar a teoria microeconômica baseada no postulado clássico de que os agentes agem em seus próprios interesses para sugerir uma lista de variáveis que pertencem ao lado direito de uma determinada programação comportamental, digamos, uma programação de demanda por um fator de produção ou uma programação de consumo. Mas, do ponto de vista da identificação de uma determinada equação estrutural por meio de restrições do tipo (a), é necessário ter informações prévias confiáveis de que certas variáveis devem ser excluídas do lado direito. A teoria microeconômica probabilística moderna quase nunca implica as restrições de exclusão sugeridas por Keynes ou aquelas impostas por modelos macroeconométricos.

Vamos considerar um exemplo com implicações extremamente graves para a identificação dos modelos macro existentes. As expectativas sobre os preços futuros, as taxas de impostos e os níveis de renda desempenham um papel crítico em muitas programações de demanda e oferta. Nos melhores modelos, por exemplo, a demanda por investimento normalmente deve responder às expectativas das empresas sobre futuros créditos fiscais, taxas de impostos e custos de fatores, e a oferta de trabalho normalmente deve depender da taxa de inflação que os trabalhadores esperam no futuro. Tais equações estruturais são geralmente identificadas pela suposição de que a expectativa sobre, digamos, preços de fatores ou a taxa de inflação atribuída aos agentes é uma função apenas de alguns valores defasados da variável que o agente supostamente está prevendo. No entanto, os próprios modelos macro contêm interações dinâmicas complicadas entre variáveis endógenas, incluindo preços de fatores e a taxa de inflação, e eles geralmente implicam que um agente sábio usaria valores atuais e muitos valores defasados de muitas e geralmente a maioria das variáveis endógenas e exógenas no modelo para formar expectativas sobre qualquer variável. Assim, praticamente qualquer versão da hipótese de que os agentes agem em seus próprios interesses contradirá as restrições de identificação impostas à formação de expectativas. Além disso, as restrições sobre as expectativas que foram usadas para alcançar a identificação são totalmente arbitrárias e não foram derivadas de nenhuma suposição mais profunda que reflita os primeiros princípios sobre o comportamento econômico. Nenhum primeiro princípio geral jamais foi estabelecido que implicaria que, digamos, a taxa de inflação esperada deve ser modelada como uma função linear de taxas de inflação defasadas sozinhas com pesos que somam a unidade, mas essa hipótese é usada como uma restrição de identificação em quase todos os modelos existentes. O tratamento casual das expectativas não é um problema periférico nesses modelos, pois o papel das expectativas é onipresente neles e exerce uma influência massiva em suas propriedades dinâmicas (um ponto que o próprio Keynes insistiu). A falha dos modelos existentes em derivar restrições sobre as expectativas de quaisquer primeiros princípios fundamentados na teoria econômica é um sintoma de uma falha mais profunda e mais geral em derivar relações comportamentais de quaisquer problemas de otimização dinâmica consistentemente colocados.

Quanto à segunda categoria, restrições do tipo (b), os modelos macro keynesianos existentes impõem severas restrições a priori nos $R/s$. Tipicamente, os $R/s$ são supostos serem diagonais de modo que a correlação serial defasada entre equações é ignorada, e também a ordem do processo $et$ é assumida como curta de modo que apenas a correlação serial de baixa ordem é permitida. Atualmente, não há fundamentos teóricos para introduzir essas restrições, e por boas razões, há pouca perspectiva de que a teoria econômica em breve fornecerá tais fundamentos. Em princípio, a identificação pode ser alcançada sem impor tais restrições. Abandonar o uso de restrições da categoria (b) aumentaria as restrições necessárias das categorias (a) e (c). De qualquer forma, os modelos macro existentes restringem fortemente os $R/s$.

Passando para a terceira categoria, todos os grandes modelos existentes adotam uma classificação a priori de variáveis como variáveis estritamente endógenas, os $yt's$, ou variáveis estritamente exógenas, os $xt's$. Cada vez mais, está sendo reconhecido que a classificação de uma variável como exógena com base na observação de que ela poderia ser definida sem referência aos valores atuais e passados de outras variáveis não tem nada a ver com a questão economicamente relevante de como essa variável de fato se relacionou com outras ao longo de um determinado período histórico. Além disso, à luz dos recentes desenvolvimentos em econometria de séries temporais, sabemos que esse procedimento de classificação arbitrário não é necessário. Christopher Sims (1972) mostrou que, em um contexto de séries temporais, a hipótese de exogeneidade econométrica pode ser testada. Ou seja, Sims mostrou que a hipótese de que $xt$ é estritamente exógeno econométrico em (1) necessariamente implica certas restrições que podem ser testadas dadas séries temporais nos $y's$ e $x's$. Testes ao longo das linhas de Sims deveriam ser usados rotineiramente para verificar classificações em conjuntos de variáveis exógenas e endógenas. Até o momento, eles não foram. Construtores proeminentes de grandes modelos econométricos até negaram a utilidade de tais testes. (Veja, por exemplo, Ando 1977, pp. 209-10, e L. R. Klein em Okun e Perry 1973, p. 644.)

\subsubsection{\textbf{Falha da Macroeconometria Keynesiana}}

Portanto, existem várias razões teóricas para acreditar que os parâmetros identificados como estruturais pelos métodos macroeconômicos atuais não são, de fato, estruturais. Ou seja, não vemos motivo para acreditar que esses modelos tenham estruturas isoladas que permanecerão invariantes em toda a classe de intervenções que figuram nas discussões contemporâneas de política econômica. No entanto, a questão de saber se um modelo específico é estrutural é uma questão empírica, não teórica. Se os modelos macroeconométricos tivessem compilado um registro de estabilidade de parâmetros, particularmente diante de quebras no comportamento estocástico das variáveis exógenas e perturbações, seria cético quanto à importância das objeções teóricas prévias do tipo que levantamos.

Na verdade, no entanto, o histórico dos principais modelos econométricos é, em qualquer dimensão que não seja a previsão incondicional de muito curto prazo, muito pobre. Testes estatísticos formais para instabilidade de parâmetros, conduzidos subdividindo séries passadas em períodos e verificando a estabilidade de parâmetros ao longo do tempo, invariavelmente revelam grandes mudanças. (Para um exemplo, veja Muench et. al. 1974.) Além disso, essa dificuldade é implicitamente reconhecida pelos próprios construtores de modelos, que rotineiramente empregam um sistema elaborado de fatores adicionais na previsão, na tentativa de compensar a deriva contínua do modelo para longe da série real.

Embora não, é claro, projetados como tal por ninguém, os modelos macroeconométricos foram submetidos a um teste decisivo na década de 1970. Um elemento-chave em todos os modelos keynesianos é um trade-off entre inflação e produção real: quanto maior a taxa de inflação, maior é a produção (ou, equivalentemente, menor é a taxa de desemprego). Por exemplo, os modelos do final da década de 1960 previam uma taxa de desemprego nos EUA sustentada de 4 por cento como consistente com uma taxa anual de inflação de 4 por cento. Com base nessa previsão, muitos economistas naquela época defendiam uma política deliberada de inflação. Certamente o caráter errático de "acertos e erros" da política dos EUA na década de 1970 não pode ser atribuído a recomendações baseadas em modelos keynesianos, mas o viés inflacionário em média da política monetária e fiscal neste período deveria, de acordo com todos esses modelos, ter produzido as menores taxas médias de desemprego para qualquer década desde a década de 1940. Na verdade, como sabemos, eles produziram as maiores taxas de desemprego desde a década de 1930. Isso foi uma falha econométrica em grande escala.

Esse fracasso não levou a conversões generalizadas de economistas keynesianos para outras crenças, nem deveria ter sido esperado. Em economia, como em outras ciências, uma estrutura teórica é sempre mais ampla e flexível do que qualquer conjunto particular de equações, e sempre há a esperança de que, se um modelo específico falhar, pode-se encontrar um modelo mais bem-sucedido baseado em ideias aproximadamente as mesmas. No entanto, o fracasso já teve algumas consequências importantes, com implicações sérias tanto para a formulação de políticas econômicas quanto para a prática da ciência econômica.

Para a política, o fato central é que as recomendações de política keynesiana não têm base mais sólida, em um sentido científico, do que as recomendações de economistas não keynesianos ou, por falar nisso, não economistas. Para notar uma consequência do amplo reconhecimento disso, a atual onda de sentimento protecionista direcionada a "salvar empregos" teria sido respondida dez anos atrás com o contra-argumento keynesiano de que a política fiscal pode alcançar o mesmo fim, mas de forma mais eficiente. Hoje, é claro, ninguém levaria essa resposta a sério, então ela não é oferecida. De fato, economistas que há dez anos defendiam a política fiscal keynesiana como uma alternativa aos controles diretos ineficientes favorecem cada vez mais tais controles como suplementos à política keynesiana. A ideia parece ser que, se as pessoas se recusam a obedecer às equações que ajustamos ao seu comportamento passado, podemos aprovar leis para fazê-las agir assim.

Cientificamente, o fracasso keynesiano da década de 1970 resultou em uma nova abertura. Cada vez menos economistas estão envolvidos no monitoramento e refinamento dos principais modelos econométricos; cada vez mais estão desenvolvendo teorias alternativas do ciclo de negócios, baseadas em diferentes princípios teóricos. Além disso, mais atenção e respeito são concedidos às baixas teóricas da Revolução Keynesiana, às ideias dos contemporâneos de Keynes e dos economistas anteriores cujo pensamento foi considerado ultrapassado por anos.

Ninguém pode prever para onde esses desenvolvimentos levarão. Alguns, é claro, continuam a acreditar que os problemas dos modelos keynesianos existentes podem ser resolvidos dentro do quadro existente, que esses modelos podem ser adequadamente refinados mudando algumas equações estruturais, adicionando ou subtraindo algumas variáveis aqui e ali, ou talvez desagregando vários blocos de equações. Enfatizamos nossas críticas em termos tão gerais precisamente para enfatizar seu caráter genérico e, portanto, a futilidade de perseguir variações menores dentro deste quadro geral. Uma segunda resposta ao fracasso dos métodos analíticos keynesianos é renunciar completamente aos métodos analíticos, retornando aos métodos de julgamento.

A primeira dessas respostas identifica os objetivos quantitativos e científicos da Revolução Keynesiana com os detalhes dos modelos particulares desenvolvidos até agora. A segunda renuncia tanto a esses modelos quanto aos objetivos que foram projetados para atingir. Acreditamos que existe um curso intermediário, para o qual agora nos voltamos.

\subsubsection{\textbf{Teoria do Ciclo de Negócios de Equilíbrio}}

Antes da década de 1930, os economistas não reconheciam a necessidade de um ramo especial da economia, com seus próprios postulados especiais, projetado para explicar o ciclo de negócios. Keynes fundou essa subdisciplina, chamada macroeconomia, porque pensava que explicar as características dos ciclos de negócios era impossível dentro da disciplina imposta pela teoria econômica clássica, uma disciplina imposta por sua insistência na aderência aos dois postulados (a) de que os mercados se equilibram e (b) que os agentes agem em seu próprio interesse. Os fatos notáveis que pareciam impossíveis de conciliar com esses dois postulados eram a duração e a severidade das depressões econômicas e o desemprego em larga escala que elas acarretavam. Uma observação relacionada era que as medidas de demanda agregada e preços estavam positivamente correlacionadas com medidas de produção real e emprego, em aparente contradição com o resultado clássico de que mudanças em uma magnitude puramente nominal como o nível geral de preços eram mudanças puras de unidade que não deveriam alterar o comportamento real.

Depois de se libertar da camisa de força (ou disciplina) imposta pelos postulados clássicos, Keynes descreveu um modelo no qual regras práticas, como a função de consumo e a programação de preferência por liquidez, ocuparam o lugar das funções de decisão que um economista clássico insistiria ser derivado da teoria da escolha. E em vez de exigir que salários e preços sejam determinados pelo postulado de que os mercados se equilibram - o que para o mercado de trabalho parecia patentemente contraditado pela severidade das depressões econômicas - Keynes adotou como postulado não examinado que os salários monetários são pegajosos, o que significa que eles são definidos em um nível ou por um processo que poderia ser considerado como não influenciado pelas forças macroeconômicas que ele propôs analisar.

Quando Keynes escreveu, os termos equilíbrio e clássico carregavam certas conotações positivas e normativas que pareciam descartar qualquer modificador sendo aplicado à teoria do ciclo de negócios. O termo equilíbrio era pensado para se referir a um sistema em repouso, e alguns usavam tanto equilíbrio quanto clássico de forma intercambiável com ideal. Assim, uma economia em equilíbrio clássico seria tanto inalterada quanto inaprimorável por intervenções políticas. Com termos usados dessa maneira, não é de se admirar que poucos economistas considerassem a teoria do equilíbrio como um ponto de partida promissor para entender os ciclos de negócios e projetar políticas para mitigar ou eliminá-los.

Nos últimos anos, o significado do termo equilíbrio mudou tão dramaticamente que um teórico da década de 1930 não o reconheceria. Uma economia seguindo um processo estocástico multivariado é agora rotineiramente descrita como estando em equilíbrio, o que significa nada mais do que em cada ponto no tempo, os postulados (a) e (b) acima são satisfeitos. Este desenvolvimento, que se originou principalmente do trabalho de K. J. Arrow (1964) e G. Debreu (1959), implica que simplesmente olhar para qualquer série temporal econômica e concluir que é um fenômeno de desequilíbrio é uma observação sem sentido. De fato, uma conjectura mais provável, com base no trabalho recente de Hugo Sonnenschein (1973), é que a hipótese geral de que uma coleção de séries temporais descreve uma economia em equilíbrio competitivo é sem conteúdo.

A linha de pesquisa que alguns de nós estão seguindo envolve a tentativa de descobrir uma teoria particular do ciclo de negócios em equilíbrio, testável econometricamente, que pode servir como base para a análise quantitativa da política macroeconômica. Não há como negar que essa abordagem é contrarrevolucionária, pois pressupõe que Keynes e seus seguidores estavam errados em desistir da possibilidade de que uma teoria de equilíbrio pudesse explicar o ciclo de negócios. Até agora, nenhum modelo macroeconométrico de equilíbrio bem-sucedido no nível de detalhe de, digamos, o modelo Federal Reserve-MIT-Penn foi construído. Mas pequenos modelos teóricos de equilíbrio foram construídos que mostram potencial para explicar algumas características-chave do ciclo de negócios há muito consideradas inexplicáveis dentro dos limites dos postulados clássicos. Os modelos de equilíbrio também fornecem razões para entender por que os modelos keynesianos estimados falham em se manter fora da amostra sobre a qual foram estimados. Agora passamos a descrever alguns dos fatos-chave sobre os ciclos de negócios e a maneira como os novos modelos clássicos os confrontam.

Por muito tempo, a maior parte da profissão de economista, com alguma razão, seguiu Keynes na rejeição de modelos macroeconômicos clássicos porque eles pareciam incapazes de explicar algumas características importantes das séries temporais que medem agregados econômicos importantes. Talvez a falha mais importante do modelo clássico tenha sido sua aparente incapacidade de explicar a correlação positiva nas séries temporais entre preços e/ou salários, por um lado, e medidas de produção agregada ou emprego, por outro. Uma segunda falha relacionada foi sua incapacidade de explicar as correlações positivas entre medidas de demanda agregada, como o estoque de dinheiro, e produção agregada ou emprego. A análise estática de modelos macroeconômicos clássicos normalmente implicava que os níveis de produção e emprego eram determinados independentemente tanto do nível absoluto de preços quanto da demanda agregada. Mas a presença onipresente de correlações positivas nas séries temporais parece consistente com conexões causais fluindo da demanda agregada e inflação para produção e emprego, contrariamente às proposições de neutralidade clássica. Modelos macroeconométricos keynesianos implicam tais conexões causais.

Agora temos modelos teóricos rigorosos que ilustram como essas correlações podem surgir enquanto mantêm os postulados clássicos de que os mercados se equilibram e os agentes otimizam (Phelps 1970 e Lucas 1972, 1975). O passo-chave para obter tais modelos tem sido relaxar o postulado auxiliar usado em grande parte da análise econômica clássica de que os agentes têm informações perfeitas. Os novos modelos clássicos ainda assumem que os mercados se equilibram e que os agentes otimizam; os agentes tomam suas decisões de oferta e demanda com base em variáveis reais, incluindo preços relativos percebidos. No entanto, presume-se que cada agente tenha informações limitadas e receba informações sobre alguns preços com mais frequência do que outros preços. Com base em suas informações limitadas - as listas que eles têm dos preços absolutos atuais e passados de vários bens - presume-se que os agentes façam a melhor estimativa possível de todos os preços relativos que influenciam suas decisões de oferta e demanda.

Porque eles não têm todas as informações necessárias para calcular perfeitamente os preços relativos de que se importam, os agentes cometem erros na estimativa dos preços relativos pertinentes, erros que são inevitáveis dadas suas informações limitadas. Em particular, sob certas condições, os agentes tendem temporariamente a confundir um aumento geral em todos os preços absolutos como um aumento no preço relativo do bem que estão vendendo, levando-os a aumentar sua oferta desse bem além do que haviam planejado anteriormente. Como, em média, todos estão cometendo o mesmo erro, a produção agregada aumenta acima do que teria sido. Esse aumento da produção acima do que teria sido ocorre sempre que o nível de preços médio da economia neste período está acima do que os agentes esperavam que fosse com base em informações anteriores. Simetricamente, a produção agregada diminui sempre que o preço agregado acaba sendo menor do que os agentes esperavam. A hipótese das expectativas racionais está sendo imposta aqui: presume-se que os agentes façam o melhor uso possível das informações limitadas que têm e conheçam as distribuições de probabilidade objetivas pertinentes. Essa hipótese é imposta por meio da adesão aos princípios da teoria do equilíbrio.

Na nova teoria clássica, perturbações na demanda agregada levam a uma correlação positiva entre mudanças inesperadas no nível de preços agregados e revisões na produção agregada de seu nível planejado anteriormente. Além disso, é fácil mostrar que a teoria implica correlações entre revisões na produção agregada e mudanças inesperadas em quaisquer variáveis que ajudem a determinar a demanda agregada. Na maioria dos modelos macroeconômicos, a oferta de dinheiro é um determinante da demanda agregada. A nova teoria pode facilmente explicar correlações positivas entre revisões na produção agregada e aumentos inesperados na oferta de dinheiro.

Embora tal teoria preveja correlações positivas entre a taxa de inflação ou a oferta de dinheiro, por um lado, e o nível de produção, por outro, ela também afirma que essas correlações não retratam compensações que podem ser exploradas por uma autoridade de política. Ou seja, a teoria prevê que não há maneira de a autoridade monetária seguir uma política ativista sistemática e alcançar uma taxa de produção que seja, em média, maior ao longo do ciclo de negócios do que o que ocorreria se simplesmente adotasse uma regra de X por cento sem feedback, do tipo que Friedman (1948) e Simons (1936) recomendaram. Pois a teoria prevê que a produção agregada é uma função das mudanças inesperadas atuais e passadas na oferta de dinheiro. A produção será alta apenas quando a oferta de dinheiro é e tem sido maior do que se esperava que fosse, ou seja, maior do que a média. Simplesmente não há maneira de que, em média, ao longo de todo o ciclo de negócios, a oferta de dinheiro possa ser maior do que a média. Assim, embora a teoria possa explicar algumas das correlações há muito consideradas para invalidar a teoria macroeconômica clássica, ela é clássica tanto em sua aderência aos postulados teóricos clássicos quanto no sabor não ativista de suas implicações para a política monetária.

Modelos econométricos de pequena escala no sentido padrão foram construídos que capturam algumas das principais características da nova teoria clássica. (Veja, por exemplo, Sargent 1976a.) Em particular, esses modelos incorporam a hipótese de que as expectativas são racionais ou que os agentes usam todas as informações disponíveis. Em certo grau, esses modelos alcançam a identificação econométrica invocando restrições em cada uma das três categorias (a), (b) e (c). No entanto, uma característica distintiva desses modelos "clássicos" é que eles também dependem fortemente de uma importante quarta categoria de restrições de identificação. Esta categoria (d) consiste em um conjunto de restrições que são derivadas da teoria econômica probabilística, mas não desempenham nenhum papel no arcabouço keynesiano. Essas restrições, em geral, não assumem a forma de restrições de zero do tipo (a). Em vez disso, eles normalmente assumem a forma de restrições de equação cruzada entre os parâmetros $Aj, Bj, Cj$. A fonte dessas restrições é a implicação da teoria econômica de que as decisões atuais dependem das previsões dos agentes sobre variáveis futuras, combinadas com a implicação de que essas previsões são formadas de maneira ideal, dado o comportamento das variáveis passadas. As restrições não têm uma expressão matemática tão simples quanto simplesmente definir um número de parâmetros igual a zero, mas sua motivação econômica é fácil de entender. Maneiras de utilizar essas restrições na estimação e teste econométrico estão sendo rapidamente desenvolvidas.

Outra característica importante do trabalho recente em modelos macroeconométricos de equilíbrio é que a dependência de categorizações inteiramente a priori (c) de variáveis como estritamente exógenas e endógenas foi notavelmente reduzida, embora não totalmente eliminada. Este desenvolvimento decorre conjuntamente do fato de que os modelos atribuem papéis importantes às previsões ótimas dos agentes de variáveis futuras e da demonstração de Christopher Sims (1972) de que há uma estreita conexão entre o conceito de exogeneidade econométrica estrita e as formas dos preditores ótimos para um vetor de séries temporais. Construir um modelo com expectativas racionais necessariamente força a considerar qual conjunto de outras variáveis ajuda a prever uma determinada variável, digamos, renda ou a taxa de inflação. Se a variável $y$ ajuda a prever a variável $x$, os teoremas de Sims implicam que $x$ não pode ser considerada exógena em relação a $y$.

O resultado dessa conexão entre previsibilidade e exogeneidade tem sido que, em modelos macroeconométricos de equilíbrio, a distinção entre variáveis endógenas e exógenas não foi feita em uma base inteiramente a priori. Além disso, casos especiais dos modelos teóricos, que muitas vezes envolvem restrições laterais nos $$R-s$$ não extraídos da teoria econômica, têm fortes previsões testáveis quanto às relações de exogeneidade entre as variáveis.

Uma característica-chave dos modelos macroeconométricos de equilíbrio é que, como resultado das restrições entre os $A/s$, $B/s$ e $C/s$, os modelos preveem que, em geral, os parâmetros em muitas das equações mudarão se houver uma intervenção política que tome a forma de uma mudança em uma equação que descreve como uma variável de política está sendo definida. Como eles ignoram essas restrições de equação cruzada, os modelos keynesianos em geral assumem que todas as outras equações permanecem inalteradas quando uma equação que descreve uma variável de política é alterada. Acreditamos que essa é uma razão importante pela qual os modelos keynesianos falharam quando as equações que governam as variáveis de política ou as variáveis exógenas mudaram significativamente. Esperamos que os novos métodos que descrevemos nos dêem a capacidade de prever as consequências para todas as equações de mudanças nas regras que governam as variáveis de política. Ter essa capacidade é necessário antes que possamos afirmar ter uma base científica para fazer declarações quantitativas sobre a política macroeconômica.

Até agora, esses novos desenvolvimentos teóricos e econométricos não foram totalmente integrados, embora claramente estejam muito próximos, tanto conceitualmente quanto operacionalmente. Consideramos os melhores modelos de equilíbrio atualmente existentes como protótipos de modelos futuros melhores que, esperamos, se mostrarão de uso prático na formulação de políticas.

Mas não devemos subestimar o sucesso econométrico já alcançado pelos modelos de equilíbrio. Versões iniciais desses modelos foram estimadas e submetidas a alguns testes econométricos rigorosos por McCallum (1976), Barro (1977, em breve) e Sargent (1976a), com o resultado de que eles parecem capazes de explicar algumas características amplas do ciclo de negócios. Novos modelos mais sofisticados envolvendo restrições de equação cruzada mais complicadas estão em andamento (Sargent 1978). O trabalho até o momento já mostrou que os modelos de equilíbrio podem atingir ajustes dentro da amostra tão bons quanto os obtidos pelos modelos keynesianos, tornando concreto o ponto de que os bons ajustes dos modelos keynesianos não fornecem uma boa razão para confiar nas recomendações de política derivadas deles.

\subsubsection{\textbf{Crítica da Teoria do Equilíbrio}}
A ideia central das explicações de equilíbrio dos ciclos de negócios esboçadas acima é que as flutuações econômicas surgem à medida que os agentes reagem a mudanças não antecipadas em variáveis que afetam suas decisões. Claramente, qualquer explicação desse tipo geral deve implicar severas limitações na capacidade da política governamental de compensar essas mudanças iniciadoras. Primeiro, os governos devem de alguma forma ser capazes de prever choques invisíveis para os agentes privados, mas ao mesmo tempo ser incapazes de revelar essas informações antecipadas (portanto, neutralizando os choques). Embora não seja difícil projetar modelos teóricos nos quais essas duas condições são assumidas, é difícil imaginar situações reais nas quais tais modelos se aplicariam. Em segundo lugar, a política contracíclica governamental deve ser ela mesma imprevisível para os agentes privados (certamente uma condição frequentemente realizada historicamente) e ao mesmo tempo estar sistematicamente relacionada ao estado da economia. A eficácia, então, repousa na incapacidade dos agentes privados de reconhecer padrões sistemáticos na política monetária e fiscal.

Em grande parte, a crítica aos modelos de equilíbrio é simplesmente uma reação a essas implicações para a política. Tão amplo é (ou era) o consenso de que a tarefa da macroeconomia é a descoberta das políticas monetárias e fiscais particulares que podem eliminar flutuações reagindo à instabilidade do setor privado que a afirmação de que essa tarefa não deve ou não pode ser realizada é considerada frívola, independentemente de qualquer raciocínio e evidência que possam apoiá-la. Certamente deve-se ter alguma simpatia por essa reação: uma fé infundada na curabilidade de um mal particular serviu muitas vezes como estímulo para a descoberta de curas genuínas. No entanto, confundir uma fé possivelmente funcional na existência de políticas monetárias e fiscais eficazes com evidências científicas de que tais políticas são conhecidas é claramente perigoso, e usar tal fé como critério para julgar a extensão em que teorias particulares se ajustam aos fatos é ainda pior.

Existem, é claro, questões legítimas sobre o quão bem as teorias de equilíbrio podem se ajustar aos fatos do ciclo de negócios. De fato, esta é a razão para nossa insistência no caráter preliminar e provisório dos modelos particulares que temos agora. No entanto, esses modelos provisórios compartilham certas características que podem ser consideradas essenciais, por isso não é irracional especular sobre a probabilidade de que qualquer modelo desse tipo possa ter sucesso ou perguntar o que os teóricos do ciclo de negócios de equilíbrio terão em dez anos se tivermos sorte.

Quatro razões gerais para o pessimismo foram proeminentemente avançadas:

(a) Modelos de equilíbrio postulam de maneira irrealista mercados liberados.

(b) Esses modelos não podem explicar a "persistência" (correlação serial) dos movimentos cíclicos.

(c) Modelos implementados econométricamente são lineares (em logaritmos).

(d) O comportamento de aprendizado não foi incorporado nesses modelos.

\textbf{Mercados Líquidos}

Uma característica essencial dos modelos de equilíbrio é que todos os mercados se equilibram, ou seja, todos os preços e quantidades observados são vistos como resultados de decisões tomadas por empresas individuais e famílias. Na prática, isso significou uma suposição convencional de oferta igual à demanda, embora outros tipos de equilíbrios possam ser facilmente imaginados (se não tão facilmente analisados). Portanto, se alguém toma como um "fato" básico que os mercados de trabalho não se equilibram, chega-se imediatamente a uma contradição entre teoria e fato. Os fatos que realmente temos, no entanto, são simplesmente a série temporal disponível sobre emprego e salários, além das respostas às nossas pesquisas de desemprego. Mercados equilibrados é simplesmente um princípio, não verificável por observação direta, que pode ou não ser útil na construção de hipóteses bem-sucedidas sobre o comportamento dessas séries. Princípios alternativos, como o postulado da existência de um leiloeiro de terceiros induzindo rigidez salarial e mercados não equilibrados, são igualmente "irrealistas", no sentido não especialmente importante de não oferecer uma boa descrição das instituições do mercado de trabalho observadas.

Um refinamento do postulado inexplicado de um mercado de trabalho não equilibrado foi sugerido pelo fato indiscutível de que existem contratos de trabalho de longo prazo com horizontes de dois ou três anos. No entanto, o comprimento per se ao longo do qual os contratos são executados não tem relação com a questão, pois sabemos, a partir de Arrow e Debreu, que se contratos de longo prazo infinitamente longos são determinados de modo que os preços e salários sejam contingentes à mesma informação que está disponível sob a suposição de equilíbrio de mercado período a período, então precisamente o mesmo processo de preço-quantidade resultará com o contrato de longo prazo como ocorreria sob o equilíbrio de mercado período a período. Assim, a teorização do equilíbrio fornece uma maneira, provavelmente a única que temos, de construir um modelo de um contrato de longo prazo. O fato de que contratos de longo prazo existem, então, não tem implicações sobre a aplicabilidade da teorização do equilíbrio.

Em vez disso, a questão real aqui é se os contratos reais podem ser adequadamente contabilizados dentro de um modelo de equilíbrio, ou seja, um modelo no qual os agentes estão agindo em seus próprios melhores interesses. Stanley Fischer (1977), Edmund Phelps e John Taylor (1977) e Robert Hall (1978) mostraram que algumas das conclusões não ativistas dos modelos de equilíbrio são modificadas se se substituir o equilíbrio de mercado período a período pela imposição de contratos de longo prazo desenhados contingentes a conjuntos de informações restritas que são impostos exogenamente e que se supõe serem independentes dos regimes monetários e fiscais. A teoria econômica nos leva a prever que os custos de coleta e processamento de informações tornarão ótimo que os contratos sejam feitos contingentes a um pequeno subconjunto das informações que possivelmente poderiam ser coletadas em qualquer data. Mas a teoria também sugere que o conjunto particular de informações sobre as quais os contratos serão feitos contingentes não é imutável, mas depende da estrutura de custos e benefícios de coletar vários tipos de informações. Essa estrutura de custos e benefícios mudará com cada mudança nos processos estocásticos exógenos que os agentes enfrentam. Essa presunção teórica é apoiada por um exame da maneira como os contratos de trabalho diferem entre países de alta inflação e baixa inflação e a maneira como evoluíram nos EUA nos últimos 25 anos.

Então, a questão aqui é realmente a mesma fundamental envolvida na disputa entre Keynes e os economistas clássicos: Devemos considerar certas características superficiais dos contratos salariais existentes como dadas ao analisar as consequências de regimes monetários e fiscais alternativos? A teoria econômica clássica diz que não. Para entender as implicações dos contratos de longo prazo para a política monetária, precisamos de um modelo de como esses contratos provavelmente responderão a regimes de política monetária alternativos. Uma extensão dos modelos de equilíbrio existentes nessa direção pode levar a variações interessantes, mas parece-nos improvável que modificações importantes das implicações desses modelos para a política monetária e fiscal resultem disso.

\textbf{Persistência}
Uma segunda linha de crítica surge da observação correta de que, se as expectativas dos agentes são racionais e se seus conjuntos de informações incluem valores defasados da variável sendo prevista, então os erros de previsão dos agentes devem ser um processo aleatório não correlacionado serialmente. Ou seja, em média, não deve haver relações detectáveis entre o erro de previsão de um período e qualquer período anterior. Essa característica levou vários críticos a concluir que os modelos de equilíbrio não podem explicar mais do que uma parte insignificante dos movimentos altamente correlacionados serialmente que observamos na produção real, emprego, desemprego e outras séries. Tobin (1977, p. 461) colocou o argumento de forma sucinta:

Uma explicação atualmente popular das variações no emprego é a confusão temporária de preços relativos e absolutos. Empregadores e trabalhadores são enganados em muitos empregos pela inflação inesperada, mas apenas até aprenderem que ela afeta outros preços, não apenas os preços do que vendem. O inverso acontece temporariamente quando a inflação fica aquém da expectativa. Este modelo dificilmente pode explicar mais do que o desequilíbrio transitório nos mercados de trabalho.

Então, como os fiéis podem explicar os ciclos lentos de desemprego que realmente observamos? Apenas argumentando que a própria taxa natural flutua, que as variações nas taxas de desemprego são substancialmente mudanças no desemprego voluntário, friccional ou estrutural, em vez de desemprego involuntário devido à demanda geralmente deficiente.

Os críticos normalmente concluem que a teoria atribui apenas um papel muito menor às flutuações da demanda agregada e necessariamente depende de perturbações na oferta agregada para explicar a maior parte das flutuações na produção real ao longo do ciclo de negócios. "Em outras palavras", como disse Modigliani (1977), "o que aconteceu com os Estados Unidos na década de 1930 foi um ataque severo de preguiça contagiosa".

Essa crítica é falaciosa porque não distingue adequadamente entre fontes de impulsos e mecanismos de propagação, uma distinção enfatizada por Ragnar Frisch em um artigo clássico de 1933 que forneceu muitas das bases técnicas para os modelos macroeconométricos keynesianos. Mesmo que a nova teoria clássica implique que os erros de previsão que são os impulsos da demanda agregada são não correlacionados serialmente, é certamente logicamente possível que mecanismos de propagação estejam em ação que convertam esses impulsos em movimentos correlacionados serialmente em variáveis reais como produção e emprego. De fato, o trabalho teórico detalhado já mostrou que dois mecanismos de propagação concretos fazem exatamente isso.

Um mecanismo decorre da presença de custos para as empresas de ajustar rapidamente seus estoques de capital e trabalho. Sabe-se que a presença desses custos torna ótimo para as empresas espalhar ao longo do tempo sua resposta aos sinais de preços relativos que recebem. Ou seja, tal mecanismo faz com que uma empresa converta os erros de previsão não correlacionados serialmente na previsão de preços relativos em movimentos correlacionados serialmente nas demandas de fatores e produção.

Um segundo mecanismo de propagação já está presente nos modelos de crescimento econômico mais clássicos. Os planos de acumulação ótimos das famílias para reivindicações sobre capital físico e outros ativos convertem impulsos não correlacionados serialmente em demandas correlacionadas serialmente para a acumulação de ativos reais. Isso acontece porque os agentes normalmente querem dividir quaisquer mudanças inesperadas na renda parcialmente entre consumir e acumular ativos. Assim, a demanda por ativos no próximo período depende dos estoques iniciais e das mudanças inesperadas nos preços ou na renda que os agentes enfrentam. Essa dependência faz com que surpresas não correlacionadas serialmente levem a movimentos correlacionados serialmente nas demandas por ativos físicos. Lucas (1975) mostrou como esse mecanismo de propagação aceita prontamente erros na previsão da demanda agregada como uma fonte de impulso.

Um terceiro mecanismo de propagação provável foi identificado por trabalhos recentes na teoria da busca. (Veja, por exemplo, McCall 1965, Mortensen 1970, e Lucas e Prescott 1974.) A teoria da busca tenta explicar por que os trabalhadores que, por algum motivo, estão sem emprego, acham racional não necessariamente aceitar a primeira oferta de emprego que aparece, mas sim permanecer desempregados por um tempo até que uma oferta melhor se materialize. Da mesma forma, a teoria explica por que uma empresa pode achar ótimo esperar até que apareça um candidato a emprego mais adequado, de modo que as vagas persistam por algum tempo. Principalmente por razões técnicas, modelos teóricos consistentes que permitem que este mecanismo de propagação aceite erros na previsão da demanda agregada como um impulso ainda não foram elaborados, mas o mecanismo parece provavelmente desempenhar um papel importante em um modelo bem-sucedido do comportamento da série temporal da taxa de desemprego.

Em modelos onde os agentes têm informações imperfeitas, qualquer um dos dois primeiros mecanismos e provavelmente o terceiro podem fazer movimentos correlacionados serialmente em variáveis reais decorrentes da introdução de uma sequência de erros de previsão não correlacionados serialmente. Assim, modelos teóricos e econométricos foram construídos nos quais, em princípio, o processo de erros de previsão não correlacionados serialmente pode explicar qualquer proporção entre zero e um da variância de estado estável da produção real ou emprego. O argumento de que tais modelos devem necessariamente atribuir a maior parte da variância na produção real e no emprego às variações na oferta agregada é simplesmente errado logicamente.

\textbf{Linearidade}

A maior parte do trabalho econométrico que implementa modelos de equilíbrio envolveu o ajuste de modelos estatísticos que são lineares nas variáveis (mas muitas vezes altamente não lineares nos parâmetros). Essa característica está sujeita a críticas com base no princípio indiscutível de que geralmente existem modelos não lineares que fornecem melhores aproximações do que os modelos lineares. Mais especificamente, modelos que são lineares nas variáveis não fornecem maneira de detectar e analisar efeitos sistemáticos de momentos de ordem superior aos choques e às variáveis exógenas nos momentos de primeira ordem das variáveis endógenas. Tais efeitos sistemáticos estão geralmente presentes onde as variáveis endógenas são definidas por agentes avessos ao risco.

Não há razões teóricas para que a maioria dos trabalhos aplicados tenha usado modelos lineares, apenas razões técnicas convincentes dada a tecnologia de computadores de hoje. O requisito técnico predominante do trabalho econométrico que impõe expectativas racionais é a capacidade de escrever expressões analíticas que dão as regras de decisão dos agentes como funções dos parâmetros de suas funções objetivas e como funções dos parâmetros que governam os processos aleatórios exógenos que enfrentam. Problemas máximos estocásticos dinâmicos com objetivos quadráticos, que produzem regras de decisão lineares, atendem a esse requisito essencial - essa é a sua virtude. Apenas algumas outras formas funcionais para as funções objetivas dos agentes em problemas de ótimo estocástico dinâmico têm essa mesma capacidade analítica necessária. A tecnologia de computadores no futuro previsível parece exigir trabalhar com tal classe de funções, e a classe de regras de decisão lineares pareceu mais conveniente para a maioria dos propósitos. Nenhuma questão de princípio está envolvida na seleção de uma entre a classe muito restrita de funções disponíveis. Teoricamente, sabemos como calcular, com métodos recursivos caros, as regras de decisão não lineares que resultariam de uma classe muito ampla de funções objetivas; nenhum novo princípio econométrico estaria envolvido na estimativa de seus parâmetros, apenas uma conta de computador muito mais alta. Além disso, como Frisch e Slutsky enfatizaram, as equações de diferenças estocásticas lineares são um dispositivo muito flexível para estudar ciclos de negócios. É uma questão em aberto se, para explicar as características centrais do ciclo de negócios, haverá uma grande recompensa para ajustar modelos não lineares.

\textbf{Modelos Estacionários e a Negligência da Aprendizagem}

Benjamin Friedman e outros criticaram os modelos de expectativas racionais aparentemente com base no fato de que muito trabalho teórico e quase todo o trabalho empírico assumiu que os agentes têm operado por um longo tempo em um ambiente estocasticamente estacionário. Portanto, geralmente se presume que os agentes tenham descoberto as leis de probabilidade das variáveis que desejam prever. Modigliani (1977, p. 6) colocou o argumento desta maneira:

No nível lógico, Benjamin Friedman chamou a atenção para a omissão de um modelo explícito de aprendizagem nos [modelos macroeconômicos de equilíbrio] e sugeriu que, como resultado, só pode ser interpretado como uma descrição não de equilíbrio de curto prazo, mas de equilíbrio de longo prazo no qual nenhum agente desejaria renegociar. Mas então as implicações dos [modelos macroeconômicos de equilíbrio] estão claramente longe de serem surpreendentes, e sua relevância política é quase nula.

Mas tem sido apenas uma questão de conveniência analítica e não de necessidade que os modelos de equilíbrio tenham usado a suposição de choques estocasticamente estacionários e a suposição de que os agentes já aprenderam as distribuições de probabilidade que enfrentam. Ambas essas suposições podem ser abandonadas, embora a um custo em termos da simplicidade do modelo. (Por exemplo, veja Crawford 1971 e Grossman 1975.) De fato, dentro do arcabouço de funções objetivas quadráticas, nas quais o "princípio da separação" se aplica, pode-se aplicar a fórmula de filtragem de Kalman para derivar regras de decisão lineares ótimas com coeficientes dependentes do tempo. Neste arcabouço, o filtro de Kalman permite uma aplicação elegante da aprendizagem bayesiana para atualizar regras de previsão ótimas de período para período à medida que novas informações se tornam disponíveis. O filtro de Kalman também permite a derivação de regras de decisão ótimas para uma classe interessante de processos exógenos não estacionários assumidos para enfrentar os agentes. A teorização do equilíbrio neste contexto, portanto, prontamente leva a um modelo de como a não estacionariedade do processo e a aprendizagem bayesiana aplicada pelos agentes às variáveis exógenas levam a coeficientes dependentes do tempo nas regras de decisão dos agentes.

Embora modelos que incorporam aprendizado bayesiano e não estacionariedade estocástica sejam tecnicamente viáveis e consistentes com a estratégia de modelagem de equilíbrio, sabemos de quase nenhum trabalho aplicado bem-sucedido ao longo dessas linhas. Uma provável razão para isso é que os modelos de séries temporais não estacionários são pesados e vêm em muitas variedades. Outra é que a hipótese de aprendizado bayesiano é vazia até que se impute arbitrariamente uma distribuição a priori aos agentes ou se desenvolva um método de estimativa dos parâmetros da priori a partir de dados de séries temporais. Determinar uma distribuição a priori a partir dos dados envolveria a estimativa de condições iniciais e proliferaria parâmetros de incômodo de uma maneira muito desagradável. Se essas técnicas compensarão em termos de explicação de séries temporais macroeconômicas é uma questão empírica: não é uma questão que distingue modelos macroeconométricos de equilíbrio de keynesianos. De fato, nenhum modelo macroeconométrico keynesiano existente incorpora um modelo econômico de aprendizado ou um modelo econômico que restrinja de alguma forma o padrão de não estacionariedades de coeficientes entre equações.

Os modelos macroeconométricos criticados por Friedman e Modigliani, que assumem que os agentes pegaram os processos aleatórios estacionários que enfrentam, dão origem a sistemas de equações de diferenças estocásticas lineares da forma (1), (2) e (4). Como se sabe há muito tempo, tais equações de diferenças estocásticas geram séries que "parecem" séries temporais econômicas. Além disso, se vistos como estruturais (ou seja, invariantes em relação às intervenções de política), os modelos têm algumas das implicações para a política contracíclica que descrevemos acima. Se essas implicações de política são corretas ou não depende de se os modelos são estruturais e não de se os modelos podem ser caricaturados com sucesso por termos como "longo prazo" ou "curto prazo".

Vale a pena reforçar que não desejamos que nossas respostas a essas críticas sejam confundidas com uma afirmação de que os modelos de equilíbrio existentes podem explicar satisfatoriamente todas as principais características do ciclo de negócios observado. Em vez disso, simplesmente argumentamos que ainda não foram apresentadas razões sólidas que sequer sugerem que esses modelos são, como classe, incapazes de fornecer uma teoria satisfatória do ciclo de negócios.

\subsubsection{\textbf{Resumo e Conclusões}}

Vamos tentar apresentar de forma compacta os principais argumentos avançados neste artigo. Em seguida, comentaremos brevemente as principais implicações desses argumentos para a maneira como podemos pensar de maneira útil sobre a política econômica.

Nosso primeiro e mais importante ponto é que os modelos macroeconométricos keynesianos existentes não podem fornecer orientação confiável na formulação de políticas monetárias, fiscais ou de outros tipos. Essa conclusão é baseada em parte nos fracassos recentes espetaculares desses modelos e em parte na falta de uma base teórica ou econométrica sólida. Em segundo lugar, com base nesse último ponto, não há esperança de que a modificação menor ou mesmo maior desses modelos leve a uma melhoria significativa em sua confiabilidade.

Em terceiro lugar, modelos de equilíbrio podem ser formulados que estão livres dessas dificuldades e que oferecem um conjunto diferente de princípios para identificar modelos econométricos estruturais. Os elementos-chave desses modelos são que os agentes são racionais, reagindo às mudanças de política de uma maneira que está em seus melhores interesses privados, e que os impulsos que desencadeiam as flutuações de negócios são principalmente choques não antecipados.

Em quarto lugar, os modelos de equilíbrio já desenvolvidos explicam as principais características qualitativas do ciclo de negócios. Esses modelos estão sendo submetidos a críticas contínuas, especialmente por aqueles envolvidos em seu desenvolvimento, mas argumentos de que as teorias de equilíbrio são, em princípio, incapazes de explicar uma parte substancial das flutuações observadas parecem ser devidos principalmente a mal-entendidos simples.

As implicações políticas das teorias de equilíbrio são às vezes caricaturadas, por comentaristas amigáveis e não amigáveis, como a afirmação de que "a política econômica não importa" ou "não tem efeito". Essa implicação certamente surpreenderia os economistas neoclássicos que aplicaram com sucesso a teoria do equilíbrio ao estudo de inúmeros problemas envolvendo efeitos importantes das políticas fiscais na alocação de recursos e na distribuição de renda. Nossa intenção não é rejeitar essas realizações, mas sim tentar imitá-las ou estender os métodos de equilíbrio que foram aplicados a muitos problemas econômicos para cobrir um fenômeno que até agora resistiu à sua aplicação: o ciclo de negócios.

Se essa arbitragem intelectual se mostrar bem-sucedida, sugerirá mudanças importantes na maneira como pensamos sobre a política. Mais fundamentalmente, focará a atenção na necessidade de pensar na política como a escolha de regras estáveis do jogo, bem compreendidas pelos agentes econômicos. Somente em tal cenário a teoria econômica ajudará a prever as ações que os agentes escolherão tomar. Essa abordagem também sugerirá que políticas que afetam o comportamento principalmente porque suas consequências não podem ser corretamente diagnosticadas, como a instabilidade monetária e o financiamento do déficit, têm a capacidade apenas de perturbar. A provisão deliberada de informações errôneas não pode ser usada de maneira sistemática para melhorar o ambiente econômico.

Os objetivos da teoria do ciclo de negócios de equilíbrio são tomados, sem modificação, do objetivo que motivou a construção dos modelos macroeconométricos keynesianos: fornecer um meio baseado cientificamente para avaliar, quantitativamente, os prováveis efeitos de políticas econômicas alternativas. Sem os sucessos econométricos alcançados pelos modelos keynesianos, esse objetivo seria simplesmente inconcebível. No entanto, a menos que os limites agora evidentes desses modelos também sejam francamente reconhecidos e tomadas direções novas e radicalmente diferentes, as realizações reais da Revolução Keynesiana serão perdidas tão certamente quanto aquelas que agora sabemos ser ilusórias.

\section{\textbf{Duarte (2023)}}

\subsection{\texbj{A Ascensão da Hipótese das Expectativas Racionais}}

\subsubsection{\textbf{Resumo}}

Se os agentes tomam decisões baseadas em suas crenças sobre o futuro, a maneira como eles formam expectativas é central para a economia. Uma visão é que os agentes usam da melhor maneira as informações disponíveis para não cometer erros ou não persistir neles. Essa é a ideia geral das expectativas racionais, que parece descrever como os indivíduos reais decidem. No entanto, Robert Lucas adaptou a formulação original de John Muth para modelos macroeconômicos, transformando-a em um axioma de consistência: os agentes em um modelo formam expectativas consistentes com esse modelo. Portanto, não tem nada a ver com agentes do mundo real e tudo a ver com propriedades matemáticas particulares dos modelos econômicos. Este capítulo mostra as origens da hipótese das expectativas racionais na década de 1960 e sua subsequente disseminação na macroeconomia, não apenas em círculos acadêmicos, mas também na formulação de políticas. A tensão entre ser uma hipótese sobre o comportamento real ou o comportamento de agentes fictícios em um modelo persistiu ao longo do tempo e alcançou debates recentes envolvendo hipóteses de expectativas alternativas discutidas em outros capítulos deste livro. O capítulo também discute algumas resistências iniciais à hipótese das expectativas racionais e termina com seu domínio na literatura de macroeconomia desde a década de 1980 até o início dos anos 2000.

\subsubsection{\textbf{Introdução}}
Entender como as expectativas sobre o futuro afetam as decisões dos agentes econômicos tem sido uma preocupação na economia há bastante tempo. Uma visão possível é que os agentes utilizam da melhor maneira as informações disponíveis para não cometer erros ou não persistir neles. Essa é a ideia geral das expectativas racionais, que parece descrever como os indivíduos reais decidem. No entanto, Robert Lucas adaptou a formulação original de John Muth para modelos macroeconômicos, transformando-a em um axioma de consistência: os agentes em um modelo formam expectativas consistentes com esse modelo. Portanto, não tem nada a ver com agentes do mundo real e tudo a ver com propriedades matemáticas particulares dos modelos econômicos.

Tal contribuição frequentemente ganha reconhecimento nos comentários históricos que os macroeconomistas fazem ao avançar a fronteira da pesquisa. A razão é que a formulação de Lucas da hipótese das expectativas racionais inaugurou a abordagem moderna para modelos macroeconômicos que se tornou mainstream na década de 1980. Esta nova macroeconomia queria pôr fim a um período em que, supostamente, a norma era a intervenção governamental voltada para suavizar as flutuações econômicas (que são tipicamente referidas como "políticas keynesianas").

A hipótese das expectativas racionais conquistou a macroeconomia com implicações muito fortes e em um momento de grande turbulência econômica, a década de 1970. Com expectativas racionais em seus modelos, Lucas argumentou que as políticas macroeconômicas sistemáticas não têm efeitos sobre a produção real; apenas surpresas poderiam ter efeitos reais. Este resultado de ineficácia da política tornou-se a marca registrada dos novos modelos macroeconômicos sendo criados na década de 1970. Ao ponto de que foi além dos debates acadêmicos e alcançou a mídia especializada, que o promoveu. Em 8 de novembro de 1976, a Business Week publicou um artigo intitulado "Como as Expectativas Derrotam a Política Econômica". Nele, a "controversa nova teoria chamada expectativas racionais" que "está varrendo a profissão econômica" e "diz que a política econômica é impotente" porque "o público ... toma ações que compensam [mudanças de política sistemáticas]" (citado em Duarte 2012, p. 204). Até mesmo o economista e aclamado jornalista econômico Leonard Silk discutiu as expectativas racionais em sua coluna no The New York Times em junho de 1977, apresentando-a como uma nova teoria do ciclo de negócios urgentemente necessária após a recessão de 1973-1975, com uma visão anti-keynesiana que restaurou as ideias do ciclo de negócios que prevaleceram antes da década de 1930 (Silk 1977; veja também Silk 1980). O jornalista econômico David Warsh, em sua coluna no The Boston Globe de 7 de outubro de 1979, notou que a economia "está sendo varrida por uma 'teoria das expectativas racionais' elegante, mas misteriosa", e que "a escola das expectativas racionais é praticamente a coisa mais quente na economia técnica moderna" que está varrendo a profissão "por causa de seu rigor, não de seu realismo". Em 1979, o economista de mercado livre e jornalista Herbert Stein explicou as expectativas racionais em sua coluna no The New York Times, "Verbal Windfall:" "Esta é a coisa mais quente na economia em 1979. Parece que as pessoas não podem esperar racionalmente o sucesso de uma política cujo sucesso depende das pessoas não entenderem a política. A dama na caixa não pode ser enganada pelo ilusionista que finge serrar ela ao meio."

Embora a hipótese das expectativas racionais estivesse fortemente conectada desde o início a uma preocupação muito concreta e do mundo real sobre os efeitos reais das políticas econômicas, e enquanto os argumentos apresentados a favor desta hipótese e contra os então prevalecentes modelos keynesianos envolviam argumentar que as pessoas não são estúpidas, Lucas insistiu em suas contribuições acadêmicas que esta era uma hipótese para a consistência dos modelos macroeconômicos, e não diretamente relacionada ao comportamento real dos indivíduos no mundo real (daí a descrição muito apropriada de Warsh de que estava se espalhando por causa de seu rigor, não de seu realismo). Esta tensão entre ser uma afirmação sobre agentes no mundo real ou sobre agentes em um modelo, e as implicações derivadas desses modelos, tornou as expectativas racionais muito controversas desde o início.

No debate acadêmico, as reações às expectativas racionais variaram desde disputas empíricas sobre sua validade no mundo real, até discussões empíricas sobre a plausibilidade das implicações de modelos macro com expectativas racionais. Fora da academia, a hipótese foi colocada em contato próximo com decisões de formulação de políticas através do Federal Reserve Bank de Minneapolis em 1977-1980, quando seu presidente, Mark Willes, tornou-se um defensor declarado da teoria das expectativas racionais. Este capítulo mostra as origens da hipótese das expectativas racionais na década de 1960 e sua subsequente disseminação na macroeconomia, não apenas em círculos acadêmicos, mas também na formulação de políticas. O capítulo também discute algumas resistências iniciais à hipótese das expectativas racionais e termina com seu domínio na literatura de macroeconomia desde a década de 1980 até o início dos anos 2000.

\subsubsection{\textbf{A Hipótese das Expectativas Racionais: Suas Origens e Suas Implicações}}

"O final dos anos 1960 foram bons tempos para ser um jovem macroeconomista. (...) [Os] modelos macroeconômicos eram influentes, grandes e avançados econometricamente. Eles incorporavam dinâmicas cada vez mais sofisticadas e atraíam os esforços dos melhores economistas." Essa é a visão de Thomas Sargent (1996) sobre o tempo em que ele estava estudando economia (com um BA de Berkeley em 1964 e um PhD de Harvard em 1968). Naquela época, uma abordagem comum para modelar expectativas era através do conceito de expectativas adaptativas. De acordo com isso, os indivíduos revisam suas expectativas quando cometem um erro em suas expectativas do período anterior. Pegue, por exemplo, o preço como a variável sob consideração pelo agente. Se o preço que um agente esperava para hoje difere do preço efetivo atual, então este agente corrige suas expectativas para o nível de preço de amanhã. Por exemplo, se os preços hoje são mais altos do que o esperado, isso indica ao agente que ele tem que aumentar suas expectativas para o nível de preço de amanhã. Importante notar que aqui as expectativas são formadas com base no que aconteceu no passado (ou seja, os preços esperados para amanhã dependem dos preços de hoje).

Embora sendo uma aplicação de uma teoria matemática muito conhecida em engenharia (um atraso distribuído geometricamente), o uso de expectativas adaptativas na economia tornou-se estreitamente associado a Milton Friedman (1957) e seu aluno Philip Cagan (1956), embora tenha sido articulado alguns anos antes por Marc Nerlove (1954) a partir de ideias anteriores de John R. Hicks (Young e Darity Jr. 2001, 774). Um ponto do ataque multifacetado de Friedman contra a macroeconomia keynesiana tinha a ver com argumentar que a propensão marginal estimada para consumir (quanto o nível de consumo responde a uma pequena mudança na renda atual) é menor do que os valores típicos usados nos modelos keynesianos. Uma implicação crítica disso é que os efeitos reais das políticas fiscais seriam menores do que o que os modelos keynesianos propunham. A crítica de Friedman equivaleu a desafiar a maneira como os modelos dinâmicos foram estimados, e as expectativas adaptativas foram usadas para representar estatisticamente a hipótese de renda permanente (o principal determinante do consumo de acordo com Friedman). Por sua vez, Cagan (1956) usou expectativas adaptativas para representar a formação de expectativas por agentes no contexto de hiperinflação. Em ambos os casos, o valor esperado de uma determinada variável hoje é uma média ponderada de valores passados dessa variável, com pesos que diminuem exponencialmente ao longo do tempo (ou seja, as expectativas adaptativas são um atraso distribuído geometricamente que torna as expectativas "olhando para trás").

As expectativas adaptativas foram realmente a "primeira tentativa de modelar a revisão sistemática das expectativas à luz de novas informações. (...) No entanto, embora se pensasse que as pessoas com expectativas adaptativas usavam seus próprios erros de previsão para derivar suas próximas previsões, nenhuma teoria econômica amplamente aceita foi oferecida para explicar a magnitude do parâmetro de ajuste. Além disso, confiando em regras extrapolativas mecanicistas que olham para trás, as expectativas adaptativas foram criticadas por ignorar a capacidade das pessoas de aprender com a experiência" (Sent 2002, 303). E porque os agentes corrigem parcialmente suas expectativas quando tiveram um erro de previsão, eles levarão esses erros para períodos futuros, cometendo erros sistemáticos.

A motivação da hipótese das expectativas racionais era eliminar a possibilidade de erros de previsão sistemáticos. John Muth foi quem primeiro desenvolveu esta hipótese. Depois de publicar um artigo no qual investigou as condições sob as quais as expectativas adaptativas de Friedman e Cagan são consistentes com as expectativas racionais (Muth 1960), foi o artigo publicado um ano depois que eventualmente fez sua fama em macroeconomia (Muth 1961). No trabalho de 1961, Muth começou reconhecendo que vários economistas fizeram o ponto de que as expectativas importam para entender "flutuações nos mercados ou na economia", mas que o problema era que as teorias econômicas não "incluem uma explicação de como as expectativas são formadas" (Muth 1961, 315). Muth (1961, 316) então apontou duas "principais conclusões de estudos de dados de expectativas:" "1) As médias de expectativas em uma indústria são mais precisas do que modelos ingênuos e tão precisas quanto sistemas de equações elaborados, embora haja diferenças consideráveis de opinião entre os setores; 2) As expectativas relatadas geralmente subestimam a extensão das mudanças que realmente ocorrem."

Com essa motivação de expectativas formadas por pessoas do mundo real, Muth (1961, 316) argumentou:

Para explicar esses fenômenos, gostaria de sugerir que as expectativas, uma vez que são previsões informadas de eventos futuros, são essencialmente as mesmas que as previsões da teoria econômica relevante. Correndo o risco de confundir essa hipótese puramente descritiva com uma declaração sobre o que as empresas deveriam fazer, chamamos essas expectativas de "racionais". (...)

A hipótese pode ser reescrita um pouco mais precisamente da seguinte forma: que as expectativas das empresas (ou, mais geralmente, a distribuição de probabilidade subjetiva dos resultados) tendem a ser distribuídas, para o mesmo conjunto de informações, em torno da previsão da teoria (ou as "distribuições de probabilidade objetivas" dos resultados).

A hipótese afirma três coisas: (1) A informação é escassa, e o sistema econômico geralmente não a desperdiça. (2) A maneira como as expectativas são formadas depende especificamente da estrutura do sistema relevante que descreve a economia. (3) Uma "previsão pública", no sentido de Grunberg e Modigliani [1954], não terá nenhum efeito substancial sobre a operação do sistema econômico (a menos que seja baseada em informações internas).

Enquanto o primeiro e o terceiro parágrafos desta passagem estabelecem expectativas racionais em termos gerais, significando que poderia ser aplicado tanto a modelos microeconômicos quanto a modelos macroeconômicos, é claro a partir do segundo e do restante do artigo de Muth (1961) que ele desenvolveu para estudar um problema microeconômico no nível da empresa (ou no nível de um mercado individual). Em contraste com os interesses de Muth, a hipótese das expectativas racionais ganhou tração na economia quando foi aplicada a modelos macroeconômicos por Lucas e outros, cerca de dez anos após sua formulação original.

Além de definir expectativas racionais, Muth (1961, 317) deixou claro o que não é:

Não afirma que o trabalho de rascunho dos empresários se assemelha ao sistema de equações de qualquer maneira; nem afirma que as previsões dos empresários são perfeitas ou que suas expectativas são todas iguais.

Muth desenvolveu seu trabalho como estudante de pós-graduação na Graduate School of Industrial Administration no Carnegie Institute of Technology (hoje Carnegie-Mellon University; Sent 2002). Em seu PhD, ele teve Herbert Simon como seu orientador oficial, mas recebeu a maior parte da ajuda e orientação de Franco Modigliani, e obteve um diploma em administração industrial em 1962 (McCallum 2016, 343-344). Antes disso, ele "passou o ano acadêmico de 1957-58 como professor visitante na Universidade de Chicago [quando Lucas ainda era estudante de graduação em história] (...), e o ano acadêmico de 1961-62 na Cowles Foundation na Universidade de Yale (...). Ele foi afiliado ao Carnegie como pesquisador associado de 1956 até 1959, como professor assistente de 1959 até 1962, e como professor associado sem titularidade de 1962" até 1965, mas em licença do Carnegie em 1964-65 (Sent 2002, 295; McCallum 2016, 342).
Sua pesquisa e interesses foram moldados mais diretamente por projetos-chave desenvolvidos no Carnegie por Modigliani, Simon e Charles Holt (PhD Chicago 1955), e pelos trabalhos de outros afiliados ao Carnegie, como Albert Ando (PhD em economia matemática no Carnegie em 1959) e o matemático A. Charnes. No Carnegie, Modigliani, Simon e Holt, e vários de seus estudantes de pós-graduação (incluindo Muth e Ando), desenvolveram ao longo de cinco anos o projeto de pesquisa "Planejamento e Controle de Operações Industriais" (1953-1958), contratado com o Office of Naval Research (Sent 2002, Duarte 2009, Rancan 2013). Eles estudaram uma empresa de tintas para derivar regras de decisão científicas que poderiam substituir os "procedimentos de julgamento e regra de ouro predominantes", argumentando que o resultado resultante traria custos mais baixos e lucros mais altos. Como o grupo anunciou na manchete de um artigo da Harvard Business Review: "Aqui está outra área onde as novas técnicas estatísticas podem subtrair adivinhações, adicionar precisão, dividir riscos e multiplicar a eficiência" (citado em Duarte 2009, 20).

O ambiente de Carnegie e os projetos de pesquisa levaram ao desenvolvimento de duas visões alternativas sobre a racionalidade humana: a racionalidade limitada de Simon e a hipótese das expectativas racionais de Muth. Como Esther-Mirjam Sent (2002, 302) argumentou, "Muth expôs a racionalidade escondida" na racionalidade limitada, vendo o comportamento satisfatório como um caso especial de tomada de decisão racional sob incerteza. Por outro lado, ao discutir a hipótese das expectativas racionais, Muth (1961, 321-322) fez um esforço considerável para mostrar importantes desvios da racionalidade:

Certas imperfeições e vieses nas expectativas também podem ser analisados com os métodos deste artigo. (...) Examinaremos o efeito de superdescontar as informações atuais e das diferenças nas informações possuídas por várias empresas na indústria. Se tais vieses nas expectativas são empiricamente importantes, ainda resta ser visto. Eu só quero enfatizar que os métodos são flexíveis o suficiente para lidar com eles. (Muth 1961, 321)

Enquanto Muth foi pioneiro na hipótese das expectativas racionais, usando-a em um contexto de microeconomia, essa visão da racionalidade humana se tornou um tópico quente na economia quando foi transposta para problemas macroeconômicos alguns anos depois por Lucas, Edward Prescott, Sargent, Neil Wallace, entre outros. Esses economistas lançaram um ataque ao keynesianismo então dominante, a partir de duas principais fortalezas acadêmicas: Carnegie Mellon e a Universidade de Minnesota (que tinha o economista keynesiano Walter Heller como um membro distinto). Eventualmente, com a mudança de Lucas de Carnegie para a Universidade de Chicago em 1974, a primeira fortaleza mudou de endereço e a hipótese das expectativas racionais começou uma associação duradoura com Chicago.

As origens do uso das expectativas racionais na macroeconomia são bem conhecidas (Hoover 1988, cap. 2, Sent 1998, Young e Darity 2001, De Vroey 2016, cap. 9, Silva 2017, Boumans 2020). Em 1963, um ano antes de receber seu PhD em economia da Universidade de Chicago, Lucas tornou-se membro do corpo docente da Graduate School in Industrial Administration (GSIA) na Carnegie-Mellon, onde Muth era então professor associado - com ambos se sobrepondo em Carnegie até 1965 (McCallum 2016). No mesmo ano, Prescott entrou no programa de doutorado da GSIA, fez uma aula com Lucas e teve-o como membro de seu comitê de dissertação (Silva 2017, 149). No final dos anos 1960, Lucas desenvolveu sua agenda de pesquisa sobre o mercado de trabalho juntamente com Leonard Rapping, no qual eles recorreram às expectativas adaptativas, mesmo que Lucas não estivesse contente com essa formulação (Hoover 1988, 28). Naquela época, Lucas havia trabalhado em um modelo com expectativas estáticas e racionais que recebeu a colaboração de Prescott e se tornou seu artigo de 1971, onde estudaram a determinação do investimento, produção e preços no nível da empresa (Lucas e Prescott 1971). Depois de apresentar um rascunho anterior na reunião de 1968 da Sociedade Econométrica, quando a hipótese das expectativas racionais e a previsão perfeita implícita na distribuição de preços foram criticadas, Prescott trabalhou para adaptar a definição de expectativas racionais de Muth ao seu modelo (Silva 2017, 150). No artigo publicado, Lucas e Prescott (1971, 660) assumiram "que os preços reais e antecipados têm a mesma distribuição de probabilidade, ou que as expectativas de preço são racionais". Eles então adicionaram uma nota de rodapé onde explicaram quão diferente era o uso deles das expectativas racionais de Muth:

Este termo é tirado de Muth [1961], que o aplicou ao caso em que o preço esperado e o preço real (ambos variáveis aleatórias) têm um valor médio comum. Como a discussão de Muth sobre este conceito se aplica igualmente bem à nossa suposição de uma distribuição comum para essas variáveis aleatórias, parece natural adotar o termo aqui. (Lucas e Prescott 1971, 660, fn. 4)

Prescott foi fundamental na caracterização matemática das expectativas racionais no modelo que ele desenvolveu com Lucas. Para eles, usar expectativas racionais em seu modelo permitiu que eles "obtivessem uma teoria operacional de investimento ligando o investimento atual a variáveis explicativas atuais e passadas, em vez de variáveis futuras 'esperadas' que devem, na prática, ser substituídas por várias 'variáveis proxy'" (Lucas e Prescott 1971, 660). Além de entregar resultados operacionais, a hipótese das expectativas racionais era razoável, "mais plausível do que qualquer esquema adaptativo simples" (Lucas e Prescott 1971, 664, fn. 9). Mas como Silva (2017, 151) apontou, "[a]o adotar a hipótese das expectativas racionais, Lucas e Prescott (1971, 660) evitam intencionalmente discutir o processo pelo qual as empresas traduzem informações em previsões de preços, um debate típico sobre esquemas de expectativas adaptativas." No entanto, um agente racional para Lucas é um robô de processamento de informações "sem referências antropocêntricas a intenções, propósitos, autoridade ou mesmo 'significado'", como Brockway McMillan caracterizou a nova matemática surgindo dos "novos domínios da teoria estatística da comunicação, a teoria do feedback e controle, a teoria dos jogos e a programação linear" (Boumans 2020, 146). Em seus trabalhos usando expectativas racionais, Lucas trouxe para a economia ferramentas matemáticas do campo emergente da engenharia da informação (Boumans 2020).

A culminação da agenda de pesquisa de Lucas nos anos 1960 em Carnegie foram os artigos que ele produziu no início dos anos 1970 (Lucas [1972a] 1981, 1972b, 1973), levando ao seu artigo de 1975 (Lucas 1975). Neles, ele articulou uma abordagem de equilíbrio com expectativas racionais para entender a relação entre mudanças de preços e desemprego (a curva de Phillips), a neutralidade do dinheiro (o que significa que mudanças monetárias não afetam variáveis reais, sendo traduzidas apenas em mudanças no nível geral de preços), e o ciclo de negócios. Para Lucas, a economia tende a operar em seu nível de equilíbrio, na taxa natural de desemprego (uma taxa de desemprego positiva explicada por atritos no mercado de trabalho). Lucas não estava sozinho em impulsionar a agenda de pesquisa das expectativas racionais em macroeconomia. Na fortaleza de Minnesota, Sargent fez importantes contribuições, sozinho ou em conjunto com Wallace (Sargent 1972, 1973, Sargent e Wallace 1973, 1975).

Nesses trabalhos posteriores, Lucas trouxe novamente uma defesa das expectativas racionais como uma hipótese plausível

Minha preocupação neste artigo será mostrar que as expectativas racionais podem levar a modelos de ciclo de negócios viáveis e testáveis. Para o argumento de que esta hipótese também é plausível e consistente com uma variedade de evidências, o leitor é encaminhado para Muth (1961). (Lucas 1972b, 96, fn. 7)

Então, embora Lucas tenha apelado para sua razoabilidade e operacionalidade, a hipótese das expectativas racionais era uma escolha de modelagem, não algo sobre como as pessoas reais se comportam: havia diferentes maneiras de modelar a formação de expectativas, e a hipótese das expectativas racionais era mais razoável e operacional do que as alternativas. Com isso, Lucas construiu modelos de equilíbrio do ciclo de negócios e derivou implicações testáveis sobre fenômenos de ciclo de negócios. Portanto, o debate empírico seria centrado nessas implicações, e não no realismo da hipótese das expectativas racionais per se.

Desses trabalhos anteriores sobre expectativas racionais, surgiu um resultado importante: que políticas sistemáticas (políticas fiscais ou monetárias) não afetam o valor de equilíbrio das variáveis reais (produção e desemprego, por exemplo). A única maneira de as políticas econômicas terem efeitos reais é surpreendendo os agentes, com choques de política inesperados. Mas, neste caso, os efeitos reais das políticas não sistemáticas não duram muito tempo, e a economia retorna ao equilíbrio. O corolário dessa visão é que a ideia keynesiana de que o governo pode estabilizar as flutuações cíclicas por meio de políticas sistemáticas está errada. A visão inicial era que esse resultado de ineficácia da política veio da hipótese das expectativas racionais, como os destaques da mídia mencionados na introdução atestam. De acordo com essa visão, as variáveis reais no modelo são afetadas apenas por erros de expectativa, e as políticas sistemáticas não podem gerar tais erros porque são incorporadas aos valores esperados das variáveis:

Neste sistema, não há sentido em que a autoridade tenha a opção de conduzir uma política contracíclica. Para explorar a curva de Phillips, ela deve de alguma forma enganar o público. Mas, por virtude da suposição de que as expectativas são racionais, não há regra de feedback que a autoridade possa empregar e esperar ser capaz de enganar sistematicamente o público. Isso significa que a autoridade não pode esperar explorar a Curva de Phillips nem mesmo por um período. (Sargent e Wallace 1976, 177-178)

Outra implicação muito poderosa da agenda de pesquisa coletiva desenvolvida em Carnegie (e Chicago após 1974) e Minnesota foi a chamada "Crítica de Lucas" (Lucas 1976a). Foi outra crítica ao keynesianismo predominante dos anos 1960. Naquela época, os economistas desenvolveram grandes modelos macroeconômicos estimados para estudar diferentes questões e também para prescrever políticas econômicas. Lucas (1976a) argumentou que o uso desses modelos macroeconométricos de grande escala para comparar políticas alternativas era injustificado. Nesses exercícios, os economistas pegavam um modelo estimado e o simulavam para políticas alternativas de interesse. Com base nessas simulações e usando uma função objetivo que caracteriza as preferências da autoridade de política, esses economistas poderiam derivar qual é a melhor dessas políticas alternativas. Normalmente, as preferências da autoridade de política eram representadas por funções de perda quadráticas que tinham como argumentos o quadrado do desvio da inflação de um nível alvo, e o quadrado do desvio da produção (ou desemprego) de seu nível de equilíbrio ou pleno emprego. Isso nos diz que a autoridade de política não gosta de taxas de inflação diferentes de um nível alvo, e não gosta de produção (ou desemprego) diferente de seu nível de equilíbrio ou pleno emprego. Tendo os caminhos simulados das variáveis econômicas de cada uma das políticas alternativas em consideração, esses economistas poderiam calcular o valor da função de perda e escolher a política que entregava a menor perda.

O problema desse exercício, segundo Lucas (1976a), era que os parâmetros estimados são considerados invariantes à política considerada - o mesmo conjunto de parâmetros foi usado para simular o modelo para as diferentes políticas. No entanto, nem todas as relações macroeconômicas nesses modelos macroeconométricos de grande escala eram estruturais, ou seja, invariantes às mudanças de política. Um exemplo de tal relação é a curva de Phillips, que vários economistas, de Friedman a Edmund Phelps e Lucas, argumentaram que é deslocada por mudanças de política. Portanto, a possibilidade de ter relações não invariantes à política (chamadas de "relações de forma reduzida") nesses modelos macroeconométricos de grande escala invalidou esses exercícios de escolha das políticas que minimizavam a função de perda. Para Lucas e Sargent (1979, 2), as dificuldades dos modelos macroeconométricos keynesianos eram "fatais: que os modelos macroeconômicos modernos não têm valor para orientar a política e que essa condição não será remediada por modificações ao longo de qualquer linha que esteja sendo perseguida atualmente". No entanto, os modelos macroeconométricos de grande escala continuaram a ser desenvolvidos apesar da crítica de Lucas (Goutsmedt et al. 2019).

Portanto, as expectativas racionais vieram para a macroeconomia em modelos destinados a combater os modelos keynesianos predominantes. A "revolução das expectativas racionais" trouxe consigo uma hipótese controversa que estava ligada a resultados controversos: que políticas sistemáticas não têm efeitos sobre variáveis reais, e que a análise de políticas baseada em modelos estimados era insustentável. Não surpreendentemente, um debate acalorado começou cedo.

\subsubsection{\textbf{O Suporte Não Acadêmico das Expectativas Racionais}}

A fortaleza de Minnesota da hipótese das expectativas racionais acabou sendo muito estratégica para garantir a essa hipótese um lugar central na macroeconomia. Enquanto os trabalhos acadêmicos escritos por Sargent e Wallace, além daqueles de Lucas e outros, apoiavam as expectativas racionais, o impacto que causou no debate de políticas também foi estratégico para apoiar os novos modelos propostos em Minnesota, Carnegie Mellon e Chicago. E para este aspecto muito específico, uma característica chave foi a simbiose eventualmente estabelecida entre a Universidade de Minnesota e o Federal Reserve Bank de Minneapolis na década de 1960.

Em 1965, John Kareken (PhD MIT 1956 sob Paul Samuelson; professor da Universidade de Minnesota durante toda a sua carreira) era consultor e conselheiro econômico do Federal Reserve Bank de Minneapolis, quando o presidente de Minneapolis era Hugh Galusha. Kareken também atuou por muitos anos como membro do Comitê Federal de Mercado Aberto do Conselho de Governadores do Sistema da Reserva Federal. Segundo Wallace, Kareken queria construir "um modelo da macroeconomia que seria útil para os formuladores de políticas" e foi fundamental para trazer dois outros economistas de Minnesota para o Fed de Minneapolis (Clement 2013). Em 1970, o Banco "lançou um grande programa de pesquisa explorando a melhor maneira de o Comitê Federal de Mercado Aberto (FOMC) fazer a política monetária. Em 1970 e 1971, respectivamente, Neil Wallace e Thomas Sargent, professores de economia da Universidade de Minnesota, ingressaram no departamento de pesquisa do Banco como conselheiros econômicos para auxiliar nesse programa" (Federal Reserve Bank de Minneapolis 1977, 3) - em 1971 o Fed de Minneapolis tinha um novo presidente, Bruce K. MacLaury. Wallace obteve seu PhD em Chicago em 1964 (sendo colega de classe de Lucas) e foi professor em Minnesota desde 1964. Sargent havia sido pesquisador associado em Carnegie Mellon em 1967 e tornou-se professor associado em Minnesota em setembro de 1971.

A partir desse momento, o Fed de Minneapolis e a Universidade de Minnesota iniciaram uma colaboração muito próxima. Conferências e publicações patrocinadas pelo Fed não apenas fomentaram a pesquisa acadêmica, mas também tentaram fazer com que ela influenciasse as decisões de política. Isso foi possível graças a várias reformas no Sistema da Reserva Federal, como argumentaram Michael Bordo e Prescott (2019). O Acordo do Tesouro-Fed de 1951 deu à Reserva Federal independência monetária porque encerrou os anos de fixação da taxa de juros para o financiamento de guerra e fez de William McChesney Martin o 9º presidente do Fed. No entanto, o papel dos bancos regionais era menor porque o FOMC se reunia irregularmente e as decisões eram tomadas por um comitê composto pelo Fed de Nova York e alguns outros membros. Mas, em meados da década de 1950, o presidente Martin reformou o FOMC e fez com que a política monetária fosse decidida por todo o comitê, dando mais poder a outros bancos regionais do que o de Nova York. Além disso, a partir do final da década de 1950, houve uma crescente profissionalização do Fed, com um número crescente de economistas na liderança em todo o sistema. De acordo com Bordo e Prescott (2019), essas mudanças incentivaram os bancos regionais a inovar na política monetária. O primeiro caso de sucesso foi o Fed de St. Louis, que ficou conhecido como um promotor de ideias monetaristas (de economistas como Karl Brunner, Friedman, Allan Meltzer e Anna Schwartz).

O segundo caso de sucesso foi o Fed de Minneapolis e a disseminação de ideias de expectativas racionais no debate sobre formulação de políticas. Os resultados dos trabalhos de Wallace e Sargent no Fed levaram o Fed a patrocinar uma conferência em 1974 sobre Expectativas Racionais e Política Macroeconômica (Toma e Toma 1985, 183). Nesta conferência, Sargent e Wallace apresentaram uma versão anterior de um artigo que publicaram em 1976 em um jornal líder, o Journal of Monetary Economics, popularizando o então recente trabalho de macroeconomia sobre expectativas racionais (Sargent e Wallace 1976; seu working paper foi lançado em 1974). Este foi apenas o início de uma dinâmica regular de pesquisa entre o Fed de Minneapolis e a Universidade de Minnesota. Em 1975, o Fed conduziu uma série de seminários sobre a formulação de políticas do FOMC, e outro working paper de Sargent e Wallace foi produzido. Neste mesmo ano, Sargent também apresentou sua pesquisa como um working paper e depois a publicou posteriormente em um jornal líder, desta vez o Journal of Political Economy (Sargent 1976). E uma grande conferência foi organizada em 1978 por Kareken e Wallace intitulada "Modelos de Economias Monetárias". Para eles, não foi um acidente que o Fed de Minneapolis organizou esta conferência: ele a patrocinou "precisamente porque alguns de seus pesquisadores eram tão céticos sobre os modelos macroeconômicos, ou sobre as implicações de política monetária e fiscal desses modelos. A esperança era que o Banco pudesse, ao financiar a conferência, ajudar no desenvolvimento de modelos mais satisfatórios" (Kareken e Wallace 1980, 1).

O ano de 1974 também marcou uma mudança importante nas publicações do Fed de Minneapolis. Como explicaram Toma e Toma (1985, 183), "começou a publicar seu Ninth District Quarterly. Em vez de relatórios estatísticos [como típico das publicações anteriores deste Fed], alguns artigos neste jornal abordavam questões de política monetária. O Quarterly citou a literatura das expectativas racionais pela primeira vez em um artigo, 'Política Monetária em Águas Inexploradas', que representava observações de um discurso do presidente do Federal Reserve de Minneapolis, Bruce Maclaury, em setembro de 1975. (...) Em 1977, o Fed de Minneapolis nomeou um novo presidente, Mark Willes, e introduziu um novo periódico, The Quarterly Review, que continha uma declaração editorial declarando 'a nova publicação apresentará principalmente pesquisas econômicas voltadas para a melhoria da formulação de políticas pelo Sistema da Reserva Federal e outras autoridades governamentais'."

Willes tornou-se famoso na época por ser um defensor franco das expectativas racionais nos círculos de política. Tendo obtido um PhD pela Universidade de Columbia em 1967 (com a tese "Os Atrasos Internos da Política Monetária", sob Giulio Pontecorvo), ele se tornou professor assistente na Wharton School da Universidade da Pensilvânia e logo depois consultor do departamento de pesquisa do Federal Reserve Bank na Filadélfia (e seu diretor em 1970). Em 1971, ele se tornou seu primeiro vice-presidente aos 30 anos (um cargo encarregado das operações do Fed, não relacionado ao departamento de pesquisa). Seis anos depois, em abril de 1977, ele se tornou o presidente mais jovem de um banco regional, no Fed de Minneapolis. Ele chegou lá quando o programa de pesquisa de Wallace e Sargent, sob Kareken, estava a todo vapor, e ele se fascinou pela nova teoria das expectativas racionais. Ele não apenas apoiou seu desenvolvimento adicional no departamento de pesquisa, mas também o levou ao debate de políticas e às decisões do FOMC em Washington, DC em 1978-1979. O jornal Star Tribune de Minneapolis em outubro de 1978 caracterizou Willes e sua defesa das expectativas racionais desta maneira:

Mark Willes, o jovem presidente de fala mansa do Federal Reserve de Minneapolis, dificilmente se encaixa na imagem de um rebelde. (...) Ele é uma aura, em suma, de reserva convencional.

Mas Willes, ignorando a noção tradicional de que os presidentes dos bancos regionais da reserva devem ser vistos e não ouvidos, está fazendo ondas altas o suficiente para ser identificado em grandes publicações de negócios nacionais como um dos principais críticos da política monetária do Federal Reserve. (Youngblood 1978, 13C)

Willes não apenas foi vocal em apoiar a controversa teoria das expectativas racionais, como também ficou conhecido por ser um dissidente recorrente da maioria do FOMC, sete vezes em seu mandato de um ano (veja também Sparsam e Pahl 2022). Além disso, ele não teve receio de usar o nexo academia-política para promover a teoria das expectativas racionais em diferentes redes, incluindo acadêmicas. Ele escreveu alguns artigos no Minneapolis Fed Quarterly Review, discutiu expectativas racionais em seus discursos, incluindo uma palestra em novembro de 1977 na Universidade de Minnesota ("Política Econômica Efetiva: Alguns Requisitos-Chave"), e em um artigo intitulado "O Futuro da Política Monetária", apresentado em dezembro de 1979 na Reunião Anual da Allied Social Science Association (a principal reunião de economia nos Estados Unidos). Na palestra de 1977, ele usou um argumento que parece ecoar a "visão de tomada de decisão econômica de Lucas (1976b, 62) na qual nossos concidadãos também participam de forma informada:"

Em uma democracia como a que temos nos Estados Unidos, é claro que as políticas econômicas que são seguidas devem ter o apoio da maioria dos cidadãos. Não importa quão "correta" uma determinada política econômica possa ser do ponto de vista teórico, se ela não tiver o apoio geral do povo, ela não funcionará.

Concluindo que:

Políticas baseadas nas teorias [de Keynes] não podem lidar efetivamente com as condições econômicas atuais. De alguma forma, a visão monetarista de Milton Friedman não parece completamente convincente. Talvez os expectativistas racionais aqui na Universidade de Minnesota tenham a resposta final. Neste ponto, apenas o Céu, Neil Wallace e Tom Sargent sabem com certeza. Devo dizer que estou muito impressionado com o trabalho deles; eles fizeram uma contribuição substancial - e estimulante - para a pesquisa sendo feita no Fed de Minneapolis. Só espero que ela receba o benefício de um debate aberto e genuíno para que seja confirmada e estendida ou substituída por algo melhor.

Foi neste ambiente de forte simbiose que mais um artigo foi escrito e amplamente divulgado: "Depois da Macroeconomia Keynesiana" de Lucas e Sargent (1979). Sargent escreveu um artigo intitulado "A Economia Keynesiana é um Beco sem Saída?" para a reunião de novembro de 1977 da Associação de Economia de Minnesota - uma associação criada em 1961 composta pelo Fed de Minneapolis e várias universidades de Minnesota. Lucas reformulou como o argumento foi apresentado e produziu seus artigos conjuntos para a conferência de 1978 patrocinada pelo Federal Reserve Bank de Boston "Depois da Curva de Phillips: Persistência de Alta Inflação e Alto Desemprego". O mesmo texto de Lucas e Sargent foi então reimpresso em 1979 pelo Fed de Minneapolis. Este artigo animou o debate sobre expectativas racionais em círculos acadêmicos e de formulação de políticas. Um ano depois, Willes deixou o Fed de Minneapolis para se juntar à gigante alimentícia General Mills, Inc. Até então, a estagflação havia piorado em todo o mundo e o Fed tinha então um presidente, Paul Volcker, que compartilhava a visão de Willes de que a inflação era o maior problema econômico da nação. Mas não foi nos círculos de formulação de políticas que a hipótese das expectativas racionais foi debatida, e houve importantes críticas levantadas na década de 1970 por economistas acadêmicos.

\subsubsection{\textbf{Algumas Reações Acadêmicas Iniciais à Hipótese das Expectativas Racionais}}

Conforme relatado por De Vroey (2016, 212, fn. 11), logo em 1968, após Lucas apresentar sua pesquisa com Prescott que levou ao artigo deles de 1971, Lucas narrou em uma carta para Prescott o que aconteceu no seminário:

O resto do tempo foi gasto na suposição de expectativas racionais: o que significa, se é razoável, e assim por diante. No geral, acho que conquistei muito poucos convertidos para a nossa visão sobre isso: a maioria das pessoas achava que era uma suposição irracional - e que nós "trapaceamos" ao postular equilíbrio em cada ponto no tempo.

Dois anos depois, Lucas apresentou uma versão anterior de seu artigo de 1972 (Lucas [1972a] 1981) em uma conferência organizada pelo Comitê de Consultores para Pesquisa de Preços do Conselho de Governadores do Sistema da Reserva Federal e pelo Comitê de Estabilidade Econômica do Conselho de Pesquisa em Ciências Sociais, intitulado "A Econometria da Determinação de Preços" (cujas atas foram publicadas como Eckstein 1972). Nessa conferência, a hipótese das expectativas racionais de Lucas foi contestada. O economista de Yale, James Tobin (1972, 13), argumentou que as expectativas racionais eram uma suposição muito forte, e que os tomadores de decisão reais nas economias reais não são capazes de fazer cálculos sofisticados para basear suas decisões:

Os participantes [da economia] não só devem receber a informação correta sobre a estrutura [da economia], mas também devem usar todos os dados corretamente na estimativa de preços e na tomada de decisões quantitativas. Esses participantes devem ser melhores econometristas do que qualquer um de nós na Conferência.

A ideia de que dados de pesquisa sobre expectativas de agentes no mundo real poderiam ser usados para testar a hipótese das expectativas racionais foi posteriormente firmemente rejeitada por Prescott (1977, 30; destaque no original):

O paradigma das expectativas racionais pode ser considerado no mesmo espírito que a suposição de maximização, uma vez objeto de muito debate em economia, mas agora considerado fundamental. A suposição das expectativas racionais aumentou a suposição de maximização ao hipotetizar que os agentes usam seus conjuntos de informações de maneira eficiente ao maximizar. Como a utilidade, as expectativas não são observadas, e pesquisas não podem ser usadas para testar a hipótese das expectativas racionais. Só se pode testar se alguma teoria, seja ela incorpora expectativas racionais ou, por falar nisso, expectativas irracionais, é ou não consistente com as observações.

Na mesma conferência de 1970, o professor de economia do MIT, Franklin M. Fisher, argumentou que em casos em que uma política implementada permanece a mesma por um longo período, poderia ser razoável esperar que os agentes tenham, em média, expectativas corretas, mas isso não seria o caso no curto prazo após uma mudança de política:

Uma visão adequada das expectativas racionais parece-me ser que o limite do valor esperado dos preços esperados é o mesmo que o limite do valor esperado dos preços reais, dado que não há mudanças de política. No entanto, não vejo motivo pelo qual os dois valores esperados devem ser iguais em todos os momentos. (Fisher 1972, 113)

Mais tarde, as mesmas críticas de Tobin foram expressas por Robert Shiller (1978) e Benjamin Friedman (1979) - que falaram sobre as suposições extremas de informação dos modelos de expectativas racionais (veja De Vroey 2016, 212-213). Para Friedman (1979), havia duas maneiras possíveis de interpretar as expectativas racionais. Na primeira, a interpretação fraca, presume-se que os agentes utilizam eficientemente as informações disponíveis. Na segunda, a forte, os agentes usam a estrutura do modelo para formar suas expectativas. O problema, segundo Friedman (1979), era que nada é dito sobre como os agentes aprendem sobre a estrutura da economia.

Em uma veia diferente, Franco Modigliani (1977), comentando um artigo de Prescott (1977), argumentou que enquanto as expectativas racionais podem descrever bem alguns mercados particulares, como os mercados especulativos, é mais duvidoso para outros mercados, como o mercado de trabalho. Portanto, a relevância ampla dessa hipótese para análises macroeconômicas e de políticas é injustificada.

A maioria dessas reações contra as expectativas racionais acabou sendo sobre visões de como o mundo econômico funciona, como modelá-las e como os agentes decidem no mundo real. Alguns fatos empíricos foram trazidos à mesa, como "a alta correlação serial no desemprego ou nos desvios da produção do 'pleno emprego'", como Modigliani (1977, 89) fez. Do lado da defesa, Lucas e Sargent (1979, 1) usaram a estagflação (recessão econômica acompanhada de alta inflação) dos anos 1970 como um argumento contra o consenso keynesiano a ser substituído pela nova teoria que estavam propondo. Segundo eles, naquele período, o governo teve "déficits orçamentários massivos e altas taxas de expansão monetária" sem causar crescimento econômico e baixas taxas de desemprego: "Que essas previsões estavam terrivelmente incorretas e que a doutrina na qual se baseavam é fundamentalmente falha são agora simples questões de fato que não envolvem novidades na teoria econômica."

Então outra discussão empírica importante foi sobre a forma da curva de Phillips: uma curva vertical era a consistente com os modelos de expectativas racionais. Aqui vemos mais testes econométricos formais sendo desenvolvidos, como o próprio Lucas (1973) iniciou mostrando que dados internacionais indicavam que não existia trade-off entre inflação e produção (ou seja, a curva de Phillips seria vertical).

Outra discussão econométrica se concentrou no resultado da ineficácia da política, uma implicação central dos modelos de expectativas racionais do início dos anos 1970. Sargent (1976, 236) usou um modelo de expectativas racionais e realizou testes empíricos que mostraram apoio à afirmação de que as variáveis de política econômica "não causam desemprego ou a taxa de juros". Barro (1978) também argumentou que a evidência empírica é que apenas mudanças não antecipadas na política monetária têm efeitos reais. Tais descobertas foram questionadas por outros trabalhos empíricos: Boschen e Grossman (1982) e Mishkin (1982), entre outros, realizaram testes empíricos que indicaram que mudanças antecipadas na política monetária afetam variáveis econômicas reais.

Os debates empíricos sobre modelos de expectativas racionais continuaram ao longo dos anos, com novas técnicas econométricas (como testes de raiz unitária e modelos de autorregressão vetorial de econometria de séries temporais) chegando para moldá-lo. Mas isso aconteceu quando a hipótese das expectativas racionais não era mais contestada e passou a ser usada por economistas anti-keynesianos e keynesianos.

\subsubsection{\textbf{A Incorporação das Expectativas Racionais}}

O destino da hipótese das expectativas racionais estava emaranhado com os resultados controversos a ela associados e promovidos academicamente e nos círculos de formulação de políticas: o resultado da ineficácia da política e a crítica de Lucas. Mas já na segunda metade dos anos 1970, o resultado da ineficácia da política se desvinculou dela, esse resultado dependia, em vez disso, do equilíbrio com flexibilidade de preços que veio junto com as expectativas racionais nos trabalhos de Lucas, Sargent, Wallace e outros. Essa cisão permitiu que os economistas que entendiam que as políticas sistemáticas têm efeitos reais também adotassem a hipótese das expectativas racionais. Alan Blinder (citado em Duarte 2012, 205-206) resumiu bem isso:

Demorou um pouco, e alguma ajuda de Fischer (1977) e Phelps e Taylor (1977), para a profissão entender claramente que as expectativas racionais (RE) são uma suposição sobre o comportamento que pode estar certa ou errada, mas que está logicamente desconectada da hipótese de que os preços se movem instantaneamente para limpar os mercados. É mais a partir do último do que do primeiro que a nova economia clássica (...) deriva suas implicações distintivas [como o resultado da ineficácia da política].

Separar essas duas ideias ajudou a espalhar o evangelho das RE, uma vez que os testes econométricos formais da hipótese conjunta de RE e limpeza de mercado quase sempre a rejeitavam. A maioria dos economistas tinha uma forte suspeita de que a hipótese de limpeza de mercado era o elo fraco na parceria.

Como resultado, tanto os economistas anti-keynesianos quanto os keynesianos abraçaram a hipótese das expectativas racionais que se tornou um ponto de referência central na macroeconomia a partir dos anos 1980 (Duarte 2012, De Vroey 2016). Os chamados modelos de Ciclo de Negócios Reais, desenvolvidos por Prescott (em Minnesota), Fin Kydland e outros continuaram a agenda de Lucas, Sargent e Wallace de entender as flutuações econômicas de curto prazo como resultado das escolhas ótimas de agentes com expectativas racionais e em equilíbrio, em um ambiente de preços flexíveis e choques estocásticos. Por outro lado, uma coorte mais jovem de economistas keynesianos, chamados de novos keynesianos, também adotou a hipótese das expectativas racionais, mas em um ambiente de preços rígidos em que o governo pode implementar políticas que melhoram o bem-estar.

Ambos os grupos são seguidores de Lucas não apenas porque desenvolveram modelos com expectativas racionais, mas também porque adotaram a crítica de Lucas como um requisito metodológico na formulação de modelos econômicos: se o modelador constrói relações macroeconômicas a partir das escolhas dos agentes individuais, então os modelos macroeconômicos terão microfundamentos que supostamente os tornam imunes à crítica de Lucas - em termos gerais, ao comparar políticas alternativas, os agentes levarão essas políticas em consideração e ajustarão seu comportamento de acordo, evitando o problema de usar relações macroeconômicas de forma reduzida que Lucas (1976a) criticou (veja Goutsmedt et al. 2019). E as expectativas racionais se tornaram parte integrante desse requisito de microfundamentação que molda a macroeconomia mainstream moderna com seus modelos de equilíbrio geral dinâmico estocástico (DSGE) do presente.

Isso não quer dizer que não existiam alternativas às expectativas racionais na macroeconomia. Com certeza, essa foi e é uma hipótese muito contestada por vários macroeconomistas. No entanto, essa hipótese se tornou central para a macroeconomia mainstream, um benchmark ensinado a todos os estudantes de pós-graduação. Por sua vez, isso não implica que os macroeconomistas mainstream não desenvolveram modelos com racionalidade limitada e limitada ao longo dos anos. Duas áreas receberam maior atenção, especialmente após a crise financeira de 2008: macroeconomia experimental e comportamental. A primeira é um subcampo da economia experimental que usa métodos de laboratório controlados para estudar fenômenos econômicos agregados e testar suposições e implicações de modelos macroeconômicos. A segunda é uma grande área que considera comportamentos não racionais em modelos macroeconômicos, incluindo agentes com habilidades cognitivas limitadas, processos de aprendizagem, modelos baseados em agentes que especificam a rede de interações entre agentes heterogêneos e os resultados agregados resultantes, entre outras coisas. Em ambos os casos, a hipótese das expectativas racionais é contestada, seja porque não é uma descrição empiricamente adequada da tomada de decisão no mundo real, ou porque dispensá-la permite que os modelos macroeconômicos descrevam melhor fatos importantes do mundo real, uma dualidade que acompanhou essa hipótese desde sua criação na macroeconomia.


\end{document}