\documentclass[a4paper,12pt]{article}[abntex2]
\bibliographystyle{abntex2-alf}
\usepackage{siunitx} % Fornece suporte para a tipografia de unidades do Sistema Internacional e formatação de números
\usepackage{booktabs} % Melhora a qualidade das tabelas
\usepackage{tabularx} % Permite tabelas com larguras de colunas ajustáveis
\usepackage{graphicx} % Suporte para inclusão de imagens
\usepackage{newtxtext} % Substitui a fonte padrão pela Times Roman
\usepackage{ragged2e} % Justificação de texto melhorada
\usepackage{setspace} % Controle do espaçamento entre linhas
\usepackage[a4paper, left=3.0cm, top=3.0cm, bottom=2.0cm, right=2.0cm]{geometry} % Personalização das margens do documento
\usepackage{lipsum} % Geração de texto dummy 'Lorem Ipsum'
\usepackage{fancyhdr} % Customização de cabeçalhos e rodapés
\usepackage{titlesec} % Personalização dos títulos de seções
\usepackage[portuguese]{babel} % Adaptação para o português (nomes e hifenização
\usepackage{hyperref} % Suporte a hiperlinks
\usepackage{indentfirst} % Indentação do primeiro parágrafo das seções
\sisetup{
  output-decimal-marker = {,},
  inter-unit-product = \ensuremath{{}\cdot{}},
  per-mode = symbol
}
\DeclareSIUnit{\real}{R\$}
\newcommand{\real}[1]{R\$#1}
\usepackage{float} % Melhor controle sobre o posicionamento de figuras e tabelas
\usepackage{footnotehyper} % Notas de rodapé clicáveis em combinação com hyperref
\hypersetup{
    colorlinks=true,
    linkcolor=black,
    filecolor=magenta,      
    urlcolor=cyan,
    citecolor=black,        
    pdfborder={0 0 0},
}
\usepackage[normalem]{ulem} % Permite o uso de diferentes tipos de sublinhados sem alterar o \emph{}
\makeatletter
\def\@pdfborder{0 0 0} % Remove a borda dos links
\def\@pdfborderstyle{/S/U/W 1} % Estilo da borda dos links
\makeatother
\onehalfspacing

\begin{document}

\begin{titlepage}
    \centering
    \vspace*{1cm}
    \Large\textbf{INSPER – INSTITUTO DE ENSINO E PESQUISA}\\
    \Large ECONOMIA\\
    \vspace{1.5cm}
    \Large\textbf{Tradução Tópico 8.1 - HPE}\\
    \vspace{1.5cm}
    Prof. Pedro Duarte\\
    Prof. Auxiliar Guilherme Mazer\\
    \vfill
    \normalsize
    Hicham Munir Tayfour, \href{mailto:hichamt@al.insper.edu.br}{hichamt@al.insper.edu.br}\\
    4º Período - Economia B\\
    \vfill
    São Paulo\\
    Abril/2024
\end{titlepage}

\newpage
\tableofcontents
\thispagestyle{empty} % This command removes the page number from the table of contents page
\newpage
\setcounter{page}{1} % This command sets the page number to start from this page
\justify
\onehalfspacing

\pagestyle{fancy}
\fancyhf{}
\rhead{\thepage}

\section{\textbf{Morgan (2008)}}
\subsection{\textbf{Classificação JEL B4}}
A modelagem se tornou a metodologia dominante da economia durante o século 20.

No entanto, apesar de seu uso onipresente na economia moderna, o termo 'modelo' foi introduzido relativamente recentemente. No final do século 19, os 'modelos' nem sequer eram uma categoria reconhecida nas discussões sobre metodologia (como por exemplo no Dicionário de Economia Política de Palgrave dos anos 1890), embora alguns existissem como objetos de trabalho práticos. O uso efetivo do termo 'modelo' na economia está associado ao movimento de econometria do período entre guerras, um movimento cujo objetivo era tanto desenvolver quanto fundir abordagens matemáticas e estatísticas para a economia. Dessa noção original ampla, na década de 1950 surgiram campos separados de economistas matemáticos e econométricos, e ambos mantiveram a modelagem como uma ferramenta central de sua prática científica. Tornou-se convencional então pensar em modelos na economia moderna como objetos matemáticos usados na teoria econômica ou como objetos econométricos (envolvendo propriedades estatísticas e matemáticas) no trabalho empírico. As contas históricas de modelos na economia moderna podem começar convenientemente então com essa divisão.

Os comentários filosóficos, também, na maioria das vezes tendem a seguir essa divisão, tratando os modelos da teoria econômica como diferentes tipos de criaturas, com diferentes papéis, daqueles que são aplicados aos dados. O último papel dos modelos, o de 'adequar teorias ao mundo', é exemplificado na modelagem empírica, no trabalho econométrico e nas declarações metodológicas de Jan Tinbergen na década de 1930. Em contraste, os modelos matemáticos da economia moderna são vistos principalmente como uma maneira pela qual a construção da teoria econômica acontece. Essa visão de 'modelagem como teorização' é exemplificada nas declarações programáticas de Tjalling Koopmans em 1957. Um terceiro quadro metodológico apresenta 'modelos como instrumentos de investigação': ferramentas para aprender sobre a teoria econômica ou o mundo econômico, uma posição tipificada no trabalho de Irving Fisher no final do século 19 e início do século 20, que pode ser visto como outro dos fundadores da modelagem na economia.

Este artigo aborda o surgimento histórico e os papéis dos modelos na economia de acordo com essas três diferentes contas metodológicas, e discute como essas abordagens se encaixam na ciência econômica moderna.

\subsection{\textbf{Modelagem como Ajuste de Teorias ao Mundo}}
Embora uma vibrante comunidade de econometria tenha se desenvolvido nas duas décadas até a década de 1920, seus produtos (regressões de demanda, contas estatísticas de ciclos de negócios e assim por diante) foram apresentados como descrições diretas das relações econômicas subjacentes, em vez de como modelos propostos de forma tentativa para representá-los (veja Morgan 1990). A diferença é sutil, mas iluminada pelo uso do termo 'econométrico natural' de Philippe Le Gall (2007) para aqueles economistas do século 19 que acreditavam, em paralelo às ciências naturais, que as leis que governavam a economia eram escritas em matemática, e a manipulação inteligente de dados estatísticos (sem, deve-se dizer, muito em termos de técnicas analíticas) revelaria essas leis.

Nesse quadro estatístico descritivo, Jan Tinbergen não apenas introduziu o termo 'modelo' em 1935 (veja Boumans 1993), mas também foi responsável - juntamente com Ragnar Frisch - pelo desenvolvimento de tais objetos matemático-estatísticos na econometria da década de 1930. (Antes disso, o raro uso do termo 'modelo' normalmente se referia a modelos de objetos físicos como Boltzmann os definiu em 1911. Paul Ehrenfest é a provável fonte de uma ampliação no escopo do termo para incluir modelos matemáticos, e Tinbergen foi seu assistente durante a metade da década de 1920; veja Boumans 2005, cap. 2.) Frisch em 1933 desenvolveu - no contexto da pesquisa de ciclo de negócios - a noção de um 'esquema macrodinâmico': um modelo de três equações com erros aleatórios. Ele até simulou para mostrar que poderia reproduzir as características genéricas dos dados de séries temporais de seu tempo. Mas foi Tinbergen quem desenvolveu o design de Frisch em um modelo econométrico - um modelo que poderia ser ajustado aos dados reais da economia. Como é bem conhecido, ele construiu a primeira geração de modelos macroeconométricos (veja Tinbergen 1937; 1939; e Bodkin et al. 1991), e ao fazer isso ele tornou explícita a noção de um modelo como um veículo para preencher a lacuna entre teorias do ciclo de negócios e dados estatísticos específicos (tempo e lugar) do ciclo, como Morgan (1990, cap. 4) argumenta. Para apreciar a tarefa, é preciso lembrar que a maioria das teorias existentes do ciclo era expressa verbalmente, e as teorias matemáticas nascentes do ciclo eram muito pequenas e simplificadas para representar as características dos ciclos reais, então até mesmo construir um sistema de equações a partir dessas teorias era uma tarefa considerável. Os dados também desempenharam um papel na decisão da sequência temporal das relações e quais variáveis deveriam ser incluídas ou omitidas, pois ambos esses elementos foram determinados no trabalho estatístico. Em outras palavras, Tinbergen criou um conjunto de relações matemático-estatísticas utilizáveis que incorporavam tanto ideias teóricas sobre como a economia funcionava quanto representavam empiricamente as diferentes partes da economia. Tendo ajustado teorias e dados juntos no formato do modelo econométrico, ele então usou o modelo para testar a viabilidade de várias teorias do ciclo, para explicar eventos na economia e para executar o modelo para a frente com diferentes opções de política relevantes para os anos da Grande Depressão - tudo isso na era pré-computador usando calculadoras manuais! Essa 'nova prática' de modelos, como Boumans (2005) a denomina, envolveu uma construção criativa da teoria econômica matemática em relação aos dados estatísticos do mundo econômico e da habilidade artesanal no uso desses modelos. Para Frisch e Tinbergen, a modelagem era um projeto para explicar como o mundo econômico funcionava.

A próxima etapa na história pode ser marcada pelo famoso plano de Trygve Haavelmo para a econometria de 1944, que trouxe outra mudança sutil de foco para a tarefa de modelagem econométrica. Ele sugeriu que a econometria deveria se preocupar, não com um processo de correspondência entre teoria e dados em um processo iterativo, mas com a busca do modelo correto para os dados observados usando raciocínio de probabilidade (veja Morgan 1990, cap. 8). Ele efetivamente introduziu na econometria não apenas a noção do modelo teórico (o modelo matemático derivado da teoria a priori), mas também a do modelo 'verdadeiro' (mas desconhecido): 'o mecanismo 'verdadeiro' sob o qual os dados considerados estão sendo produzidos' (Haavelmo 1944, p. 49). No entanto, ele estava longe de ser um 'econométrico natural' (no sentido de Le Gall para o século 19), argumentando sobre modelos de comportamento econômico que 'seja qual for as 'explicações' [dos fenômenos econômicos] que preferimos, não deve ser esquecido que todos são nossas próprias invenções artificiais em busca de uma compreensão da vida real; eles não são verdades ocultas a serem "descobertas"' (Haavelmo 1944, p. 3). Embora ele tenha insistido que um modelo econométrico bem ajustado (uma teoria que se ajusta bem aos dados) pode não ser o modelo 'verdadeiro', no entanto, sua abordagem de probabilidade de plano foi destinada a alterar a tarefa aceita de econometria. A abordagem da Comissão Cowles que se seguiu (cujas contribuições são analisadas por Qin 1993, e Epstein 1987) enfatizou o uso dos métodos corretos de identificação e estimação do modelo estrutural completo derivado teoricamente como o meio de descobrir esse modelo verdadeiro. O 'apriorismo forte' de sua abordagem para a econometria, na qual a teoria propõe o modelo e os dados dispõem (ou não) dessas hipóteses, provocou o famoso debate 'medição sem teoria' com o ramo mais empirista do campo sobre como fazer econometria no final da década de 1940.

É tentador ver a provisão de Haavelmo de uma base filosófica para a econometria como pavimentando o caminho para uma divisão de trabalho pós-1950 no uso de modelos - ou seja, os economistas fornecem modelos matemáticos da teoria econômica, e o trabalho do econométrico é usar estatísticas para estimativa de modelos e teste de teorias. Até certo ponto, essa divisão de trabalho é confirmada, pois é neste período que surge uma distinção muito mais clara entre economia teórica e aplicada (como visto em Backhouse 1998). No entanto, apesar da retórica da econometria pós-1950 que fala de 'confrontar teoria com dados', ou 'aplicar teoria aos dados', do ponto de vista da modelagem econométrica, a divisão prática não é tão clara. Existem várias razões. Primeiro, permanece um comentário prosaico, mas geralmente válido, que a teoria raramente fornece todos os recursos necessários para fazer modelos que podem ser imediatamente aplicados aos dados do mundo. É precisamente por isso que os modelos econométricos se destacaram como um intermediário necessário, um dispositivo de correspondência, entre eles. Segundo, esse processo de correspondência de ajustar teorias ao mundo é feito com muitos propósitos diferentes - para testar teorias, para medir relações, para explicar eventos, e assim por diante - cada um precisando de recursos diferentes da teoria e com critérios diferentes. Terceiro, não existem regras científicas gerais para modelagem. Houve argumentos acirrados dentro da comunidade de econometria nas últimas décadas sobre vários princípios científicos para modelagem (e critérios associados): se os modelos devem ser orientados pela teoria ou orientados pelos dados; se o processo de modelagem deve ser simples para geral ou geral para específico; e assim por diante (veja Pagan 1987; Heckman 2000). Independentemente de quais princípios são seguidos, o elemento criativo ainda é muito evidente onde quer que ocorra a modelagem econométrica aplicada, seja essa modelagem no extremo de busca de padrões ou modelagem orientada pela teoria, e seja o campo macro ou microeconométrico.

Uma mudança de foco mais recente, particularmente no campo macroeconométrico e associada a Robert Lucas, implica em desistir do objetivo de usar a teoria para fazer modelos que representem a verdadeira estrutura geral como uma maneira de descobrir essa estrutura. Como ele escreveu:

Uma 'teoria' não é uma coleção de afirmações sobre o comportamento da economia real, mas sim um conjunto explícito de instruções para construir um sistema paralelo ou análogo - uma economia de imitação mecânica. Um 'bom' modelo, deste ponto de vista, não será exatamente mais 'real' do que um pobre, mas fornecerá melhores imitações. (Lucas 1980, p. 697)

Essa mudança altera a relação entre modelos e teoria, pois agora a tarefa da teoria é produzir modelos como analogias do mundo, em vez de usá-los para explicar o comportamento do mundo (veja Boumans 1997). Ao mesmo tempo, muda o foco do 'ajuste': o objetivo não é mais ajustar a teoria ao mundo, mas ajustar o modelo ao mundo no sentido particular de ser capaz de imitar certos tipos de características de dados.

Outra conta recente, desenvolvida desta vez na microeconometria por John Sutton (2000), se valida em relação à agenda econométrica anterior mantida por Frisch e Tinbergen, pois, como esses pioneiros iniciais, ele pensa em modelos não como dispositivos para a descoberta do modelo geral verdadeiro como na interpretação da Comissão Cowles do projeto de Haavelmo, nem como máquinas matemáticas que imitam o mundo como na conta de Lucas, mas como dispositivos investigativos para descobrir sobre o mundo. Na visão de Sutton, o mundo econômico produz regularidades ou variabilidades razoavelmente estáveis apenas dentro de uma classe de casos, não em todos os casos; portanto, procurar um modelo geral é muito ambicioso. O objetivo da modelagem é descrever os mecanismos econômicos que produzem as características dos dados que são compartilhadas dentro de um subconjunto de todos os casos e, assim, explicar as regularidades observadas dentro dessa subclasse. Sutton descreve isso como uma abordagem de 'classe de modelos'. Mais uma vez, os modelos aparecem como um dispositivo intermediário entre a teoria e os dados, mas desta vez funcionam para classificar casos semelhantes no mundo e, assim, oferecer explicações para seu comportamento característico.

Os modelos aparentemente desempenham um papel epistemológico crítico na econometria - mas existem diferentes maneiras de caracterizar isso. A econometria pode ser vista como cumprindo a função de experimentos de laboratório em algumas outras ciências - uma afirmação que está implícita na discussão de Haavelmo sobre os dados da economia como sendo o resultado da observação passiva dos experimentos da natureza e explícita em sua discussão sobre modelagem econométrica como projetando experimentos (veja Haavelmo 1944, cap. 1 e 2). Sua conceituação de econometria apela para a importância da probabilidade e do raciocínio estatístico como as bases para o design do modelo e a inferência estatística: os modelos têm que ser projetados para corresponder aos dados que poderiam ser observados e ser enquadrados em termos de probabilidade. A noção de 'design de experimentos' requer que o econométrico pense sobre o problema de ajuste, enquanto a configuração de probabilidade dá regras para inferências a partir do experimento do modelo, que são de fato muito melhor especificadas do que aquelas para experimentos de laboratório na maioria das ciências. Assim, o plano de Haavelmo compra explicitamente uma tradição de pensamento estatístico como um modo válido de raciocínio científico, mas o reinterpreta como uma forma de trabalho experimental.

Uma caracterização mais recente da função epistemológica dos modelos em econometria é entendê-los como instrumentos de observação e medição que permitem aos economistas identificar fenômenos estáveis no mundo da atividade econômica. A conta de Kevin Hoover de 'econometria como observação' descreve 'cálculos econométricos' como 'o telescópio do economista' (1994, p. 74) onde as regras para focar o telescópio vêm da teoria estatística e onde a teoria econômica, e o propósito engajado, guiam o processo de observação. Marcel Boumans (2005) entende os modelos como o instrumento primário neste processo, sem o qual os economistas não poderiam 'modelar o mundo em número'. Em vez de um meio de observação, ele retrata os modelos como instrumentos científicos complexos que geram os números para aqueles objetos econômicos, conceitos e relações que não podem ser observados diretamente e que ainda não são medidos. Como Haavelmo, a conta de trabalho do modelo de Boumans invoca um design cuidadoso de experimentos, mas ele fornece uma discussão mais concreta de como a modelagem econométrica fornece estruturas de medição para lidar com cláusulas ceteris paribus; como critérios estatísticos e outros fornecem maneiras de avaliar a confiabilidade dos instrumentos do modelo (via calibração, filtragem e assim por diante); e como a precisão e o rigor são obtidos no processo de medição.

Nem Boumans nem Hoover são instrumentalistas sobre modelos no sentido que passou a ser associado ao argumento de Milton Friedman de 1953 de que os modelos precisam ser eficientes apenas para previsão, não para explicação. (O ensaio de Friedman tem sido muito discutido, e as interpretações deste ponto particular variam; veja particularmente Hirsch e De Marchi 1990; instrumentalismo e operacionalismo; e Mäki 2007.) Nem são operacionistas no sentido Bridgmaniano (que informou, por exemplo, o trabalho inicial de Paul Samuelson em economia; veja Bridgeman 1927), ou seja, que um conceito é definido por seu processo de medição (como um modelo econométrico). Tanto Hoover quanto Boumans podem ser chamados de 'instrumentalistas sofisticados' pois consideram cálculos econométricos ou modelos como instrumentos habilmente projetados para observar e medir as relações da economia, e assim entender e explicar, o mundo.

\subsection{\textbf{Modelagem como Teorização}}
O termo 'modelo' raramente foi usado em economia antes da década de 1930, embora coisas que agora rotularíamos como 'modelos' tenham sido desenvolvidas e usadas para teorizar antes disso. Podemos certamente reconhecer alguns exemplos anteriores de modelagem no final do século 19; por exemplo, podemos felizmente denotar o diagrama de caixa de Edgeworth-Bowley, e os diagramas de comércio e tesoura de oferta-demanda de Alfred Marshall como modelos. Esses exemplos sinalizam que a modelagem era um elemento não reconhecido no processo de matematização desse período anterior (veja Morgan 2008). No entanto, foi apenas depois da década de 1950 que a modelagem se tornou uma maneira amplamente reconhecida de usar matemática na economia e se tornou uma das formas dominantes de teorização econômica. Enquanto o estabelecimento da noção estatístico-econométrica está associado a Tinbergen, o matemático-teorizante pode estar associado a outro econométrico holandês, Tjalling Koopmans, cuja conta, dada em um conjunto de três ensaios em 1957, é amplamente entendida como uma declaração paradigmática da abordagem de modelagem da economia matemática moderna. Koopmans desenvolveu as ideias anteriores de Tinbergen sobre modelagem para se ajustar às discussões contemporâneas sobre o papel da matemática na economia nas décadas de 1940 e 1950 e com a ideia formal matemática de um modelo naquela época. Como tal, sua declaração se encaixa em uma história mais ampla de matemática e economia tratada particularmente em Weintraub (2002) e Ingrao e Israel (1987).

Koopmans definiu uma teoria econômica como um conjunto de postulados com os quais raciocinamos para elaborar e tornar explícitos os efeitos, de outra forma implícitos, do conjunto de postulados tomados em conjunto: uma prática de raciocínio que aparentemente envolve modelos. Para Koopmans, esse raciocínio era uma parte importante da teorização, uma vez que essas implicações não são autoevidentes, nem qualquer conjunto particular de postulados é necessariamente frutífero. Sua representação de 'Teoria Econômica como uma Sequência de Modelos' (para citar seu título de seção de 1957, p. 142) é apresentada como sua resposta ao argumento contínuo de seu dia sobre o status das suposições e as previsões da economia, no qual ele explicitamente definiu o papel dos modelos quase como um aparte:

nem os postulados da teoria econômica são totalmente autoevidentes [como Robbins argumentou em 1932], nem as implicações de vários conjuntos de postulados são prontamente testadas pela observação [como Friedman argumentou em 1953]. Nesta situação, é desejável que organizemos e registremos nossas deduções lógicas de tal maneira que qualquer conclusão ou implicação refutável observacionalmente possa ser rastreada até os postulados em que se baseia ... Considerações desta ordem sugerem que olhemos para a teoria econômica como uma sequência de modelos conceituais que buscam expressar em forma simplificada diferentes aspectos de uma realidade sempre mais complicada. A princípio, esses aspectos são formalizados tanto quanto possível isoladamente, depois em combinações de realismo crescente. (Koopmans 1957, p. 142)

Koopmans sugere, então, que os modelos são um elemento essencial na teorização, e que seu papel vem em sua capacidade sequenciada de expressar diferentes e combinados aspectos de uma realidade simplificada. Mas sua projeção de que tal sequência de modelos representaria 'combinações de realismo crescente' parece não ter se concretizado. Embora a tratabilidade sugira que o realismo crescente em alguns aspectos terá que ser compensado pela simplificação em outros, a história da modelagem sugere que as sequências de modelos são mais frequentemente impulsionadas por mudanças em problemas, em questões e nas ferramentas matemáticas disponíveis. Esta última foi uma possibilidade que o próprio Koopmans discute no contexto da mudança de formas de teorização aritmética para diagramática para algébrica. E, como acabamos de notar com Lucas, algumas modelagens modernas não visam mais representar o mundo como ele é, mas desenvolver sistemas artificiais que imitam as saídas do mundo.

Existem várias maneiras de caracterizar o uso de modelos matemáticos na teoria econômica. Para Daniel Hausman, a conexão dos modelos com a formação de conceitos é tanto mais explícita quanto mais importante do que Koopmans sugere, pois a modelagem econômica é onde o desenvolvimento da teoria acontece:

Uma teoria deve identificar regularidades no mundo. Mas a ciência não avança principalmente por identificar correlações entre várias propriedades conhecidas das coisas. Um passo absolutamente crucial é construir novos conceitos - novas maneiras de classificar e descrever fenômenos. Grande parte da teorização científica consiste em desenvolver e pensar sobre tais novos conceitos, relacioná-los a outros conceitos e explorar suas implicações.

Esse tipo de empreendimento é particularmente proeminente na economia, onde os teóricos dedicam muito esforço para explorar as implicações da racionalidade perfeita, informação perfeita e competição perfeita. Essas explorações, que são separadas de questões de aplicação e avaliação, são, acredito eu, o que os economistas (mas não os econométricos) chamam de 'modelos'. (Hausman 1984, p. 13)

Hoje em dia, as explorações seriam em racionalidade limitada, informação imperfeita e competição imperfeita: a agenda avançou, mas o modo de teorizar por meio da modelagem permanece o mesmo. A atenção de Hausman ao papel dos modelos na inovação conceitual é dada credibilidade e profundidade em sua própria análise do modelo de 'gerações sobrepostas' de Samuelson, uma história sobre exploração criativa no reino teórico. A história da Caixa de Edgeworth (veja Humphrey 1996, e Morgan 2004a) fornece outro bom exemplo de como a modelagem está associada a novos conceitos e descrições - afinal, é onde as curvas de indiferença, curvas de contrato e assim por diante foram introduzidas pela primeira vez.

O desenvolvimento do diagrama de oferta e demanda que encontramos nos Princípios de Marshall (1890) exemplifica as afirmações de Hausman. Não é apenas que os diagramas de Marshall descrevem de novas maneiras algumas ideias mais antigas sobre os fenômenos de oferta e demanda que remontam muito tempo nas literaturas puramente verbais de economia, mas que em suas mãos essas curvas são moldadas para representar vários tipos de mercados e relações, resultando em novos conceitos e classificação de tipos de oferta ou demanda em um nível que fica entre qualquer teoria geral e casos únicos (veja Morgan 2002). É essa função de modelagem como um dispositivo de classificação que Sutton (2000) retoma em uma forma diferente em seu trabalho de 'classe de modelos' sobre competição industrial (discutido acima). E, historicamente entre esses dois economistas, podemos situar, como apenas um exemplo, o trabalho de Martin Shubik (1959) que usou modelos teóricos de jogos para classificar tipos de competição e estrutura da indústria de acordo com o tipo de jogo que mais se assemelha à situação econômica envolvida.

Hausman está ansioso para fazer sua conta da metodologia da economia não apenas se ajustar à prática da economia moderna, mas filosoficamente sensata, então ele separa a atividade de modelagem das afirmações e reivindicações de verdade mais gerais das teorias. À primeira vista, essa separação estrita pode parecer curiosa para os economistas que muitas vezes falam de 'testar modelos' em vez de teorias, e não se preocupam em separar as categorias de teorias e modelos em seu trabalho científico cotidiano. Essa confluência pode ocorrer porque, como Hausman sugere, 'Os modelos não são eles próprios aplicações empíricas, mas têm a mesma estrutura' (Hausman 1992, p. 80). Ter a mesma estrutura pode permitir a aplicação empírica por econométricos, embora essa não seja a maneira como os economistas usam principalmente modelos matemáticos para argumentar sobre o mundo: ao contrário, eles estão mais frequentemente ligados ao mundo de uma maneira muito mais casual.

De fato, 'aplicação casual' é exatamente o termo usado por Alan Gibbard e Hal Varian para descrever como os modelos matemáticos são aplicados 'para explicar aspectos do mundo que podem ser notados ou conjecturados sem técnicas explícitas de medição' (1978, p. 672). Em sua visão, os modelos matemáticos são projetados apenas para aproximar o mundo, e, ao contrário dos modelos econométricos que passam por um processo sério de ajuste ao mundo, eles estão casualmente conectados ao mundo por 'histórias' que interpretam os termos no modelo para elementos no mundo. Mas eles enfatizam que tais aplicações de modelos não se referem a situações ou coisas particulares no mundo. Em contraste, Hausman (1990) argumenta que os economistas costumam usar seus modelos dessa maneira para discutir eventos particulares do mundo real, e eles usam narrativas para preencher as descrições dadas no modelo para fornecer explicações desses eventos no mundo. Morgan (2001, 2007) assume uma posição mais forte em relação a essas histórias, sugerindo que elas formam uma parte integral da aplicação de modelos ao mundo - tanto em geral quanto para casos particulares - e igualmente formam uma parte essencial da identidade do modelo. Steven Rappaport (1998), como Hausman, acha que os modelos matemáticos são bastante flexíveis em função: em trabalho conceitual, em trabalho normativo (por exemplo, em discussões de política) e em trabalho explicativo heurístico. No entanto, em outros aspectos, a conta de Rappaport dos modelos e sua função contrasta com a de Hausman e com a de Morgan, pois ele retrata os modelos como 'mini-teorias' dentro de um programa de pesquisa que funcionam em formato contrafactual: ou seja, sua função é fornecer contas do que poderia acontecer se o modelo fosse uma descrição verdadeira do mundo.

Essas contas de como os modelos matemáticos se conectam ao mundo sugerem todos uma dependência de elementos cognitivos, intuitivos ou informais da teorização dos economistas em relação ao mundo, em forte contraste com os critérios estatísticos e econômicos que atendem à maneira como os econométricos usam modelos para ajustar teorias ao mundo. Por outro lado, os modelos matemáticos parecem cumprir uma variedade mais ampla de funções, variando de dispositivos para a formação de novos conceitos e trabalho classificatório na teorização a dispositivos de inferência que pretendem dar explicações de eventos gerais ou particulares. O uso de políticas geralmente envolve modelos matemáticos para análise de intervenções de política e para fins de design de mecanismos - como, por exemplo, no design de leilões. Até agora, há pouca literatura filosófica histórica ou reflexiva sobre este lado do trabalho do modelo (embora veja Guala 2001). Em contraste, há uma considerável literatura reflexiva sobre as atividades de política associadas a modelos empíricos ou econométricos (veja exemplos e referências em Den Butter e Morgan 2000).

\subsection{\textbf{Modelos como Instrumentos de Investigação}}

Já vimos várias maneiras pelas quais os modelos são entendidos como dispositivos de investigação. Nos comentários sobre econometria, encontramos modelos retratados como ferramentas ou instrumentos de observação e medição, e no trabalho econométrico inicial, os modelos também foram entendidos como ferramentas para ajudar a explicar o mundo. A ideia de modelos como instrumentos também está presente na literatura de modelagem matemática, mas está associada a um sentido mais ativo de investigação. Irving Fisher, para sua tese, construiu fisicamente um modelo analógico hidráulico de equilíbrio geral de três bens e três consumidores:

O mecanismo recém-descrito é o análogo físico do mercado econômico ideal. Os elementos que contribuem para a determinação dos preços são representados cada um com seu papel apropriado e abertos ao escrutínio do olho. Assim, somos capazes não apenas de obter uma imagem clara e analítica da interdependência dos muitos elementos na causação dos preços, mas também de empregar o mecanismo como um instrumento de investigação e por ele, estudar algumas variações complicadas que dificilmente poderiam ser seguidas com sucesso sem sua ajuda. (Fisher 1892, p. 44)

Isso combina bem com o comentário de Scott Gordon, que, a partir de sua análise histórica e filosófica da economia, afirma que 'o propósito de qualquer modelo é servir como uma ferramenta ou instrumento de investigação científica' (1991, p. 108).

A noção de ferramentas na economia não foi bem desenvolvida. Arthur Pigou (1929) introduziu a distinção entre 'fabricantes de ferramentas' e 'usuários de ferramentas', rotulando Francis Edgeworth como um fabricante de ferramentas, e Marshall como ambos, fabricante e usuário. Para Pigou, o termo 'ferramentas' não se referia a processos de indução em oposição à dedução, ou mesmo ao método matemático em oposição ao método literário, mas a algo que ele se referiu como um movimento analítico 'mais amplo' envolvendo técnicas estatísticas e matemáticas específicas (como o método de análise de demanda e oferta). Foi seguindo-o que Joan Robinson (1933), em comentários frequentemente citados, escreveu sobre a 'caixa de ferramentas da economia' que ela apresentou como consistindo em 'suposições' (teoria) e 'geometria' (métodos), embora possamos pensar mais naturalmente nesses elementos combinando para formar modelos. Koopmans (1957) também escreveu sobre ferramentas, referindo-se não apenas a exemplos numéricos e representações diagramáticas, mas também a matemática formal, técnicas de computação, análise de entrada-saída e assim por diante, assim (para o nosso tempo) misturando métodos ou modos de análise (aqueles que agora associamos à modelagem) e tipos de modelos. No entanto, há uma semelhança impressionante entre a maneira como Fisher se referiu e usou seu modelo hidráulico físico e a maneira como os economistas modernos usam seus modelos matemáticos equivalentes da economia moderna como ferramentas de investigação. Ambos parecem estar bem cobertos pelas noções de uso de ferramentas que Pigou introduziu.

De fato, a atenção às funções dos modelos enfatizou que grande parte do trabalho de classificação e desenvolvimento conceitual da teorização discutido na seção anterior ocorre não tanto na construção de modelos matemáticos quanto em seu uso. Por exemplo, os modelos desenvolvidos por Hicks, Samuelson, Meade e outros no final da década de 1930 com base na Teoria Geral de Keynes foram usados para explorar, desenvolver e entender essa teoria de maneiras que envolveram trabalho conceitual e classificatório substantivo de sua autoria (veja Darity e Young 1995). Ao derivar soluções para problemas teóricos, ou ao explorar os limites do comportamento implícito pelas relações teóricas representadas nos modelos, e ao aplicar seus modelos para pensar sobre problemas do mundo econômico representado no modelo, esses economistas usaram seus modelos como instrumentos de investigação. Essas investigações aparecem como experimentos de pensamento glorificados, muito complicados para fazer na mente e, portanto, exigindo uma representação do caso ou sistema na forma do modelo e modos matemáticos associados de raciocínio sobre ele. No caso de Fisher, ele tinha um objeto material para experimentar. Os modelos matemáticos em economia também normalmente fornecem tais recursos internos para manipulação experimental. Morgan (2002) argumenta o caso para considerar a atividade de modelagem matemática como trabalho experimental em modelos matemáticos em paralelo com experimentos estatísticos praticados em modelos econométricos. Mas enquanto temos regras estatísticas bem fundamentadas para fazer inferências a partir de experimentos econométricos, a aplicação de modelos matemáticos ao mundo (ou inferências de tais experimentos de modelo) é mais casual ou aproximada, como já vimos.

Essa noção de que o trabalho de modelagem matemática é uma forma de atividade experimental é mais evidente na literatura fundadora sobre simulação em economia por volta de 1960 (pesquisada na época por Shubik 1960a, b). Em alguns outros campos da ciência, a simulação foi introduzida principalmente como um método de solução numérica, em vez de analítica. Mas em economia, a simulação tem sido mais comumente apresentada e usada como um processo de experimento em modelos, um processo que efetivamente investiga de maneira sistemática toda a gama de comportamentos do sistema ou dos atores retratados no modelo. Houve exemplos isolados de simulação anteriormente na história da economia - mais particularmente a simulação de Tinbergen de 1936 de seu modelo macroeconométrico, a simulação de Paul Samuelson (1939) de um pequeno sistema matemático keynesiano e os famosos modelos de choque aleatório de Eugen Slutsky (1927) que imitavam ciclos de negócios. As possibilidades de simulação foram então exploradas de maneira mais eficaz durante as atividades da Guerra Fria das décadas de 1950 e 1960 que reuniram as ciências sociais e a matemática.

O nascimento da simulação em economia geralmente tem sido atribuído a Herbert Simon, mas igualmente importantes foram os desenvolvimentos simultâneos relacionados a outros pioneiros, particularmente Frank e Irma Adelman, Martin Shubik e Guy Orcutt (veja Morgan 2004b). Os projetos de simulação de Simon em economia envolveram, por exemplo, programar computadores para imitar decisões e escolhas da mesma maneira que os banqueiros de investimento tomavam essas decisões e escolhas, ou seja, com as mesmas informações e pelos mesmos processos de comparação e avaliação (veja Clarkson e Simon 1960). O trabalho dos Adelmans foi particularmente importante no desenvolvimento de métodos de simulação em econometria seguindo o exemplo do trabalho anterior de Tinbergen (veja também Duesenberry et al. 1960), enquanto na economia naquela época as simulações envolviam tanto ‘jogos’, significando experimentos nos quais as pessoas desempenhavam papéis fazendo decisões econômicas onde o modelo simulava o ambiente e todo o interesse estava no comportamento das pessoas (por exemplo, gerentes tomando decisões), e simulações de modelos matemáticos nas quais o comportamento era dado (por exemplo, comportamento econômico racional) e o ambiente variava para ver como isso alterava os resultados projetados pelo modelo. (Esta ampla categoria de simulações por volta de 1960, portanto, incluía algumas coisas que agora rotularíamos como experimentos.) Shubik esteve envolvido em muitos desses diferentes tipos de simulações, variando de experimentos de jogos, jogos de negócios, a experimentos de modelos. Orcutt (1960), por sua vez, foi pioneiro no método de microsimulação, no qual ele construiu uma amostra virtual representativa da população, dotou os indivíduos da amostra com características da população real e, em seguida, simulou seu comportamento ao longo do tempo para explorar as características do sistema agregado, bem como as partes individuais. Este é um trabalho experimental de modelo complicado que só foi possível com o novo poder de computação daquele dia. Todos esses economistas ampliaram significativamente as maneiras pelas quais os modelos funcionavam como instrumentos de investigação por meio de diferentes formas de atividade experimental, em que cada ‘execução’ do modelo fornecia um experimento ligeiramente diferente com o modelo. A simulação, desde sua introdução na economia, tem sido caracterizada como uma forma de experimento com modelos que visa imitar uma variedade de comportamentos econômicos diferentes, em diferentes níveis e de diferentes maneiras.

\subsection{\textbf{Construção de Modelos}}
A criação de modelos (em oposição às definições formais ou informais de modelos) tem sido um terreno fértil para comentaristas filosóficos sobre economia que a apresentaram como um processo de 'idealização', um termo que abrange uma série de coisas, incluindo abstração, simplificação e isolamento (veja Hamminga e De Marchi 1994). Essa ideia geral remonta ao conceito de 'tipo ideal' definido por Max Weber (1904, 1913) para as ciências sociais. Sua discussão incluiu noções do tipo ideal de comportamento econômico individual e a noção de tipo ideal de um mercado. Certamente é fácil ver o retrato do homem econômico do final do século 19 como tipo ideal, divorciado de todas as suas motivações econômicas puras sem qualquer psicologia mais profunda. O termo 'idealização' sugere que os modelos são obtidos por processos de abstração ao nível de ideias ou conceitos; de simplificar o caso ou sistema tratado omitindo influências irrelevantes ou negligenciáveis; de isolar os elementos que realmente se pensa serem importantes por cláusulas ceteris paribus; e assim por diante (veja Morgan 2006). Esses processos podem ser entendidos como trabalhando em teorias (por exemplo, passando de uma conta de equilíbrio total para um único mercado particular) ou como começando com o mundo complicado e isolando uma pequena parte dele para representação do modelo. Leszek Nowak (por exemplo, 1994) apresenta uma análise bastante geral na qual a 'idealização' leva do mundo à teoria e a 'concretização' da teoria ao mundo em dois processos paralelos bastante contínuos. Essa conta conhecida como a 'abordagem de Poznań' (nomeada após a Universidade que hospedou seu desenvolvimento: veja Hamminga 1998), foi formulada para a economia marxiana, mas pode muito bem ser aplicada de maneira mais geral. Dois outros comentaristas particularmente associados a questões de idealização na modelagem econômica são Nancy Cartwright e Uskali Mäki. Cartwright (1989) está interessada no que foi chamado de 'idealização causal', ou seja, em isolar as capacidades causais que realmente funcionam no mundo.

Ela associa esse objetivo tanto com o funcionamento da modelagem econométrica quanto com as tendências millianas (a conta das leis de tendência na economia fornecida por John Stuart Mill em meados do século 19). Mäki (1992) está mais interessado na ‘idealização de construção’, ou seja, em como a teorização econômica ocorre construindo versões de teoria com mais ou menos escopo ao longo de diferentes dimensões de isolamento. (A distinção entre idealização construtiva e causal usada aqui é devida a McMullin 1985.) Podemos encontrar ambos esses tipos de processo acontecendo na história da construção de modelos. A construção de Von Thünen (1826) de seu modelo diagramático de um ‘estado isolado’ fornece um exemplo claro de construção de modelos isolando os fatores que determinam a lucratividade da fazenda. Suas isolações podem ser interpretadas como a criação de um modelo teórico (ou seja, ele construiu um modelo idealizado), mas ele também estava interessado em chegar às causas reais, pois ajustou esse modelo aos dados estatísticos de sua própria fazenda (ou seja, ele isolou as causas, usando procedimentos econométricos informais).

A idealização em si pode envolver não apenas simplificações ou isolações, mas a adição de elementos falsos. Max Weber (1904) discute como os tipos ideais apresentam certos recursos de forma exagerada, não apenas acentuando esses recursos deixados pela omissão de outros, mas como uma estratégia para apresentar a forma mais ideal do tipo. Essa noção de exagero surge novamente na noção de modelagem de caricatura de Gibbard e Varian (1978) na economia, onde o exagero é projetado para permitir que o economista investigue a robustez do modelo (a virtude que Friedman, é claro, associou anteriormente ao uso de suposições irreais). Mas se interpretarmos esse processo de caricaturização para envolver não apenas um grau extremo de exagero, mas a adição de recursos, então temos uma idealização de um tipo qualitativamente diferente daqueles que vêm de métodos de isolamento ou simplificação. Por exemplo, a suposição de Frank Knight (1921) de informação perfeita envolve a adição de um recurso ao retrato do homem econômico; a suposição pode ser especificada de diferentes maneiras, cada uma cria um modelo diferente. Modelos de caricatura não devem ser confundidos com as construções artificiais dos modelos de Lucas, que não são derivados por idealização de teoria ou do mundo. As idealizações, mesmo na forma de caricatura, ainda são entendidas como representações do sistema ou comportamento do homem (por mais irreais ou positivamente falsas que possam ser), enquanto os modelos de mundo artificial não buscam representar o sistema ou o comportamento do agente - ao contrário, o objetivo é imitar a saída de tais sistemas ou comportamentos. Ao imitar as saídas do sistema, pode-se, é claro, argumentar que o poder representacional é buscado em um ponto diferente.

Na própria economia, ao contrário das análises dos comentaristas, esses processos de construção de modelos podem estar acontecendo todos juntos ao mesmo tempo. Ou seja, os modelos podem ser construídos para representar as versões idealizadas de teorias mais grandiosas, serem abstraídos das particularidades da vida econômica e fornecer simplificações do mundo mais complicado. Esses recursos estão todos em jogo no famoso Tableau économique do século 18 de François Quesnay, uma construção que pode ser considerada o ancestral geral dos modelos em economia. Mas esse modelo faz um exemplo revelador, pois como uma construção é apenas em parte uma derivação ou isolamento de um conjunto geral de ideias ou teoria, apenas em parte uma simplificação das relações no mundo ou abstração em um quadro conceitual mais amplo. Não parece ser derivado inteiramente da teoria, nem aparece como uma descrição de seus dados contemporâneos. No entanto, embora incorpore elementos de todas essas coisas, também é uma construção por si só (veja Charles 2004). Quesnay moldou esses elementos juntos para criar uma maravilhosa tabela-cum-imagem que representa a economia francesa de seu dia, que poucos economistas posteriores conseguem entender facilmente (pelo menos sem traduzi-la para uma forma diferente, o que, é claro, muda seu significado e funcionamento).

Essa interpretação da modelagem de Quesnay pressupõe que os modelos não são apenas derivados da teoria nem construídos exclusivamente a partir de dados, pois normalmente envolvem pedaços de ambos e muitas vezes outras coisas também, como metáforas, formas matemáticas importadas, e assim por diante. A noção de que os modelos econométricos são construídos a partir de relações teóricas e elementos estatísticos provavelmente não é tão controversa. A mistura de elementos também é óbvia em um caso como o modelo Phillips-Newlyn, uma máquina hidráulica real na qual a água vermelha, representando os vários estoques e fluxos agregados da economia, circulava pela máquina e às vezes se derramava na sala de aula (veja Leeson 2000; Boumans e Morgan 2004). Mas essas misturas são igualmente características em modelos matemáticos, de acordo com a conta de trabalho de construção de modelos de Boumans (1999), que argumenta que devemos pensar na construção de modelos como cozinhar novas receitas, nas quais a matemática fornece os meios de integrar vários, às vezes díspares, elementos em novos modelos. Essa conta de construção de modelos vai contra muita filosofia tradicional, mesmo por economistas, sobre a construção de modelos. No entanto, mais recentemente, os economistas começaram a escrever sobre seu trabalho de modelagem como uma atividade muito mais ad hoc na qual práticas passadas, novas intuições e até especulações orientam a construção de seus modelos (veja, por exemplo, Krugman 1993; Sugden 2000).

Entender a construção de modelos de acordo com a conta de receitas de Boumans sugere que os modelos - por construção - são parcialmente independentes tanto da teoria quanto do mundo (ou de seus dados), e isso explica sua aparente existência autônoma como objetos de trabalho na economia moderna. Essa conta de construção faz parte da visão de ‘modelos como mediadores’ do papel dos modelos, que analisa seu uso como instrumentos de investigação (veja Morrison e Morgan 1999). De acordo com essa conta, os modelos podem funcionar de maneira autônoma intermediária por causa de sua construção. No entanto, a possibilidade de aprender com o uso de modelos depende de outro elemento em sua construção, ou seja, que os modelos são dispositivos feitos para representar de alguma forma ou forma algo em nossas teorias econômicas ou no mundo econômico ou ambos de uma vez. É essa qualidade representativa, incorporada na etapa de construção, que torna possível usar um modelo não apenas como um instrumento de previsão, mas como um instrumento de investigação para aprender algo sobre o mundo ou a teoria que ele representa. Essa conta pode se aplicar até mesmo aos modelos de mundo artificial propostos por Lucas, que são construídos não para representar o funcionamento do sistema, mas as saídas do sistema, embora aqui as ambições dos modeladores de aprender com a modelagem para entender o sistema econômico e explicar os fenômenos de resultado que eles imitam parecem um tanto reduzidos.

Essa conta recente de construção de modelos contrasta acentuadamente com as contas de como a construção de modelos ocorre de acordo com os comentaristas do meio do século 20 discutidos anteriormente. Lembre-se de que Koopmans rotulou os modelos matemáticos como 'definidos por um conjunto de postulados' onde o conjunto completo de postulados forma a teoria - uma definição consistente com a abordagem axiomática das teorias então atual. Em econometria, a Comissão Cowles apresentou modelos econométricos como sendo derivados - diretamente dados em algum sentido - da teoria a priori. De fato, foi a base de sua posição no debate 'medição sem teoria' de que a econometria precisava de modelos que fossem claramente versões de teorias para chegar a qualquer lugar, contra os modelos derivados de dados do National Bureau of Economic Research (NBER) que eles denunciaram como não científicos. Outra descrição que se encaixa nas inclinações filosóficas do meio do século 20, mas é mais voltada para o modelo, foi dada por Friedman, que definiu uma teoria como consistindo em duas partes: 'um mundo conceitual ou modelo abstrato mais simples do que o "mundo real" e contendo apenas as forças que a hipótese [teoria] afirma serem importantes' e uma segunda parte definindo a 'classe de fenômenos para os quais o "modelo" pode ser considerado uma representação adequada do "mundo real"' junto com as regras de correspondência que ligam os termos do modelo e os fenômenos (Friedman 1953, p. 24). Friedman aqui retrata o modelo de forma ordenada como uma versão da teoria e ao mesmo tempo uma representação do mundo real, mas as regras de correspondência estão longe de serem indiscutíveis. Embora se possa argumentar que o principal trabalho da econometria tem sido desenvolver tanto a teoria quanto as práticas de tais regras de correspondência para modelos, para modelos matemáticos, em contraste, as contas metodológicas muitas vezes fracassaram em como tais critérios de correspondência podem ser formulados. Apesar da longa sombra dessas definições bastante formais do meio do século 20, está de acordo com nossas observações sobre como os modelos são usados na ciência econômica moderna que eles agora podem ser entendidos como objetos de trabalho autônomos, em vez de como proto-teorias ou versões de dados.

\subsection{\textbf{Conclusão}}
Há mais a ser dito e pesquisado sobre a filosofia da modelagem, por exemplo, sobre a natureza do raciocínio com modelos matemáticos; sobre o papel dos modelos matemáticos no design de experimentos em sala de aula/laboratório em economia; sobre o uso de modelos em consultoria de políticas e intervenção; e sobre a ausência de critérios formais para trabalhar com modelos matemáticos que são equivalentes aos critérios estatísticos associados ao trabalho de modelagem econométrica. Também há muito a ser feito para preencher a história esquelética da modelagem oferecida aqui: em separar a história da modelagem tanto da história da economia matemática quanto da história da econometria; em demarcar o alcance histórico do escopo da modelagem; e em discernir por que e como o método se firmou. No entanto, a trajetória básica da história é clara: a modelagem se tornando definida como um modo de raciocínio e trabalho para a economia na década de 1930, foi desenvolvida e usada de várias maneiras nas décadas de 1940 e 1950, preparando o cenário para a modelagem se tornar uma metodologia dominante na última parte do século. E uma vez definido, podemos olhar para trás e reconhecer protótipos anteriores para tal método remontando a Quesnay no século 18. Quando olhamos para trás e consideramos a visão de mundo científica que perdemos na economia ao adotar a modelagem como um de nossos métodos preferidos de fazer economia, o que se destaca é que a ciência é radicalmente diferente. Os economistas não acreditam mais e não investigam algumas leis governantes grandiosas, nem mesmo propõem teorias gerais de grande alcance - em vez disso, a economia se tornou uma ciência de muitos modelos diferentes e particulares.

\section{\textbf{Weintraub (2002)}}
\subsection{\textbf{Pags : 9-25}}
[Problema:] para analisar o movimento de uma moeda lisa e plana rolando dentro da superfície áspera de um elipsoide oco equilibrado nas costas de uma tartaruga hemisférica caminhando a uma velocidade constante morro acima de um gradiente uniforme em Saturno.
- I. Grattan-Guinness, A História Norton das Ciências Matemáticas

Os economistas acreditam que o último terço do século XIX desempenhou um papel fundamental na evolução de sua disciplina moderna. 'A Revolução Marginalista', com sua introdução do homo oeconomicus tomando decisões de consumo na margem, reformulou a economia em uma ciência moderna. As controvérsias sobre o status científico da economia estavam bastante vivas no final do século XIX, quando a Escola Histórica Alemã, os Institucionalistas Americanos, a Escola Austríaca e outros contestaram a natureza e os limites da ciência econômica. A melhor maneira de fazer economia era, de fato, uma questão em aberto em 1900. Neste capítulo, argumentarei que essa disputa refletiu disputas semelhantes em matemática e física, e sua resolução foi moldada pelas resoluções que eventualmente estabilizaram esses outros campos de investigação.

Semelhante às crenças dos economistas sobre suas próprias origens disciplinares modernas, um dos tropos centrais da história da matemática diz respeito às crises na matemática e na física no final do século XIX que induziram físicos e matemáticos a reconstruir suas disciplinas no século XX. Para a matemática, a crise é frequentemente entendida como tendo se preocupado com os fundamentos da matemática. Havia três tópicos principais: 1) os fundamentos da geometria, especificamente as falhas da geometria euclidiana para domesticar as geometrias não euclidianas; 2) as falhas da teoria dos conjuntos manifestas através das novas ideias de Georg Cantor sobre "infinito" (ou seja, cardinais transfinitos e o contínuo dos números reais); e 3) paradoxos nos fundamentos da aritmética e lógica, associados a Gottlieb Frege e Guiseppe Peano. Geralmente se assume que a resposta a esses problemas deixou a comunidade de matemáticos insegura sobre o que era certo e adequado e verdadeiro e duradouro na matemática. Por exemplo, uma exposição popular nos diz que:

Contra o pano de fundo do progresso constante nos grandes centros científicos da Inglaterra, França, Alemanha, Itália e Rússia, três desenvolvimentos empolgantes no último quarto do século XIX prepararam o terreno para a explosão massiva de novas ideias em matemática pura no início do século XX: a criação (basicamente, de forma independente) da teoria dos conjuntos infinitos por George Cantor (1845-1918); o anúncio de Felix Klein (1849-1925) em 1872 do Programa Erlanger, que propôs a geometria como uma disciplina preocupada com o estudo de um objeto abstrato invariante sob grupos de transformação dados; a aparição em 1899 de Grundlagen der Geometrie por David Hilbert (1862-1943) axiomatizando a geometria euclidiana. . . . Todos os três vieram da Alemanha. Eles provocaram uma mudança fundamental tanto na posição da matemática entre outras disciplinas do conhecimento, quanto na maneira como os matemáticos pensam sobre si mesmos. Os abalos secundários duraram bem até a década de 1930 e além.... Como resultado, a matemática se separou do corpo das ciências naturais. (Woyczynski 1996, 107-8)

Na imaginação popular, no entanto, ainda mais crítico foi o fracasso da física, particularmente a mecânica racional, para lidar com os novos problemas levantados pela termodinâmica, quanta e relatividade. Isso levou a uma crise na física e, a fortiori, na física matemática. Ou seja, os tipos de matemática do século XIX baseados em equações diferenciais, e propriedades quantitativas e qualitativas de sistemas dinâmicos, estavam fundamentalmente ligados aos problemas da mecânica. Se o modo de argumentação mecânica determinística fosse substituído por uma teoria física alternativa, algumas áreas estabelecidas da matemática não estariam mais conectadas a um modelo físico geralmente aceito.

Com Plank e Einstein, houve o nascimento de uma nova física: a mecânica estatística, a mecânica quântica e a teoria da relatividade forçaram os físicos a pensar em termos de novos modelos do universo, tanto grandes quanto pequenos. Modelar a nova física exigia uma nova matemática, matemática baseada menos em sistemas dinâmicos determinísticos e mais em argumentação estatística e álgebra. Consequentemente, a física matemática estava para se conectar com ideias matemáticas mais novas em álgebra (por exemplo, teoria dos grupos) e teoria da probabilidade (por exemplo, teoria da medida), à medida que os matemáticos assumiam o desafio de trabalhar em ideias matemáticas que poderiam facilitar a compreensão do mundo.

Assim como os objetos do mundo físico pareciam mudados - os bilhares se foram, os quanta estavam recém-presentes - o universo dos objetos matemáticos mudou. Conjuntos transfinitos e novas geometrias, juntamente com o reconhecimento de que os paradoxos da teoria dos conjuntos e da lógica estavam entrelaçados, levaram os matemáticos no início do século XX a buscar novos fundamentos para seu assunto com base na axiomatização e na modelagem formal dos fundamentos da teoria dos conjuntos, lógica e aritmética. Até os anos 1920 e 1930, a matemática voltaria a ser clara e coerente após a "crise dos fundamentos" da virada do século. Em particular, parece ser uma parte estabelecida da história geral da ciência do século XX que os problemas, paradoxos e confusões da matemática da virada do século deveriam ser resolvidos pela reconceituação dos objetos fundamentais da matemática, assim como a física reformulou os blocos de construção do mundo natural.

Quer se aceite ou não essa história de crises, olhando primeiro para o trabalho matemático feito antes de 1900, depois para o trabalho feito nos anos 1920 e 1930, e finalmente para o trabalho feito nos anos 1950, é claro que a paisagem matemática havia sido transformada. Não é nossa tarefa construir a história dessas transformações, embora no capítulo 3 vamos olhar de perto as histórias concorrentes dessas mudanças na matemática. Em vez disso, como nossa preocupação é a transformação da economia, precisamos prestar atenção nas características em mudança da paisagem matemática como um pano de fundo contra o qual podemos entender como a economia foi reformulada, ao longo dos dois primeiros terços do século XX, como uma disciplina matemática.

\subsubsection{\textbf{Matemática de Cambridge}}

Como muitos economistas de língua inglesa do século XX traçam sua genealogia profissional de volta aos Princípios de Economia de Alfred Marshall, vamos entrar no mundo da Universidade de Cambridge de Marshall, na Inglaterra. Como esta universidade era o lar intelectual de muitos dos economistas ingleses, é um lugar apropriado para começar a olhar para as imagens do pensamento matemático.

No início do século XIX, a abordagem georgiana solta deu lugar a uma estrutura acadêmica um pouco mais rigorosa. Em um período de aumento das matrículas, os exames começaram a desempenhar um papel cada vez mais importante tanto em Cambridge quanto em Oxford. Em Oxford, os exames eram em clássicos, cujo foco era justificado como uma maneira de ampliar as mentes jovens, em vez de transmitir conhecimento especializado. O mesmo tipo de justificativa foi aplicado em Cambridge, onde, no entanto, o exame central [chamado Tripos] era em matemática. Até meados do século, era necessário passar neste exame para poder fazer o exame paralelo nos clássicos. Mesmo que o Tripos se tornasse cada vez mais exigente matematicamente, a justificativa para exigir que os alunos estudassem para ele continuou a ser amplamente humanística, em vez de específica ou profissional. Ao longo do século, o centro da educação matemática da Inglaterra perseguiu o assunto como uma maneira de ajudar os alunos a se tornarem seres humanos plenamente formados. (Richards 1991, 307-8)

O Tripos era um conjunto final de exames dados aos estudantes de Cambridge que buscavam um diploma. O nome pode derivar do banquinho medieval de três pernas em que o candidato se sentava enquanto era examinado (Ball 1889), ou pode ter sua origem no "Trivium" escolástico de gramática, lógica e retórica. No uso costumeiro, fala-se de cada Tripos particular, por exemplo, Ciências Naturais, como definindo um grande campo de estudo no qual se poderia receber um diploma de Cambridge.

O que os próprios economistas ingleses do final do século XIX estudaram? A resposta geralmente não é conhecida, exceto para especialistas do período histórico, mas em Cambridge, especificamente, eles geralmente estudavam matemática. Como Richards aponta,

Antes de 1848, o Tripos era um exame [de matemática] indiferenciado de seis dias. Na reforma de 1848, foi prolongado para oito dias e dividido em duas partes. Os primeiros três dias foram projetados para cobrir o material essencial para qualquer pessoa receber um diploma comum. . . .

[Depois de 1851,] quando o Tripos de Ciências Morais e o Tripos de Ciências Naturais foram adicionados, os alunos poderiam tentar receber honras em qualquer Tripos depois de fazer apenas a primeira parte do Tripos Matemático. Assim, até o final do século, a primeira parte do Tripos Matemático permaneceu o núcleo sólido da educação de qualquer graduado em Cambridge. (Richards 1988, 40)

Isso significava que não apenas a carreira de graduação, na principal universidade da Inglaterra, era passada em grande parte estudando matemática, mas na verdade se estudava um tipo particular de matemática. O exame Tripos Parte I testava

as partes de Euclides geralmente lidas; Aritmética; partes da Álgebra, abrangendo o Teorema Binomial e o Princípio dos Logaritmos; Trigonometria Plana, até incluir a solução de Triângulos; Seções Cônicas, tratadas geometricamente; as partes elementares de Estática e Dinâmica, tratadas sem o Cálculo Diferencial; as Primeiras Três Seções de Newton, as Proposições a serem provadas à maneira de Newton; as partes elementares de Hidrostática, sem o Cálculo Diferencial; as proposições mais simples de Óptica, tratadas geometricamente, as partes da Astronomia necessárias para a explicação dos fenômenos mais simples, sem cálculo. (Relatório do Conselho de Exame para 1849, conforme citado em Richards 1988,40-41)

A característica curiosa deste programa é a ênfase no que agora é considerado como “matemática aplicada”, na verdade mecânica racional. Este ponto é melhor compreendido quando se considera a história do treinamento dos próprios examinadores matemáticos, pois nos anos desde Newton, até os primeiros anos do século XX, a matemática inglesa se destacava das tradições continentais no campo da Análise. O Tripos definia as preocupações da matemática inglesa de uma maneira fundamental. O próprio exame, mas não a matemática que era importante aos olhos dos melhores matemáticos europeus, definia as preocupações dos alunos e do programa. Na verdade, os melhores matemáticos de Cambridge, homens como J.J. Sylvester e Arthur Cayley, deram palestras que nenhum aluno, ou muito poucos alunos, jamais frequentaram. Por que eles deveriam ter frequentado, já que esse material nunca apareceria em nenhum exame? Em vez disso, treinadores matemáticos como Hopkins e Routh prepararam os alunos para o Tripos. Os treinadores organizavam os alunos em pequenas turmas de no máximo dez.

Cada turma era feita três vezes por semana, em dias alternados durante as oito semanas de cada um dos três trimestres e as seis ou sete semanas das Longas Férias; e no total havia dez termos e três Longas Férias no curso completo de graduação. Cada turma era de uma hora exatamente. Os tópicos eram tratados, não em conexão com os princípios gerais subjacentes que poderiam caracterizar o assunto, mas de uma maneira que o aluno deveria enquadrar sua resposta no exame. . . . No final da hora, algumas perguntas… [foram entregues]; sua solução teve que ser trazida para a próxima palestra. (Forsyth 1935, 89)

Assim, a matemática foi definida, na Inglaterra, por um conjunto de truques e detalhes, baseados em Newton, que estavam ligados à física aplicada e à mecânica, e que podiam ser testados de forma limitada no tempo. A função do exame realmente era fornecer uma ordenação fixa dos candidatos ao grau. O melhor desempenho até o último encontrou seu lugar na lista de resultados postada. Desde o Senior Wrangler (primeiro lugar) até o Second Wrangler (lugar de Marshall) até o Third, etc., até o Twelfth Wrangler (lugar de Keynes) até a Wooden Spoon (última nota de aprovação), a ordem de chegada do Tripos definia as opções no mundo da bolsa de estudos pelo menos. Keynes, lembre-se, não conseguiu um cargo acadêmico em Cambridge, o desejo mais fervoroso de seu pai, porque sua posição de Twelfth Wrangler simplesmente não era boa o suficiente naquele ano. Consequentemente, ele se preparou para o Exame do Serviço Civil em vez de receber um cargo como Fellow de Cambridge.

A “pergunta típica do Tripos, que foi parodiada repetidamente, era uma abstração irreal, muitas vezes fantasticamente irreal, do problema físico que a sugeriu, cujo único objetivo era torná-la tratável para os candidatos [O Tripos se tornou] de longe o teste matemático mais difícil que o mundo já conheceu” (Roth 1971, 99,97).

Para tornar um pouco mais específicos os argumentos que busco desenvolver, considere as questões do Tripos Matemático para o ano de 1878. Neste exame, encontramos perguntas como:

viii. Descreva a teoria da máquina de Atwood e explique como ela é usada para verificar as leis dos movimentos. Se a ranhura na polia em que a corda corre for cortada até a profundidade em que se descobre que a inércia da polia pode ser dividida igualmente entre os pesos móveis, e se Q for o peso necessário para ser adicionado para superar o atrito do eixo da polia quando pesos iguais P são pendurados nas extremidades da corda, prove que um peso adicional R produzirá aceleração R dividido por 2P + 2Q + R + W [todas as vezes] g onde W é o peso da polia

A diferença entre as pressões em quaisquer dois pontos de um líquido homogêneo em repouso sob a gravidade é proporcional à distância entre os planos horizontais em que os pontos se encontram. Um tetraedro regular de metal fino, cujo peso é igual ao peso da água que ele conteria, é esvaziado de água e cortado em duas metades por uma seção central paralela a duas arestas opostas. Se uma metade for mantida firme em qualquer posição, mostre que a força necessária para afastar a outra metade dela será a mesma, desde que o centro do tetraedro esteja sempre no mesmo plano horizontal.

Essas questões do exame também foram incluídas:

viii. Determine o movimento inicial de um corpo rígido que recebe um impulso dado; e encontre o parafuso em torno do qual ele começará a girar. Um anel pesado, inelástico e áspero rola, com seu plano vertical, descendo um plano inclinado, no qual se encontram uma série de obstáculos pontiagudos que são iguais e a distâncias iguais uns dos outros, e que são suficientemente altos para impedir que o anel toque o plano. Se os anéis começarem a partir do repouso de uma posição em que estão em contato com dois obstáculos, prove que sua velocidade angular ao deixar o (n + l)th obstáculo é dada por

$$ w^2 = 2 \frac{g}{a} \sin i \sin \gamma \cos^4\gamma \frac{1 - \cos^{4n}\gamma}{1 - \cos^4\gamma}  $$

onde a é o raio do anel, i é a inclinação do plano em relação ao horizonte, e 2g/a é o ângulo que dois obstáculos adjacentes subtendem no centro do anel quando ele está em contato com ambos. (Ibid., 78)

Obtenha as equações gerais de equilíbrio de uma placa elástica de pequena espessura, sob forças dadas. [Em seguida, assuma uma] casca esférica fina e uniforme de material isotrópico, cujo peso pode ser negligenciado, é feita para realizar vibrações na direção do raio, simétricas em relação a um diâmetro. Mostre como elas podem ser encontradas. (Ibid., 211)

1. Se a órbita na qual um corpo se move gira em torno do centro da força com uma velocidade angular que sempre mantém uma proporção fixa com a do corpo; prove, pelo método de Newton, que o corpo pode ser feito para se mover na órbita giratória da mesma maneira que a órbita em repouso pela ação da força tendendo ao mesmo centro. (Ibid., 220)

A partir dessas questões de amostra, podemos ver como o exame Mathematical Tripos se tornou artificial, irreal e até bizarro no final do século XIX. Eles nos mostram uma comunidade de matemática de Cambridge isolada das preocupações dos matemáticos continentais. A caricatura apresentada por Grattan-Guinness no epígrafe deste capítulo revela o jogo. O Mathematical Tripos estava associado ao "grande período da física matemática de Cambridge: Ferrers, Green, Stokes, Kelvin, Clerk Maxwell, G. H. Darwin, Rayleigh, Larmor, J. J. Thompson" (Roth 1971, 223). No entanto, o Tripos estabeleceu uma visão específica da matemática, que retardou a compreensão da matemática pura como uma disciplina lógica ou estrutural. E o Tripos no final do século XIX foi mantido em parte pela eminência do Professor Sadlerian de Matemática, o notável Andrew Russell Forsyth.

Sobre Forsyth, foi dito que ele

teve o infortúnio de nascer cem anos tarde demais; em sua perspectiva e técnica matemática, ele era um homem do século XVIII. . . . Forsyth olha para trás para Lagrange em vez de para frente para Cauchy. . . . [Sua Teoria das Funções de uma Variável Complexa de 1893] é magistral, Johnsoniana; os poderes de assimilação do autor são quase incríveis - e ainda assim, estranho dizer, apesar de suas intenções e sua absorção do material, ele nunca chega perto de compreender o que a análise moderna é realmente: de fato, grandes trechos do livro parecem ter sido escritos por Euler. (Roth 1971, 225)

Eu trago este material à atenção do leitor, em uma discussão sobre economia matemática, para sugerir que a matemática inglesa era a antítese do que agora consideramos como matemática rigorosa. Para todos os efeitos e propósitos, não houve matemática pura feita na Inglaterra no século XIX, pois mesmo Cayley e Sylvester não estavam, como estavam Cauchy, Riemann, Weierstrass, Klein e Lie et al., preocupados com questões fundamentais de análise. As ideias matemáticas modernas na Inglaterra, como preocupações compartilhadas da maior comunidade matemática mundial, fizeram sua aparição com Hardy e Littlewood na segunda década do século XX. A matemática inglesa estudada pelo Second Wrangler Marshall e pelo Twelfth Wrangler Keynes, por outro lado, era uma mistura de física aplicada, termodinâmica, óptica, geometria, etc. Estava tão longe das ideias modernas de rigor quanto William Morris estava de Piet Mondrian.

Mas rigoroso era, no entanto, como o rigor era então entendido para significar "baseado em um substrato de raciocínio físico". O oposto de rigoroso era irrestrito, como com um argumento matemático irrestrito por instanciação em um modelo de ciência física/natural. Como Giorgio Israel (1981) argumentou brilhantemente, a matemática do final do século XIX considerava "rigor" e "axiomatização" antitéticos, enquanto essas duas noções são praticamente indistinguíveis na matemática do final do século XX.

A afirmação moderna de que Marshall não forneceu uma "fundamentação matemática axiomática rigorosa" para a teoria econômica dificilmente é surpreendente, pois ele não teria sido capaz de compreender nossa ideia atual da natureza e do significado dessa frase, em seu vocabulário um oxímoro. Assim, notar a falta de uma economia matemática formal/axiomatizada, que geralmente significa economia escrita em inglês, antes do século XX, é atender às peculiaridades do Cambridge Tripos, que instanciou a ideia inglesa do século XIX de argumentação matemática rigorosa. Para correr o risco de ser repetitivo, não posso enfatizar demais que embora toda a matemática, pelo menos durante grande parte do século XIX, exigisse raciocínio físico conectado para ser considerada rigorosa, no final do século esse vínculo estava sendo quebrado em quase todos os países europeus, exceto a Inglaterra. A causa do atraso da matemática inglesa foi o exame Tripos de visão retrospectiva. Sua importância para a estrutura institucional, desenvolvida para definir uma ordem fixa de mérito na educação "superior" inglesa, apoiou essa rigidez.

\subsubsection{\textbf{Alfred Marshall}}
Na biografia de Alfred Marshall escrita por Peter Groenewegen (Groenewegen 1995), somos apresentados a um retrato detalhado do papel que o Mathematical Tripos desempenhou na própria educação de Alfred Marshall, e como seu trabalho subsequente maduro estava entrelaçado com essa instituição peculiar de Cambridge. Marshall fez o exame em sua forma pós-1848: "Nos primeiros três dias, seis trabalhos elementares foram tentados. Estes decidiram se uma pessoa era elegível para se sentar para a parte avançada do tripos a ser examinada ao longo de cinco dias e dez trabalhos adicionais, seguindo um intervalo de uma semana após os exames da primeira parte" (Groenewegen 1995, 80).

Groenewegen apresenta o que Marshall era responsável por preparar para esses exames. Entre as partes usuais de álgebra, trigonometria e seções cônicas, há partes elementares de estática e dinâmica, esta última tratada sem cálculo diferencial; as primeiras, segunda e terceira seções do Principia de Newton com proposições "a serem provadas à maneira de Newton"; partes elementares de hidrostática, óptica e astronomia também eram necessárias (82). O segundo ano exigia mais trabalho em cálculo, equações diferenciais, estática e dinâmica. Finalmente, o aluno passou para a geometria sólida, hidrostática, dinâmica e óptica em um nível mais avançado. Groenewegen observa que a matéria enfatizava a "aplicação contínua pelo aluno durante o período e as férias" (83). Sua própria discussão sobre o Tripos se concentra quase exclusivamente na necessidade de coaching, habilidade rápida para resolver problemas e a necessidade de desenvolver o domínio de várias peças fixas que os examinadores costumavam colocar nos exames ano após ano.

Groenewegen cita a visão de Arthur Berry de 1912 sobre o Tripos e a visão de Berry de que

o bom matemático que naturalmente nesta fase de sua carreira teria uma inclinação para certos departamentos de matemática era muito desencorajado de qualquer tipo de especialização. O matemático puro inclinado a prosseguir o estudo da análise superior seria verificado pela necessidade de ser capaz de responder perguntas sobre óptica geométrica.... No exame... nenhum lugar é atribuído a qualquer tipo de pesquisa original. Os exames testaram o conhecimento naquela forma limitada de originalidade que consistia em aplicar o conhecimento muito rapidamente a tal aplicação da teoria que poderia tomar a forma de perguntas de exame. . . . Outro defeito sério... é o divórcio quase completo entre matemática e física experimental. (Berry 1912, conforme citado em Groenewegen 1995, 86)

Na autobiografia de Sir J. J. Thompson, ele lembrou sua própria experiência no Mathematical Tripos de janeiro de 1880:

O exame para o Mathematical Tripos quando eu me sentei para ele em janeiro de 1880 foi uma experiência árdua, ansiosa e muito desconfortável. Foi realizado no auge do inverno no Senate House, uma sala na qual não havia aparelhos de aquecimento de qualquer tipo. Certamente estava horrivelmente frio, embora a tinta não tenha congelado como se diz ter feito uma vez.

O exame foi dividido em dois períodos: o primeiro durou quatro dias. No final do quarto dia, houve um intervalo de dez dias em que os examinadores elaboraram uma lista daqueles que, por seu desempenho nos três primeiros dias, se haviam comportado de tal maneira a merecer honras matemáticas. Estes, e somente estes, poderiam fazer a segunda parte do Tripos, que durou cinco dias começando na segunda-feira seguinte, mas uma após o início dos primeiros quatro dias. (56-57)

Depois de discutir parte do material que apareceu no exame, Thompson lembra que

era de grande importância que o aluno não cometesse erros. . . . Um erro em aritmética ou um sinal errado em um pedaço de álgebra, envolveria passar pelo trabalho novamente, e a perda de tempo quando não havia tempo a perder. A precisão na manipulação era talvez a condição mais importante nesta parte do exame... [as] qualidades, ter o conhecimento na ponta dos dedos, concentração, precisão e mobilidade devem sua importância ao exame ser competitivo, a haver uma ordem de mérito, a termos que galopar todo o caminho para ter uma chance de ganhar. (58)

Thompson conclui argumentando que o Tripos não era um rito de passagem útil para todos os alunos:

O Tripos... era, na minha opinião, um exame muito bom para os melhores homens. No entanto, era muito ruim para a maioria que não tinha habilidade matemática excepcional. Muitos desses homens poderiam lidar com os assuntos mais elementares e se beneficiar estudando-os, mas com esses eles tinham disparado seu parafuso; eles acharam os assuntos mais elevados além deles e o tempo que passaram com eles foi desperdiçado. (60)

Como evidência adicional em apoio ao ponto de Thompson, temos notas autobiográficas, publicadas em 1916, de Edward Carpenter, que foi o 10º Wrangler por volta de 1870, e que deveria ocupar uma bolsa de estudos em matemática (para tutoria em astronomia) no Trinity Hall. Carpenter (que também nos forneceu nessas notas uma memória de F. D. Maurice) memorializou o Mathematical Tripos desta forma:

Ao chegar a Cambridge, nunca me ocorreu no início ir para um diploma de honras; minha opinião sobre a Universidade era muito alta para isso. Mas depois de um ou dois termos, o tutor, para minha surpresa, me recomendou seriamente a ler para o tripos matemático. Eu estava, é claro, terrivelmente atrasado em minhas matérias, mas contratei um coach particular, passei pela rotina de estudo intensivo e, finalmente, obtive uma bolsa de estudos.

A matemática me interessava e eu a lia com muito prazer - mas às vezes lamento que três anos da minha vida tenham sido - na medida em que o estudo estava preocupado - quase inteiramente absorvidos por um assunto tão especial e, no geral, tão infrutífero. Acho que todo menino (e menina) deveria aprender um pouco de Geometria e Mecânica; sem esses, a mente carece de forma e definição e sua aderência ao mundo externo não é tão forte quanto deveria ser; mas as matemáticas superiores (certamente como são lidas em Cambridge) são, na maior parte, meros exercícios de ginástica não aplicados à vida real e aos fatos, e facilmente passíveis de serem prejudiciais, como todos esses exercícios são.

Depois do meu diploma, embora mantendo um certo interesse geral no assunto, nunca mais abri um livro de matemática com a intenção de prosseguir seriamente seu estudo. (Carpenter 1916, 48-49)

No mesmo período geral descrito por Thompson e Carpenter, Marshall emergiu de um estudo detalhado e preparação com um resultado notável como Second Wrangler. É digno de nota que Marshall não entrou em um exame de prêmio para o Prêmio Smith, que exigia alguma habilidade em pesquisa matemática original. Groenewegen conclui sua própria discussão tentando avaliar "quão bom matemático o Tripos fez dele?" Ele oferece a avaliação de que "Alfred Marshall aplicou sua matemática à economia com cuidado, com cautela e com um grau considerável de habilidade, um benefício de sua experiência no Tripos e no período de graduação que não deve ser subestimado, embora ele preferisse a geometria mais para esse papel do que a linguagem concisa da álgebra e do cálculo" (94). Em resumo, Groenewegen cita John Whitaker (1975, 4-5) que argumentou que "apesar de uma inclinação anterior por Euclides, há desde o primeiro uma estranheza e hesitação nos esforços de Marshall em economia matemática que argumenta contra ele já ter respirado livremente nos pináculos da abstração. Tanto Jevons quanto Edgeworth pareciam ter habitado mais confortavelmente no reino da lógica abstrata, apesar de sua inferioridade a Marshall no treinamento matemático."

No entanto, há algo mais a ser dito. Whitaker, por exemplo, continua a dizer que "a visão comum [entre os economistas] de Marshall como um gigante matemático que exerceu grande autocontrole em resistir, por economia, à inclinação natural de sua própria mente, pode ter se tornado exagerada." No entanto, as discussões de Whitaker e Groenewegen pressupõem que a matemática é um monólito, embora a ideia de "ser um matemático" não tenha um conjunto de referentes estáveis. O próprio Marshall, como é bem conhecido, era tanto um defensor quanto um oponente das ideias matemáticas em economia. Essa ambivalência certamente tem suas raízes, como argumenta Groenewegen, em seu próprio treinamento matemático, conforme definido pela preparação para o Tripos. A matemática para Marshall era uma competição, um local para obter prêmios e prêmios e liderar o campo. Que ele não conseguiu fazer exames que testassem a originalidade matemática é evidência de seu próprio movimento afastado de uma carreira matemática quando ele "apenas" conseguiu alcançar a posição de Second Wrangler. No entanto, a ambivalência que se prolongou desde a sua juventude não é toda a história dos comentários posteriores de Marshall sobre a economia matemática.

Do apoio de Marshall ao trabalho, e aos trabalhadores, em economia matemática, temos não apenas a evidência de seu principal livro, mas uma ampla gama de documentos de apoio escritos em auxílio às carreiras daqueles que impulsionaram a economia nessa direção. Uma peça inicial é uma nota que Marshall enviou a Edgeworth em 8 de fevereiro de 1880, sobre o livro de Edgeworth Novos e Antigos Métodos de Ética: "Agora quase li todo o livro que você me enviou e estou extremamente encantado com muitas coisas nele. Parece haver um acordo muito próximo entre nós quanto à promessa da matemática nas ciências que se relacionam com a ação do homem. Quanto à interpretação do dogma utilitarista, acho que você fez um grande avanço. Mas ainda tenho uma inclinação por um modo de exposição em que o caráter dinâmico do problema seja mais óbvio, que pode de fato representar a noção central de felicidade como um processo e não uma condição estática" (como citado em Whitaker 1996, 401).

Em contraste, temos a famosa carta para Arthur Bowley de 27 de fevereiro de 1906:

Mas eu sei que tive um sentimento crescente nos últimos anos do meu trabalho no assunto de que um bom teorema matemático lidando com a hipótese econômica era muito improvável de ser uma boa economia: e fui mais e mais pelas regras - (1) use matemática como uma linguagem de atalho, em vez de um motor de investigação. (2) Mantenha-os até que você tenha terminado. (3) Traduza para o inglês. (4) Em seguida, ilustre com exemplos que são importantes na vida real. (5) Queime a matemática. (6) Se você não conseguir ter sucesso em quatro, queime três. Isso último eu fiz muitas vezes. Acho que você deveria fazer tudo o que pudesse para impedir que as pessoas usassem matemática em casos em que a língua inglesa é tão curta quanto a matemática. (Groenewegen 1995,413)

Groenewegen apresenta esta carta observando que "[Schumpeter] supôs que essa prática refletia a ambição peculiar de Marshall de ser lido por empresários" (ibid.). No entanto, Groenewegen sugere que "o mais crucial para a decisão foi a crescente percepção de Marshall dos perigos em perseguir as consequências lógicas do raciocínio matemático na economia até o limite. . . . A 'ganância' de um economista por fatos era uma força contraproducente essencial para a emoção da perseguição que o raciocínio matemático proporcionava, se o contato com a realidade da economia fosse preservado" (ibid.). O Marshall de Groenewegen aparece nesta interpretação para assumir as opiniões dos estudiosos da era de Cambridge de Joan Robinson, para quem o mundo da economia matemática do final do século XX foi um desvio errado.

Acho que há uma explicação mais convincente para essa ambivalência, embora seja uma explicação que nos leva para fora do próprio Marshall. Na época da escrita de Marshall para Bowley, temos uma economia matemática emergente com as obras de Pareto, Panteloni e outros. Cournot, mesmo com seu erro (na visão de Marshall) sobre retornos crescentes, estava começando a ser lido, e o livro de Irving Fisher não só estava impresso, mas amplamente elogiado. Esses livros refletiam uma sofisticação matemática e o uso da matemática de maneiras essencialmente novas. Para um produto do antigo Tripos como Marshall, que cresceu pensando na matemática como preocupada em derivar certas conclusões de argumentos geométricos, tendo como modelo a memória de resolver peças fixas de mecânica newtoniana pelos próprios métodos geométricos (euclidianos) de Newton sob coação, essa nova maneira de usar a matemática pode ter sido desconfortável. O ponto é que para Marshall, sua imagem do que era a matemática, e como deveria ser feita, e especialmente como deveria ser aplicada aos problemas, foi forjada pelo Mathematical Tripos de seus anos de estudante em Cambridge, e seus dias pré-professorais lá.

Groenewegen observa, 'A apreciação do conhecimento matemático como verdade necessária e inevitável, derivada axiomática, era um aspecto do treinamento matemático de Cambridge que justificava sua preeminência no currículo de honras da universidade, combinado como estava com os métodos pelos quais tais verdades poderiam ser dominadas. Este foi um ponto enfatizado por Whewell em sua defesa do valor da especialização matemática. Um wrangler de alto nível em particular teria sido fortemente impregnado por esse recurso especializado do conhecimento matemático" (116). Este ponto feito por Groenewegen é extremamente importante. Ou seja, para Marshall, a matemática era de fato parte do programa de Whewell (Henderson 1996), pelo qual servia como um exemplo do caminho para a verdade, para construir argumentos indubitavelmente verdadeiros. É por isso que teve um lugar tão central no processo educacional de Cambridge vitoriano. No entanto, como Richards mostrou, esse papel foi minado pelo primeiro conjunto de crises na matemática no século XIX que resultou da construção de geometrias não euclidianas. Consequentemente, a matemática, particularmente uma matemática baseada na geometria euclidiana e na mecânica newtoniana abordada através da geometria euclidiana, não era mais um caminho seguro para a verdade. O conhecimento matemático não era mais um modelo para o conhecimento seguro. A imagem da matemática com a qual Marshall cresceu não era mais sustentável.

Essa mudança na matemática, baseada em uma nova concepção do que poderia significar a verdade matemática, ocorreu durante o segundo terço do século XIX e foi bem incorporada na tradição continental em matemática. Ou seja, fora da Inglaterra houve uma mudança na matemática entre o tempo da defesa de Whewell da matemática no processo educacional, o tempo de estudante de Marshall e o tempo posterior de Marshall como Professor de Economia Política. A era matemática de Whewell não era a de Marshall. O surgimento de geometrias não euclidianas fez o argumento de Whewell sobre axiomas e verdade inevitável soar vazio muito antes de 1906 e da carta de Marshall para Bowley. No tempo das novas geometrias, a dificuldade de vincular a verdade matemática a uma geometria (euclidiana) particular produziu uma crise de confiança para a prática educacional vitoriana, um ponto bem documentado (Richards 1988). Na verdade, foi essa crise que preparou a mente vitoriana tardia para a nova ideia de que o rigor matemático tinha que estar associado à argumentação física. Além disso, como veremos, foi essa imagem da matemática na ciência que moldou as preocupações de indivíduos como Edgeworth e Pareto.

Mas, no final do século, as imagens e os estilos de fazer matemática mudaram novamente em resposta aos estudos de Klein sobre geometria, à teoria dos conjuntos de Cantor e aos novos desafios à mecânica newtoniana surgidos na física. À medida que as influências continentais finalmente se infiltravam no mundo insular da matemática inglesa, Marshall foi pego. Sua imagem da matemática foi formada pelo Mathematical Tripos vitoriano de geometria simples, o desenho de segmentos de corda e seções cônicas, estática simples, dinâmica e afins. Sua concepção de matemática era incompatível com a matemática do final do século XIX de análise baseada em modelos físicos, ou aquela que a suplantaria por sua vez, a mudança do início do século XX para a axiomática e análise baseada em modelos matemáticos. A primeira mudança teria exigido uma economia matemática baseada em medição, enquanto a última teria exigido um afastamento do estudo do "homem nos negócios comuns da vida".

O paradoxo de Marshall, o Segundo Wrangler, cada vez mais desconfiado da matemática, tem sido visto como um problema para os historiadores da economia sob a perspectiva de uma matemática inalterada e um Marshall em mudança, um pequeno Das Alfred Marshall Problem: a visão de Marshall da matemática era contínua ao longo de sua vida, ou ele mudou de ideia sobre o papel da matemática na economia? Se o último, o historiador da economia então precisa de alguma explicação para as mudanças de Marshall. O que estou sugerindo é uma inversão da imagem usual. Eu sugiro que há um poder explicativo considerável na sugestão de que a imagem de Marshall da matemática foi formada em sua própria experiência do Mathematical Tripos e foi geralmente inalterada ao longo de sua vida. O "conselho" de Marshall a Bowley foi dado por um estudioso de sessenta e três anos perto da aposentadoria; Lembro-me da peroração na Teoria Geral de Keynes, na qual ele observa que "no campo da filosofia econômica e política, não há muitos que são influenciados por novas teorias depois de terem 25 ou 30 anos de idade" (Keynes 1936,383-84). O mesmo vale para a matemática.

\subsection{\textbf{Pags : 37-40}}
\subsubsection{\textbf{Remodelando a Matemática para o Novo Século}}
Os capítulos subsequentes explorarão a interconexão da análise econômica com as mudanças nas imagens da matemática que ocorreram no século XX; especificamente, como os matemáticos pensavam sobre a matemática mediava o próprio pensamento dos economistas sobre a matemática e, assim, moldava como os economistas passaram a entender o empreendimento de tornar a economia mais matemática. Por 1900, é claro, argumentos a favor de tornar a economia uma ciência matemática circulavam há décadas. Os apelos para transformar a economia em uma ciência, que cresceram a partir dos sucessos e prestígio da ciência em muitos países ao longo de grande parte do século XIX, deram lugar a um novo entendimento de que, para a economia ocupar seu lugar como a rainha das ciências sociais, ela precisava emular a rainha das ciências em si. Os capítulos subsequentes abordarão, de diferentes maneiras, exatamente o que isso significa. No entanto, antes de prosseguir para abordar tais questões em uma generalidade maior, deixe-me reiterar como essas questões teriam aparecido na época para Volterra e Klein. Como, à medida que a paisagem matemática estava sendo transformada por volta de 1900, esses matemáticos poderiam ter pensado sobre uma economia matemática e como ela estaria relacionada às suas visões sobre a própria matemática?

Tanto Volterra quanto Klein, antes do final do século, compartilhavam uma perspectiva que sugeriu que as aplicações da matemática a outras disciplinas (eles especificamente nomearam economia e biologia e física) dependeriam de medições e observações precisas e estratégias de modelagem nas disciplinas aplicadas. Tais estratégias permitiriam o uso da matemática, aparentemente a matemática dos sistemas dinâmicos determinísticos conectados à argumentação mecânica racional, para definir um programa aplicado razoável. Tal reducionismo determinístico seria bastante não observado no final do século pelos matemáticos. Essas ideias começariam a mudar por volta dessa época, mas não haviam mudado de forma tão clara e distinta que todos os matemáticos concordariam que a mudança havia ocorrido. Para um economista olhando para a matemática então, poderia-se construir uma teoria econômica e baseá-la em uma estrutura de raciocínio matemático, que por si só seria consistente e coerente e, tanto quanto a palavra pudesse ser usada, verdadeira.

É esse tipo de argumentação reducionista que atrai a atenção crítica de Philip Mirowski em More Heat Than Light, onde ele mostra como esse argumento funcionou nas tentativas de Leon Walras de conectar sua teoria às visões de Poincare, que por 1900 havia começado a abandonar tais ideias reducionistas. Contrastando Volterra com Poincare, Ingrao e Israel (1990) argumentam que:

Volterra seguiu Poincare ao considerar a relação entre a teoria e os dados empíricos (e a questão conexa da mensurabilidade das magnitudes básicas da teoria) como crucial, mas sua atitude ansiosa estava muito longe da adotada pelo matemático francês. Isso se devia às suas diferentes visões sobre o assunto da matemática aplicada e sobre a possibilidade de transferir o modelo formal-explicativo da física matemática para outros setores. Embora ambos estivessem cientes da crise que a ciência estava passando na época, suas reações foram diferentes. Poincare abordou de frente os temas cruciais da física contemporânea;

Volterra evitou-os e procurou consolidar o ponto de vista clássico, estendendo seu campo de aplicação a outros setores. (163)

Enquanto Klein buscava a aplicabilidade da matemática e olhava para a física, Volterra trabalhava nela tanto na economia quanto na biologia. Mirowski detalha as dificuldades que Walras teve com um Poincare que não foi persuadido pela perspectiva reducionista; podemos entender o argumento de Mirowski em nosso arcabouço para ser que Walras estava lidando com um matemático menos apegado ao passado do que Volterra. De fato, uma vez que Walras tomou conhecimento da própria perspectiva de Volterra, ele compartilhou a boa notícia com seu próprio filho, escrevendo para ele "você compreendeu perfeitamente a importância do artigo de Volterra. Como um matemático habilidoso, ele reconhece imediatamente que a revolução que tentamos e até mesmo realizamos na economia política e social é absolutamente a mesma que aquelas realizadas por Descartes, Lagrange, Maxwell e Helmholtz em geometria, mecânica, física e fisiologia" (Ingrao e Israel 1990,162).

Visto sob essa luz, os argumentos pessimistas de Marshall sobre a matemática parecem direcionados a um tipo particular de reducionismo, uma visão específica da aplicação da matemática comumente entendida nos anos antes do final do século XIX, mas uma imagem de matemática inconveniente para alguém criado nas ideias de Whewell sobre o papel da matemática. Não podemos interpretar esse argumento como desfavorecendo a noção da aplicabilidade da matemática em si: de fato, nosso ponto envolveu a inadmissibilidade de qualquer ideia essencialista. É por isso que os comentários de Marshall sobre a necessidade de "queimar a matemática" são tão curiosos para um economista moderno. No entanto, se o "queimar" se refere à recusa de Marshall em aceitar um reducionismo de mecânica racional porque sua imagem formada pelo Tripos do conhecimento matemático obriga "matemática aplicável igual a redução a um sistema mecânico newtoniano-euclidiano", seus comentários na carta a Bowley parecem fazer algum sentido. Mas então Marshall se torna ainda mais impossível de reconstruir a partir da perspectiva ainda mais tardia sobre a matemática na qual um problema econômico pode ser interpretado em termos de um modelo matemático (em oposição a um modelo físico-mecânico). Com nossa imagem atual da matemática, a carta para Bowley mostra que Marshall deve ter "se afastado" de seu otimismo juvenil.

Assim, 1900 prova ser um ponto de partida difícil para nossa discussão, pois a própria matemática muda novamente no início do século XX. Não apenas encontramos Marshall olhando para trás para os velhos dias do Tripos, mas Edgeworth e Volterra e Pareto estavam argumentando sobre a necessidade de uma ciência econômica empiricamente fundamentada que pudesse empregar com sucesso ideias matemáticas, enquanto ao mesmo tempo Klein estava procurando manter a nova matemática de axiomas à distância. Todos eles estavam escrevendo sobre matemática por volta de 1900, mas havia pelo menos três imagens de matemática emaranhadas nessas discussões sobre a natureza e o papel da matemática na economia no início do século XX.

A natureza das mudanças de uma matemática baseada na criação de verdade, para uma matemática baseada na argumentação física, para uma matemática moldada pelo que é referido como formalismo matemático, são questões extremamente complexas e controversas na história da matemática. No entanto, ao refletirmos essas questões de volta à economia, precisamos ver com mais precisão como elas começaram a se resolver nas práticas dos matemáticos que faziam economia. Deixe-me agora passar para um exame mais abrangente do italiano que acabamos de conhecer que desistiu da economia matemática, e seu discípulo matemático americano que tentou reformulá-la: Vito Volterra e Griffith Conrad Evans.


\section{\textbf{Weintraub e Mirowski (1994)}}
\subsection{\textbf{O Puro e o Aplicado: Bourbakismo Chega à Economia Matemática}}
\subsubsection{\textbf{O Argumento}}
Na mente de muitos, a tendência Bourbakista na matemática foi caracterizada pela busca de rigor em detrimento da preocupação com aplicações ou concessões didáticas ao não matemático, o que pareceria tornar o conceito de uma incursão Bourbakista em um campo de matemática aplicada um paradoxo. Argumentamos que tal conjuntura de fato aconteceu na economia matemática do pós-guerra, e descrevemos a carreira de Gerard Debreu para ilustrar como isso aconteceu. Usando o trabalho de Leo Corry sobre o destino do programa Bourbakista na matemática, demonstramos que muitos dos mesmos problemas da busca por uma estrutura formal com a qual fundamentar a prática matemática também aconteceram no caso de Debreu. Vemos este estudo de caso como um exemplar alternativo para discussões convencionais sobre a "eficácia irracional" da matemática na ciência.

Jorge Luis Borges, "A Casa de Asterion":
"O fato é que eu sou único. Não estou interessado no que um homem possa transmitir a outros homens; como o filósofo, penso que nada é comunicável pela arte da escrita. Detalhes incômodos e triviais não têm lugar no meu espírito, que está preparado para tudo o que é vasto e grandioso; nunca retive a diferença entre uma letra e outra."

Gerard Debreu, "A Matematização da Teoria Econômica":
"Antes do período contemporâneo das últimas cinco décadas, a física teórica havia sido um ideal inacessível ao qual a teoria econômica às vezes se esforçava. Durante esse período, esse esforço se tornou um poderoso estímulo na matematização da teoria econômica. As grandes teorias da física cobrem uma imensa gama de fenômenos com suprema economia de expressão. . . . Essa extrema concisão é possibilitada pela relação privilegiada que se desenvolveu ao longo de vários séculos entre física e matemática. . . . Os benefícios dessa relação especial foram grandes para ambos os campos; mas a física não se rendeu completamente ao abraço da matemática e à sua compulsão inerente para o rigor lógico. .. . Nessas direções, a teoria econômica não pôde seguir o modelo oferecido pela teoria física. . . . Sendo negada uma base experimental suficientemente segura, a teoria econômica tem que aderir às regras do discurso lógico e deve renunciar à facilidade da inconsistência interna."

\subsubsection{\textbf{Pureza e Perigo}}
Construir a história de uma área moderna de "matemática aplicada", como a economia matemática do século XX, parece condenada desde o início. O projeto é primeiro acorrentado por suspeitas sisíficas sobre escrever uma história de algo cujo modo de expressão se pensa não ter história, já que a matemática representa para muitos o epítome da verdade atemporal. Além disso, os próprios matemáticos abrigam uma hierarquia implícita do "puro" sobre o "aplicado" em seus campos de atuação, tornando este último quase invisível nas narrativas de seus heróis e vilões. E então há o problema de que a história da matemática persiste como o subcampo mais teimosamente internalista da história da ciência, defendendo sua pureza contra realistas e relativistas. Como se tudo isso não fosse ruim o suficiente, as histórias existentes de economia ou conseguem negligenciar o componente matemático por completo ou então o tratam como um presságio não problemático da economia ortodoxa moderna, desprovido de qualquer relação com a história da matemática - ou de fato qualquer outra coisa.

O primeiro autor deste artigo argumentou persistentemente na última década que a história da economia moderna deve levar em conta a história da matemática (Weintraub 1985, 1991, 1992). O segundo autor se esforçou para colocar a história da teoria econômica moderna em um conto de moldura da história de ciências como física, psicologia e ciência cognitiva (Mirowski 1989,1993,1994). Ambas as nossas preocupações convergem para a construção da hegemonia americana pós-guerra na teoria econômica matemática no período de 1930 a 1950, uma história que permanece em grande parte não escrita. Tal história não poderia existir em um vácuo, já que tanto naquele período quanto no presente, um dos maiores pontos de contenda dentro da economia tem sido o impacto e a significância do aumento substancial dos padrões de sofisticação matemática dentro da profissão (Beed e Kane 1992; Klamer e Colander 1990; Debreu 1991). No entanto, essas disputas metodológicas foram processadas de uma maneira profundamente a-histórica e internalista, sem dúvida devido à sua natureza inflamatória. Portanto, os fatores que militam contra uma narrativa histórica mais satisfatória parecem tão assustadores a ponto de precluir qualquer público para tal exercício.

Neste artigo, propomos tentar quebrar o impasse semiótico reformulando tanto a questão quanto o público. Em vez de tentar acalmar qualquer uma das constituições acima, este trabalho é direcionado a um público generalista de estudiosos de estudos científicos que possam estar interessados em um estudo de caso de como um modo distinto de matemática poderia fazer incursões em um campo aparentemente distante e subsequentemente transformar a autoimagem desse campo, bem como sua própria concepção de investigação. Para ser mais preciso, apresentaremos uma narrativa de como a escola Bourbakist de matemática migrou rapidamente para a economia matemática neoclássica. Cruzar essa fronteira disciplinar estabeleceu, para os economistas, o imponente edifício da teoria do equilíbrio geral de Walras, o marco da alta teoria em economia pelas próximas quatro décadas. 'Como acontece, nossa narrativa é tornada viável pelo fato de que a história pode ser contada principalmente por meio de uma única metonímia, a biografia intelectual de um ator individual excepcional - o vencedor do Prêmio Nobel Gerard Debreu.

Por que a história das atividades de um economista deveria ter alguma significância para a comunidade de estudos científicos? Acreditamos que a resposta reside na maneira como ilustra a interseção de preocupações técnicas, filosóficas e históricas quando se trata de contar o que acontece quando a sublimidade da matemática pura (a "música da razão", como Dieudonne a chamou) encontra a impureza do discurso científico - um confronto que só pode ser adiado, nunca completamente prevenido. Muitas vezes, esses problemas são tratados apenas como questões para a metamatemática, ou talvez a estranha especulação sobre as razões para a "eficácia irracional" da matemática nas ciências físicas. Mas como qualquer leitor de Mary Douglas pode atestar, a reflexão sobre o impuro envolve a reflexão sobre a relação da ordem com a desordem. Pode até ser que haja uma metáfora matemática oculta em sua própria insistência de que "rituais de pureza e impureza criam unidade na experiência. .. . Por meio deles, padrões simbólicos são elaborados e exibidos publicamente. Dentro desses padrões, elementos díspares são relacionados e experiências díspares são dadas significado" (Douglas 1989, 2-3).

Para nossos propósitos, a escola de Bourbaki servirá para representar os campeões da pureza dentro da casa da matemática do século XX. Embora Bourbaki dificilmente seja uma palavra familiar entre os historiadores, muitos matemáticos mais ou menos concordariam com nossa atribuição:

Por algumas décadas, no final dos anos trinta, quarenta e início dos cinquenta, a visão predominante nos círculos matemáticos americanos era a mesma de Bourbaki: a matemática é um assunto abstrato autônomo, sem necessidade de qualquer entrada do mundo real, com seus próprios critérios de profundidade e beleza, e com uma bússola interna para orientar o crescimento futuro. A maioria dos criadores da matemática moderna - certamente Gauss, Riemann, Poincare, Hilbert, Hadamard, Birkhoff, Weyl, Wiener, von Neumann - teria considerado essa visão como totalmente equivocada. (Lax 1989, 455-56)

E novamente,

O século XX foi, até recentemente, uma era de "matemática moderna" em um sentido bastante paralelo à "arte moderna" ou "arquitetura moderna" ou "música moderna". Ou seja, voltou-se para uma análise da abstração, glorificou a pureza e tentou simplificar seus resultados até que as raízes de cada ideia fossem manifestas. Essas tendências começaram no trabalho de Hilbert na Alemanha, foram grandemente estendidas na França pelo clube matemático secreto conhecido como "Bourbaki", e encontraram solo fértil no Texas, na escola topológica de R. L. Moore. Eventualmente, eles conquistaram essencialmente todo o mundo da matemática, até tentando romper as paredes do ensino médio no episódio desastroso da "nova matemática". (Mumford 1991)

Assim, Bourbaki veio a defender a primazia do puro sobre o aplicado, do rigoroso sobre o intuitivo, do essencial sobre o frívolo, do fundamental sobre o que um membro de Bourbaki chamou de "lixo axiomático". Eles também vieram a definir o isolamento disciplinar do departamento de matemática na América pós-guerra. É essa reputação de pureza e isolamento que atraiu a ira de muitos cientistas naturais nos últimos anos. Por exemplo, o físico Murray Gell-Mann escreveu: "A aparente divergência da matemática pura da ciência foi em parte uma ilusão produzida pela linguagem obscurantista e ultra-rigorosa usada pelos matemáticos, especialmente aqueles de uma persuasão Bourbakista, e por sua relutância em escrever exemplos não triviais em detalhes explícitos.... A matemática pura e a ciência estão finalmente sendo reunidas e, felizmente, a praga Bourbaki está morrendo" (Gell-Mann 1992,7). Ou se poderia citar o caso de Benoit Mandelbrot, ainda mais comovente por causa de sua relação de sangue com um membro de Bourbaki:

O estudo do caos e dos fractais... deveria provocar uma discussão sobre as profundas diferenças que existem... entre a abordagem "de cima para baixo" para o conhecimento e as várias abordagens "de baixo para cima" ou auto-organizadas. As primeiras tendem a ser construídas em torno de um princípio ou estrutura chave, isto é, em torno de uma ferramenta. E eles se sentem corretamente livres para modificar, restringir e limpar seu próprio escopo, excluindo tudo que não se encaixa. As últimas tendem a se organizar em torno de uma classe de problemas.... A abordagem de cima para baixo se torna típica da maioria das partes da matemática, depois que elas se tornam maduras e totalmente auto-referenciais, e encontra seu excesso de realização e caricatura destrutiva em Bourbaki. As questões sérias eram estratégia intelectual, em matemática e além, e poder político bruto. Uma manifestação óbvia da estratégia intelectual diz respeito ao "gosto". Para Bourbaki, os campos a serem incentivados eram poucos em número, e os campos a serem desencorajados ou suprimidos eram muitos. Eles foram tão longe a ponto de excluir (de fato, embora talvez não em lei) a maior parte da análise clássica dura. Também indigno era a maior parte da ciência descuidada, incluindo quase tudo de relevância futura para o caos e para os fractais. (Mandelbrot 1989, 10-11)

Para muitos cientistas, Bourbaki se tornou a palavra de ordem para o abismo que se abriu entre a matemática e suas aplicações, entre "rigor" e seu homeostato alternativo, os ditames da situação concreta do problema (Israel 1981). Nesse mundo, não pareceria que uma disciplina de "matemática aplicada" inspirada em Bourbakist constituiria um oxímoro? Tal fenômeno poderia ser muito mais do que uma simples contradição em termos? É nossa tese que tal coisa ocorreu na economia e, de fato, que ela se enraizou e floresceu no ambiente americano pós-guerra. A gêmula transoceânica era Gerard Debreu;

o leito para a economia foi a Comissão Cowles (Christ 1952, 1994) na Universidade de Chicago. Embora a história natural da economia matemática exija uma certa quantidade de trabalho detalhado, a narrativa resultante pode provar de interesse mais amplo para a comunidade de estudos científicos, na medida em que pode demonstrar como o puro e o impuro estavam constantemente intercalados na prática matemática, sugerir algumas das atrações e perigos que fertilizaram o transplante, e talvez também abrir a estufa da matemática para uma busca historiográfica pela influência de Bourbaki e outras versões de "imagens da matemática" (Corry 1989) em toda a gama das ciências no século XX.

\subsubsection{\textbf{Estruturas Puras para um Mundo Impuro}}
Quem ou o que era "Bourbaki" para que pudessem transformar tão completamente o mundo sóbrio da matemática? Embora os materiais primários sejam escassos e não exista uma história abrangente em inglês, basearemos nossa breve narrativa nos textos publicados por Bourbaki, algumas declarações sobre Bourbaki por ex-membros (Dieudonne 1970, 1982b; Cartan 1980; Guedj 1985; Adler 1988) e os importantes trabalhos de Corry (1992a, 1992b). Nossa intenção é principalmente preparar o palco para a aparição de nosso protagonista, Gerard Debreu, e não fornecer uma visão geral abrangente do fenômeno Bourbaki.

Em 1934-35, Claude Chevalley e Andre Weil decidiram tentar reintroduzir o rigor no ensino do cálculo reescrevendo um dos tratados clássicos franceses. Como Chevalley lembrou, "O projeto, naquela época, era extremamente ingênuo: a base para o ensino do cálculo diferencial era o Traité de Goursat, muito insuficiente em vários pontos. A ideia era escrever outro para substituí-lo. Isso, pensávamos, seria uma questão de um ou dois anos" (Guedj 1985, 19). O projeto (que continua até hoje) foi adotado como o trabalho do grupo original de sete: Henri Cartan, Claude Chevalley, Jean Delsarte, Jean Dieudonne, Szolem Mandelbrojt, Rene de Possel e Andre Weil; na nomenclatura de Bourbaki, eles são chamados de "fundadores". Continuando uma piada elaborada que havia sido jogada, ao longo do tempo, na Ecole Normale, eles se deram o nome de um obscuro general francês do século XIX, Nicolas Bourbaki, e concordaram em operar como um clube ou sociedade secreta. No início, eles concordaram que o modelo para o livro que desejavam fazer era a Álgebra de B. L. Van der Waerden, que havia aparecido em alemão em 1930:

Então, pretendíamos fazer algo desse tipo. Agora, Van der Waerden usa uma linguagem muito precisa e tem uma organização extremamente apertada do desenvolvimento de ideias e das diferentes partes do trabalho como um todo. Como isso nos parecia ser a melhor maneira de apresentar o livro, tivemos que elaborar muitas coisas que nunca antes haviam sido tratadas em detalhes. (Dieudonne 1970, 137)

A dificuldade era que este projeto era imenso. Assim, "rapidamente percebemos que havíamos nos precipitado em uma empresa que era consideravelmente mais vasta do que imaginávamos" (ibid.). O trabalho foi feito em reuniões ocasionais em Paris, mas principalmente em "congressos" - o mais longo dos quais ocorreu no campo francês a cada verão. As regras de Bourbaki rapidamente se estabeleceram, tanto as formais quanto as informais. Das regras formais, havia apenas uma, que era que nenhum membro do grupo deveria ter mais de cinquenta anos, e que ao atingir essa idade, um membro desistiria de seu lugar. No entanto, certos comportamentos se tornaram convencionais. Passou a haver duas reuniões por ano, além do congresso mais longo. O trabalho foi feito por indivíduos que concordaram em submeter rascunhos de capítulos ao grupo para leitura pública e para serem desmontados. Se o resultado não fosse aceito, e a aceitação exigisse unanimidade, o rascunho era dado a outra pessoa para ser refeito e reapresentado em um congresso subsequente. Até dois visitantes podiam participar dos congressos, desde que participassem plenamente; às vezes isso era uma maneira de ver se uma pessoa poderia ser considerada um potencial novo Bourbaki.

Nunca houve um exemplo de um primeiro rascunho sendo aceito. As decisões não ocorreram em um bloco. Nos congressos de Bourbaki, lia-se os rascunhos. Em cada linha havia sugestões, propostas de mudança escritas em um quadro negro. Desta forma, uma nova versão não nasceu de uma simples rejeição de um texto, mas sim emergiu de uma série de melhorias suficientemente importantes que foram propostas coletivamente. (Guedj 1985, 20)

A questão de que tipo de livro eles deveriam escrever rapidamente veio à tona em suas discussões. O que distingue o projeto Bourbaki é o resultado da decisão de Bourbaki de criar um livro "básico" para matemáticos.

A ideia que logo se tornou dominante era que o trabalho tinha que ser principalmente uma ferramenta. Tinha que ser algo utilizável não apenas em uma pequena parte da matemática, mas também no maior número possível de lugares matemáticos. Então, se quiser, tinha que se concentrar nas ideias matemáticas básicas e na pesquisa essencial. Tinha que rejeitar completamente qualquer coisa secundária que tivesse aplicação imediatamente conhecida [em matemática] e que não levasse diretamente a concepções de importância conhecida e comprovada. .. . Então, como escolhemos esses teoremas fundamentais? Bem, foi aí que surgiu a nova ideia: a da estrutura matemática. Não digo que era uma nova ideia de Bourbaki - não há dúvida de que Bourbaki contém algo original. Desde Hilbert e Dedekind, sabemos muito bem que grandes partes da matemática podem se desenvolver logicamente e de maneira frutífera a partir de um pequeno número de axiomas bem escolhidos. Ou seja, dadas as bases de uma teoria em forma axiomática, podemos desenvolver toda a teoria de uma forma mais compreensível do que poderíamos de outra forma. Isso é o que deu a ideia geral de estrutura matemática. Uma vez que essa ideia foi esclarecida, tivemos que decidir quais eram as estruturas matemáticas mais importantes. (Dieudonne 1970, 138, 107)

Em 1939, o primeiro livro apareceu, Elements de Mathematique I (Fascicule de Resultats). Este livro foi a primeira parte do primeiro volume, que é a teoria dos conjuntos. Ele apresentou o plano do trabalho e delineou as conexões entre as várias partes principais da matemática de uma maneira funcional, ou o que Bourbaki chamou de maneira estrutural. Ele continha

sem qualquer prova todas as notações e fórmulas em teoria dos conjuntos a serem usadas nos volumes subsequentes. Agora, quando cada novo volume aparece, ele assume sua posição lógica no conjunto da obra. . . . Bourbaki frequentemente coloca um relatório histórico no final de um capítulo. Nunca há referências históricas no próprio texto, pois Bourbaki nunca permitiu o menor desvio da organização lógica do trabalho em si. (Cartan [1958] 1980, 8)

Assim, em vez da divisão em álgebra, análise e geometria, os assuntos fundamentais, dos quais os outros poderiam ser derivados, deveriam ser teoria dos conjuntos, álgebra geral, topologia geral, análise clássica, espaços vetoriais topológicos e integração. Esta organização aparece nos próprios volumes, porque os seis primeiros livros, cada um compreendendo vários capítulos com numerosos exercícios, correspondem a essas seis divisões. Os vinte e um volumes publicados no final dos anos 1950 pertencem todos à Parte I, "As Estruturas Fundamentais da Análise".

"Uma média de 8-12 anos é necessária desde o primeiro momento em que começamos a trabalhar em um capítulo até o momento em que ele aparece em uma livraria" (Dieudonne 1970,142,110). O tempo parece ser um resultado tanto da regra de unanimidade para os congressos quanto da complexidade da tarefa em si.

O que se previa era um repertório das definições e teoremas mais úteis (com provas completas...) de que os matemáticos pesquisadores poderiam precisar . . . apresentados com uma generalidade adequada à mais ampla gama possível de aplicações. Em outras palavras, o tratado de Bourbaki foi planejado como um conjunto de ferramentas, um kit de ferramentas, para o matemático trabalhador."(Dieudonne 1982b, 618)

Esta visão levou à ideia organizadora fundamental do trabalho: "Era nosso propósito produzir a teoria geral primeiro antes de passar para as aplicações, de acordo com o princípio que adotamos de ir 'do mais geral (generalissime) ao particular'" (Guedj 1985,20). Chevalley lembrou as visões dos "fundadores" do projeto quanto à sua importância assim:

Parecia muito claro que ninguém era obrigado a ler Bourbaki... uma bíblia em matemática não é como uma bíblia em outras disciplinas. É um cemitério muito bem organizado com uma bela variedade de lápides.  Havia algo que nos oprimia a todos: tudo o que escrevíamos seria inútil para o ensino. (Guedj 1985, 20)

Foi através do Seminaire Bourbaki que os matemáticos franceses, após a guerra, se reconectaram à comunidade matemática mundial. O projeto dos Elementos ganhou impulso, e os convites para vir dar palestras em Paris foram apreciados. A imensa força dos matemáticos franceses em várias áreas importantes fez com que o projeto fosse cada vez mais notado entre os matemáticos nos Estados Unidos. A natureza internacional da comunidade matemática e as conexões pré-guerra dos poucos homens mais velhos, particularmente Andre Weil, facilitaram o reconhecimento do trabalho. O mistério de Bourbaki, e a ambição do projeto, provavelmente atraíram atenção também.

Bourbaki teve o grande problema, ao escrever os Elementos, de organização, de relacionar as várias partes da matemática uma com a outra. Este "problema" foi abordado através da noção de "estrutura matemática" - do qual falaremos mais adiante. A segunda questão que Bourbaki teve que enfrentar foi a da abordagem a ser tomada dentro de cada seção do todo, e isso foi tratado pela regra "do geral ao específico". Assim, à medida que os livros e capítulos emergiam do editor, e o imenso projeto tomava forma em impressão ao longo das décadas, a matemática era apresentada como autocontida, no sentido de que crescia a partir de si mesma - das estruturas básicas àquelas mais derivadas, das "estruturas-mãe" àquelas das áreas específicas da matemática. Por exemplo:

Na ordem lógica do sistema Bourbaki, os números reais não podiam aparecer no início do trabalho. Eles aparecem em vez disso no quarto capítulo do terceiro livro. E com razão, pois subjacente à teoria dos números reais está a interação simultânea de três tipos de estruturas. Uma vez que o método de Bourbaki de deduzir casos especiais do mais geral, a construção dos números reais a partir dos racionais é para ele um caso especial de uma construção mais geral: a conclusão de um grupo topológico (Capítulo 3 no Livro III.) E esta conclusão é baseada na teoria da conclusão de um "espaço uniforme" (Capítulo 2 no Livro III). (Cartan [1958] 1980, 178)

O que esses princípios organizadores realizaram, ao tornar o trabalho em si coerente, não pode ser subestimado. As escolhas que Bourbaki fez foram razoáveis para a imensa tarefa de escrever um manual de matemática para matemáticos trabalhadores. A imposição desta ordem, e coerência, levou a um livro com a elegância e graça de uma obra-prima, uma versão moderna dos Elementos de Euclides. Mas as ideias de estrutura e o movimento do livro do geral para o específico tiveram grandes consequências.

A palavra “estrutura”, seja em francês ou em inglês, pode significar muitas coisas para muitas pessoas. A tentação imediata é associá-la ao movimento filosófico e cultural francês conhecido como “estruturalismo” (Caws 1988); e há alguma justificativa para essa inclinação, como as conexões entre Andre Weil e um dos gurus do movimento, Claude Levi-Strauss. De fato, o título de um dos poucos pronunciamentos metodológicos explicitamente de Bourbaki foi “A Arquitetura da Matemática”, publicado em francês em 1948, antecipando o título e parte do conteúdo do próprio L’archeologie du savoir de Michel Foucault por duas décadas (Gutting 1989). Outros insistem em um contexto filosófico mais estritamente definido de um “estruturalismo matemático” específico (Chihara 1990, cap. 7). Lamentavelmente, passamos por essas questões históricas tentadoras e optamos por nos concentrar mais estreitamente na própria conta de Bourbaki sobre o significado da estrutura, e no esclarecimento dessas questões fornecido por Corry (1992a).

A questão que motivou Bourbaki foi: “Temos hoje uma matemática ou temos várias matemáticas?” ([1948] 1950,221). O medo da desordem, ou “sujeira”, como Mary Douglas diria, era a ordem do dia, com Bourbaki se perguntando “se o domínio da matemática não está se tornando uma Torre de Babel?” (Ibid.). Bourbaki não gostaria de fazer essa pergunta àqueles trabalhadores habituais e lixeiros do conhecimento, os filósofos, mas sim a um tipo ideal, que eles identificaram como o “matemático trabalhador”. Este homme moyen foi supostamente definido por seu recurso ao “formalismo matemático”: “O método de raciocinar estabelecendo cadeias de silogismos Para estabelecer as regras dessa linguagem, para estabelecer seu vocabulário e para esclarecer sua sintaxe” (ibid., 223). Bourbaki continua a afirmar, no entanto, que isso é

mas um aspecto do método axiomático… [que] estabelece como seu objetivo essencial… a profunda inteligibilidade da matemática Onde o observador superficial vê apenas duas, ou várias, teorias bastante distintas. O método axiomático nos ensina a procurar as razões profundas para tal descoberta, encontrar as ideias comuns dessas teorias, enterradas sob o acúmulo de detalhes pertencentes a cada uma delas, trazer essas ideias à tona e colocá-las na luz adequada. (Ibid.)

Ele então prossegue para sugerir que o ponto de partida do método axiomático é uma preocupação com “estruturas” e desenvolve essa ideia de estrutura através de exemplos. De acordo com a definição informal, “estrutura” é um nome genérico que pode ser aplicado a conjuntos de elementos cuja natureza* não foi especificada; para definir uma estrutura, toma-se como dado uma ou várias relações, nas quais esses elementos entram* no caso de grupos, essa era a relação z = xty entre os três elementos arbitrários); então se postula que a relação dada, ou relações, satisfazem certas condições (que são explicitamente declaradas e que são os axiomas da estrutura em consideração). Para estabelecer a teoria axiomática de uma determinada estrutura, equivale à dedução das consequências lógicas dos axiomas da estrutura, excluindo qualquer outra hipótese sobre os elementos em consideração (em particular, qualquer hipótese quanto à sua própria natureza). (Ibid., 225-26)

Esta passagem notável é, de fato, a pedra angular da empresa, pois contém nela e fora dela, pelo que exclui, a matemática Bourbaki.

Primeiro, observe a nota de rodapé anexada a “natureza” (*). Bourbaki comenta que as preocupações filosóficas devem ser evitadas aqui, nos debates sobre fundamentos formalistas, idealistas, intuicionistas. Em vez disso, “Deste novo ponto de vista, as estruturas matemáticas se tornam, propriamente ditas, os únicos ‘objetos’ da matemática” (ibid., 225-226, nota). Ou seja, a matemática se preocupa com objetos matemáticos, chamados estruturas, se quiser, e o trabalho dos matemáticos é fazer matemática atendendo a essas estruturas. Bourbaki continua a dizer, na próxima nota de rodapé (#), ligada à palavra “entrar”, que “esta definição de estruturas não é suficientemente geral para as necessidades da matemática” por causa da necessidade de considerar estruturas de ordem superior ou, em efeito, estruturas cujos elementos são estruturas. As questões de incompletude de Godel são deixadas de lado; pois os matemáticos simplesmente fazem matemática, e quando surge uma inconsistência, a regra é enfrentá-la e fazer matemática em torno dela quase em um sentido empírico.

O que tudo isso significa é que a matemática tem menos do que nunca sido reduzida a um jogo puramente mecânico de fórmulas isoladas; mais do que nunca a intuição domina no gênio das descobertas. Mas doravante, possui as poderosas ferramentas fornecidas pelos grandes tipos de estruturas; em uma única visão, varre domínios imensos, agora unificados pelo método axiomático, mas que antes estavam em um estado completamente caótico. (Ibid., 228)

Em seu artigo de 1949, Bourbaki apresenta seu programa real para fundamentos no mundo da lógica pós-Gödel:

Qual será a atitude do matemático trabalhador quando confrontado com tais dilemas [Gödel]? Não precisa, acredito eu, ser outra coisa senão estritamente empírica. Não podemos esperar provar que toda definição, todo símbolo, toda abreviação que introduzimos está livre de ambiguidades potenciais, que não traz a possibilidade de uma contradição que talvez não teria sido presente de outra forma. Que as regras sejam tão formuladas, as definições tão dispostas, que toda contradição possa ser mais facilmente rastreada até sua causa, e esta última seja removida ou tão cercada por sinais de advertência a fim de evitar problemas. Isso, para o matemático, deve ser suficiente; e é com esse objetivo comparativamente modesto e limitado em vista que busquei estabelecer as bases para meu tratado matemático." (Bourbaki 1949, 3)

O que temos aqui é uma “admissão”, por assim dizer, de que não há mais segurança a ser encontrada na ideia magistral de “estrutura” do que havia na ideia de “conjunto” ou “número” como o alicerce sobre o qual uma matemática segura poderia ser construída. No entanto, Bourbaki apresenta neste artigo a “linguagem de sinais” de objetos, sinais, relações, etc., para acabar com uma linguagem na qual ele pode proceder para fazer matemática. Que isso não seja necessariamente consistente não é uma preocupação para o matemático trabalhador, pois basta fazer a matemática de Bourbaki. Temos então uma disjunção entre a estrutura em itálico de Corry e “estrutura”.

Leo Corry (1989) sugeriu que a matemática deveria ser separada de outras ciências porque se esforça persistentemente para aplicar as ferramentas e critérios de suas práticas atuais a si mesma de maneira meta-analítica, mascarando assim a distinção entre o “corpo de conhecimento”, que é característico de uma conjuntura histórica particular, e as “imagens do conhecimento”, que são implantadas para organizar e motivar a investigação. Para Corry, são as imagens do conhecimento e não o corpus real de provas e refutações que são derrubadas e transformadas sempre que as escolas e modas matemáticas mudam ao longo da história. A principal ilustração desta tese de Corry é sua descrição (1992a) das variantes de “estrutura” e estrutura de Bourbaki.

No Fascículo de 1939, doravante citado como a Teoria dos Conjuntos, Bourbaki propôs apresentar no Capítulo IV as fundações - a base formalmente rigorosa de todo o seu empreendimento. Esta coleção de formalismos, que Corry designa como a palavra em itálico estrutura, envolveu conjuntos base e um esquema de construção de escalão que se destinava a gerar estruturas-mãe, que por sua vez gerariam o resto da matemática como Bourbaki a via. No entanto, houve uma disjunção entre este capítulo e o resto do livro, bem como com todos os outros volumes do corpus de Bourbaki.

O objetivo declarado de Bourbaki ao introduzir tais conceitos é expandir o aparato conceitual sobre o qual o desenvolvimento unificado das teorias matemáticas se basearia posteriormente. No entanto, todo esse trabalho acaba sendo bastante redundante, já que… esses conceitos são usados de uma maneira muito limitada - e certamente não altamente esclarecedora ou unificadora - no restante do tratado. (Corry 1992a, 324).

Parece que o conceito de estrutura não tem uso matemático palpável no restante do trabalho de Bourbaki, e os vínculos entre o aparato formal e o matemático trabalhador estão em grande parte ausentes. “Nenhum novo teorema é obtido através da abordagem estrutural e os teoremas padrão são tratados das maneiras padrão” (ibid., 329). No entanto, como já testemunhamos, o ideal de “estrutura” e a realização de Bourbaki permaneceram identificados nas mentes daqueles que vieram depois. Como isso pode ser?

Corry responde que isso tem a ver com a diferença entre o corpo real de resultados e a imagem do conhecimento. “Se o objetivo declarado do livro era mostrar que podemos estabelecer formalmente uma base sólida para a matemática, o propósito do fascículo é nos informar sobre o léxico que usaremos no que segue e sobre o significado informal dos termos dentro dele. A mudança repentina de abordagem, de um estilo estritamente formal para um estilo completamente informal, é claramente admitida” (ibid., 326). Este é o significado prático da “estrutura” não itálica de Corry: a principal contribuição de Bourbaki teve a ver com a maneira como os matemáticos interpretavam seu trabalho matemático, e não com os fundamentos formais desse trabalho em si. Era, se quiser, uma questão de estilo, de gosto, de opiniões compartilhadas sobre o que era valioso na matemática, de todas aquelas coisas que não deveriam realmente importar para o platonista ou o formalista ou o intuicionista. Se a matemática é a música da razão, então Bourbaki acabou sendo seu Sol Hurok ou Brian Epstein, e não seu Pierre Boulez ou Pink Floyd. (O que se aplica ao coletivo não precisa se aplicar aos membros individuais, no entanto.) Ou como Corry colocou, “O estilo de Bourbaki é geralmente descrito como um de rigor intransigente sem concessões heurísticas ou didáticas ao leitor. . [Mas na Teoria dos Conjuntos] a linguagem formal que foi introduzida passo a passo no Capítulo 1 é quase abandonada e rapidamente substituída pela linguagem natural” (ibid., 321).

O legado final de Bourbaki é mais curioso. Como Corry resumiu em 1992b, (p. 15): “Bourbaki não adotou o formalismo com total compromisso filosófico, mas sim como uma fachada para evitar dificuldades filosóficas.” Outros agora concordam com essa avaliação (veja Mathias 1992). Bourbaki deu a impressão de elevar suas escolhas em matemática acima de toda disputa: mas era tudo o que era - apenas uma impressão. “Está [agora] claro que os primeiros desenvolvimentos da formação categórica, mais flexíveis e eficazes do que os fornecidos pelas estruturas, tornaram questionáveis as esperanças iniciais de Bourbaki de encontrar a melhor fundação única para cada ideia matemática e lançaram dúvidas sobre a universalidade inicialmente pretendida da empresa de Bourbaki. [Como escreveu Saunders Mac Lane] uma boa teoria geral não busca a generalidade máxima, mas a generalidade certa” (Corry 1992a, 336). Mas essa percepção levou tempo, acontecendo possivelmente tão tarde quanto os anos 1970; e no ínterim, o juggernaut de Bourbaki continuou produzindo mais volumes. O momento desses eventos é de alguma importância para nossa narrativa subsequente.

Esses detalhes sobre a história de Bourbaki e a leitura de Corry sobre ela, aparentemente tão distantes da economia, são absolutamente centrais para entender sua evolução pós-guerra. A razão é que quase tudo o que foi dito sobre Bourbaki se aplica com igual força a Gerard Debreu.

\subsubsection{\textbf{Gerard Debreu e a Criação de uma Economia Pura}}
Quando se aborda o lugar da matemática na economia, é Debreu quem é sempre mencionado com admiração, e não pouca apreensão. “Debreu é conhecido por sua abordagem direta e sem rodeios ao assunto”, escreve Samuelson (1983, 988). “As contribuições de Debreu podem parecer, à primeira vista, incompreensivelmente ‘abstratas’… Nesse aspecto, Debreu nunca fez concessões, assim como nunca seguiu modas na pesquisa econômica”, escreve seu memorialista Werner Hildenbrand. “Debreu apresenta suas contribuições científicas da maneira mais honesta possível, declarando explicitamente todas as suposições subjacentes e evitando, em qualquer estágio da análise, interpretações floridas que possam desviar a atenção da restrição das suposições e levar o leitor a tirar conclusões falsas” (Hildenbrand 1983b, 2-3). Quando George Feiwel tentou conduzir uma história oral, ele foi reduzido a prever muitas de suas perguntas com a cláusula, “Para o benefício dos não educados. …” Em resposta à pergunta, “Por que a questão da existência do equilíbrio econômico geral é tão profundamente importante?” Debreu respondeu, “Como eu não vi sua pergunta discutida nos termos que eu gostaria de usar, eu não vou te dar uma resposta concisa” (Feiwel 1987,243). No entanto, quando um dos autores atuais o entrevistou em 1992, ele foi gracioso e prestativo em responder a muitas perguntas sobre sua carreira.

Debreu é talvez mais conhecido por sua prova conjunta de 1954 com Kenneth Arrow da existência de um equilíbrio competitivo geral de Walras (Weintraub 1985), e sua monografia de 1959 A Teoria do Valor, que ainda se mantém como a axiomatização referência do modelo de equilíbrio geral de Walras. Em retrospecto, o livro de 1959 ostentava suas credenciais bourbakistas na manga, embora possa ter havido poucos economistas naquela conjuntura que teriam entendido as implicações desta declaração:

A teoria do valor é tratada aqui com os padrões de rigor da escola formalista contemporânea de matemática. O esforço em direção ao rigor substitui raciocínios e resultados corretos por incorretos, mas também oferece outras recompensas. Geralmente leva a uma compreensão mais profunda dos problemas aos quais é aplicado, e isso não deixou de acontecer no presente caso. Também pode levar a uma mudança radical de ferramentas matemáticas. Na área em discussão, foi essencialmente uma mudança do cálculo para a convexidade e propriedades topológicas, uma transformação que resultou em ganhos notáveis na generalidade e na simplicidade da teoria. A lealdade ao rigor dita a forma axiomática da análise onde a teoria, no sentido estrito, está logicamente completamente desconectada de suas interpretações. Para destacar plenamente essa desconexão, todas as definições, todas as hipóteses e os principais resultados da teoria, no sentido estrito, são distinguidos por itálicos; além disso, a transição da discussão informal das interpretações para a construção formal da teoria é frequentemente marcada por uma das expressões: “na linguagem da teoria”, “pelo bem da teoria”, “formalmente”. Tal dicotomia revela todas as suposições e a estrutura lógica da análise. (Debreu 1959, x)

Embora fosse o caso de que a maioria dos economistas estaria desconhecida naquela época com as novas ferramentas de teoria dos conjuntos, teoremas de ponto fixo e pré-ordenações parciais, havia algo mais que os teria surpreendido: uma certa atitude de não fazer prisioneiros quando se tratava de especificar o conteúdo “econômico” do exercício. Embora tenha havido saltos quânticos de sofisticação matemática antes na história da economia, nunca houve nada parecido com isso. Por exemplo, poucos teriam reconhecido prontamente o retrato de uma “economia” esboçada na monografia:

Uma economia E é definida por: para cada i = l ma um subconjunto não vazio xt de Rl completamente pré-ordenado por <;; para cada y= 1,…,«; um subconjunto não vazio de j7of /?‘; um ponto <o de/?1 . Um estado de/sis um (m+ri) -tuple de pontos de /?’. (Ibid., 75)

Enquanto mais de um membro da profissão poderia ter pensado que essa espécie de economista havia caído de Marte, na verdade, ele apenas migrou da França. A maneira como isso aconteceu pode ajudar a explicar o caráter totalmente sem precedentes desse tipo de economia matemática.

Gerard Debreu nasceu em 4 de julho de 1921 em Calais, França. Ele teve uma carreira escolar inicial bem-sucedida, preparando-se para o bacharelado estudando física e matemática. Seus planos de estudar em um liceu para ingressar em uma das Grandes Ecoles foram interrompidos pelo início da guerra, mas ele conseguiu uma preparação adicional em matemática em Grenoble; ele ganhou o Concours General em física em 1939, e mais tarde a admissão na Ecole Normale Superieure.

O grupo que ingressou na Ecole Normale Superieure foi dividido aproximadamente pela metade, com cerca de cinquenta alunos cada nas humanidades e ciências. Cerca de vinte eram, portanto, estudantes de matemática.

As ciências eram basicamente divididas entre matemática de um lado e física e química do outro (as duas andavam juntas) e havia uma terceira possibilidade (mas muito poucos estudantes seguiram esse caminho), que era a biologia. E imagino que em nosso grupo talvez apenas um ou dois tenham seguido o caminho da biologia, enquanto a divisão entre matemática e física e química era mais ou menos igual. Todos os estudantes de ciências fizeram o mesmo exame para entrar na escola, e então decidimos qual caminho seguir. Em matemática, era normalmente um curso de três anos e em física acho que era quatro. E em um momento eu pensei que queria me distanciar da matemática porque era muito abstrata, e como escrevi em outro lugar eu estava interessado em várias outras direções. Uma delas era a economia, como você bem sabe, mas uma era astrofísica, embora eu não tenha ido muito longe. O problema na astrofísica era que, em primeiro lugar, o corpo docente da Universidade de Paris foi esgotado durante a Segunda Guerra Mundial. Acho que alguns deles eram judeus e não era prudente para eles ficarem em Paris. E outros eram comunistas (e alguns eram ambos!) Então o que aconteceu na astrofísica é que quando eu olhei em volta, eu encontrei - talvez minha busca não tenha sido longa o suficiente, profunda o suficiente - mas eu tive a impressão de que não havia corpo docente, então não era um campo muito promissor porque eu teria que estudar inteiramente por conta própria para permanecer nesse campo. (Debreu 1992b)

O treinamento matemático que Debreu recebeu na Ecole Normale Superieure foi muito diferente do que ele teve anteriormente. A instrução era realizada, em matemática, de uma maneira complicada.

É muito estranho. Novamente, é único. Se você pegar outra Grande Ecole como a Ecole Polytechnique, eles têm todo o seu ensino dentro da escola apenas para os alunos de lá. Não na Ecole Normale Superieure. Está perto da Sorbonne, geograficamente perto, e deveríamos fazer os cursos padrão na Sorbonne. E o que tínhamos na Ecole Normale Superieure eram seminários muito pequenos; é aí que fomos ensinados por Cartan. Não havia currículo fixo, e era frequentado por cerca de 10 pessoas, enquanto nos cursos fundamentais na Sorbonne a frequência era inicialmente de centenas. Eu me lembro de um curso ministrado pelo físico…, acredito que ele é o pai de um Primeiro Ministro, e descobri que, como havia tantos alunos (e as palestras estavam disponíveis por escrito), parei de ir a elas completamente. O que faltava nelas então era o entusiasmo que Cartan gerava. (Ibid.)

O instrutor que Debreu mais se lembra é Henri Cartan, um dos “fundadores” de Bourbaki.

É muito provável que eu o tenha conhecido em 1941, mas posso estar errado por um ano. De qualquer forma, eu estava ciente de quais volumes de Bourbaki já haviam aparecido, que de fato em 1941 era muito pouco, acho que eram apenas dois volumes. E mesmo assim, um era um resumo. (Ibid.)

Para Debreu, o trabalho matemático era interessante, mas ele já tinha alguma ideia de que talvez fosse se envolver mais com matemática em outra disciplina. Talvez isso tenha sido por causa de seu sucesso anterior em física, talvez seja porque ele atingiu um limite, por assim dizer, em sua capacidade de sustentar o interesse em matemática pura sob as condições da Paris em tempo de guerra. De qualquer forma, Debreu parece ter entendido, bastante cedo em sua carreira na Ecole Normale Superieure, que seu próprio caminho seria um pouco diferente daquele dos matemáticos.

O objetivo na Ecole Normale Superieure era basicamente produzir professores de matemática; e isso foi entendido nos dias em que eu estava lá, para significar professores de matemática no nível Mathematiques Speciale Preparatoire e Mathematiques Sp6ciale. Os alunos tinham que tomar decisões então se queriam se tornar professores ou pesquisadores, e alguns deles foram de um jeito e outros foram de outro. Eu não sei se a decisão foi tomada quando entramos, ou se descobrimos dois anos depois que talvez quiséssemos fazer pesquisa. Depois de um ano ou mais (entrei no outono de 41), comecei a me perguntar se a matemática, naquela época, estava se tornando muito abstrata sob a influência de Bourbaki - embora não tão dominante quanto se tornou mais tarde (embora talvez eu tenha antecipado esse desenvolvimento). Eu tinha que decidir se queria passar toda a minha vida fazendo pesquisa em um assunto muito abstrato. Você também deve se lembrar que durante aquele último ano do ensino médio, quando fui influenciado por meu professor de física, pensei que a física seria minha área. (Ibid.)

Seu treinamento na Ecole Normale Superieure estava no nível universitário mais alto, e na verdade pode ser melhor comparado ao trabalho feito no nível de pós-graduação na maioria das outras universidades, porque os alunos tinham que fazer o currículo universitário padrão de matemática por conta própria, por assim dizer. Na Sorbonne

eram palestras, palestras bastante polidas. Eu assisti fielmente às palestras de Garnier, e Garnier ensinou geometria diferencial. Valiron ensinou análise clássica, e mais tarde eu assisti a palestras de Gaston Julia sobre Espaço de Hilbert. Tenho certeza de que assisti a outras palestras também. E então no seminário [na ENS] era uma mistura; ocasionalmente tínhamos uma palestra de Elie Cartan, o pai de Henri, que é claro já era naquela época um matemático muito reverenciado. Tivemos uma palestra de De Broglie, o físico, ganhador do Prêmio Nobel. Então o seminário era um pouco de coisas diferentes por pessoas diferentes. Henri Cartan ainda era jovem, e fez grandes coisas mais tarde, e o seminário deveria simplesmente revisar a matemática, e fez isso; também era para nos dar um gostinho de uma variedade de pesquisas matemáticas, e nenhum texto foi usado. Por conta própria, li a maior parte de Goursat. (Ibid.)

O ponto da história, é claro, é que Debreu foi tão bem treinado em matemática quanto era possível para qualquer estudante ter sido naquela época. Ele teve a notável boa sorte de estar no lugar, na hora, quando a própria matemática estava sendo representada por Bourbaki como uma disciplina definida por sua busca pelas implicações e pela investigação e exposição da ideia de estrutura matemática. Neste ambiente fervilhante de matemática, isolado por causa da guerra e das deslocações que ela produziu, Debreu perseguiu a matemática, mas não queria que ela definisse sua vida intelectual. Mas não havia alternativas reais; ele era um estudante de matemática em primeiro lugar, e outras possibilidades teriam que ser adiadas para o fim da guerra, já que a única alternativa aplicada que ele poderia ter considerado na ENS, astrofísica, parecia ser descartada pela ausência de qualquer programa instrucional real - o professor não estava presente. Preso na matemática, por volta de 1943 ele olhou para as possibilidades de trabalho futuro:

Quando me interessei pela economia como uma possibilidade (como antes eu me interessava pela astrofísica) eu peguei o texto padrão estudado pelos estudantes de economia na universidade. Eu não me lembro quem era o autor, mas era muito não teórico (alguém que eu nunca conheci). Eu sei que o livro didático era popular então - eu não acredito que eu tenha guardado uma cópia - mas em qualquer caso, minha primeira impressão da economia foi muito decepcionante porque eu estava vindo de um mundo de matemática muito sofisticada e rarefeita e encontrei apenas uma abordagem muito pedestre para a economia. (Ibid.)

Debreu contou, em alguns lugares onde o material autobiográfico está disponível, a casualidade de receber uma cópia do livro de Allais de 1943, A la recherche d’une discipline economique. Em conversa, ele observou que sua mudança para a economia foi uma característica de suas próprias mudanças intelectuais, bem como uma circunstância dos tempos:

Em grande medida, foi pura chance porque Allais enviou seu livro para um amigo meu, que era um humanista, que era o presidente de sua turma na ENS - ele na verdade não estava na minha turma, mas um ano depois, mas éramos amigos e ele me deu sua cópia. Suponho que, caso contrário, se eu tivesse perseverado em meu interesse em economia, talvez não tivesse conhecimento do livro de Allais por meses, e talvez até então seria tarde demais. Mas uma parte do meu interesse em economia - embora não fosse um campo muito elegante - era simplesmente que a economia de guerra na França era especial. Acreditávamos, embora tenhamos achado que demorou muito para chegar, que a Alemanha seria derrotada. Estava claro que haveria muita reconstrução, em particular na França. Haveria muito trabalho de reconstrução a fazer após a guerra e isso se provou ser o caso. E talvez seja por isso que eu vim a conhecer pessoas como Pierre Massey, que foi em um momento presidente da Electricite de France, mas que também escreveu um livro sobre gestão de estoque. Eu o conheci muito bem e o vi regularmente até sua morte. Ele foi sucedido como Presidente da Electricite de France por um amigo meu, Marcel Boiteaux, que também teve sua carreira perturbada pela guerra. Ele era um oficial em algum lugar na Itália e mais tarde, após a Agregação, lançou sua sorte com a Electricite de France. Compartilhamos a extraordinária história do sorteio da bolsa Rockefeller. Então, uma série de eventos aleatórios foram certamente muito importantes [na minha mudança para a economia].

O livro de Allais chegou às mãos de Debreu em um momento muito crucial, pois Debreu estava procurando um trabalho significativo, como muitos jovens procuram naquela idade. O livro de Allais foi mais ou menos apenas enviado para indivíduos; não foram impressas muitas cópias. Estava muito fora dos canais estabelecidos da economia francesa. Em retrospecto, é não apenas notável que tenha sido possível imprimi-lo sob aquelas condições de guerra, mas também que tenha recebido qualquer atenção, dado o número de livros incomuns que simplesmente desaparecem. Mesmo que tivesse ido para todos os interessados em economia matemática naquela época, haveria problemas. Era uma matemática muito primitiva do ponto de vista de Bourbaki, embora fosse mais sofisticada do que a maioria dos textos neoclássicos. No entanto, em Debreu, encontrou um leitor amigável.

Primeiro de tudo, eu vi que a matemática poderia ser usada na economia de uma maneira rigorosa, mesmo que não fosse o tipo de matemática de que eu mais gostava. E talvez eu sentisse que havia muito a ser feito com matemática mais sofisticada na economia. Meu interesse em economia não estava pronto. Eu me interessei pela economia em 1943 ou talvez um ano antes. E as circunstâncias eram tais que Bonpere me deu aquele livro por volta de abril de 1944. Mas acho que as circunstâncias não eram ideais para eu lê-lo naquele momento - lembre-se de que o Dia D foi em 6 de junho. Foi só em setembro, acho, quando ficou claro em primeiro lugar que eu não começaria outro ano acadêmico normalmente. As coisas estavam muito caóticas na França e foi então, acho, que me tornei sério sobre a economia; eu posso ter olhado para ele quando Bonpere me deu, mas eu não o estudei. Como algumas pessoas conseguiram cópias, meu palpite seria que a maioria delas deve ter sido repelida por todas as características do livro, era longo, técnico, etc. Mas lembre-se de que eu estava em uma situação muito bizarra porque ficou claro para mim que eu não seria um matemático profissional. Foi tarde quando decidi isso, e minha carreira foi perturbada pelos eventos da guerra. Eu deveria fazer o exame final na primavera de 1944 após 3 anos, mas foi exatamente quando ocorreu o Dia D. Eu fiz esse exame eventualmente em 1946. Então eu terminei meus estudos dois anos atrasado: eu deveria ter terminado em 1944 e eu realmente terminei em 1946. Então é difícil dizer o que teria acontecido se o Dia D não tivesse ocorrido porque esse atraso de dois anos (parte do qual passei no Exército Francês) me deu a chance de me familiarizar muito melhor com a economia do que teria sido se meu currículo tivesse seguido seu curso normal. Mas, em qualquer caso, quando fiz a Agregação, foi um exame de matemática pura. Foi uma situação um tanto bizarra. Foi um exame de natureza um tanto escolástica, que foi ainda mais para mim; era muito clássico, enquanto Cartan, em particular, fez matemática contemporânea, o que não era o caso com a Agregação, então eu tive uma série de problemas. (Ibid.)

A história, em resumo, é clara neste ponto. Debreu era um matemático de pesquisa muito bem treinado, moldado no modelo Bourbaki. No entanto, ele também se encaixa em um perfil bastante comum de muitas figuras-chave na economia política dos anos 1930 e 1940: alguém com muito pouco conhecimento em economia, passando para esse campo após um treinamento acadêmico completo em física ou matemática (Mirowski 1991). O engenheiro autodidata Allais não era de forma alguma representativo do estado da economia política na França na época, e como consequência, Debreu posteriormente teve que encontrar seu próprio caminho na literatura de economia matemática; ele foi, por sua própria conta, particularmente impressionado pela Teoria dos Jogos e Comportamento Econômico (1944) de John von Neumann e Oskar Morgenstern neste período. De 1946 a 1948, ele ocupou o cargo de pesquisador associado no Centre National de la Recherche Scientifique; e ao receber uma bolsa de viagem da Rockefeller, ele visitou Harvard, Berkeley, Chicago, Uppsala e Oslo: a visita mais fatídica foi a estadia na Comissão Cowles em Chicago.

Sua aparição na Cowles em 1949 foi fortuita. A Comissão Cowles até aquele ponto era conhecida principalmente como um centro para o desenvolvimento de econometria - a aplicação de estatísticas matemáticas a questões econômicas empíricas - mas várias crises relacionadas a decepções em seu programa de estimação estrutural e disputas territoriais com o departamento de economia em Chicago estavam fazendo a unidade contemplar uma mudança na direção da pesquisa (Epstein 1987, 110; Mirowski 1993). O diretor de pesquisa na época era Tjalling Koopmans, um refugiado holandês da física quântica cujo trabalho anterior envolvia principalmente estimação estatística. A reorientação da pesquisa, longe do trabalho empírico e em direção à teoria matemática, já havia começado sob Koopmans em 1949, mas claramente faltava direção. Debreu se sentiu em casa entre os defensores sofisticados matematicamente da economia neoclássica, muitos deles também expatriados europeus com diplomas em ciências naturais. No entanto, havia outro lado serendipitoso para Chicago. Desprezado pelos economistas, Koopmans começou a fazer aberturas para o departamento de matemática para estabelecer uma unidade de estatística matemática. O presidente do departamento era Marshall Stone, um dos principais impulsionadores do Bourbakismo no contexto americano. Stone vinha reformulando o departamento em uma direção Bourbakista desde 1947, atraindo Andre Weil e construindo uma faculdade de pesquisa matemática de primeira classe (Stone 1989, Browder 1989). Koopmans manteve contato próximo com Stone através do Comitê de Estatísticas.

Os vetores exatos de influência não estão claros, mas depois que Debreu se juntou permanentemente à Comissão Cowles em junho de 1950, o Bourbakismo rapidamente se tornou a doutrina da casa da Comissão Cowles. Identificaríamos os textos filosóficos primários que afirmam este ponto de virada como Três Ensaios sobre o Estado da Ciência Econômica (1957) de Koopmans e a Teoria do Valor (1959) de Debreu. O primeiro era o livro didático da nova abordagem, com discussões metodológicas explícitas sobre a natureza do rigor matemático e a relação da economia com as práticas em física; enquanto a Teoria mais austera pretendia mostrar como a pesquisa de ponta seria feita no futuro. Debreu sinalizou explicitamente Três Ensaios como facilitando a compreensão de seu próprio trabalho (Debreu 1959, x). Enquanto Koopmans e Debreu eram os principais defensores dessa nova abordagem, ambos ganharam posteriormente prêmios Nobel por seu trabalho datado desta era, também se pode observar a nova orientação no trabalho de outros associados à Cowles neste período: John Chipman, Murray Gerstenhaber, I. N. Herstein, Leonid Hurwicz, Edmond Malinvaud, Roy Radner e Daniel Waterman. Quando a Comissão Cowles se mudou para Yale em 1955, as atitudes Bourbakistas em relação à teoria matemática começaram a se espalhar por toda a educação de pós-graduação em teoria econômica americana, à medida que os homens de Cowles se espalhavam pelos principais departamentos de economia.

Não podemos nem começar a abordar adequadamente exatamente por que a orientação Bourbakista em relação à economia matemática se espalhou tão rapidamente no contexto americano, uma vez que foi cristalizada dentro da Comissão Cowles; isso exigiria uma história muito mais ambiciosa da economia matemática na América. Mas algumas generalizações podem ser sugeridas, antes de voltarmos a focar diretamente em Debreu. Primeiro, como Cowles havia se desiludido com seus compromissos empiricistas anteriores, o programa Bourbakista de isolamento da teoria de sua inspiração empírica provou ser conveniente e oportuno. O ceticismo sobre a qualidade da empiricidade econômica se tornou uma característica daqueles que adotaram o programa de formalização matemática. Segundo, às vezes se esquece que a década de 1940 foi um período de muita rivalidade e dissensão entre diversas escolas de pensamento econômico, e que Cowles muitas vezes se encontrava no meio da controvérsia. Por exemplo, a própria Cowles estava lutando contra os institucionalistas de Wesley Clair Mitchell no NBER por financiamento e legitimidade no debate “Medição sem Teoria”; Keynesianos como Lawrence Klein em Cowles estavam em desacordo com Milton Friedman e outros no departamento de economia de Chicago. Bourbakismo prometia elevar-se acima de tudo, oferecendo um ponto de vista a partir do qual se poderia ficar alheio à Babel que ameaçava afogar o discurso racional. Quase se tornou uma medalha de honra sugerir que não se estava familiarizado com as tradições anteriores em economia: por exemplo, Koopmans escreveu em uma proposta à Fundação Ford, “Com uma possível exceção (Simon) o atual corpo docente da Comissão… pode reivindicar nenhuma competência especial nos tomos da literatura de ciências sociais, nem achamos que isso deveria ser um critério primário na seleção de pessoal adicional. Pretendemos, ao contratar pessoal, dar maior peso à combinação de imaginação criativa e tratamento rigoroso lógico e/ou matemático dos problemas” (“Aplicação à Fundação Ford”, enviada em 17 de setembro de 1951, p. 14; Arquivos da Fundação Cowles, Universidade de Yale). E terceiro, não se deve esquecer que o Bourbakismo estava varrendo a profissão matemática americana neste mesmo período. Muitas das ciências sociais fizeram esforços concertados para matematizar suas doutrinas no período imediatamente pós-guerra, mas foram apenas os economistas que pareciam estar fazendo matemática de um tipo que um matemático reconheceria. De fato, Debreu foi nomeado para um cargo no departamento de matemática em Berkeley em 1975, além da cátedra em economia que ocupava lá desde 1962.

O surgimento do formalismo matemático na economia não é um fenômeno simples do imperativo da matéria, como às vezes se afirma; ao contrário, é o produto de contingências da interseção de diversas disciplinas e, como Debreu é o primeiro a reconhecer, de numerosos acidentes pessoais e encontros fortuitos.

\subsubsection{\textbf{Ajustando as Estruturas Corretamente}}
Quando Debreu recebeu o Prêmio Nobel em 1983, muitos repórteres e comentaristas ficaram perplexos com seu encontro com este economista austero. Seu trabalho era abstruso e impenetrável, seu comportamento reservado, e sua resistência em usar o púlpito para comentar sobre eventos econômicos atuais era sem precedentes. Muitos dentro da profissão econômica também encontraram seu programa inescrutável, porque insistem em tentar enquadrá-lo em seus próprios termos locais. Gostaríamos de sugerir que um progresso interpretativo melhor poderia ser feito se as analogias com o programa Bourbakist em matemática fossem levadas muito mais a sério. De fato, muitos aspectos do programa Bourbakist cobertos na seção 2 deste artigo encontram correspondências diretas nos detalhes da versão de Debreu da economia matemática.

Parece claro que Debreu pretendia que sua Teoria do Valor servisse como o análogo direto da Teoria dos Conjuntos de Bourbaki, até mesmo no título. A monografia de Debreu deveria estabelecer a estrutura-mãe analítica definitiva a partir da qual todo o trabalho futuro em economia partiria, principalmente por “enfraquecer” suas suposições ou então por superpor novas “interpretações” ao formalismo existente. Mas isso exigia uma manobra muito crucial que não foi explicitamente declarada - ou seja, que o modelo de equilíbrio geral de Walras era a estrutura raiz a partir da qual todo o trabalho científico futuro em economia se originaria. Como observado de forma perspicaz por Bruna Ingrao e Giorgio Israel: “Na interpretação de Debreu, a teoria do equilíbrio geral perde assim seu status de ‘modelo’ para se tornar uma estrutura formal auto-suficiente” (Ingrao e Israel 1990, 286). O objetivo não era mais representar a economia, seja lá o que isso signifique, mas sim codificar a própria essência dessa entidade elusiva, o sistema Walrasiano. Esta mudança fundamental no objetivo explica muitos aspectos, de outra forma intrigantes, da carreira de Debreu, como a mudança progressiva de sua dependência inicial dos conceitos de teoria dos jogos, seu desprezo pelas tentativas (como a de Kenneth Arrow e Frank Hahn) de forjar links explícitos entre o modelo Walrasiano e as preocupações teóricas contemporâneas em macroeconomia ou teoria do bem-estar, e sua auto-negação ao lidar com questões de estabilidade e dinâmica.

Assim como com Bourbaki, o problema era justificar a identificação inicial das estruturas. No caso de Debreu, deve-se insistir que isso não era uma conclusão precipitada: a teoria Walrasiana não era amplamente respeitada na França ou na América; havia versões alternativas do programa neoclássico, como o aparato Marshalliano de demanda e oferta, com mais adeptos substanciais na América naquela época; existiam alguns rivais da ortodoxia neoclássica, como o marxismo e o institucionalismo; e foi apenas com a História da Análise Econômica (1954) de Joseph Schumpeter que Walras foi identificado como “o maior economista de todos os tempos”. Acreditar que a estrutura de toda a economia analítica estava meio obscurecida na variante Walrasiana/Paretiana relativamente adormecida em 1950 foi um salto de fé ousado. Uma consideração que pode ter tornado o salto menos improvável foi o fato de que Walras apresentou seu próprio trabalho como os Elementos da Economia Pura. Aqui encontramos uma noção bem articulada da separação da teoria pura de seu aspecto aplicado; isso certamente ressoaria com a inclinação de Debreu em aceitar uma separação da matemática pura da aplicada. Mas outro fator, operante para Allais e muitos dos membros da Comissão Cowles, foi a semelhança da matemática Walrasiana com estruturas usadas na física (Mirowski 1989)

A importância da analogia entre extremos de teorias de campo na física e otimização restrita de utilidade na economia neoclássica foi reconhecida em várias ocasiões por Koopmans (Mirowski 1991) e, como pode ser observado a partir de um dos epígrafes que preparam este artigo, por Debreu. Como muitos dos expatriados tinham pouco conhecimento em economia, as semelhanças na matemática inicialmente serviram para acelerar suas migrações para o campo. No entanto, a analogia poderia cortar de duas maneiras, pois, ao contrário dos casos de indivíduos como Edgeworth e Jevons, ninguém no século XX queria manter que a utilidade e a energia eram ontologicamente idênticas. Isso deixou o programa Walrasiano desprovido de uma explicação das semelhanças com a física. Cowles desenvolveu uma resposta interessante para este enigma - ou seja, que as novas técnicas matemáticas importadas por Koopmans, Debreu e outros libertaram a economia de sua dependência do cálculo clássico e das analogias físicas. Debreu, como observado, levou esta posição ainda mais longe, alegando que seu programa Bourbakista marcou a ruptura definitiva com as metáforas físicas, já que a física dependia de seu sucesso em conjecturas ousadas e refutações experimentais, mas a economia não tinha nada em que se apoiar além do rigor matemático. Isso é totalmente consistente com o credo Bourbakista, que reconhece que a inspiração matemática pode se originar nas ciências especiais, mas que uma vez que a estrutura analítica é extraída, as condições de sua gênese são irrelevantes.

Em suma, o formato de cada livro espelha o do outro, com a Teoria do Valor exemplificando o ideal de rigor intransigente, desprovido de todas as concessões heurísticas ou didáticas ao leitor. Assim como Bourbaki estava interessado e considerava seu projeto como fornecendo um manual para o matemático trabalhador, Debreu é melhor lido como fornecendo um manual para o teórico econômico trabalhador dos componentes neoclássicos da teoria econômica. Em retrospecto, é difícil ler a Teoria do Valor como qualquer outra coisa, uma vez que também não fornece "novos" teoremas ou resultados; é o "cemitério muito bem organizado com uma bela variedade de lápides" de Chevalley (Guedj 1985, 20). Assim, o evidente entusiasmo de Debreu no Capítulo 7 sobre sua capacidade de incorporar "incerteza" no modelo axiomatizado mantendo os formalismos matemáticos idênticos, mas redefinindo a "interpretação" da mercadoria, não deve ser considerado como uma nova contribuição para a teoria econômica do risco ou ignorância; em vez disso, nesta leitura, Debreu desenvolveu isso como ratificação do caráter estrutural de seus axiomas. No entanto, de uma maneira indubitavelmente não pretendida por Debreu, a monografia também compartilha muitos dos mesmos problemas de estruturas e "estruturas" experimentados por Bourbaki.

O impedimento, se podemos colocá-lo assim, é multicamadas, mas essencialmente semelhante em cada nível. Bourbaki afirmou que as estruturas que consideravam fundamentais compartilhavam algumas características analíticas unificadoras; mas sua afirmação acabou não se sustentando. Debreu pareceu argumentar que a teoria do equilíbrio geral de Walras deveria ser tratada como possuindo o mesmo status estrutural privilegiado na economia que os grupos têm em "estruturas algébricas" e como a relação de ordem tem em "estruturas topológicas"; mas essa afirmação foi finalmente problematizada pela geração de economistas matemáticos treinados nos padrões de rigor de Debreu - nos referimos aqui ao que é frequentemente citado como os resultados "Sonnenschein/Mantel/Debreu", a importância dos quais foi tornada moeda corrente nos anos 1980. Mas, é claro, em ambos os casos, o conjunto de práticas já havia reunido seu próprio ímpeto naquela data tardia, na medida em que tanto o formalismo Bourbakista quanto o Debreuvian se tornaram representativos de um certo gosto refinado e estilo na expressão matemática, de tal forma que os aspectos "estruturais" de sua inovação ainda exemplificavam a atividade de melhor prática, muito depois de terem abandonado o papel de fornecer fundamentação filosófica para o programa. E então havia a simples questão do atraso de fase entre as disciplinas de matemática e economia: a desilusão com Bourbaki era evidente na matemática dos anos 1970; um questionamento de alma semelhante está chegando apenas agora à economia. Quando Debreu leu o Fascículo pela primeira vez nos anos 1940, ele não tinha como saber como o programa estrutural Bourbakista se sairia nos anos 1960. Isso talvez ajude a explicar o tom bastante reservado das últimas declarações de Debreu sobre o lugar da matemática na economia (veja Debreu 1991).

Debreu, como observado acima, nunca pareceu muito interessado em descrever a dinâmica de convergência de uma economia para o equilíbrio de Walras. No entanto, a questão do movimento não poderia ser evitada para sempre, e houve um longo intervalo no período pós-guerra em que "dinâmicas" foram redefinidas para significar "estabilidade" dentro da comunidade de economia matemática (Weintraub 1991). Nesse contexto, a questão foi levantada por Hugo Sonnenschein se a "estrutura" básica dos modelos de equilíbrio geral de Walras colocava quaisquer restrições substanciais sobre a unicidade e estabilidade dos equilíbrios resultantes, e ele propôs a resposta surpreendente: não, fora de algumas restrições globais triviais e inúteis. O efeito devastador que isso teve no antigo programa "estrutural" de Debreu foi bem capturado por seu principal protegido, Werner Hildenbrand:

Quando li nos anos setenta as publicações de Sonnenschein, Mantel e Debreu sobre a estrutura da função de demanda excessiva de uma economia de troca, fiquei profundamente consternado. Até aquele momento, eu tinha a ilusão ingênua de que a fundamentação microeconômica do modelo de equilíbrio geral, que eu tanto admirava, não só nos permite provar que o modelo e o conceito de equilíbrio são logicamente consistentes, mas também nos permite mostrar que o equilíbrio está bem determinado. Essa ilusão, ou devo dizer antes essa esperança, foi destruída, de uma vez por todas, pelo menos para o modelo tradicional de economias de troca. Fui tentado a reprimir essa percepção e continuar a encontrar satisfação em provar a existência de equilíbrio para modelos mais gerais sob suposições ainda mais fracas. No entanto, não consegui reprimir a percepção recém-adquirida porque acredito que uma teoria do equilíbrio econômico é incompleta se o equilíbrio não estiver bem determinado. (Hildenbrand 1994, ix)

Este impasse é algo mais substancial do que os tipos de obstáculos que são periodicamente encontrados no curso de qualquer ciência vibrante; esses resultados foram vistos como prejudiciais precisamente porque questionam toda a estrutura Walrasiana como a estrutura apropriada para a elaboração da teoria econômica matemática. O Bourbakismo propagado pela Cowles tornou essa identificação algo muito além da adoção usual de alguns modelos característicos por uma escola acadêmica; tornou-se confundido com o próprio padrão de rigor matemático no pensamento econômico. De fato, isso definiu o programa Cowles desde o seu início: Por que precisamente a estrutura Walrasiana deveria ser tomada como a única "estrutura" da qual todo o trabalho matemático deveria partir? E qual era o modelo Walrasiano "correto"? Era o que realmente se encontrava nos textos de Walras ou Pareto ou Edgeworth ou Hicks ou Allais? Ou, para colocá-lo nos termos de Saunders Mac Lane: Não seria melhor fazer um caso para o "nível certo" de generalidade do que reivindicar que se atingiu o nível máximo?

A resposta para Debreu, como no caso de Bourbaki, era que o rigor era mais uma questão de "estrutura", de estilo (e política, como Mandelbrot corretamente insistiu), e de gosto; mas, em última análise, estilos e gostos mudam, por razões que só podem ser parcialmente contabilizadas pelas críticas internas geradas pelas atividades da comunidade fechada de matemáticos. Embora Debreu esperasse que os padrões elevados de economia matemática colocassem o discurso econômico em uma base mais estável, nunca houve uma razão formal para acreditar que seria assim.

\end{document}