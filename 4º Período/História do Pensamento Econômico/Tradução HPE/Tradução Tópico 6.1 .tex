\documentclass[12pt]{article}
\usepackage{siunitx} % Fornece suporte para a tipografia de unidades do Sistema Internacional e formatação de números
\usepackage{booktabs} % Melhora a qualidade das tabelas
\usepackage{tabularx} % Permite tabelas com larguras de colunas ajustáveis
\usepackage{graphicx} % Suporte para inclusão de imagens
\usepackage{newtxtext} % Substitui a fonte padrão pela Times Roman
\usepackage{ragged2e} % Justificação de texto melhorada
\usepackage{setspace} % Controle do espaçamento entre linhas
\usepackage[a4paper, left=3.0cm, top=3.0cm, bottom=2.0cm, rigH=2.0cm]{geometry} % Personalização das margens do documento
\usepackage{lipsum} % Geração de texto dummy 'Lorem Ipsum'
\usepackage{fancyhdr} % Customização de cabeçalhos e rodapés
\usepackage{titlesec} % Personalização dos títulos de seções
\usepackage[portuguese]{babel} % Adaptação para o português (nomes e hifenização)
\usepackage{hyperref} % Suporte a hiperlinks
\usepackage{indentfirst} % Indentação do primeiro parágrafo das seções
\usepackage{siunitx} % (Este pacote está duplicado, você pode querer removê-lo)
\sisetup{
  output-decimal-marker = {,},
  inter-unit-product = \ensuremath{{}\cdot{}},
  per-mode = symbol
}
\DeclareSIUnit{\real}{R\$}
\newcommand{\real}[1]{R\$#1}
\usepackage{float} % Melhor controle sobre o posicionamento de figuras e tabelas
\usepackage{footnotehyper} % Notas de rodapé clicáveis em combinação com hyperref
\usepackage{hyperref} % (Este pacote está duplicado, você pode querer ajustar isso)
\hypersetup{
    colorlinks=true,
    linkcolor=black,
    filecolor=magenta,      
    urlcolor=cyan,
    pdfborder={0 0 0},
}
\usepackage[normalem]{ulem} % Permite o uso de diferentes tipos de sublinhados sem alterar o \emph{}
\makeatletter
\def\@pdfborder{0 0 0} % Remove a borda dos links
\def\@pdfborderstyle{/S/U/W 1} % Estilo da borda dos links
\makeatother
\onehalfspacing

\begin{document}

\begin{titlepage}
    \centering
    \vspace*{1cm}
    \Large\textbf{INSPER – INSTITUTO DE ENSINO E PESQUISA}\\
    \Large ECONOMIA\\
    \vspace{1.5cm}
    \Large\textbf{Tradução Tópico 6.1 - HPE}\\
    \vspace{1.5cm}
    Prof. Pedro Duarte\\
    Prof. Auxiliar Guilherme Mazer\\
    \vfill
    \normalsize
    Hicham Munir Tayfour, \href{mailto:hichamt@al.insper.edu.br}{hichamt@al.insper.edu.br}\\
    4º Período - Economia B\\
    \vfill
    São Paulo\\
    Março/2024
\end{titlepage}

\newpage
\tableofcontents
\thispagestyle{empty} % This command removes the page number from the table of contents page
\newpage
\setcounter{page}{1} % This command sets the page number to start from this page
\justify
\onehalfspacing

\pagestyle{fancy}
\fancyhf{}
\rhead{\thepage}

\section{\textbf{BackHouse (2002)}}
\subsection{\textbf{Econometria e Economia Matemática, 1930 até o Presente}}
\subsubsection{\textbf{A Matematização da Economia}}

Entre as décadas de 1930 e 1970, a economia se matematizou no sentido de que se tornou prática normal para os economistas desenvolverem seus argumentos e apresentarem seus resultados, pelo menos uns aos outros, usando matemática. Isso geralmente envolvia geometria (particularmente importante no ensino) e álgebra (particularmente cálculo diferencial e álgebra matricial). Na década de 1930, apenas uma pequena minoria de artigos publicados nas principais revistas acadêmicas usava matemática, enquanto na década de 1970 era incomum encontrar artigos influentes que não o fizessem. Embora a velocidade da mudança tenha variado de um campo para outro, ela afetou toda a disciplina - trabalho teórico e aplicado.

A matemática é usada de duas maneiras na economia. Uma é como uma ferramenta de pesquisa teórica. Álgebra, geometria e até exemplos numéricos permitem que os economistas deduzam conclusões que de outra forma poderiam não ver, e o fazem com maior rigor do que se tivessem usado apenas raciocínio verbal. Este uso da matemática tem uma longa história. Quesnay e Ricardo fizeram uso tão extenso de exemplos numéricos no desenvolvimento de suas teorias que foram criticados da mesma forma que o uso da matemática na economia atual é criticado - os críticos argumentaram que a matemática tornava seus argumentos incompreensíveis para os de fora. Marx também fez uso extensivo de exemplos numéricos. O uso da álgebra remonta pelo menos ao início do século XIX, embora, em retrospecto, o desenvolvimento mais significativo tenha sido o uso do cálculo diferencial por Thünen (1826) e Cournot (1838). Com o trabalho de Jevons, Walras e seus seguidores de virada de século - notavelmente Fisher - o uso da matemática, em particular cálculo e equações simultâneas, foi claramente estabelecido como um importante método de investigação teórica.

O segundo uso da matemática - com o qual este capítulo está preocupado - é como uma ferramenta na pesquisa empírica. Isso envolve generalizar a partir de observações (indução) e testar teorias econômicas usando dados estatísticos sobre o mundo real. Dado que calcular médias ou proporções é uma técnica matemática, isso tem uma história muito longa. Uma pré-condição para o uso de tais métodos é a disponibilidade de estatísticas. Isso significou que o escopo para tal trabalho aumentou dramaticamente com a extensa coleta de tais dados no início do século XIX por economistas e estatísticos como McCulloch, Tooke e William Newmarch (1820-82), colaborador de Tooke em sua História dos Preços (1838-57). Técnicas estatísticas mais formais, incluindo análise de correlação e regressão, foram desenvolvidas no final do século XIX por Francis Galton (1822-1911), Karl Pearson (1857-1936) e Edgeworth. Jevons especulou que um dia poderia ser possível calcular curvas de demanda usando dados estatísticos, e no início do século XX vários economistas tentaram fazer isso, tanto na Europa quanto nos Estados Unidos. No período antes da Primeira Guerra Mundial, os economistas começaram a abordar o problema de como escolher entre as diferentes curvas que poderiam ser ajustadas aos dados.

Apesar dessas longas histórias do uso da matemática em argumentos dedutivos e indutivos, a matematização da economia desde a década de 1930 representa uma grande mudança no assunto. A razão é que isso levou a uma mudança profunda na maneira como o assunto foi concebido. A economia passou a ser estruturada não em torno de um conjunto de problemas do mundo real, mas em torno de um conjunto de técnicas teóricas e técnicas empíricas. As técnicas teóricas envolvem não apenas ferramentas matemáticas como otimização restrita ou álgebra matricial, mas também suposições recebidas sobre como se representa o comportamento de indivíduos ou organizações para que possa ser analisado usando métodos padrão. Da mesma forma, as técnicas empíricas envolvem suposições sobre como se relaciona conceitos teóricos a dados empíricos, bem como métodos para gerar e analisar dados estatísticos.

Essa nova abordagem pode ser resumida dizendo que os economistas agora falam sobre 'modelos'. Estes podem ser modelos teóricos, descrevendo mundos abstratos construídos para focar em problemas ou mecanismos específicos, ou podem ser modelos empíricos, fornecendo uma representação numérica de uma situação específica do mundo real. Por exemplo, um modelo do mercado de sorvetes pode permitir prever quanto mais altas serão as vendas em dias quentes e ensolarados do que em dias frios e chuvosos, ou qual será o efeito da imposição de um imposto sobre vendas. Um modelo da economia dos EUA pode prever o que aconteceria com o desemprego e a inflação se o Federal Reserve aumentasse sua taxa de juros em um ponto percentual. Onde um modelo teórico gerará apenas previsões qualitativas - que as variáveis irão subir ou descer, ou mudar em relação a outras variáveis - os modelos empíricos gerarão previsões que são numéricas: que, por exemplo, um aumento de um ponto percentual nas taxas de juros levará a um aumento de três pontos percentuais na taxa de desemprego.

Essa mudança em direção à modelagem teve efeitos profundos na estrutura da disciplina. O assunto passou a ser considerado como compreendendo um 'núcleo' de teoria (tanto teoria econômica quanto técnicas econométricas) cercado por campos nos quais essa teoria é aplicada. A teoria foi separada das aplicações e, ao mesmo tempo, a pesquisa teórica e empírica se tornaram separadas. Os mesmos indivíduos frequentemente se envolvem em ambos (as habilidades matemáticas são altamente transferíveis), mas essas são, no entanto, empresas separadas. Como essa mudança em direção à modelagem gera problemas técnicos que precisam ser resolvidos, algumas pesquisas foram impulsionadas por uma agenda interna à disciplina, mesmo quando isso não ajudou a resolver nenhum problema do mundo real. Isso levou alguns críticos a argumentar que a economia se divorciou da realidade, uma acusação veementemente negada por outros.

\subsubsection{\textbf{A Revolução na Contabilidade da Renda Nacional}}

Um dos desenvolvimentos mais importantes nas décadas de 1930 e 1940 foi a coleta sistemática em larga escala de estatísticas econômicas e a criação de contas nacionais. Sem contas nacionais precisas, não sabemos quanto de produção está sendo gerada ou como essa produção está sendo usada. Na década de 1920, contas abrangentes de renda nacional não existiam para nenhum país. As tentativas pioneiras de pessoas como Petty e King envolveram suposições inspiradas tanto quanto evidências detalhadas, e não se baseavam em nenhum arcabouço conceitual sistemático. Mesmo no século XIX e início do século XX, quando estimativas de renda nacional foram feitas em vários países, incluindo os Estados Unidos e a Grã-Bretanha, as lacunas nos dados eram tão amplas que contas detalhadas eram impossíveis. Nos Estados Unidos, a tentativa mais abrangente foi A Riqueza e Renda do Povo dos Estados Unidos (1915) por Willford I. King (1880-1962), um aluno de Irving Fisher. King mostrou que a renda nacional triplicou em sessenta anos, e que a parcela de salários e salários na renda total aumentou de 36 para 47 por cento. Ele concluiu que, ao contrário do que os socialistas estavam alegando, o sistema econômico existente estava funcionando bem. Na Grã-Bretanha, A. L. Bowley (1869-1957) estava produzindo estimativas baseadas em dados fiscais, censos populacionais, o censo de produção de 1907 e informações sobre salários e emprego. No entanto, este trabalho, como o que estava sendo realizado em outros lugares, permaneceu muito limitado em seu escopo.

No período entre guerras, as estatísticas de renda nacional foram construídas em toda a Europa. O interesse nelas foi estimulado pelos imensos problemas da reconstrução pós-guerra, as enormes mudanças no poder econômico relativo de diferentes nações, a Depressão da década de 1930 e a necessidade de mobilizar recursos em antecipação a outra guerra. Durante a década de 1930, a Alemanha estava produzindo estimativas anuais de renda nacional com um atraso de apenas um ano. A União Soviética construiu tabelas de entrada-saída (mostrando quanto cada setor da economia comprou de todos os outros setores) durante a maior parte da década de 1920 e início da década de 1930. Itália e Alemanha elaboraram uma base conceitual para a contabilidade nacional que era tão avançada quanto qualquer outra no mundo. Em 1939, dez países estavam produzindo estimativas oficiais de renda nacional. No entanto, por causa da guerra, os países que tiveram mais influência a longo prazo foram a Grã-Bretanha e os Estados Unidos. Ao contrário desses dois, a Alemanha nunca usou a renda nacional para o planejamento de guerra e parou de produzir estatísticas. Em completo contraste, na década de 1950, as estatísticas de renda nacional estavam sendo construídas por governos nacionais e coordenadas através das Nações Unidas e existiam estimativas para quase cem países.

Nos Estados Unidos, houve três vertentes para o trabalho inicial sobre a contabilidade da renda nacional. A primeira estava associada ao National Bureau of Economic Research (NBER), estabelecido por Mitchell em 1920. Seu primeiro projeto foi um estudo sobre as variações anuais na renda nacional e a distribuição de renda. Publicado em 1921, seu relatório forneceu estimativas anuais de renda nacional para o período 1909-19. Essas foram estendidas durante a década de 1920 e foram complementadas em 1926 por estimativas feitas pela Federal Trade Commission. No entanto, a FTC não conseguiu continuar este trabalho. Com o início da Depressão, o governo federal se envolveu. Em junho de 1932, uma resolução do Senado proposta por Robert La Follette, senador de Wisconsin, comprometeu o Bureau of Foreign and Domestic Commerce (BFDC) a preparar estimativas de renda nacional para 1929, 1930 e 1931.

Pouco foi alcançado nos primeiros seis meses após a resolução, e em janeiro de 1933 o trabalho do BFDC foi entregue a Simon Kuznets (1901-85), que vinha trabalhando na renda nacional no NBER desde 1929. No NBER, ele havia preparado planos para estimar a renda nacional, posteriormente resumidos em um artigo amplamente lido sobre o assunto na Enciclopédia das Ciências Sociais (1933). Em um ano, Kuznets e sua equipe produziram estimativas para 1929-32. (Reconhecendo a importância das estatísticas atualizadas, eles incluíram 1932, bem como os anos exigidos pela resolução de La Follette.) Kuznets voltou para o NBER, onde trabalhou em poupança e acumulação de capital, e posteriormente em problemas de crescimento de longo prazo. O estudo do BFDC sobre a renda nacional tornou-se permanente sob a direção de Robert Nathan (1908-2001). As estimativas originais foram revisadas e estendidas, e novas séries foram produzidas (por exemplo, números mensais foram produzidos em 1938).

Nesse momento, a própria definição de renda nacional era controversa. Kuznets e sua equipe publicaram duas estimativas: 'renda nacional produzida', que se referia ao produto líquido de toda a economia, e 'renda nacional recebida', que cobria os pagamentos feitos àqueles que produziram o produto líquido. Para basear as estimativas em dados confiáveis, eles excluíram muitos dos itens então controversos. Essas estimativas de renda nacional cobriam apenas a economia de mercado (bens que foram comprados e vendidos), e os bens eram avaliados a preços de mercado. A distinção básica subjacente ao arcabouço de Kuznets era entre despesa do consumidor e formação de capital.

Ao mesmo tempo, Clark Warburton (1896-1979), na Brookings Institution, produziu estimativas do produto nacional bruto (PNB). Isso foi definido como a soma dos produtos finais que emergem dos processos de produção e marketing e são repassados aos consumidores e empresas (ou seja, excluindo produtos que são remanufaturados para fazer outros produtos). Isso era muito maior do que a figura de Kuznets para a renda nacional, porque também incluía bens de capital comprados para substituir os que haviam se desgastado, serviços governamentais aos consumidores e compras governamentais de bens de capital. Warburton argumentou que o PNB menos a depreciação era a maneira correta de medir os recursos disponíveis para serem gastos. Ele produziu, pela primeira vez, evidências de que os gastos com bens de capital eram mais erráticos do que os gastos com bens de consumo. Os economistas há muito tempo estavam cientes disso, mas anteriormente tinham apenas evidências indiretas.

A terceira vertente no trabalho americano sobre a renda nacional foi o trabalho associado a Lauchlin Currie. Em 1934-5 ele começou a calcular o 'déficit de bombeamento'. Isso se baseava na ideia de que, para o setor privado gerar demanda suficiente por bens para curar o desemprego, o governo tinha que 'preparar a bomba' aumentando seus próprios gastos. Currie e seus colegas se concentraram na contribuição de cada setor para o poder de compra nacional - a diferença entre os gastos e a renda de cada setor. Uma contribuição positiva do governo (ou seja, um déficit) era necessária para compensar a poupança líquida de outros setores.

Na Grã-Bretanha, o cálculo das estatísticas de renda nacional foi o trabalho de um pequeno número de estudiosos sem assistência governamental durante todo o período entre guerras. De particular importância foi Colin Clark (1905-89). Em 1932, Clark usou o conceito de produto nacional bruto e estimou os principais componentes da demanda agregada (consumo, investimento e gastos governamentais). Este trabalho aumentou em importância após a publicação da Teoria Geral de Keynes (1936), e logo após sua publicação Clark estimou o valor do multiplicador (ele pensou que era cerca de dois, uma figura que Keynes considerou plausível). Seu principal trabalho foi Renda Nacional e Despesa (1937). Um de seus seguidores escreveu sobre este livro que ele 'restaurou a visão dos aritméticos políticos [Petty e Davenant]... [Ele] reuniu estimativas de renda, produção, despesa do consumidor, receita e despesa governamental, formação de capital, poupança, comércio exterior e balança de pagamentos. Embora ele não tenha definido suas figuras em um arcabouço contábil, está claro que elas chegaram bastante perto da consistência.'

O trabalho de Clark não foi apoiado pelo governo. (Quando ele foi nomeado para o secretariado do Conselho Consultivo Econômico em 1930, o Tesouro até se recusou a comprar uma máquina de somar para ele.) Questões de distribuição de renda eram muito sensíveis para o governo querer publicar números. Os industriais não queriam que os lucros fossem revelados. O governo calculou as cifras de renda nacional para 1929, mas negou sua existência porque as estimativas de salários eram inferiores às já disponíveis. O envolvimento oficial na contabilidade da renda nacional não começou até a Segunda Guerra Mundial. Keynes usou os números de Clark em Como Pagar pela Guerra (1940).

No verão de 1940, Richard Stone (1913-91) juntou-se a James Meade (1907-94) no Serviço Central de Informações Econômicas do Gabinete de Guerra. Durante o resto do ano, encorajados e apoiados por Keynes, eles construíram um conjunto de contas nacionais para 1938 e 1940 que foi publicado em um Livro Branco acompanhando o Orçamento de 1941. A falta de recursos disponíveis para eles é ilustrada por uma história sobre sua cooperação. Eles começaram com Meade (o parceiro sênior) lendo números que Stone digitava em sua calculadora mecânica, mas logo descobriram que era mais eficiente que seus papéis fossem invertidos. Embora o Chanceler do Tesouro tenha dito que a publicação de suas cifras não estabeleceria um precedente, as estimativas foram publicadas anualmente a partir de então.

Durante a Segunda Guerra Mundial, as estimativas de renda nacional foram transformadas em sistemas de contas nacionais nos quais diferentes contas estavam conectadas. Sua posição no esforço de guerra, juntamente com o trabalho de Kuznets e Nathan no Conselho de Produção de Guerra, garantiu que os Estados Unidos fossem o país dominante nesse processo. No entanto, o sistema que foi eventualmente adotado deveu muito ao trabalho britânico. Em 1940, Hicks introduziu a equação que se tornou básica para a contabilidade da renda nacional: PNB = C + I + G (renda igual a consumo mais investimento mais gastos governamentais em bens e serviços). Ele também foi responsável pela distinção entre preços de mercado e custo de fator (preços de mercado menos impostos indiretos). Talvez mais importante, Meade e Stone forneceram uma base conceitual mais sólida para as contas nacionais, apresentando-as como uma conta de produção de dupla entrada para toda a economia. Em uma coluna estavam os pagamentos de fatores (renda nacional), e na outra coluna as despesas (despesa nacional). Como em todas as contas de dupla entrada, quando calculadas corretamente, as duas colunas se equilibram.

A partir de 1941, os Estados Unidos se afastaram das contas nacionais conforme construídas por Kuznets e Nathan para aquelas construídas nas linhas Keynesianas, usando o arcabouço Meade-Stone. Este foi o trabalho de Martin Gilbert (1909-79), um ex-aluno de Kuznets, que foi chefe da Divisão de Renda Nacional do Departamento de Comércio dos EUA de 1941 a 1951. Uma razão para essa mudança foi a rápida disseminação da economia keynesiana, que forneceu uma base teórica para o novo sistema de contas. Não havia uma teoria econômica formal subjacente às categorias de Kuznets, que derivavam de considerações puramente empíricas. A mudança também parecia desejável por outros motivos. Em tempo de guerra, quando a preocupação era com a disponibilidade de recursos a curto prazo, não era necessário manter o capital, o que significava que o PNB era a medida relevante de produção. Além disso, devido à escala dos gastos governamentais durante a guerra, era importante ter uma medida de renda que incluísse o gasto governamental. Finalmente, o sistema Meade-Stone forneceu um arcabouço dentro do qual uma gama mais ampla de contas poderia ser desenvolvida. Após a guerra, em 1947, um relatório da Liga das Nações, no qual Stone desempenhou um papel importante, forneceu o arcabouço dentro do qual vários governos começaram a compilar suas contas para que fosse possível fazer comparações entre países. Posteriormente, Stone também se envolveu no trabalho da Organização para a Cooperação Econômica Europeia e das Nações Unidas, que em 1953 produziu um sistema padrão de contas nacionais.

\subsubsection{\textbf{A Sociedade Econométrica e as Origens da Econometria Moderna}}

Para que a economia se tornasse uma disciplina quantitativa, não era suficiente que os dados estivessem disponíveis; era necessário que os economistas desenvolvessem técnicas para analisar os dados. Aqui, um papel central foi desempenhado pela Sociedade Econométrica, formada em 1930, em Chicago, por iniciativa de Charles Roos (1901-58), Irving Fisher e Ragnar Frisch (1895-1973). Sua constituição descreveu seus objetivos nos seguintes termos:

A Sociedade Econométrica é uma sociedade internacional para o avanço da teoria econômica em relação à estatística e à matemática... Seu principal objetivo será promover estudos que visem a unificação da abordagem teórico-quantitativa e empírico-quantitativa para problemas econômicos e que sejam permeados por um pensamento construtivo e rigoroso semelhante ao que passou a dominar nas ciências naturais.

Ao comentar esta declaração, Frisch enfatizou que o aspecto importante da econometria, como o termo era usado na sociedade, era a unificação da teoria econômica, estatística e matemática. A matemática, por si só, não era suficiente.

Em seus primeiros anos, a Sociedade Econométrica era muito pequena. Vinte anos antes, Fisher havia tentado gerar interesse na criação de uma sociedade como essa, mas falhou. Então, quando Roos e Frisch o abordaram sobre a possibilidade de formar uma sociedade, ele estava cético sobre se havia interesse suficiente no assunto. No entanto, ele disse a eles que apoiaria a ideia se eles pudessem produzir uma lista de cem membros potenciais. Para surpresa de Fisher, eles encontraram setenta nomes. Com alguns outros adicionados por Fisher, isso forneceu a base para a sociedade.

Pouco depois da formação da sociedade, ela entrou em contato com Alfred Cowles (1891-1984). Cowles era um empresário que havia criado uma agência de previsão, mas que se tornou cético sobre se os previsores estavam fazendo algo mais do que adivinhar o que poderia acontecer. Ele, portanto, desenvolveu um interesse em pesquisa quantitativa. Quando ele escreveu um artigo sob o título 'Os previsores do mercado de ações podem prever?' (1933), ele deu a ele o resumo de três palavras 'É duvidoso'. Sua evidência veio de uma comparação dos retornos obtidos ao seguir o conselho oferecido por dezesseis provedores de serviços financeiros e o desempenho de vinte companhias de seguros com os retornos que teriam sido obtidos seguindo previsões aleatórias. No período de 1928 a 1932, não havia evidências de que as previsões profissionais fossem melhores do que as aleatórias. Com o apoio de Cowles, a Sociedade Econométrica conseguiu estabelecer uma revista, Econometrica, em 1933. Além disso, Cowles apoiou a criação, em 1932, da Comissão Cowles, um centro de pesquisa matemática e estatística em economia. De 1939 a 1955, foi baseado na Universidade de Chicago, distinto do departamento de economia, após o qual se mudou para Yale. Este instituto provou ser importante no desenvolvimento da econometria.

A econometria cresceu a partir de duas tradições distintas - uma americana, representada na Sociedade Econométrica por Fisher e Roos, e a outra europeia, representada por Frisch (um norueguês). Uma vertente na tradição americana foi a análise estatística do dinheiro e do ciclo econômico. Fisher e outros buscaram testar a teoria quantitativa do dinheiro, buscando encontrar medidas independentes de todos os quatro termos na equação de troca (dinheiro, velocidade de circulação, transações e nível de preços). Mitchell, em vez de encontrar evidências para apoiar uma teoria particular do ciclo, redefiniu o problema como tentar descrever o que acontecia nos ciclos econômicos. Este programa inerentemente quantitativo, estabelecido em seus Ciclos Econômicos e suas Causas (1913), foi assumido pelo National Bureau of Economic Research, sob a direção de Mitchell. Isso resultou em um método de cálculo de 'ciclos de referência' com os quais as flutuações em qualquer série poderiam ser comparadas. Uma abordagem alternativa foi o 'barômetro de negócios' desenvolvido em Harvard por Warren Persons (1878-1937) como um método de previsão do ciclo. Havia também Henry Ludwell Moore (1869-1958) na Universidade de Columbia, que procurou, como Jevons, estabelecer uma ligação entre o ciclo econômico e o clima. Alguns anos depois, em 1923, ele mudou do clima para o movimento do planeta Vênus como sua explicação. O trabalho de Moore é notável pelo uso de uma gama mais ampla de técnicas estatísticas do que as empregadas por outros economistas naquela época.

A outra vertente na tradição americana era a análise da demanda. Moore e Henry Schultz (1893-1938) estimaram curvas de demanda para bens agrícolas e outros. Este trabalho não foi muito longe em unir a teoria econômica matemática com a análise estatística. A dissertação de Fisher envolveu uma análise matemática da teoria do consumidor e da demanda, mas isso permaneceu separado de seu trabalho estatístico, que era sobre taxas de juros e dinheiro. Mitchell era cético sobre o valor de perseguir teorias simplificadas do ciclo econômico que enfatizavam uma causa particular do ciclo. Para ele, o trabalho estatístico fornecia uma maneira de integrar diferentes teorias e sugerir novas linhas de investigação, o que significava que ele não conseguiu desenvolver uma ligação estreita entre teoria e dados. Mitchell também era cético sobre a teoria do consumidor padrão. Ele esperava que estudos empíricos do comportamento dos consumidores tornassem obsoletos os modelos teóricos, nos quais os consumidores eram tratados como chegando ao mercado com escalas prontas de preços de oferta e demanda. Em outras palavras, o trabalho estatístico substituiria a teoria abstrata em vez de complementá-la. Da mesma forma, Moore criticou as curvas de demanda padrão por serem estáticas e por suas suposições ceteris paribus (suposições sobre as variáveis, como gostos e rendas, que foram mantidas constantes). O resultado foi que, enquanto a atitude dos estatísticos era de ceticismo em relação à teoria matemática, essa teoria era improvável de ser integrada de perto com o trabalho estatístico. Essa improbabilidade foi reforçada pelo ceticismo expresso por muitos economistas (incluindo Keynes e Morgenstern) sobre a precisão e relevância de muitos dados estatísticos.

A tradição europeia, que se sobrepunha à americana em muitos pontos, incluindo pesquisa sobre ciclos econômicos e demanda, tinha ênfases diferentes. O trabalho de uma variedade de autores no final da década de 1920 levou a uma consciência de alguns dos problemas envolvidos na aplicação de técnicas estatísticas, como correlação, a dados de séries temporais. George Udny Yule (1871-1951), aluno de Karl Pearson, explorou o problema das 'correlações sem sentido' - relações aparentemente fortes entre séries temporais que não deveriam ter relação entre si, como a chuva na Índia e o comprimento das saias em Paris. Ele argumentou que tais correlações muitas vezes não refletiam uma causa comum a ambas as variáveis, mas eram puramente acidentais. Ele também usou métodos experimentais para explorar a relação entre choques aleatórios e flutuações periódicas em séries temporais. O russo Eugen Slutsky (1880-1948) foi ainda mais longe ao mostrar que a soma de números aleatórios (gerados pela loteria estadual) poderia produzir ciclos que se pareciam notavelmente com o ciclo econômico: parecia haver flutuações regulares e periódicas. Frisch também abordou o problema das séries temporais, de uma maneira mais próxima de Mitchell e Persons do que de Yule ou Slutsky, tentando decompor os ciclos em suas partes componentes.

\subsubsection{\textbf{Frisch, Tinbergen e a Comissão Cowles}}

Slutsky e Frisch foram pioneiros na modelagem estatística do ciclo econômico, mas o primeiro modelo estatístico de uma economia inteira foi construído pelo economista holandês Jan Tinbergen (1903-94), que chegou à economia depois de obter um doutorado em física e passou grande parte de sua carreira no Central Planning Bureau na Holanda. Para entender o que Tinbergen estava fazendo com este modelo, é útil considerar a teoria do ciclo econômico que Frisch publicou em 1933. Ele adotou a ideia (retirada de Wicksell) de que o problema do ciclo econômico tinha que ser dividido em duas partes - os problemas de 'impulso' e 'propagação'. O problema do impulso dizia respeito à fonte de choques para o sistema, que poderiam ser mudanças na tecnologia, guerras ou qualquer coisa fora do sistema. O problema de propagação dizia respeito ao mecanismo pelo qual os efeitos desses choques eram propagados pela economia. Frisch produziu um modelo que, se deixado por si só sem choques externos, produziria oscilações amortecidas - ciclos que se tornavam progressivamente menores, eventualmente desaparecendo - mas que produzia ciclos regulares porque estava sujeito a choques periódicos. Seguindo Wicksell, ele descreveu isso como um 'modelo de cavalinho de balanço'. Se deixado por si só, o movimento de um cavalinho de balanço gradualmente desaparecerá, mas se perturbado de vez em quando, o cavalo continuará a balançar. Tal modelo, argumentou Frisch, produziria os ciclos regularmente ocorrentes, mas irregulares, que caracterizam o ciclo econômico.

A distinção entre problemas de propagação e impulso se traduziu facilmente nas técnicas matemáticas que Frisch estava usando. O mecanismo de propagação dependia dos valores dos parâmetros nas equações e na estrutura da economia. Em 1933, Frisch simplesmente fez suposições plausíveis sobre o que poderiam ser esses, embora tenha expressado confiança de que em breve seria possível obter tais números usando técnicas estatísticas. Os choques foram representados pelas condições iniciais que tiveram que ser assumidas ao resolver o modelo. Usando seus coeficientes adivinhados e condições iniciais adequadas, Frisch empregou simulações para mostrar que seu modelo produzia ciclos que pareciam realistas.

Em 1936, Tinbergen produziu um modelo da economia holandesa. Isso foi significativamente além do modelo de Frisch em dois aspectos. A estrutura da economia holandesa foi descrita em dezesseis equações mais identidades contábeis suficientes para determinar todas as suas trinta e uma variáveis. As variáveis que explicou incluíam preços, quantidades físicas, rendas e níveis de gastos. Portanto, era muito mais detalhado do que o modelo de Frisch, que continha apenas três variáveis (produção de bens de consumo, novos bens de capital iniciados e produção de bens de capital transferidos de períodos anteriores). Mais importante, enquanto Frisch simplesmente fez suposições plausíveis sobre os números que aparecem em suas equações, Tinbergen estimou a maioria dos seus usando técnicas estatísticas. Ele conseguiu mostrar que, deixado por si só, seu modelo produzia oscilações amortecidas e que poderia explicar o ciclo.

Três anos depois, Tinbergen publicou dois volumes intitulados "Testes Estatísticos das Teorias do Ciclo Econômico", o segundo dos quais apresentou o primeiro modelo econométrico dos Estados Unidos (que continha três vezes mais equações do que seu modelo anterior dos Países Baixos). Este trabalho foi patrocinado pela Liga das Nações, que o havia contratado para testar as teorias do ciclo econômico pesquisadas no "Prosperidade e Depressão" de Haberler (1936). No entanto, embora Tinbergen tenha conseguido construir um modelo que pudesse ser usado para analisar o ciclo econômico nos Estados Unidos, a tarefa de fornecer um teste estatístico de teorias concorrentes do ciclo econômico provou ser muito ambiciosa. Os dados estatísticos disponíveis eram limitados. A maioria das teorias do ciclo eram expressas verbalmente e não eram completamente precisas. Mais importante, como Mitchell havia observado, a maioria das teorias discutia apenas um aspecto do problema, o que significava que elas tinham que ser combinadas para obter um modelo adequado. Era impossível testá-los individualmente. O que Tinbergen conseguiu fazer, no entanto, foi esclarecer os requisitos que deveriam ser atendidos se uma teoria fosse formar a base para um modelo econométrico. O modelo tinha que ser completo (contendo relações suficientes para explicar todas as variáveis), determinado (cada relação deve ser totalmente especificada) e dinâmico (com defasagens temporais totalmente especificadas).

Com o início da Segunda Guerra Mundial em 1939, o trabalho europeu sobre a modelagem econométrica do ciclo praticamente cessou e o principal trabalho em econometria foi realizado nos Estados Unidos por membros da Comissão Cowles. No entanto, muitos dos economistas que trabalhavam lá eram emigrantes europeus. Um período particularmente importante começou quando Jacob Marschak (1898-1977) se tornou o diretor de pesquisa da comissão em 1943. (Marschak ilustra a extensão em que as carreiras de muitos economistas foram mudadas pelos eventos mundiais. Um judeu ucraniano, nascido em Kiev, ele experimentou a agitação de 1917-18. Ele estudou economia na Alemanha e começou uma carreira acadêmica lá, mas em 1933 a perspectiva do domínio nazista o fez se mudar para Oxford. Em 1938, ele visitou os Estados Unidos por um ano, e quando a guerra estourou, ele ficou.)

Sob Marschak, a pesquisa na Comissão Cowles se afastou da busca por resultados concretos para o desenvolvimento de novos métodos que levavam em conta as principais características da teoria econômica e dos dados econômicos, dos quais havia quatro. (1) A teoria econômica é sobre sistemas de equações simultâneas. O preço de uma mercadoria, por exemplo, depende da oferta, demanda e do processo pelo qual o preço muda quando a oferta e a demanda são desiguais. (2) Muitas dessas equações incluem termos 'aleatórios', pois o comportamento é afetado por choques e por fatores que as teorias econômicas não podem lidar. (3) Muitos dados econômicos estão na forma de séries temporais, onde o valor de um período depende dos valores em períodos anteriores. (4) Muitos dados publicados referem-se a agregados, não a indivíduos únicos, os exemplos óbvios sendo a renda nacional (ou qualquer outro item nas contas nacionais) e o nível de emprego. Nenhuma dessas quatro características era nova - todas eram bem conhecidas. O que era novo era a maneira sistemática pela qual os economistas associados à Comissão Cowles buscavam desenvolver técnicas que levassem em conta todas as quatro.

Embora muitos membros e associados da Comissão Cowles estivessem envolvidos no desenvolvimento desses novos métodos, a contribuição chave foi a de Trygve Haavelmo (1911-99). Haavelmo argumentou que o uso de métodos estatísticos para analisar dados era sem sentido, a menos que fossem baseados em um modelo de probabilidade. Economistas anteriores haviam rejeitado modelos de probabilidade, porque acreditavam que eles eram relevantes apenas para situações como loterias (onde probabilidades precisas podem ser calculadas) ou para situações experimentais controladas (como a aplicação de fertilizantes em diferentes parcelas de terra). Haavelmo contestou isso, afirmando que 'nenhuma ferramenta desenvolvida na teoria da estatística tem qualquer significado - exceto, talvez, para fins descritivos - sem ser referida a algum esquema estocástico [algum modelo das probabilidades subjacentes]'. Igualmente significativo, ele argumentou que a incerteza entra nos modelos econômicos não apenas por causa do erro de medição, mas porque a incerteza é inerente à maioria das relações econômicas:

A necessidade de introduzir 'termos de erro' nas relações econômicas não é meramente um resultado de erros estatísticos de medição. É tanto um resultado da própria natureza do comportamento econômico, sua dependência de um número enorme de fatores, em comparação com aqueles que podemos contabilizar, explicitamente, em nossas teorias.

Durante a década de 1940, portanto, Haavelmo e outros desenvolveram métodos para atribuir números aos coeficientes em sistemas de equações simultâneas. A suposição de um modelo de probabilidade subjacente significava que eles poderiam avaliar esses métodos, perguntando, por exemplo, se as estimativas obtidas eram imparciais e consistentes.

No final da década de 1940, este programa começou a produzir resultados que eram potencialmente relevantes para os formuladores de políticas. A aplicação mais importante foi de Lawrence Klein (1920-2013), que usou modelos da economia dos EUA para prever a renda nacional. Os modelos de Klein eram representativos da abordagem estabelecida por Marschak em 1943. Eles eram sistemas de equações simultâneas, destinados a representar a estrutura da economia dos EUA, e foram concebidos usando as mais recentes técnicas estatísticas desenvolvidas pela Comissão Cowles. A abordagem de Klein levou aos modelos macroeconométricos em grande escala, muitas vezes compostos por centenas de equações, que foram amplamente utilizados para previsões nas décadas de 1960 e 1970.

Este programa de integração de matemática, economia e estatística promovido pelos fundadores da Sociedade Econométrica foi apenas parcialmente bem-sucedido. Matemática e estatística se tornaram uma parte integral da economia, mas a integração esperada da teoria econômica e do trabalho empírico nunca aconteceu. Permaneceram dúvidas sobre o valor de tentar modelar a estrutura de uma economia usando os métodos desenvolvidos em Cowles. Não estava claro se os modelos estruturais, com toda a sua sofisticação matemática, eram superiores aos mais simples baseados em métodos mais 'ingênuos'. O problema da agregação (como derivar o comportamento de um agregado, como a demanda de mercado por um produto, do comportamento dos indivíduos dos quais o agregado é composto) provou ser muito difícil. O resultado foi que, no final da década de 1940, a Comissão Cowles mudou para a pesquisa em teoria econômica. (A pesquisa em econometria continuou a todo vapor, principalmente fora de Cowles, mas sem o mesmo otimismo que caracterizava o trabalho anterior.) O lema da comissão, 'Ciência é medição' (adotado de Lord Kelvin), foi alterado em 1952 para 'Teoria e medição'. Como um historiador expressou, 'Até a década de 1950, o ideal fundador da econometria, a união da economia matemática e estatística em uma economia verdadeiramente sintética, havia desmoronado.

\section{\textbf{Frisch (1933)}}

\subsection{\textbf{Editorial}}
Dois anos já se passaram desde a fundação da Sociedade Econométrica. Embora a Sociedade tenha propositalmente dado pouca publicidade aos seus assuntos durante esses anos de organização, perguntas e sugestões vieram de muitos lugares manifestando uma prontidão para, e uma grande expectativa de, algo ao longo das linhas agora seguidas pela Sociedade Econométrica. Uma fonte de energia potencial, muito maior do que originalmente antecipada pelos fundadores da Sociedade, parece existir, apenas esperando para ser liberada e direcionada para o trabalho econométrico.

Esta é a razão pela qual a Sociedade decidiu estabelecer seu próprio periódico. ECONOMETRICA será o seu nome. Ele aparecerá trimestralmente; esta é a primeira edição.

Uma palavra de explicação sobre o termo econometria pode ser necessária. Sua definição está implícita na declaração do escopo da Sociedade, na Seção I da Constituição, que diz: "A Sociedade Econométrica é uma sociedade internacional para o avanço da teoria econômica em relação à estatística e matemática. A Sociedade operará como uma organização científica completamente desinteressada, sem viés político, social, financeiro ou nacionalista. Seu principal objetivo será promover estudos que visem a unificação da abordagem teórico-quantitativa e empírico-quantitativa para problemas econômicos e que sejam permeados por um pensamento construtivo e rigoroso semelhante ao que passou a dominar nas ciências naturais. Qualquer atividade que prometa, em última instância, promover tal unificação de estudos teóricos e factuais em economia estará dentro da esfera de interesse da Sociedade."

Esta ênfase no aspecto quantitativo dos problemas econômicos tem um significado profundo. A vida econômica é uma rede complexa de relações operando em todas as direções. Portanto, enquanto nos limitarmos a declarações em termos gerais sobre um fator econômico tendo um "efeito" sobre algum outro fator, quase qualquer tipo de relação pode ser selecionado, postulado como uma lei, e "explicado" por um argumento plausível.

Assim, existe um perigo real de avançar declarações e conclusões que - embora verdadeiras como tendências em um sentido muito restrito - são, no entanto, completamente inadequadas, ou mesmo enganosas se oferecidas como uma explicação da situação. Para usar uma ilustração extrema, eles podem ser tão enganosos quanto dizer que quando um homem tenta remar um barco para a frente, o barco será impulsionado para trás por causa da pressão exercida por seus pés. A situação do barco a remo não é, é claro, explicada ao descobrir que existe uma pressão em uma direção ou outra, mas apenas comparando as magnitudes relativas de uma série de pressões e contrapressões. É essa comparação de magnitudes que dá um significado real à análise. Muitas, senão a maioria, das situações que temos que enfrentar na economia são exatamente desse tipo. A plena utilidade de um grande e importante grupo de nossas análises econômicas virá, portanto, apenas à medida que conseguirmos formular as discussões em termos quantitativos.

Existem vários aspectos da abordagem quantitativa à economia, e nenhum desses aspectos, tomado por si só, deve ser confundido com a econometria. Assim, a econometria não é de forma alguma a mesma coisa que as estatísticas econômicas. Nem é idêntica ao que chamamos de teoria econômica geral, embora uma parte considerável dessa teoria tenha um caráter definitivamente quantitativo. A econometria também não deve ser tomada como sinônimo da aplicação da matemática à economia. A experiência mostrou que cada um desses três pontos de vista, o das estatísticas, da teoria econômica e da matemática, é uma condição necessária, mas não por si só suficiente, para uma verdadeira compreensão das relações quantitativas na vida econômica moderna. É a unificação de todos os três que é poderosa. E é essa unificação que constitui a econometria.

Essa unificação é mais necessária hoje do que em qualquer estágio anterior da economia. As informações estatísticas estão atualmente se acumulando a uma taxa sem precedentes. Mas nenhuma quantidade de informações estatísticas, por mais completas e exatas que sejam, pode por si só explicar os fenômenos econômicos. Se não quisermos nos perder na massa avassaladora e desconcertante de dados estatísticos que agora estão disponíveis, precisamos da orientação e ajuda de um poderoso arcabouço teórico. Sem isso, nenhuma interpretação e coordenação significativas de nossas observações serão possíveis.

A estrutura teórica que nos ajudará nessa situação deve, no entanto, ser mais precisa, mais realista e, em muitos aspectos, mais complexa do que qualquer uma disponível até agora. A teoria, ao formular suas noções quantitativas abstratas, deve ser inspirada em maior medida pela técnica de observação. E novos estudos estatísticos e factuais devem ser o elemento saudável de perturbação que constantemente ameaça e inquieta o teórico e o impede de descansar em um conjunto herdado e obsoleto de suposições.

Essa penetração mútua da teoria econômica quantitativa e da observação estatística é a essência da econometria. E aí reside a necessidade da matemática, tanto na formulação dos princípios da teoria econômica, quanto na técnica de manipulação dos dados estatísticos. A matemática certamente não é um procedimento mágico que por si só pode resolver os enigmas da vida econômica moderna, como acreditam alguns entusiastas. Mas, quando combinada com uma compreensão profunda do significado econômico dos fenômenos, é uma ferramenta extremamente útil. De fato, essa ferramenta é indispensável em muitos casos. Muitas das coisas essenciais no novo cenário dos problemas são tão complexas que é impossível discuti-las com segurança e consistência sem o uso da matemática.

Estimular o trabalho nas direções aqui indicadas é o propósito da Sociedade Econométrica e de seu periódico ECONOMETRICA. ECONOMETRICA reportará sobre pesquisa, ensino, reuniões científicas e outras atividades de interesse econométrico nos vários países. Publicará trabalhos originais que sejam significativos, direta ou indiretamente, como contribuições para o desenvolvimento da econometria. A história do nosso assunto também receberá atenção. No entanto, nenhum esforço será feito para monopolizar todas as importantes contribuições econométricas para as colunas deste periódico. Pelo contrário, ECONOMETRICA incentivará, e provavelmente de tempos em tempos relatará, trabalhos econométricos em todos os principais periódicos econômicos do mundo. A política geral será fazer de ECONOMETRICA uma central de trabalhos econométricos.

Com o passar do tempo, talvez se desenvolva alguma forma de cooperação. ECONOMETRICA pode, por exemplo, estar pronta para assumir a publicação de trabalhos que são tão matemáticos a ponto de serem inadequados para publicação em certos outros periódicos econômicos. De fato, será um princípio editorial de ECONOMETRICA que nenhum artigo será rejeitado apenas por ser muito matemático. Isso se aplica, não importa quão envolvido seja o aparato matemático.

Isso não é, é claro, para dizer que um artigo será considerado aceitável para ECONOMETRICA apenas pelo seu uso de símbolos matemáticos. A política de ECONOMETRICA será denunciar com entusiasmo o jogo fútil com símbolos matemáticos na economia, assim como incentivar seu uso construtivo. E uma parte considerável do material que aparece em ECONOMETRICA provavelmente será totalmente não matemática.

Em trabalhos estatísticos e outros numéricos apresentados em ECONOMETRICA, os dados brutos originais serão, como regra, publicados, a menos que seu volume seja excessivo. Isso é importante para estimular a crítica, o controle e estudos adicionais. O objetivo será apresentar este tipo de artigo de forma condensada. Descrições breves e precisas de (1) o cenário teórico, (2) os dados, (3) o método e (4) os resultados, são os elementos essenciais.

Houve uma época em que consideramos organizar em ECONOMETRICA um trabalho extenso sobre resumos de artigos e livros econométricos. No entanto, descobriu-se que isso se sobrepunha em certa medida ao trabalho realizado por outros periódicos econômicos, e que ECONOMETRICA faria um serviço maior organizando pesquisas sobre os desenvolvimentos significativos nos principais campos de interesse do economista. A cada ano, aparecerão em ECONOMETRICA quatro dessas pesquisas, uma para cada um dos seguintes campos: (1) teoria econômica geral (incluindo economia pura), que aparecerá como um recurso regular na edição de janeiro, (2) teoria do ciclo econômico, na edição de abril, (3) técnica estatística, na edição de julho, (4) informação estatística, na edição de outubro.

Todas essas pesquisas serão internacionais em escopo, mas não haverá tentativa de torná-las abrangentes. Pelo contrário, elas serão seletivas. Seu objetivo será apresentar ao leitor uma avaliação dos desenvolvimentos realmente significativos nos campos em questão, sendo assim um guia útil para aqueles que estão interessados nesses campos, sem ter, talvez, tempo para acompanhar de perto a literatura. Haverá um autor responsável por cada pesquisa, que buscará, quando necessário, a cooperação de especialistas. Os nomes dos autores das quatro primeiras pesquisas serão encontrados na Seção de Anúncios no final desta edição.

A liberdade completa de pensamento reinará nas colunas de ECONOMETRICA, e a discussão franca sobre as pesquisas ou sobre outros materiais que aparecem em suas páginas será sempre bem-vinda. A julgar pelas discussões estimulantes que ocorreram tanto nas reuniões europeias quanto nas americanas da Sociedade, não há perigo de endogamia de pensamento entre os economistas. É verdade que aqueles que participaram das reuniões mostraram um entusiasmo verdadeiro pela causa comum, a econometria. Mas, juntamente com essa comunidade de interesse geral e atitude, manifestou-se uma variedade de ideias e uma franqueza na crítica mútua que garantem a amplitude e a novidade do trabalho futuro. Este espírito vigoroso, sem dúvida, será refletido nas colunas de ECONOMETRICA.

ECONOMETRICA é apresentada ao público na esperança de que possa fazer sua parte integral no trabalho construtivo traçado pela Sociedade Econométrica.

\section{\textbf{Morgan (1990)}}
\subsection{\textbf{A História das Ideias Econométricas}}
\subsubsection{\textbf{Introdução}}
A econometria foi vista por seus primeiros praticantes como uma síntese criativa de teoria e evidência, com a qual quase tudo e qualquer coisa poderia, ao que parece, ser alcançado: novas leis econômicas poderiam ser descobertas e novas teorias econômicas desenvolvidas, bem como leis antigas medidas e teorias existentes postas à prova. Esse otimismo baseava-se em uma fé extraordinária nas técnicas quantitativas e na crença de que a econometria ostentava as características de uma forma genuinamente científica de economia aplicada. Em primeiro lugar, a abordagem econométrica não era principalmente empírica: os economistas acreditavam firmemente que a teoria econômica desempenhava um papel essencial na descoberta do mundo. Mas para ver como o mundo realmente funcionava, a teoria tinha que ser aplicada; e suas evidências estatísticas ostentavam todas as credenciais científicas corretas: os dados eram numerosos, numéricos e o mais objetivo possível. Finalmente, os economistas dependiam de um método analítico baseado nos últimos avanços nas técnicas estatísticas. Esses novos métodos estatísticos foram particularmente importantes, pois deram aos economistas do início do século XX maneiras de descobrir sobre o mundo que não estavam disponíveis para seus antecessores do século XIX, maneiras que, por si mesmas, pareciam garantir respeitabilidade científica para a econometria.

Portanto, quando a econometria surgiu como uma atividade distinta no início do século XX, seu uso de métodos estatísticos e dados para medir e revelar as relações da teoria econômica ofereceu uma forma de investigação marcadamente diferente daquelas da economia do século XIX, que variava da introspecção pessoal e observação casual da economia clássica à empiria detalhada da economia histórica. A economia aplicada do século XX seria baseada em uma tecnologia mais moderna: suas ferramentas seriam métodos estatísticos, modelos matemáticos e até calculadoras mecânicas. Com o apelo dessa nova tecnologia, a econometria se tornaria firmemente estabelecida na década de 1940 e a forma dominante de ciência econômica aplicada a partir de então.

Aqueles leitores que conhecem algo sobre a econometria atual já podem suspeitar que o programa de econometria mudou em alguns aspectos de sua concepção e práticas originais. A diferença mais saliente entre a econometria antiga e a moderna é que os primeiros economistas conscientemente uniram a economia matemática e a economia estatística. De fato, a Sociedade Econométrica, fundada em 1931, tinha como principal objetivo:

promover estudos que visam a unificação da abordagem teórico-quantitativa e empírico-quantitativa para problemas econômicos e que são permeados por um pensamento construtivo e rigoroso semelhante ao que passou a dominar nas ciências naturais. Qualquer atividade que prometa, em última instância, promover tal unificação de estudos teóricos e factuais em economia estará dentro da esfera de interesse da Sociedade.

Para os economistas da primeira metade do século XX, a união de matemática e estatística com economia era a maneira ideal de praticar a economia científica.

Exatamente como o ideal econométrico, a união da economia matemática e estatística, surgiu no início do século XX é uma questão interessante. A suposição óbvia é que as raízes da econometria estão na economia matemática e estatística do século XIX. No entanto, na visão contemporânea daquela época (como na economia hoje), acreditava-se que a matemática e a estatística operavam em esferas diferentes. A matemática era considerada essencial para o avanço da economia como uma disciplina científica dedutiva. Acreditava-se que o uso da matemática, isto é, cálculo e álgebra, levaria à clareza e concisão na expressão de teorias (e das suposições envolvidas) e que o processo de raciocínio matemático tornaria os argumentos econômicos mais rigorosos. As muitas virtudes atribuídas ao método matemático ajudaram-no a ganhar aceitação entre um pequeno grupo de economistas analíticos no século XIX, com o resultado de que a economia matemática estava na vanguarda do trabalho teórico a partir da década de 1880. Mas, se levarmos em conta a gama completa de estilos e métodos em economia, o raciocínio matemático foi usado por apenas um punhado de economistas, embora influente, no último quarto do século XIX. O método só realmente passou a ser mais comum, embora ainda não generalizado, no período após 1930. Exemplos numéricos e ilustrações geométricas foram às vezes introduzidos para ajudar a explicar ideias econômicas antes que a álgebra e o cálculo diferencial fossem usados no desenvolvimento da teoria, mas mesmo esses ainda eram comparativamente raros antes de 1900.

Em comparação, o uso de dados estatísticos na economia tem uma longa história que remonta à escola de aritmética política de Petty e Graunt no final do século XVII. Mas não havia uma tradição contínua, não até o surgimento geral do pensamento estatístico no século XIX, que estava particularmente associado às ciências sociais emergentes. Acreditava-se então que a aplicação da análise estatística melhoraria o lado indutivo da economia, e à medida que o século XIX avançava, o uso casual e muitas vezes indiscriminado de dados estatísticos na economia começou a dar lugar a um uso mais cuidadoso. Os dados estatísticos provaram ser úteis na criação de regularidades econômicas, na apresentação de argumentos econômicos e, mais efetivamente, na tomada de medidas de variáveis econômicas (das quais facilmente a mais sofisticada era a construção de índices de números).

Jevons tinha a esperança de que a estatística pudesse ser usada para obter as leis numericamente precisas (ou 'concretas') que se pensava serem típicas da boa ciência física.

Não hesito em dizer também que a Economia Política poderia ser gradualmente erigida em uma ciência exata, se apenas as estatísticas comerciais fossem muito mais completas e precisas do que são atualmente, de modo que as fórmulas pudessem ser dotadas de um significado exato com a ajuda de dados numéricos.
(Jevons (1871), p. 25)

Talvez ele tivesse em mente o exemplo da astronomia, geralmente considerada a ciência mais perfeita e avançada, que já havia invocado a ajuda de matemática e estatística no meio do século XIX. Astrônomos confrontados com várias medidas diferentes do caminho de um planeta (acreditava-se ser de uma forma matemática exata) extraíram a equação do caminho, descartando os resíduos como erros de medição. Os economistas também poderiam olhar para o campo da psicologia, onde as técnicas estatísticas foram usadas pela primeira vez na década de 1860 para medir as relações de estímulo-resposta em circunstâncias experimentais (a lei de Weber-Fechner). Esses 'psicofísicos' adotaram tanto o modelo matemático quanto os métodos estatísticos da astronomia do início do século XIX, salvaguardando efetivamente tanto o determinismo nas teorias quanto a objetividade no trabalho aplicado, a fim de ganhar respeitabilidade científica para a psicologia. Não parece muito exagerado imaginar um polímata como Jevons experimentando em si mesmo para tentar medir incrementos de utilidade, assim como Fechner mediu sua própria capacidade de discriminar entre objetos de pesos diferentes. Note-se, porém, que a possibilidade de experimentação não era um pré-requisito necessário para o uso de estatísticas para medir relações econômicas; afinal, os astrônomos não podiam manipular os planetas e as estrelas.

Mas essas são especulações, pois a econometria não surgiu no século XIX. Com o benefício da retrospectiva, pode-se ver uma série de obstáculos que impediram um programa científico unificado de economia matemática e estatística do tipo que floresceu na primeira metade do século XX. Um problema óbvio era a falta de dados econômicos relevantes: nem todas as variáveis teóricas tinham medidas ou mesmo contrapartes observacionais (a utilidade sendo apenas um exemplo). Igualmente pertinente, os métodos estatísticos (discutidos com mais detalhes mais adiante nesta introdução) não estavam suficientemente avançados para poder atribuir valores numéricos às complexas leis causais de comportamento que figuravam nas teorias econômicas. Mesmo no final do século XIX, os métodos estatísticos ainda podiam fazer pouco mais pelas ciências sociais do que estabelecer regularidades descritivas em variáveis únicas, como o suicídio, ou fornecer comparações de regularidades em duas variáveis. Mesmo que os dados e os métodos apropriados estivessem disponíveis, os economistas matemáticos raramente enquadravam suas teorias com o objetivo de torná-las passíveis de tratamento estatístico. Com alguma presciência, J. N Keynes (1891) observou que as representações geométricas favorecidas pelos economistas matemáticos 'se prestam naturalmente ao registro de estatísticas', mas as representações geométricas, como já observei, ainda eram incomuns. Por último, mas não menos importante, os economistas do século XIX acreditavam que a matemática e a estatística funcionavam de maneiras diferentes: a matemática como uma ferramenta de dedução e a estatística como uma ferramenta de indução. Jevons, que foi pioneiro no uso de matemática e estatística em seu trabalho, expressou tanto o status quo daqueles que visavam uma economia científica quanto sua própria visão de econometria quando escreveu:

A ciência dedutiva da Economia deve ser verificada e tornada útil pela ciência puramente indutiva da Estatística. A teoria deve ser investida com a realidade e a vida do fato. Mas as dificuldades dessa união são imensamente grandes. (Jevons (1871), p. 26)

Pode-se argumentar que devemos olhar não para a história dos métodos econômicos, mas para a história das pessoas, os próprios economistas, para entender de onde veio a econometria. Aqui, também, podemos sugerir possibilidades em vez de encontrar respostas definitivas. Em uma das poucas descrições sistemáticas do surgimento da quantificação na economia do século XIX e início do século XX, Spengler (1961) discute tanto estilos nacionais quanto escolas de pensamento. Ele sugere que a quantificação floresceu na escola neoclássica, e que Marshall e Edgeworth foram particularmente responsáveis por seu sucesso, pois ambos acreditavam na complementaridade dos métodos dedutivos e indutivos na economia. No entanto, Marshall não tinha muito tempo para a economia estatística (e sua forte influência na economia inglesa talvez explique a fraca representação da Grã-Bretanha no trabalho econométrico inicial) e a peculiar marca de probabilidade e economia de Edgeworth não se transformou em econometria. Aquele outro grande pioneiro matemático e neoclássico, Walras, assim como seu compatriota Cournot, não tinha tempo para estatísticas (veja Menard (1980)). Jevons sozinho parece se encaixar na conta e a surpresa é que, apesar de suas declarações, ele evitou o uso complementar de modelos matemáticos e técnicas estatísticas em seu trabalho aplicado. Além disso, o mais importante economista pioneiro americano, Moore, abraçou a econometria em parte porque se desencantou com as ideias e métodos da escola neoclássica. Embora os economistas da década de 1930 se referissem à sua tradição como a de Cournot, Walras e Marshall, qualquer derivação simples da econometria do programa neoclássico é altamente duvidosa, não menos porque ignora as importantes contribuições da economia empírica.

Os economistas do final do século XIX das escolas históricas e institucionalistas tendiam a acolher a crescente riqueza de fatos estatísticos sem necessariamente adotar explicações estatísticas. Na Alemanha, a escola histórica de economistas políticos formou o núcleo de um círculo intelectual mais amplo que acolheu calorosamente o pensamento estatístico. Este círculo proporcionou ligações íntimas entre pensadores estatísticos como Lexis e Engel e economistas históricos como Schmoller. Craver e Leijonhufvud (1987) traçam como essa vertente estatística empírica entrou na escola institucionalista americana no final do século XIX (muitos de cujos membros foram treinados na Alemanha com economistas da escola histórica) e como essa abordagem estatística foi reexportada de volta para a Europa na década de 1920. Mas os economistas americanos que aceitaram a evidência estatística foram muitas vezes aqueles que mais rejeitaram fortemente os métodos e modelos matemáticos na economia. À parte dos EUA, os ambientes mais férteis para o desenvolvimento prático da econometria no início do século XX provaram ser os Países Baixos e a Escandinávia, mas Spengler sugere que os economistas holandeses do século XIX não tinham pontos fortes particulares em matemática ou estatística e os escandinavos eram bons em matemática, mas tinham pouco em termos de formação estatística. Portanto, apesar de muitos comentários interessantes sobre economia matemática e estatística separadamente, Spengler não nos ajuda a ter uma ideia mais clara do processo pessoal pelo qual a econometria surgiu.

Essa busca por um padrão que remonta à economia do século XIX pode estar condenada, pois os economistas do início do século XX não estavam ligados a nenhuma tradição particular. Eles eram um grupo decididamente internacional, de origens intelectuais diversas, e ecléticos em suas crenças econômicas. O estilo nacional e a lealdade teórica pareciam importar menos do que o entusiasmo que os economistas geravam por seu programa metodológico comum. A memória de Frisch do Primeiro Encontro Europeu da Sociedade Econométrica em 1931 em Lausanne foi vividamente lembrada em 1970:

Nós, o povo de Lausanne, estávamos de fato tão entusiasmados com a nova empreitada, e tão ansiosos para dar e receber, que mal tínhamos tempo para comer quando nos sentávamos juntos para almoçar ou jantar com todas as nossas anotações flutuando ao redor da mesa para o desespero dos garçons. (Frisch (1970), p. 152)

Traçar as raízes pessoais desse entusiasmo e as ligações entre os economistas e a comunidade científica mais ampla do século XX permanecem tarefas importantes para o futuro. O lugar mais frutífero para começar tal busca pode muito bem ser entre os pensadores estatísticos, pois o conteúdo e a evolução do programa econométrico em seus anos formativos foram muito influenciados pelos desenvolvimentos na estatística. E, a longo prazo, a economia matemática e a economia estatística se dividiram novamente, deixando a econometria firmemente do lado estatístico da cerca.

Quando olhamos para a história da estatística, descobrimos que a econometria estava longe de ser um desenvolvimento isolado, pois os métodos estatísticos e o pensamento probabilístico foram amplamente introduzidos na ciência do século XIX e início do século XX. Poucos campos científicos permaneceram inalterados, embora o pensamento estatístico e a probabilidade nem sempre entrassem no mesmo nível em cada disciplina, como eu ajudei a argumentar em outro lugar. O movimento econométrico foi paralelo em particular à biometria na biologia e à psicometria na psicologia. Como seus nomes sugerem, todos os três estavam preocupados com a medição e a inferência, em particular com o uso de métodos estatísticos para revelar e substanciar as leis de seus assuntos; e todos os três se desenvolveram aproximadamente ao mesmo tempo.

Dois livros recentes sobre a história da estatística antes de 1900 preparam o cenário para essa explosão de atividade estatística (da qual a econometria era uma pequena parte) no início do século XX. O excelente livro de T. M. Porter (1986) ilumina o terreno difícil entre o pensamento estatístico (maneiras de analisar massas de eventos) e o pensamento probabilístico (onde o acaso e a aleatoriedade têm influência). Esta conta é complementada pela história igualmente interessante de S. M. Stigler (1986) das técnicas de inferência estatística. Ambos os autores enfatizam a importância no início do século XIX da caracterização estatística do comportamento humano por Quetelet: os indivíduos se comportam de maneira imprevisível, mas, juntos, esses indivíduos aparentemente desorganizados obedecem à lei dos erros ao se desviarem do 'homem médio' ideal. Essa observação de regularidade estatística nos assuntos humanos provou ser enormemente influente. Por exemplo, Porter mostra como essas estatísticas sociais levaram os físicos, por analogia, à mecânica estatística. Para Stigler, por outro lado, as noções de Quetelet criaram barreiras para a transferência de técnicas de inferência da astronomia e geodésia para as ciências sociais emergentes.

A influência de Quetelet é refletida na interpretação típica do século XIX da regularidade estatística encontrada na discussão clássica de Mill (1872) sobre métodos de inferência científica. Assim: tomar muitas observações juntas tem o efeito de eliminar aquelas circunstâncias devidas ao acaso (ou seja, as causas variáveis ou individuais), e a regularidade estatística que emerge do conjunto de dados verifica que uma lei causal (ou causa constante) está em ação. Esta ideia de indivíduos como sujeitos a uma causa constante e muitas pequenas causas variáveis (obedecendo à lei dos erros) deixou a causa constante operando em, ou através, do meio da 'sociedade'. Mas os procedimentos de inferência estatística não eram hábeis em descobrir tal causa constante da tendência do homem médio ao suicídio, ou ao assassinato, ou qualquer que seja. Uma pluralidade de causas atuando no nível individual pode ser mais crível para nossas mentes, mas seria igualmente difícil de descobrir, pois, como Stigler aponta, não havia maneiras naturais ou externas (impostas teoricamente) de categorizar tais observações sociais nos grupos homogêneos necessários para a análise estatística existente, e ainda não havia métodos estatísticos para lidar com a pluralidade de causas.

A resolução dessas dificuldades começou no final do século XIX com a reinterpretação do comportamento das pessoas não em termos de erros, mas como variação natural ou real devido às causas complexas dos assuntos humanos e morais. Porter identifica isso como a segunda grande descoberta no pensamento estatístico, quando a 'lei dos erros' se tornou a 'distribuição normal'. Isso levou diretamente ao desenvolvimento de leis estatísticas de herança na genética, nomeadamente regressão (1877-86) e correlação (1889-96). Essas relações semelhantes a leis são importantes porque são, de acordo com Hacking (1983a), as primeiras 'leis estatísticas autônomas' genuínas: leis que oferecem uma explicação eficaz para fenômenos sem a necessidade de se referir a causas anteriores. Na conta de Porter, a adoção de tais modelos estatísticos libertou as ciências sociais e naturais do determinismo da ciência física do século XIX, e levou tanto a um florescimento do trabalho estatístico em diversos campos aplicados quanto ao desenvolvimento do campo teórico da estatística matemática. Para Stigler, é a contribuição de Yule (1897) que liga essas novas leis estatísticas ao antigo método astronômico de mínimos quadrados que foi o passo crucial para tornar a regressão de mínimos quadrados uma ferramenta geral para análise estatística. Essa transformação abriu o desenvolvimento da análise multivariada na qual havia a possibilidade de controlar estatisticamente (ou categorizar) uma variável enquanto investigava o comportamento das variáveis restantes.

Uma conta que visa uma história mais especializada das fundações conceituais da econometria, da aritmética política do final do século XVII até 1920, é oferecida por J. L. Klein (1986). Embora ela recorra às mesmas áreas de estatística aplicada que Porter e Stigler, ela conta outra história. Seu argumento é que tanto a biometria quanto a econometria foram fundadas na necessidade de tornar a variação temporal passível de análise. Assim, a variação lógica (ou não temporal) nos fenômenos humanos recebeu uma identidade estatística em termos do 'homem médio' de Quetelet e da distribuição normal de Pearson. Para prover a variação temporal entre gerações, os biométricos desenvolveram as relações de regressão e correlação. Os economistas adotaram essas novas ferramentas biométricas, juntamente com algumas noções de processos estocásticos da astronomia, para fornecer maneiras estatísticas de caracterizar as relações econômicas ao longo do tempo.

Essas três contas da história da estatística por Porter, Stigler e Klein são todas muito sugestivas, mas por que os economistas gostariam de adotar os novos métodos estatísticos? O que havia nos novos métodos que justificava o otimismo dos economistas do início do século XX sobre sua abordagem? A resposta reside na capacidade dos novos métodos estatísticos de fornecer um substituto para o método experimental. A ideia de um experimento científico ou controlado é reproduzir as condições exigidas por uma teoria e então manipular as variáveis relevantes para fazer medições de um determinado parâmetro científico ou testar a teoria. Quando os dados não são coletados sob condições controladas ou não são de experimentos repetíveis, então a relação entre os dados e as leis teóricas provavelmente não será nem direta nem clara. Este problema não era, claro, único para a economia; surgiu em outras ciências sociais e em ciências naturais onde experimentos controlados não eram possíveis. Para ver como a estatística entra aqui, precisamos voltar mais uma vez ao século XIX, quando os economistas que buscavam um perfil mais científico para a economia lamentavam regularmente o fato de que o método experimental disponível para as ciências físicas era inaplicável à economia.

Um substituto estatístico para o experimento científico no século XIX dependia da capacidade do método estatístico de extrair uma regularidade, ou padrão repetido, ou relação constante, de um conjunto de dados (em vez de tirar uma observação de um evento individualmente manipulado ou variado). Como a conta de Porter enfatiza, a capacidade da estatística de discernir ordem do caos foi uma das inovações cruciais nas estatísticas sociais do século XIX. E, já me referi ao exemplo da astronomia, o exemplo paradigmático do uso de métodos estatísticos do século XIX, no qual uma relação exata foi extraída de várias medidas variáveis da relação. Mas tais ideias e métodos estatísticos do século XIX não necessariamente forneceram soluções prontas para os problemas do século XX. Yule expressou isso de forma mais ordenada:

A investigação de relações causais entre fenômenos econômicos apresenta muitos problemas de dificuldade peculiar e oferece muitas oportunidades para conclusões falaciosas. Como o estatístico raramente ou nunca pode fazer experimentos por si mesmo, ele tem que aceitar os dados da experiência diária e discutir da melhor maneira possível as relações de um grupo inteiro de mudanças; ele não pode, como o físico, reduzir a questão ao efeito de uma variação de cada vez. Os problemas da estatística são, neste sentido, muito mais complexos do que os problemas da física. (Yule (1897), p. 812.

As leis causais altamente complexas das ciências sociais e biológicas exigiam métodos de medição projetados para neutralizar ou permitir os efeitos das circunstâncias variáveis (ou seja, incontroláveis) sob as quais os dados foram coletados em vez do controle total que define o caso ideal do experimento científico. Estas são precisamente as características que Stigler apreende em sua conta dos novos métodos estatísticos emanados da escola biométrica; métodos de medição que permitiriam ao cientista do século XX algum grau de controle sobre dados obtidos de forma não experimental e que lhes permitiam lidar com a pluralidade de causas.

Parece que as condições necessárias para a execução bem-sucedida de uma economia estatística foram cumpridas no início do século XX, mas devemos ter cuidado, pois não há necessidade de qualquer campo adotar novas formas de pensamento estatístico, como a variedade de estudos de caso em Kriiger et al. (1987, II) deixa claro. Além disso, tanto Porter quanto Stigler enfatizam as dificuldades de adaptar ferramentas e ideias estatísticas projetadas em uma arena científica para uso em outra. Este problema não desapareceu simplesmente em 1900, como evidenciado pelo fato de que a biometria, a psicometria e a econometria forjaram suas próprias versões diferentes das novas ferramentas. Em biometria, R. A. Fisher projetou técnicas para randomizar (e assim neutralizar) os efeitos de fatores não controláveis em experimentos agrícolas. Psicometristas como Thurstone desenvolveram o método de análise fatorial de Spearman para extrair medidas dos aparentemente inobserváveis 'vetores da mente' a partir de dados sobre características observáveis para resolver seu próprio hiato entre dados e teoria. Nenhuma técnica era diretamente apropriada para a economia. Fornecidos com ferramentas gerais, os economistas ainda tiveram que desenvolver suas próprias soluções estatísticas para os problemas de preencher a lacuna entre as condições exigidas pela teoria e as condições sob as quais os dados foram coletados. Seria errado sugerir que, ao fazer isso, os economistas estavam conscientemente fazendo seu próprio substituto para experimentos, em vez disso, eles responderam aos problemas particulares de medição e controle, o descompasso particular entre teoria e dados, que ocorreram em seu trabalho aplicado. Somente mais tarde eles começaram discussões teóricas e buscaram soluções gerais para seus problemas práticos.

A estrutura deste livro reflete a importância do trabalho aplicado no desenvolvimento de ideias econométricas: as aplicações formaram o catalisador necessário para os economistas reconhecerem suas dificuldades e buscarem soluções. As partes I e II do livro traçam a evolução da econometria através do trabalho prático sobre ciclos de negócios e análise de demanda de mercado, que juntos constituíram a maior parte da econometria aplicada do período até cerca de 1950. A exploração desses campos revela não apenas avanços na compreensão, mas também confusão e becos sem saída. A parte III do livro reconstrói a história dos modelos econométricos formais da relação dados-teoria. Esta parte final se baseia no trabalho econométrico aplicado discutido nas seções anteriores e na econometria teórica que começou a se desenvolver durante a década de 1930. Até a década de 1940, discussões teóricas e prática aplicada se cristalizaram em um programa que é reconhecível como econometria moderna.

\section{\textbf{Schumpeter (1933)}}
\subsection{\textbf{O Bom Senso da Econometria}}
O BOM SENSO DA ECONOMETRIA

Os objetivos desta revista, e da Sociedade da qual ela é o órgão, foram declarados acima pelo Editor com a brevidade e precisão que são características de toda declaração de um caso sólido. O que tenho a acrescentar, por meio de comentários e amplificações, espero, confirmará a impressão de que não há nada surpreendente ou paradoxal em nossa empreitada, mas que ela cresce naturalmente da situação atual de nossa ciência. Não desejamos reviver controvérsias sobre questões gerais de 'método', mas simplesmente apresentar e discutir os resultados de nosso trabalho. Não impomos nenhum credo - científico ou de outra forma -, e não temos nenhum credo comum além de afirmar: primeiro, que a economia é uma ciência, e em segundo lugar, que essa ciência tem um aspecto quantitativo muito importante. Não somos uma seita. Nem somos uma 'escola'. Para todas as possíveis diferenças de opinião sobre problemas individuais, que podem existir entre os economistas, existem, e espero que sempre existam, entre nós.

Como tudo o mais, a vida econômica pode ser vista de um grande, estritamente falando, número infinito de pontos de vista. Apenas alguns desses pertencem ao reino da ciência, ainda menos admitem ou requerem o uso de métodos quantitativos. Muitos aspectos não quantitativos são, e sempre foram, mais interessantes para a maioria das mentes. Trabalho frutífero pode ser feito em linhas inteiramente não quantitativas. Muito do que queremos saber sobre fenômenos econômicos pode ser descoberto e declarado sem refinamentos técnicos, muito menos matemáticos, sobre modos de pensamento comuns, e sem tratamento elaborado de figuras estatísticas. Nada está mais longe de nossas mentes do que qualquer crença acrimoniosa na excelência exclusiva dos métodos matemáticos, ou qualquer desejo de menosprezar o trabalho de historiadores, etnólogos, sociólogos e assim por diante. Não queremos lutar contra ninguém, ou, além do diletantismo, qualquer coisa. Queremos servir da melhor maneira que pudermos.

\subsubsection{\textbf{ECONOMIA, A CIÊNCIA QUANTITATIVA}}
Existe, no entanto, um sentido em que a economia é a mais quantitativa, não apenas das 'ciências sociais' ou 'morais', mas de todas as ciências, física não excluída. Pois massa, velocidade, corrente e similares podem, sem dúvida, ser medidos, mas para isso sempre precisamos inventar um processo distinto de medição. Isso deve ser feito antes de podermos lidar com esses fenômenos numericamente. Alguns dos fatos econômicos mais fundamentais, ao contrário, já se apresentam à nossa observação como quantidades tornadas numéricas pela própria vida. Eles carregam significado apenas em virtude de seu caráter numérico. Haveria movimento mesmo se não pudéssemos transformá-lo em quantidade mensurável, mas não podem existir preços independentes da expressão numérica de cada um deles e de relações numéricas definidas entre todos eles.

A econometria não é nada além do reconhecimento explícito desse fato bastante óbvio e a tentativa de enfrentar as consequências disso. Poderíamos até ir tão longe a ponto de dizer que, por virtude disso, todo economista é um economista, quer queira ser ou não, desde que lide com este setor de nossa ciência e não, por exemplo, com a história da organização empresarial, os aspectos culturais da vida econômica, motivo econômico, a filosofia da propriedade privada, e assim por diante. É fácil entender por que o reconhecimento explícito desse fato deve ter sido tão difícil e por que demorou tanto para acontecer. Os filósofos, que sempre se deliciaram em classificar as ciências, sempre se sentiram inseguros sobre o lugar preciso a ser atribuído à economia como um todo. Como era, eles praticamente seguiram a linha divisória empírica entre 'ciências naturais' e 'morais', e classificaram a economia com esta última. E lá, é claro, o aspecto ou setor quantitativo de nossa ciência encontrou um terreno pouco agradável.

Outra razão foi que os problemas econômicos foram na maioria das vezes abordados em um espírito prático, indiferente ou hostil às reivindicações dos hábitos de pensamento científico. Nenhuma ciência prospera, no entanto, na atmosfera de um objetivo prático direto, e mesmo os resultados práticos são apenas subprodutos do trabalho desinteressado no problema pelo bem do problema. Ainda estaríamos sem a maioria das conveniências da vida moderna se os físicos tivessem sido tão ansiosos por uma 'aplicação' imediata quanto a maioria dos economistas são e sempre foram. Isso explica a negligência da econometria, bem como o estado insatisfatório de nossa ciência em geral. Ninguém que anseia por respostas rápidas e curtas para questões urgentes do dia se importará em se envolver em dificuldades que apenas um trabalho paciente pode esclarecer ao longo de muitos anos.

No entanto, o caráter quantitativo do assunto estava destinado a se afirmar. É um dos fatos mais marcantes sobre a história de nossa ciência, que a maioria - e se excluirmos os historiadores, todos - daqueles homens que estamos justificados em chamar de grandes economistas invariavelmente exibem uma mente notavelmente matemática, mesmo quando são totalmente ignorantes de qualquer coisa além da técnica quantitativa ao alcance de um estudante; Quesnay, Ricardo, Böhm-Bawerk, são exemplos disso.

Nem é tudo. Se os economistas têm algum desejo de imitar outras pessoas e se gloriar em uma ancestralidade heroica, eles podem com justiça reivindicar o grande nome de Sir William Petty como seu. A segunda parte do século XVII está cheia de empreendimentos vigorosos no campo da econometria - basta apontar para a curva de demanda estatística de Gregory King. É uma questão de algum interesse como foi possível que tais começos promissores pudessem ter falhado em inspirar um trabalho posterior, e como seus resultados poderiam ter sido deixados para permanecer no crepúsculo, embora de forma alguma fossem esquecidos, a referência à 'regra de King' sendo parte do estoque em comércio de quase todos os livros didáticos obsoletos escritos desde então.

Na esfera dos fenômenos monetários e de sua vizinhança, a análise quantitativa e até mesmo numérica se tornou prática estabelecida já no século XVI, principalmente na Itália, e essa tradição nunca mais foi perdida; passagens dos escritores italianos do século XVIII, como Beccaria, Carli, Verri e outros, soam muito familiares ao ouvido moderno. O que temos diante de nós ali é nada menos que uma tentativa consciente de fundir em um argumento indivisível teoremas e fatos estatísticos.

E, desde que deixemos de fora a palavra 'consciente', encontramos substancialmente a mesma tendência em qualquer parte do trabalho de nossos predecessores que escolhemos olhar. Para dar apenas um exemplo; estamos acostumados a zombar da literatura da controvérsia consagrada pelo tempo sobre o Valor. Mas o que mais está no fundo disso, sobreposto é verdade por massas pesadas de verborragia especulativa, senão a busca verdadeiramente científica por uma unidade de medida econômica, ou de várias dessas unidades adaptadas a diferentes classes de fenômenos? Não havia mais especulação não empírica sobre isso do que há sobre toda ciência em sua infância. Nem há menos conexão com tais materiais estatísticos como cada época poderia comandar do que temos direito de esperar - como todos admitirão quem se deu ao trabalho de ler a resposta de Ricardo a Bosanquet.

\subsubsection{\textbf{DESENVOLVIMENTOS POSTERIORES}}

Análise essencialmente quantitativa, mas prejudicada pela falta tanto de técnica apropriada quanto de material estatístico adequado - esta é a diagnose que chegamos quando estudamos o trabalho dos economistas até aquele tempo, quando os princípios de Mill eram bastante representativos do que nossa ciência tinha a oferecer. Este, também, é o elemento de verdade que emerge da fraseologia hostil que temos o hábito de usar sobre a 'doutrina clássica'. Obviamente, portanto, o que nossa sociedade defende é tudo menos uma inovação. Não é mais do que um esforço consciente para remover obstruções ao fluxo de um fluxo que tem corrido desde que os homens começaram a pensar e escrever sobre a vida econômica.

Para ver em sua plena significância as condições que tornaram desejável, e de fato necessário, formar, sob a bandeira da Econometria, uma coalizão dos diferentes tipos de economistas que vão se unir em nossa sociedade, devemos, no entanto, agora dar uma olhada nos desenvolvimentos posteriores. A fase que poderia, até cerca de dez anos atrás, ser chamada de fase 'moderna' da economia, admite descrição em termos de três fatos e suas consequências: primeiro, o rápido crescimento de nossa riqueza de material estatístico e outro; segundo, o progresso da técnica estatística ao nosso comando - que na medida em que cresceu em grande parte fora de nosso próprio campo e sem referência às nossas necessidades, foi um golpe de boa sorte, muito como uma carona no carro de outro homem é para o andarilho em uma estrada empoeirada; e terceiro, o surgimento de um motor teórico muito superior ao antigo. Verdade, em nenhum desses pontos estamos, ou podemos estar, satisfeitos; em todos eles, parece-me, a coisa real ainda está por vir, e o desempenho atual pede desculpas em vez de congratulações. No entanto, não seria apenas mesquinho, mas positivamente falso, negar a importância do que foi alcançado, ou as possibilidades que começam a surgir no futuro.

Em tudo isso, a linha econométrica se destaca claramente. Foi definitivamente estabelecido que a teoria econômica envolve quantidade, e portanto requer a única linguagem ou método disponível para lidar com argumento quantitativo assim que ultrapassa seu estágio mais primitivo. A W. St. Jevons pertence a honra de ter falado uma dessas mensagens simples, que às vezes parecem focar tanto a história passada quanto a futura e se tornar marcos visíveis para sempre. Foi ele quem disse na introdução à sua Teoria da Economia Política (1ª ed. 1871); "Está claro que a Economia, se for uma ciência, deve ser uma matemática". Mas ainda maior tributo é devido a A. Cournot que, sem incentivo ou liderança, em um ambiente então mais inconveniente, em 1838 antecipou totalmente o programa econométrico por suas Pesquisas, uma das realizações mais marcantes do verdadeiro gênio, a qual respeitamos até hoje quase sempre começando a partir delas. Claro, seria supérfluo enfatizar a importância primordial daquele grande professor nosso cuja exposição da teoria exata brotou de sua cabeça como Minerva da cabeça de Júpiter. O que quero enfatizar é que ele construiu seu aparato analítico com uma percepção clara do objetivo econométrico final, cada parte dele sendo pensada de forma a se ajustar para agarrar o fato estatístico quando chegasse a hora. Nisso ele foi muito mais longe do que Jevons. Isso soa como um paradoxo porque Jevons realmente trabalhou em números, como no caso dos índices. Mas dentro dos limites da própria teoria pura, ele parece muito menos preocupado com esse objetivo do que Cournot, e é muito mais difícil para o cavalo numérico pular as cercas de Jevons do que trotar na estrada de Cournot.

No nosso panteão, o lugar de J. H. von Thünen está lado a lado com o de Cournot. Não é apenas - na verdade, nem mesmo principalmente - a ideia de produtividade marginal que é importante mencionar aqui, mas a peculiar relação de Thünen com um conjunto de fatos, que é tão vital para a econometria quanto as estatísticas no sentido estrito da palavra. Thünen apontou que a contabilidade de custos, a contabilidade e os títulos vizinhos, cobrem uma massa de material que os economistas negligenciaram completamente. Esta negligência tem sido tão completa que os especialistas em 'Administração de Empresas' agora realmente começaram a construir suas próprias casas teóricas que os isolam da 'teoria geral' tão completamente quanto ela, por sua vez, os excluiu, apesar do fato de que ambos os grupos de trabalhadores em grande medida - um exemplo notável é a questão das curvas de custo - cultivam o mesmo terreno. É claro que os economistas não podem indefinidamente prescindir desse vasto reservatório de fatos, nem os contadores de custos, contabilistas e assim por diante, prescindir da contribuição dos economistas. E, olhando para trás, vemos agora que já em 1826, o livro de Thünen poderia ter nos ensinado como a teoria cresce a partir da observação da prática empresarial.

Por mim, sempre olharei para Léon Walras como o maior de todos os economistas. Em sua teoria do equilíbrio, ele deu uma base poderosa para todo o nosso trabalho. É verdade que, enquanto ele deu o passo decisivo no quantitativo, ele falhou em mover-se na linha numérica, a junção das quais duas é característica da econometria. Mas temos sido ensinados recentemente a olhar com mais esperança até mesmo para as possibilidades 'numéricas' daquela parte mais geral e mais abstrata de nossa ciência que é a teoria do equilíbrio no sentido de Walras. E este fato, da mesma forma, indica as reivindicações econométricas do trabalho de Auspitz e Lieben, de Knut Wicksell, de Francis Y. Edgeworth, e do grande sucessor de Walras em Lausanne, Vilfredo Pareto.

Em um sentido um tanto diferente, podemos finalmente reivindicar como nosso aquele maior de todos os professores de economia, Alfred Marshall. Com alguns de nós, tornou-se um costume falar dele como o expoente da doutrina 'neoclássica'. Este não é o lugar para mostrar como aconteceu - não sem alguma falha do próprio Marshall - que um rótulo tão injusto e, de fato, sem sentido foi afixado ao seu nome. Mas eu gostaria de enfatizar primeiro, que ninguém pode ler seu discurso sobre "A Velha Geração de Economistas e a Nova" sem descobrir, embora talvez não sem alguma surpresa, quão claramente nosso programa estava diante de sua mente. Nem é possível para quem sabe ler seus Princípios à luz de sua Indústria e Comércio definir o que ele realmente se esforçou para realizar em qualquer termo que não seja econométrico. Mais importante de tudo, ele sempre trabalhou de olho na aplicação estatística, e ele estava no seu melhor como teórico ao construir essas ferramentas úteis, como elasticidade, quase-renda, economias externas e internas, e assim por diante, que são tantas pontes entre a ilha da teoria pura e a terra firme da prática empresarial e das estatísticas empresariais.

Não desejo falar de nenhum economista vivo. Mas os leitores provavelmente não me perdoariam se eu deixasse de fazer duas exceções e mencionar o trabalho pioneiro de Irving Fisher e Henry L. Moore.

\subsubsection{\textbf{O ESTADO ATUAL}}
Todos esses feitos foram, para dizer o mínimo, suficientes como um bom começo e para construir a partir deles. E, de fato, um trabalho cheio de promessas tem sido feito em nossa linha durante as últimas duas décadas, um trabalho que nos faz sentir, quando agora olhamos para o sistema Walrasiano, muito como nos sentimos ao contemplar o modelo de um carro motorizado construído há quarenta anos. Mas ainda assim, a maioria de nós, sem dúvida, concorda em encontrar o estado atual de nossa ciência decepcionante, não apenas em comparação com as realizações de outras ciências, mas também em comparação com o que nossa ciência poderia ser esperada para realizar. Há muitas razões para isso, mas algumas delas apenas, tendo uma relação especial com a missão desta Sociedade, chamam a atenção aqui.

Raciocinar sobre fatos econômicos significa, e sempre significou, dentro de um setor muito importante, raciocínio quantitativo. E não há ruptura lógica entre o raciocínio quantitativo de caráter elementar e o raciocínio quantitativo do tipo que envolve o uso de 'matemática avançada'. Mas nada faz uma maior ruptura prática na evolução de uma ciência do que a introdução de um hábito de pensamento que até agora tem sido estranho ao equipamento reconhecido do especialista, e que ao mesmo tempo é inacessível exceto por esforço árduo. Quando a necessidade de proceder ao uso de métodos matemáticos mais refinados, tanto na teoria econômica quanto nas estatísticas, se tornou aparente para alguns, a maioria, mesmo daqueles economistas que trabalhavam no setor quantitativo, se recusou a seguir. No início eles riram. Eles não fazem mais isso. Integrais deixam de ser hieróglifos para eles. Muitos deles tentam entender e fizeram as pazes conosco, enquanto reservam o direito de criticar nossos resultados e se opor a excessos matemáticos. Mas isso não é a cooperação plena de que precisamos. Mesmo nesta situação melhorada, a economia carece daquela ampla extensão de terreno comum profissional que, no caso da física, transmite resultados adquiridos ao público em geral. Os iniciantes estão perplexos com essa situação instável. Energia está sendo desperdiçada e o verdadeiro negócio da ciência está sendo prejudicado. O progresso recente, e ainda mais do que o progresso real, amplas possibilidades dele, atraiu para nosso campo uma série de novatos mais promissores. Mas a velha situação tendo sido fundamentalmente mudada, não tínhamos um treinamento uniforme para oferecer a eles. Daí a falta de coordenação do trabalho. Os novos homens vieram enfrentar nossos problemas de ângulos muito diferentes e com aquisições muito diferentes, cheios de impaciência para limpar o terreno e construir inteiramente de novo. O homem que a natureza moldou para se deleitar com o fato não adulterado, seja ele trabalhando em um escritório de estatística ou fazendo trabalho de campo por conta própria, muitas vezes sabia pouco e se importava menos com aquele motor de análise, que chamamos de 'teoria econômica', ou por técnica estatística refinada. Por outro lado, o mestre desta técnica, sentindo seu poder e vendo o material para agarrar com ela, tentou se apressar em suas próprias regularidades ou generalizações. E o teórico, consciente de sua própria tarefa, recusou mais vezes do que seria sábio aceitar o trabalho dos outros dois tipos como nada além de (possíveis) verificações de seus teoremas. Mas, embora descoordenado, o crescimento tem sido tropical. Poderia-se esperar que ele se estabilizasse e desse frutos a tempo, mas há caos por enquanto, no qual apenas um olho muito experiente pode ver uma tendência subjacente trabalhando seu caminho lentamente, embora poderosamente, em direção a um objetivo comum a todos.

\subsubsection{\textbf{O PROGRAMA}}
O bom senso do programa de nossa Sociedade se concentra na pergunta: Não podemos fazer melhor do que isso? Certamente não seria uma política razoável sentar e esperar até que, no final, as coisas se ajeitem por si mesmas, e enquanto isso permitir que os economistas de todos os países lutem sozinhos em sua batalha árdua. O que queremos criar é, em primeiro lugar, um fórum para o esforço econométrico de todos os tipos amplo o suficiente para dar amplo escopo a todas as possíveis visões sobre nossos problemas, mas não tão amplo a ponto de ser prejudicado pelo peso de uma audiência que mantém a discussão nas antessalas dos verdadeiros pontos em questão, e força cada orador ou escritor a passar todas as vezes pelas mesmas preliminares.

Neste fórum, que pensamos como internacional, queremos em segundo lugar criar um espírito e um hábito de cooperação entre homens de diferentes tipos de mente por meio de discussões de problemas concretos de caráter quantitativo e, na medida do possível, numérico. Os próprios problemas individuais são, por assim dizer, para nos ensinar como eles querem ser tratados. Queremos aprender como ajudar uns aos outros e entender por que, e precisamente onde, nós mesmos, teóricos, estatísticos, coletores de fatos, ou nossos vizinhos, de alguma forma não conseguem chegar aonde queremos estar. Nenhuma discussão geral sobre princípios de método científico pode nos ensinar isso. Já tivemos o suficiente disso. Sabemos que não leva a lugar nenhum e só deixa as partes no concurso onde estavam antes, talvez ainda mais exasperadas por aquelas grosserias gentis que é costume administrar uns aos outros em tais ocasiões. Nenhum argumento geral desse tipo jamais convence o homem que pretende trabalho real. Mas, confrontados com questões claras, a maioria de nós, esperamos, estará pronta para aceitar o único julgamento competente e o único critério relevante do método científico, ou seja, o julgamento ou critério do resultado. Há alta virtude remediadora no argumento quantitativo e na prova exata. Aquela parte de nossas diferenças - não importa se grandes ou pequenas - que se deve ao mal-entendido mútuo, desaparecerá automaticamente assim que mostrarmos uns aos outros, em detalhes e na prática, como nossas ferramentas funcionam e onde precisam ser melhoradas. E a acerbidade metafísica e os veredictos abrangentes desaparecerão com ela. A pesquisa teórica e 'factual' por si só encontrará suas proporções corretas, e podemos esperar razoavelmente concordar no final sobre o tipo certo de teoria e o tipo certo de fato e os métodos de tratá-los, não postulando nada sobre eles por programa, mas evoluindo-os, esperamos, por realização positiva.

Não devemos nos entregar a grandes esperanças de produzir rapidamente resultados de uso imediato para a política econômica ou a prática empresarial. Nossos objetivos são, em primeiro e último lugar, científicos. Não enfatizamos o aspecto numérico apenas porque pensamos que ele leva diretamente ao cerne das questões candentes do dia, mas sim porque esperamos, do constante esforço para lidar com as dificuldades do trabalho numérico, uma disciplina saudável, a sugestão de novos pontos de vista e ajuda na construção da teoria econômica do futuro. Mas acreditamos, é claro, que indiretamente a abordagem quantitativa será de grande consequência prática. O único caminho para uma posição em que nossa ciência possa dar conselhos positivos em larga escala para políticos e empresários, passa pelo trabalho quantitativo. Pois enquanto não formos capazes de colocar nossos argumentos em números, a voz de nossa ciência, embora ocasionalmente possa ajudar a dissipar erros grosseiros, nunca será ouvida por homens práticos. Eles são, por instinto, economistas todos eles, em sua desconfiança de qualquer coisa que não seja passível de prova exata.

\end{document}
