\documentclass[a4paper,12pt]{article}[abntex2]
\bibliographystyle{abntex2-alf}
\usepackage{siunitx} % Fornece suporte para a tipografia de unidades do Sistema Internacional e formatação de números
\usepackage{booktabs} % Melhora a qualidade das tabelas
\usepackage{tabularx} % Permite tabelas com larguras de colunas ajustáveis
\usepackage{graphicx} % Suporte para inclusão de imagens
\usepackage{newtxtext} % Substitui a fonte padrão pela Times Roman
\usepackage{ragged2e} % Justificação de texto melhorada
\usepackage{setspace} % Controle do espaçamento entre linhas
\usepackage[a4paper, left=3.0cm, top=3.0cm, bottom=2.0cm, rigH=2.0cm]{geometry} % Personalização das margens do documento
\usepackage{lipsum} % Geração de texto dummy 'Lorem Ipsum'
\usepackage{fancyhdr} % Customização de cabeçalhos e rodapés
\usepackage{titlesec} % Personalização dos títulos de seções
\usepackage[portuguese]{babel} % Adaptação para o português (nomes e hifenização
\usepackage{hyperref} % Suporte a hiperlinks
\usepackage{indentfirst} % Indentação do primeiro parágrafo das seções
\sisetup{
  output-decimal-marker = {,},
  inter-unit-product = \ensuremath{{}\cdot{}},
  per-mode = symbol
}
\DeclareSIUnit{\real}{R\$}
\newcommand{\real}[1]{R\$#1}
\usepackage{float} % Melhor controle sobre o posicionamento de figuras e tabelas
\usepackage{footnotehyper} % Notas de rodapé clicáveis em combinação com hyperref
\hypersetup{
    colorlinks=true,
    linkcolor=black,
    filecolor=magenta,      
    urlcolor=cyan,
    citecolor=black,        
    pdfborder={0 0 0},
}
\usepackage[normalem]{ulem} % Permite o uso de diferentes tipos de sublinhados sem alterar o \emph{}
\makeatletter
\def\@pdfborder{0 0 0} % Remove a borda dos links
\def\@pdfborderstyle{/S/U/W 1} % Estilo da borda dos links
\makeatother
\onehalfspacing

\begin{document}

\begin{titlepage}
    \centering
    \vspace*{1cm}
    \Large\textbf{INSPER – INSTITUTO DE ENSINO E PESQUISA}\\
    \Large ECONOMIA\\
    \vspace{1.5cm}
    \Large\textbf{Tradução Tópico 7.1 - HPE}\\
    \vspace{1.5cm}
    Prof. Pedro Duarte\\
    Prof. Auxiliar Guilherme Mazer\\
    \vfill
    \normalsize
    Hicham Munir Tayfour, \href{mailto:hichamt@al.insper.edu.br}{hichamt@al.insper.edu.br}\\
    4º Período - Economia B\\
    \vfill
    São Paulo\\
    Abril/2024
\end{titlepage}

\newpage
\tableofcontents
\thispagestyle{empty} % This command removes the page number from the table of contents page
\newpage
\setcounter{page}{1} % This command sets the page number to start from this page
\justify
\onehalfspacing

\pagestyle{fancy}
\fancyhf{}
\rhead{\thepage}

\section{\textbf{Morgan e Rutherford (1998)}}
\subsection{\textbf{Economia Americana: O Caráter da Transformação}}

Uma possível interpretação do título do nosso volume, Do Pluralismo do Entre Guerras ao Neoclassicismo do Pós-Guerra, é entender o pluralismo do entre guerras como um nome de código para os "velhos institucionalistas", o neoclassicismo do pós-guerra como matemática de equilíbrio geral em pleno funcionamento, e o caminho entre eles como a vitória natural e inexorável da matemática e da ciência sobre o historicismo desajeitado. Não é difícil reconhecer essa conta pelo que ela é: um conjunto de homens de palha prontos para serem derrubados. O pluralismo consistia em mais do que apenas institucionalismo, e não havia nada predeterminado sobre o declínio do pluralismo e o crescimento do neoclassicismo. O desafio é fornecer uma conta mais convincente. Qual era exatamente a natureza desse pluralismo e do neoclassicismo que aparentemente o substituiu? E por meio de que conjunto de processos um se transformou no outro? Essas são perguntas complexas, e os ensaios neste volume fornecem componentes críticos das respostas. Eles também colocam restrições em qualquer conta geral, particularmente com relação ao momento da transformação. Usamos este ensaio para delinear as fronteiras de tal conta e esboçar uma imagem geral de como esses componentes se encaixam de uma maneira particular. Estamos muito cientes de que ainda nos falta conhecimento de muitos dos elementos que precisamos para entender o processo completo de transformação: nossa conta permanece especulativa e incompleta.

\subsubsection{\textbf{Pluralismo}}

Os dois primeiros elementos para os historiadores do final do século XX entenderem são a extensão e as dimensões do pluralismo dentro da economia americana no período entre guerras. Um forte indicativo é fornecido pela dificuldade prática de caracterizar a economia ou os economistas do entre guerras de maneira convincente. É comum pensar na economia americana do entre guerras em termos de institucionalistas versus neoclássicos, mas quando se investiga mais de perto, a imagem se torna muito menos clara.

Embora suas raízes remontem aos anos 1880, o institucionalismo se tornou um movimento auto-identificado apenas em 1918 (Rutherford 1997). No período entre guerras, o institucionalismo fez fortes reivindicações para si mesmo como uma escola e conseguiu se tornar o grupo mais visível, se não o dominante, na economia americana. O movimento não se coadunava em torno de uma agenda teórica apertada, mas em torno de uma visão particular da ciência e de uma convicção da inadequação do mercado não regulado. Não se pode dizer que institucionalistas como Thorstein Veblen, Wesley C. Mitchell, Walton H. Hamilton, John R. Commons, J. M. Clark, Rexford Tugwell e M. A. Copeland todos seguiram exatamente o mesmo programa de pesquisa ou utilizaram as mesmas técnicas de investigação. O institucionalismo incluía os métodos quantitativos de Mitchell, as histórias documentais e entrevistas de Commons, os estudos de caso de empresas e indústrias de Hamilton e a teorização aplicada de Clark. O institucionalismo consistia em uma série de programas de pesquisa vagamente relacionados, um cluster centrado em ciclos de negócios e desemprego, com uma agenda de reforma envolvendo alguma noção de planejamento geral, e outro cluster centrado nas dimensões legais dos mercados, com uma agenda de reforma focada na lei trabalhista e na regulação empresarial. O institucionalismo também se desviou para uma teoria mais "ortodoxa". Por exemplo, J. M. Clark nunca rejeitou a contribuição teórica de J. B. Clark, mas se viu como tentando continuar os esforços de seu pai para desenvolver uma teoria dinâmica. O acelerador de J. M. Clark, o trabalho de Mitchell e Simon Kuznets sobre contabilidade de renda nacional, e o fluxo de fundos de Copeland se tornaram ferramentas padrão.

O institucionalismo, então, era um movimento amplo e bastante não exclusivo. Os institucionalistas como grupo não tinham um método para defender e nenhuma teoria econômica para vender. O que eles tinham era um compromisso com a investigação científica séria, trabalho empírico detalhado (embora sem um método único), construção de teoria séria (que evitava suposições simples) e um compromisso em entender a importância das instituições econômicas na determinação dos resultados econômicos. Este último ponto se relaciona com a visão dos institucionalistas de que novas instituições ou métodos de "controle social" eram necessários para superar os problemas econômicos e sociais criados pelo sistema de mercado existente.

Da mesma forma, é especialmente difícil definir a economia "ortodoxa" ou "neoclássica" no contexto do entre guerras e fornecer um agrupamento de indivíduos sob esses rótulos. O marginalismo de J. B. Clark foi altamente influente na América, mas a maioria dos economistas da época, incluindo o próprio Clark, sentia que sua teoria estática da competição era apenas um ponto de partida para uma análise mais completa e dinâmica. As ideias austríacas e subjetivistas também eram importantes. O grupo mais "ortodoxo" no período de 1880 até a Primeira Guerra Mundial era altamente diversificado e incluía Arthur Hadley, Frank Taussig, J. B. Clark, Frank Fetter, H. J. Davenport e Edwin Seligman. A maioria desses indivíduos continuou a contribuir durante o período do entre guerras e foi acompanhada por outros, como Frank Knight, Irving Fisher, Jacob Viner e Allyn Young. Esses indivíduos tinham um maior respeito pelo corpo existente de teoria econômica e pelo sistema de mercado do que os mais falantes dos institucionalistas, mas não adotaram todos as mesmas posições teóricas ou metodológicas; nem ignoraram as deficiências da teoria existente ou permaneceram indiferentes ao avanço de seu status científico. Hadley estudou instituições e os problemas das ferrovias, Davenport combinou influências austríacas e veblenianas, e Young trouxe para sua teorização, e para a de seus alunos, uma sensibilidade institucionalista sobre a necessidade de maior realismo. Adotar uma posição teórica mais "ortodoxa" também não implicava necessariamente falta de compromisso com a reforma ou rejeição da defesa. Fisher é apenas um de muitos exemplos disso, e vale a pena notar que Fisher e Commons poderiam professar respeito mútuo um pelo outro e unir forças na Liga do Dinheiro Estável. No entanto, a natureza da economia científica, a natureza das reformas indicadas e o lugar da defesa foram todos ativamente contestados. A profissão como um todo estava muito no processo de se definir e seus papéis sociais.

Como dissemos, o pluralismo não deve ser entendido como uma palavra de código para "institucionalismo". Era um pluralismo genuíno, a ser tomado em um sentido positivo. Pluralismo significava variedade, e essa variedade era evidente nas crenças, na ideologia, nos métodos e nos conselhos de política. Estamos acostumados a pensar nos institucionalistas como difíceis de definir por causa de seus variados interesses e abordagens práticas. Mas a variedade parece ser verdadeira em geral, pois não há linhas limpas separando as escolas; na verdade, não está claro que se possa especificar escolas. E não é mais fácil fornecer rótulos simples e precisos para muitos outros economistas ativos no período do entre guerras. Os economistas se sentiam livres para buscar suas próprias combinações individuais de ideias. O pluralismo, como Warren Samuels observou em nossa conferência, descreve não apenas a diferença entre indivíduos; o pluralismo estava em cada economista. Coats (1992), em uma pesquisa abrangente do período, sugere que o economista mais "influente" do período foi Mitchell, um institucionalista renomado por análise quantitativa, enquanto o economista mais "representativo" era J. M. Clark, que fez a ponte entre o pensamento institucionalista e neoclássico. Claramente, então, no período do entre guerras era possível ter uma série de diferentes crenças econômicas e fazer economia de muitas maneiras diferentes sem estar fora do lugar ou necessariamente perder o respeito dos colegas. Os principais institucionalistas e não institucionalistas publicaram nas principais revistas, ocuparam professorados nas principais universidades e se tornaram presidentes da American Economic Association (AEA).

Essa variedade, juntamente com uma certa tolerância, era uma característica não apenas do período do entre guerras, mas também, como mostra Bradley W. Bateman neste volume, do período antes da Primeira Guerra Mundial. Admitidamente, a declaração original de princípios da AEA excluiu uma série de economistas da "velha escola", mas a associação abandonou sua declaração para efetuar uma conciliação e se tornar uma organização mais católica. Vale a pena notar que isso foi feito por um sentimento de força e que os economistas do laissez-faire eram uma minoria pequena e desaparecendo. Este foi um momento em que uma gama muito ampla de economistas, de marginalistas a históricos, compartilhavam um compromisso com a justiça econômica. Esse compromisso foi apoiado pela população em geral, como evidenciado pelo manifesto conhecido como "Credo Social" que as igrejas protestantes adotaram antes da guerra, como aprendemos no ensaio de Bateman. Um documento extraordinário, o manifesto pedia regulação econômica e intervenção, juntamente com salários altos, para garantir o bem-estar econômico da população americana. Em vez de um credo social, era um credo econômico e um chamado para ação totalmente de acordo com o programa dos Progressistas. O credo não fornecia um conjunto de crenças econômicas teóricas nem métodos de economia, mas uma declaração concreta de fé na intervenção econômica e um conjunto de objetivos econômicos específicos.

Mas, embora o colapso do movimento do Evangelho Social e a consequente perda de ímpeto por trás do Credo Social tenham minado essa fé e deixado um vácuo em termos de ideologia e ação política, a pluralidade subjacente de abordagens econômicas não foi afetada. A economia levou suas crenças e métodos pluralistas para o período do entre guerras. Quando a Grande Depressão trouxe um chamado urgente para ação econômica, essa pluralidade floresceu em uma variedade de análises e possíveis soluções para os problemas. Propostas de intervenção ou "planejamento" de muitos tipos diferentes se tornaram a moda dos anos 1930. Como os historiadores econômicos há muito reconhecem, nenhum conjunto consistente de políticas econômicas compunha o New Deal; mesmo dentro de cada agência, os objetivos econômicos muitas vezes estavam em desacordo uns com os outros. Isso refletia a variedade de abordagens de "planejamento" mantidas pelos indivíduos envolvidos, como Márcia L. Balisciano documenta neste volume. Embora visto como um fracasso considerável tanto na época quanto na maioria das contas modernas, o New Deal fez da "economia" uma responsabilidade importante para todos os governos americanos subsequentes. A criação associada de demanda por conselhos econômicos da esfera política e sua continuação ao longo do período são partes importantes de nossa história de transformação, e voltaremos a elas mais tarde neste ensaio.

\subsubsection{\textbf{As Mudanças nos Padrões da Economia Científica}}


Entender o pluralismo do período do entre guerras em termos do conceito de variedade não necessariamente nos dá uma compreensão do processo de transformação. Para não prejulgar exatamente qual poderia ser o resultado, vamos começar com uma caracterização muito ampla das mudanças na economia americana durante o período, considerando as mudanças no que significava ser "científico". Parece que na década de 1920 muitos tipos diferentes de economistas se consideravam "cientistas". De nossa perspectiva atual, o manto era amplo: um economista era um cientista investigativo, quer ele ou ela usasse os métodos da história, estatística, dedução teórica, empirismo, matemática, ou qualquer que fosse. Não havia hegemonia de método: qualquer método que pudesse ser apropriado para uma investigação particular, ou favorecido por um economista particular, poderia ser adotado. Isso não significa que todos esses métodos eram igualmente populares, como Roger E. Backhouse nos lembra em seu ensaio neste volume, pois nem a estatística nem a matemática eram um método popular durante o período. Nem significa que o rótulo "científico" era incontestável em relação aos métodos. Pelo contrário, era fortemente contestado.

Entre os institucionalistas, o conceito de ciência parece ter sido baseado em uma visão dos métodos da ciência natural como empíricos e experimentais. A abordagem quantitativa de Mitchell foi explicitamente modelada no que ele pensava ser a abordagem mais próxima dos métodos do cientista natural que era possível alcançar na economia. Mas outros institucionalistas não deram a mesma ênfase aos métodos quantitativos como fez Mitchell; Tugwell falou de economia experimental, e Copeland falou do ponto de vista da ciência natural. Embora as técnicas específicas de investigação utilizadas pelos institucionalistas variem, em todos os casos o objetivo era investigar as condições reais e criar uma teoria que se baseasse em suposições realistas e que pudesse abordar questões e problemas do mundo real. Os institucionalistas contrastaram esses métodos com o que eles viam como a natureza excessivamente abstrata de grande parte da teoria padrão como então existia. Isso não quer dizer que a maioria, ou mesmo muitos, dos economistas mais ortodoxos eram teóricos puros, mas sugerir que suas teorias de comportamento racional e mercados competitivos eram vistas pelos institucionalistas como baseadas em suposições altamente simplificadas e irreais que limitavam sua aplicabilidade e utilidade na resolução de problemas do mundo real.

As reivindicações institucionalistas sobre a natureza da investigação científica apropriada em economia ecoaram argumentos semelhantes sendo feitos em outras ciências sociais, e tais ideias impactaram na profissão de economia mesmo além das próprias fileiras dos institucionalistas. Um resultado disso foi a natureza altamente concreta e voltada para o problema da grande maioria do trabalho em economia, seja conduzido por institucionalistas ou não. Foi neste período que as áreas especializadas de economia do trabalho e organização industrial se desenvolveram, e grande parte do trabalho nessas áreas era de natureza altamente empírica e concreta. Diferenciar o institucionalista do não institucionalista com base no trabalho produzido nessas áreas é muitas vezes extremamente difícil. Mesmo nas áreas mais teóricas da disciplina, o efeito pode ser visto no desenvolvimento de teorias mais "realistas" de concorrência imperfeita, falhas de mercado e ciclos de negócios. Claro, a concepção institucionalista de uma economia científica não foi simplesmente concedida por todos os economistas, e contra-argumentos foram feitos, aparecendo mais frequentemente no final dos anos 1930 e 1940. Indivíduos com predileções teóricas mais ortodoxas, como Knight, lançaram algo como um contra-ataque, argumentando que os métodos da ciência natural não poderiam simplesmente ser trazidos para a economia sem modificação e que a ciência natural era mais teórica e abstrata do que estava sendo contestado (Rutherford 1997). Seu sucesso foi lento em chegar, mas particularmente após a Segunda Guerra Mundial, suas noções de ciência foram reforçadas por outros fatores que impactaram na disciplina, como discutido mais tarde. O ensaio de Ross B. Emmett neste volume esboça as mudanças na natureza do currículo de economia da Universidade de Chicago dos anos 1930 aos anos 1950, que refletiu um crescente ênfase na teoria neoclássica e uma mudança de cursos de campo orientados para o problema para cursos de campo orientados para métodos que enfatizavam a aplicação de métodos teóricos e estatísticos "centrais".

Embora o pluralismo do entre guerras tenha sido caracterizado por argumentos sobre o que constituía o conjunto correto de métodos científicos para a economia, isso não significa que não havia padrões científicos compartilhados. De fato, os economistas devem ter mantido alguns padrões compartilhados que permitiram que eles discutissem questões de método e ainda compartilhassem as mesmas plataformas e contribuíssem para as mesmas revistas. Olhando para trás a partir de hoje, temos a sensação de que vários tipos de padrões estavam operando no pluralismo do entre guerras. Primeiro, é revelador que as categorias naturais de hoje da ciência econômica, "teórica" e "aplicada", simplesmente não se encaixam bem neste período anterior, como Backhouse aponta. O fato de os economistas do final do século XIX até o período do entre guerras quererem descobrir sobre o mundo com um espírito científico não significava que eles usavam um único método de abordagem. Mas isso significava que a maior parte da economia era expressa em termos concretos em vez de abstratos, seja o tópico de importância específica ou geral.

Em segundo lugar, o status científico do trabalho estava mais associado às qualidades pessoais e atitudes do economista qua cientista do que a qualquer método particular usado. Isso é consistente com várias contas recentes da história da ciência sugerindo que, em vários momentos e lugares, fatores pessoais foram particularmente importantes para estabelecer reivindicações de objetividade científica. Daston (1995) descreveu esse tipo de objetividade como dependente de uma "economia moral": um conjunto de virtudes pessoais ou valores de investigação científica (honestidade, integridade, etc.). O respeito por essas qualidades e valores parece ter sido compartilhado por todos os economistas, sejam eles trabalhando na academia, no governo (como o Bureau of Agricultural Economics), ou nos novos institutos de pesquisa econômica financiados privadamente (como o National Bureau of Economic Research [NBER]).

Terceiro, a integridade dos economistas e o compromisso com um espírito de investigação científica não necessariamente significavam que eles perseguiam uma agenda científica livre de valores ou políticas. Tanto no período antes da Primeira Guerra Mundial quanto no período do entre guerras, os economistas poderiam se caracterizar como sendo científicos em sua abordagem para seu material, enquanto mantinham fortes valores e visões sobre os objetivos da economia por meio da política econômica. Furner (1975), em sua maravilhosa conta da economia americana durante o final do século XIX e início do século XX, associa objetividade com imparcialidade. Tornou-se o ethos profissional dos economistas do período ensinar ambos os lados de um caso: tanto o livre comércio quanto o protecionismo; padrão-ouro e bimetallismo; sindicatos e capitalismo. A profissionalização exigia imparcialidade. Mas essa própria demanda reconhece a existência de diferentes análises, com resultados diferentes, baseados em diferentes crenças e valores. A imparcialidade significava reconhecer diferenças de opinião, mas também significava rejeitar imparcialmente o seccionalismo em favor da promoção do interesse social. O interesse social, é claro, poderia ser definido de várias maneiras, e diferentes economistas poderiam ter diferentes posições políticas. Portanto, a imparcialidade não implicava necessariamente silêncio ou neutralidade sobre as opções de política disponíveis, e muitos economistas argumentaram fortemente por pacotes de reforma específicos. O economista poderia ser um defensor no domínio da política, mas apenas se suas visões fossem respaldadas por uma investigação científica adequadamente objetiva.

Os economistas do início do século XX compartilhavam um tipo de economia científica (mais frequentemente concreta do que abstrata), um compromisso moral de garantir os padrões de investigação científica e uma objetividade imparcial combinada com a defesa. O pluralismo foi apoiado, não comprometido, por esses padrões. Como essas características contrastam com os tipos de padrões científicos e objetividade que associamos ao neoclassicismo do pós-guerra? A economia neoclássica moderna dá por certo a objetividade no nível da investigação, mas o que é impressionante é que agora a objetividade é pensada para se estender ao nível das crenças e ao conselho de política. Dois processos transformadores provocaram essa mudança.

No primeiro processo, a noção de objetividade associada a um conjunto de atributos pessoais que garantiam os padrões da economia científica deu lugar a uma noção de objetividade investida em um conjunto particular de métodos, a saber, matemática e estatística. Esses eram métodos que poderiam ser pronunciados inequivocamente científicos com base no fato de que tinham que ser usados de uma maneira técnica, ou seja, não subjetiva. O desenvolvimento de métodos estatísticos no final do século XIX tem sido retratado como garantindo o tratamento "objetivo" dos dados econômicos (veja Gigerenzer et al. 1989), enquanto o desenvolvimento paralelo de métodos matemáticos na economia carregava um rótulo "objetivo" equivalente para análise teórica. Esses tratamentos técnicos de argumentos indutivos e dedutivos, e de maneiras de lidar com evidências, forneceram aos economistas uma aparente neutralidade. Economistas que podiam confiar em tais métodos técnicos não precisavam mais ser tão escrupulosamente imparciais ou depender tão inteiramente de suas virtudes. Essas abordagens técnicas criaram um novo tipo de expertise profissional que permitiu aos economistas oferecer conselhos de política "objetivos", pois eles poderiam argumentar que a objetividade de seus métodos garantia a objetividade dos resultados da análise e do conselho de política associado. Porter (1995) descreveu, em detalhes convincentes, o desenvolvimento americano da análise de custo-benefício durante este período de transformação para mostrar como a virada para a expertise técnica (regras de cálculo, fórmulas matemáticas e dados estatísticos) forneceu aos economistas uma defesa de sua análise contra ataques por aqueles que promovem agendas políticas ou aqueles com fortes valores opostos.

Esses "métodos objetivos" foram lentos para pegar na economia americana, com a economia agrícola, a análise estatística de Mitchell e o NBER, e a econometria emergente (incluindo a Comissão Cowles) provavelmente sendo as áreas mais fortes entre as guerras. Mas o desenvolvimento de expertise técnica, onde a credibilidade científica depende de métodos, foi um pré-requisito necessário para a aplicação por economistas de técnicas de resolução de problemas baseadas em matemática e estatística simples em vários departamentos governamentais durante a Segunda Guerra Mundial, como discutido no ensaio de Craufurd D. Goodwin neste volume.

Claro, a reivindicação de "métodos objetivos" funciona apenas quando diferentes economistas usando métodos semelhantes produzem as mesmas respostas. Quando eles não o fazem, toda a reivindicação dos economistas à objetividade científica é duplamente minada. Portanto, o desacordo sobre como medir e o que contar entre os economistas que trabalham em análises de custo-benefício para diferentes braços do governo significava, no caso de Porter, que a expertise técnica não poderia forçar o fechamento do debate político. Mas o próprio fato de que os métodos técnicos são debatidos tão veementemente fala de sua importância para as noções modernas de economia científica. O desacordo sobre o que constituía o conjunto correto de métodos estatísticos foi um elemento no famoso debate "medição sem teoria" entre o NBER e a Comissão Cowles que epitomizou a divisão entre o uso de institucionalistas e neoclássicos de tipos alternativos de métodos estatísticos no final dos anos 1940.

O segundo e igualmente importante processo de transformação foi a crescente fé na "solução de mercado" e nas virtudes da livre competição. Como vemos nos ensaios de Bateman e Anne Mayhew neste volume, essas crenças não podem ser tomadas como garantidas como parte da tradição americana. Pelo contrário, tais crenças não eram geralmente mantidas por economistas americanos do final do século XIX. Eles podem ter sido precipitados pelas falhas percebidas da intervenção econômica no New Deal. Mas foi apenas no período do pós-guerra que os economistas começaram a ver e retratar o mercado livre, a competição perfeita e os direitos econômicos individuais como, em si mesmos, incorporando verdades objetivas e livres de valores por razões que discutiremos mais tarde. Assim, a primazia da eficiência econômica como valor orientador e a possibilidade de separar valores econômicos de outras considerações foram ambos desenvolvimentos do pós-guerra entre os economistas americanos, de acordo com a discussão de Steven G. Medema neste volume no contexto da relação de direito e economia. Em contraste, a velha ideia, dominante no período pré-Primeira Guerra Mundial e mantendo-se no período do entre guerras, era que há uma inter-relação necessária, de fato interdeterminação, de instituições legais e econômicas. As instituições incorporam valores econômicos e os valores, por definição, não podem ser livres de valor.

Pode ajudar a enfatizar o que não estamos argumentando aqui. O fato de que os compromissos éticos do período pré-Primeira Guerra Mundial saíram de moda (por causa do fracasso de sua ideologia de apoio) não significa que uma economia neoclássica "livre de valores" e técnica necessariamente preencheria o vácuo. A economia neoclássica moderna não foi a única resposta possível. O afrouxamento da ideologia ética permitiu que o compromisso com o interesse próprio tivesse maior liberdade. Isso poderia ter significado um recuo para a economia clássica do laissez-faire, também apresentada como livre de valores por seus proponentes, mas não mais técnica do que a economia histórica do século XIX. No entanto, a declaração original de princípios da AEA foi formulada exatamente para excluir os economistas clássicos de estilo antigo, como William G. Sumner. Essa opção foi descartada. Outro possível resultado desses padrões em mudança poderia ter sido uma mudança para a economia empírica estatística, uma maneira de nossos dois processos implicarem uma economia mais técnica e livre de valores para se conectar. De fato, tal imagem se encaixa na economia de Mitchell e NBER pelo menos até os anos 1950. A economia neoclássica não era a única opção viável, e demorou algum tempo para pegar.

O que emerge dos dois processos que descrevemos é que, na era do pós-guerra, os economistas adotaram cada vez mais métodos que conferiam a garantia de objetividade aos resultados de sua análise e a seus conselhos de política. Ao mesmo tempo, eles aprenderam a apresentar certas crenças econômicas como livres de valores e, portanto, objetivas. Por razões que se tornarão mais claras, o ethos profissional da economia mudou. Neste novo ethos da economia, tornou-se moda oferecer conselhos de consenso, em forte contraste com os conselhos contrários que os economistas ofereceram no New Deal. O economista se tornou o cientista profissional neutro, oferecendo conselhos especializados e livres de valores em uma linguagem que o público pudesse entender. Ao mesmo tempo, as disputas profissionais internas começaram a ser expressas em uma linguagem técnica separada.

\subsubsection{\textbf{Economistas e a Economia}}
Como é amplamente pensado que o método matemático e as crenças neoclássicas são inseparáveis, pode ser natural que as mudanças em direção a ferramentas quantitativas e crenças neoclássicas ocorreram juntas. Essas duas mudanças podem ter sido simultâneas, e possivelmente foram impulsionadas pelos mesmos fatores causais, mas os ensaios neste volume sugerem que os fatores causais operaram de maneiras diferentes.

Um fator que já discutimos é o contexto histórico econômico. A Grande Depressão, o evento chave do período de 1920 a 1960, exigiu que os economistas enfrentassem o desafio de diagnosticar e tratar a doença na economia. É difícil imaginar uma economia em que a produção caiu de 25 a 30 por cento e em que o desemprego atingiu um nível de 25 por cento e não retornou ao seu nível de 1929 até 1942 (apesar do estímulo à demanda dos aliados em tempo de guerra). Mesmo os economistas mais laissez-faire teriam dúvidas sobre a eficácia da solução de mercado. Mas essa não foi a única depressão do período do entre guerras. Uma depressão muito súbita e severa ocorreu em 1921-22. E durante os anos de guerra, a maioria dos economistas estava convencida de que após o fim da guerra, a economia retornaria à depressão dos anos 1930 ou então cairia em uma queda repentina, como após a Primeira Guerra Mundial. Em tempo de guerra, é claro, a economia exigia um planejamento considerável. Nessas circunstâncias, não é surpresa descobrir que, no geral, os economistas permaneceram pró-intervencionistas até a década de 1940.

Esses problemas históricos na economia não apenas levaram os economistas em direção à intervenção, mas também criaram a demanda por seus serviços para fazer planos concretos e sugestões para os quais as novas ferramentas técnicas de modelos matemáticos simples e técnicas estatísticas eram bem adaptadas. Não era que as antigas ferramentas não pudessem fornecer respostas ou que não fossem ferramentas técnicas. Os economistas estavam acostumados a fornecer respostas específicas para perguntas concretas. A regulamentação ferroviária dependia há muito tempo da análise econômica das tarifas, e os economistas agrícolas estavam acostumados a medir e manipular o setor agrícola. Não era nem mesmo que a Grande Depressão foi imediatamente considerada como algo diferente, exigindo novas soluções. O problema era visto como massivo, mas não novo. Foi apenas no New Deal que a demanda por soluções econômicas se ampliou: todos os aspectos da economia se tornaram abertos ao ataque econômico. Economistas americanos de todas as tendências responderam. Charles Roos, um dos econometristas ativos nos primeiros anos da Comissão Cowles, tornou-se economista-chefe de pesquisa na Administração Nacional de Recuperação (NRA), construindo modelos matemáticos e estatísticos de competição industrial. Mordecai Ezekiel, econometrista agrícola, e Tugwell, institucionalista, ambos se envolveram no planejamento geral. Mas, em geral, os economistas não consideraram o New Deal uma experiência bem-sucedida; não melhorou suas reputações. E na medida em que os economistas institucionalistas estavam envolvidos nesses esquemas, eles, juntamente com os outros economistas, compartilharam o fracasso.

Apenas nos anos de guerra, e somente depois que os Estados Unidos entraram na guerra, a economia recuperou sua antiga força. A guerra, como o ensaio de Goodwin discute, foi um divisor de águas em vários aspectos. Os economistas não apenas encontraram sua expertise técnica útil para tomar decisões sobre como lidar com as escassez econômicas (em vez de excesso de oferta como na Grande Depressão), mas também voltaram suas técnicas para várias questões de guerra, usando modelos matemáticos simples de otimização, técnicas de programação linear e dispositivos de medição estatística. Os economistas foram trazidos para lutar diretamente na guerra, planejando o design ideal de ataques aéreos e analisando estatisticamente os padrões de disparo. Os economistas descobriram que, ao usar a economia de conjunto de ferramentas e a expertise técnica neoclássica em desenvolvimento, eles poderiam responder perguntas em campos muito diferentes. A economia emergiu da guerra coberta de glória, talvez lançando o "imperialismo econômico" nas ciências sociais ao longo do último meio século.

No mundo do pós-guerra, como Goodwin mostra, as tecnologias neoclássicas nascentes continuaram a se provar úteis, mas note que isso era economia neoclássica de conjunto de ferramentas, formulada para responder a perguntas claramente especificadas e bem definidas, não teorizando o equilíbrio geral grandioso. O conselho permaneceu no nível da microeconomia básica, tanto durante quanto após a guerra, e pode não ter sido muito diferente do conselho anterior oferecido por economistas no Bureau of Agricultural Economics ou nas administrações do New Deal. Esta era a economia de Hadley (o Marshall americano) em vez do neoclassicismo sofisticado de Paul Samuelson. Assim, embora para esta nova geração os tipos concretos de perguntas investigadas e respondidas possam não ter sido diferentes das investigadas por economistas anteriores, os métodos se tornaram mais orientados tecnicamente e o papel do economista mudou. Ele (na maioria das vezes) ofereceu respostas, mas sem a defesa que acompanhava o período anterior. Essas respostas eram naturalmente "corretas" porque eram o resultado de métodos "objetivos" e porque nesse estágio os economistas estavam começando a rejeitar a intervenção e a se voltar para seu novo amor: a crença da economia neoclássica no mercado, na competição e na primazia do indivíduo auto-interessado.

O momento dessa mudança é algo enigmático. Por que os economistas se apaixonaram completamente pelo mercado e se desapaixonaram pelo controle e intervenção justamente quando se tornaram bem-sucedidos na prática deste último em resposta aos eventos da história econômica e política? Pode ser uma questão de seleção: o novo tipo de expertise em economia fornecida pelos métodos de análise do conjunto de ferramentas neoclássicas foi melhor adaptado às demandas criadas por este conjunto de eventos. De acordo com tal argumento, é por causa do sucesso de suas ferramentas que os economistas passaram a acreditar nas ideias por trás delas. Isso certamente é uma inversão interessante da história interna normal da economia que retrata as ideias ("pensamento") como a luz principal em qualquer conta. Esse argumento invertido também é consistente com a afirmação geralmente feita para a Grã-Bretanha de que o sucesso das ferramentas e conceitos keynesianos usados na condução da economia de guerra encorajou a crença dos economistas e políticos naquele sistema de ideias e levou à sua popularidade no período do pós-guerra. A experiência de guerra da intervenção ativa baseada em ferramentas é um fator causal importante na transformação, mas é apenas parte da história.

\subsubsection{\textbf{Economistas, Guerra e Sociedade}}

Pode parecer perfeitamente razoável que uma sociedade cansada de guerra e depressão e encantada com o boom do pós-guerra reaja abraçando os objetivos de mercados livres e competição saudável. Essas visões podem levar algum tempo para surgir, dado as memórias ainda vívidas da Grande Depressão, e podem levar igualmente muito tempo para se fixar na mente dos economistas. Uma reivindicação mais coerente, e que surge em vários dos artigos, se relaciona com a guerra fria, uma guerra, Goodwin nos lembra, de ideologias econômicas. Este fator explicativo tem a virtude de se encaixar em nosso quebra-cabeça de tempo, pois, no momento certo, a sociedade americana se moveu solidamente a favor das virtudes dos mercados livres e da competição aberta. Ao fazer isso, reforçou, em um ponto crítico logo após a guerra, o sistema de crenças neoclássico. Encontramos este argumento mais coerente porque, neste caso, o argumento "razoável" de reação implica que as sociedades europeias, mais sobrecarregadas pelos controles econômicos de tempo de guerra, abraçariam o mercado livre com mais entusiasmo do que os Estados Unidos, mas eles não o fizeram. Além disso, a guerra fria não estava quase tão congelada na Europa, onde o planejamento de reconstrução e o estatismo de bem-estar evoluíram para a economia mista e o efeito sobre as ideias dos cientistas sociais foi menos dramático.

O momento em que os valores da sociedade se alinham com os dos economistas é um ponto a ser observado. Pouco antes da Primeira Guerra Mundial, o Credo Social foi aceito pelas igrejas protestantes mainstream, a sociedade se alinhou atrás da economia da geração fundadora da AEA, e seu programa parecia pronto para ser posto em prática. Como Bateman nos diz, o movimento do Evangelho Social forneceu aos economistas uma linguagem e uma oportunidade para falar com a comunidade mais ampla e apoiou um pluralismo de economia sob o guarda-chuva ético. O momento foi temporariamente perdido por causa da virada nacionalista tomada pelas igrejas em resposta à Primeira Guerra Mundial. Somente no New Deal os economistas tiveram novamente a oportunidade de falar com convicção, e um pluralismo semelhante ocorreu então porque as ideias e abordagens eram aquelas do período pré-1920, trazidas em parte por uma nova geração. É uma das ironias de nossa história que, assim que algo próximo ao Credo Social das igrejas se tornou politicamente bem apoiado e política econômica oficial no final da Segunda Guerra Mundial, apareceu uma nova forma de nacionalismo. (Uma comparação dos episódios revelados nos ensaios de Bateman e Balisciano é instrutiva.) Neste segundo momento, o nacionalismo da guerra fria empurrou o compromisso da sociedade com a liberdade econômica bem à frente do corpo principal dos economistas americanos. No clima daqueles tempos, os economistas acharam mais seguro se conformar, e a economia neoclássica em desenvolvimento, que incorporava valores semelhantes, recebeu um grande impulso.

Os efeitos do macartismo certamente não podem ser ignorados em qualquer relato da transformação da economia americana. Embora tenhamos registros de economistas emigrados que chegaram à América para escapar do fascismo na década de 1930 (veja Hagemann e Krohn 1991, Craver 1986), temos pouco mais do que anedotas de economistas que deixaram a América para evitar a perseguição anticomunista. Alguns destes, é claro, tinham visões que poderiam ser comunistas, ou pró-socialistas, mas até mesmo o keynesianismo era uma heterodoxia suspeita, e aqueles que haviam defendido o planejamento para a economia do pós-guerra alguns anos antes, sem falar nos fervorosos New Dealers, poderiam muito bem ter se encontrado excluídos. Alguns desses economistas retornaram, outros não, mas sua ausência foi certamente um dos fatores que contribuíram para um estreitamento do pluralismo anterior da economia americana no período do pós-guerra.

O resultado dessas pressões sobre aqueles que permaneceram foi tanto para estreitar a gama de crenças quanto para restringir as maneiras aceitáveis de expressá-las. O ensaio de Goodwin sugere que esse estreitamento foi alcançado em parte através de uma virada para uma maior tecnicidade e para as linguagens aparentemente "neutras" da matemática. Como já observamos, essa defesa técnica do pós-Segunda Guerra Mundial por si só não implica necessariamente na economia neoclássica. Embora o keynesianismo possa ter sido considerado perigosamente próximo do marxismo, um diagrama IS/LM provavelmente parecia inofensivo para um estranho, e números estatísticos como os de Mitchell há muito mantinham seu próprio status neutro como "dados". A economia expressa em geometria, álgebra ou números poderia ser uma boa autodefesa nos dias da guerra fria e passar no teste na sala de aula, bem como no governo. De fato, em nosso quadro mais geral, essa mudança para métodos técnicos foi precisamente a mudança que Porter (1995) sugere que tornou a economia defensável contra o poder democrático, quem quer que o exercesse de qualquer lado. A guerra fria reforçou, se não criou, a tendência dos economistas em oferecer expertise profissionalmente neutra e objetiva, que contrastava fortemente com a defesa ética e fortemente mantida do economista profissional do final do século XIX. Mesmo em seu modo "imparcial", as declarações públicas do final do século XIX ofereciam considerável munição política em comparação com o jargão de especialistas e o estilo de conjunto de ferramentas da economia do pós-guerra, que poderia ser usado para disfarçar o conteúdo teórico e a ideologia para o mundo exterior.

Embora essa mudança para a matemática tenha sido em parte uma defesa autoimposta realizada por indivíduos (veja Johnson 1977), também foi incentivada por instituições acadêmicas que buscavam professores "seguros" e institutos de pesquisa que buscavam pesquisadores "aceitáveis". Como o ensaio de Goodwin sugere, os patronos da pesquisa econômica exigiram uma obrigação de correção política em linha com a guerra fria que teve o efeito de estreitar as visões que poderiam ser expressas e contar dentro das correntes principais da economia, sejam elas acadêmicas ou governamentais. Há alguma sugestão de que patronos e economistas conspiraram para esconder ideias radicais do público, pois as instituições, tanto quanto os indivíduos, buscavam segurança no clima de repressão política. Em outros lugares houve guerra aberta. Um dos poucos estudos de caso dos efeitos acadêmicos da guerra fria, de Solberg e Tomilson (1997), descreve eventos no Departamento de Economia da Universidade de Illinois. Lá, o "macartismo acadêmico" expulsou tanto os keynesianos nascentes quanto aqueles que defendiam a economia moderna baseada em ferramentas, uma combinação que incluía Margaret Reid, Leonid Hurwicz, Dorothy Brady, Robert Eisner, Don Patinkin e, finalmente, Franco Modigliani. Independentemente do processo, o efeito foi o mesmo: tanto a perseguição aberta quanto a correção de armário levaram ao estreitamento da opinião econômica permissível.

A retirada não foi total, pois, embora ser um seguidor de Keynes fosse um rótulo duvidoso para os economistas americanos na década de 1950, o keynesianismo poderia ser compatível com a economia de mercado no ambiente americano. Isso foi realizado tanto persuadindo os empresários de que eles poderiam se sair melhor sob um governo que levasse a macroeconomia a sério (como Balisciano sugere) quanto traduzindo o keynesianismo para a mesma forma técnica que a economia neoclássica, o primeiro passo para a "síntese neoclássica-keynesiana" americana. De maneira semelhante, o monetarismo institucionalista à moda antiga foi compatibilizado com o neoclassicismo acadêmico americano, como Perry Mehrling relata neste volume. Os economistas poderiam discutir com segurança as mesmas coisas antigas, mas seu debate se tornou um argumento técnico interno, não aberto ao olhar público. Os debates públicos da década de 1930 se tornaram disputas técnicas internas nas décadas de 1950 e 1960.

É claramente difícil, sem muito mais pesquisa, avaliar o impacto do macartismo na transformação da economia americana. No entanto, a pergunta contrafactual "Qual teria sido a história da economia americana sem a guerra fria?" indica as respostas potenciais para nossos enigmas sobre o tempo e sobre o grau de crença na eficácia do mercado. Lembre-se, o enigma é que, embora os institucionalistas fossem fortemente evidentes como um agrupamento dentro do pluralismo geral da década de 1930 (mas falharam em aproveitar a vantagem oferecida por esta posição para se unir em torno de um único programa), o agrupamento neoclássico realmente não se tornou evidente até a década de 1950 (veja Rutherford 1997). A mudança relativamente repentina da crença institucionalista (de fato geral) na intervenção governamental para a crença mais neoclássica no mercado livre pode ser explicada quando consideramos seriamente as visões da sociedade americana na guerra fria. O reforço da virada técnica na economia (e não apenas na economia neoclássica) surge como uma consequência importante, embora claramente não intencional, do mesmo fator causal, ou seja, o clima político.

Assim, uma transformação para uma versão de conjunto de ferramentas da economia em geral e para as crenças da economia neoclássica em particular (que era apenas uma das muitas vertentes do período entre guerras), foi reforçada e impulsionada pelos valores sociais da época, de modo que até 1960 o neoclassicismo americano estava bem estabelecido. A extensão em que essas mudanças são evidentes nas revistas nos dá algum respaldo para nossa conta. O ensaio de Backhouse sugere que a economia formalmente expressa estava em aumento a partir da década de 1930, mas a verdadeira mudança veio depois de 1945. A econometria empírica se desenvolveu um pouco mais tarde, com a verdadeira mudança ocorrendo após 1950. Assim, mais uma vez, a guerra foi um divisor de águas: depois desse tempo, vemos tanto o estilo moderno quanto as categorias de pesquisa modernas começando a surgir com força considerável, concomitantemente com a mudança dos economistas para uma abordagem tecnocrática autodefensiva.

\subsubsection{\textbf{O Novo Estilo e Suas Implicações}}

Aqueles que encontram a economia neoclássica moderna desconectada da economia do mundo real tendem a culpar o crescente formalismo de alguma forma ou de outra por esse estado de coisas e a culpar a profissão por essa mudança. Mas, em vez disso, como argumentamos, o formalismo deve primeiro ser visto como o resultado de várias contingências externas, não a causa de internas. Já discutimos como a demanda por economistas para resolver problemas de política levou ao uso crescente de ferramentas técnicas tanto na Grande Depressão quanto, mais particularmente, na guerra. Descrevemos isso como economia de conjunto de ferramentas em vez de economia neoclássica, pois é importante lembrar que nem todas as ferramentas estavam associadas à economia neoclássica; nem todos os econometristas eram economistas neoclássicos; e o planejamento de qualquer tipo exigia números e trabalho estatístico em larga escala. Também discutimos a maneira pela qual essa virada técnica foi fortemente reforçada pela necessidade de autodefesa dos economistas e patronos durante a guerra fria. Aqui, não era tanto o que as ferramentas técnicas fariam por você, mas sim que a linguagem da matemática e estatística parecia ser mais neutra e objetiva, e mais difícil para o leigo e o político, deixando os economistas menos abertos a ataques externos sobre questões de crença. O formalismo, portanto, ofereceu aos economistas tanto ferramentas para uso prático quanto linguagem neutra para expressão e argumento profissional seguro.

Mas existem efeitos de segunda rodada e mais sutis dessas mudanças que talvez sejam melhor compreendidos através do estudo de caso que Mehrling nos dá, e é por isso que o formalismo realmente carrega parte da culpa pelo declínio do pluralismo. Em sua conta de disputas dentro da economia monetária, Mehrling argumenta que a adoção conjunta de crenças neoclássicas e expressão matemática criou uma espécie de walrasianismo monetário dentro do qual os argumentos monetários não eram mais questões de crença, ou mesmo de evidência empírica, mas questões técnicas de modelagem. Podemos entender a partir deste exemplo exatamente como as mudanças na linguagem e na forma em que a economia foi expressa restringiram o que poderia ser dito e achataram o que poderia ser questionado. Os verdadeiros argumentos subjacentes sobre o dinheiro podem permanecer, mas eles não poderiam mais ser expressos de forma plena e explícita dentro do novo tipo de economia formalizada. Este processo funcionou integrando descobertas empíricas desconfortáveis ou teorias de outros tipos de economia no quadro formal de explicações neoclássicas. Esta é uma maneira de interpretar o destino da tese de preços administrados de Gardiner Means, discutida por Lee (1997), e os desenvolvimentos na teoria de custos discutidos por Naples e Aslanbeigui (1997). Em todos esses exemplos, a transformação em economia formal envolveu mudanças na linguagem, forma e ferramentas. Este novo estilo se tornou um conjunto de costumes que reduziu em si mesmo a possibilidade de pluralismo dentro da economia.

Outro fator contextual conectado precisa ser trazido: o papel da matemática na economia. Ao fornecer o contexto intelectual para o trabalho do economista matemático inicial G. C. Evans, o ensaio de E. Roy Weintraub neste volume discute a relação em mudança entre matemática e ciência. Este defensor inicial da economia matemática, juntamente com muitos daqueles comprometidos com o movimento de econometria no período entre guerras, acreditava que usar com sucesso a matemática significava conectá-la ao mundo econômico, uma visão atual nas ciências no final do século XIX. No meio do século XX, os economistas eram mais propensos a serem cativados por uma virada na própria matemática que via a matemática como uma maneira de escrever teorias consistentes. A matemática se tornou a linguagem para a expressão de teoria abstrata e geral, em vez de uma ferramenta para descobrir e escrever descrições verdadeiras do mundo econômico. Assim, por volta do momento em que o corpo principal de economistas americanos começou a preferir a linguagem da matemática em terreno defensivo, o papel científico da matemática estava se libertando de suas conexões com o mundo científico (econômico).

Se os primeiros economistas matemáticos em geral compartilhavam as visões de Evans, então a mudança nas percepções sobre a matemática na ciência ajuda a explicar tanto a formação quanto o colapso do movimento econômico, concebido no período entre guerras como a integração de matemática e estatística na economia. Se as descrições matemáticas devem se conectar ao mundo, elas precisam vincular o observável ou mensurável ao hipotético. Por volta de 1950, esse sonho da tradição econômica de que a teoria econômica matemática tinha que corresponder a algo observável (mesmo que não sejam dados estatísticos reais) desabou, e a econometria se dividiu em economia matemática e econometria como a conhecemos hoje. É pertinente que outro elemento do debate "medição sem teoria" do final dos anos 1940 entre institucionalistas e neoclássicos se concentrasse no papel correto da matemática na economia. Sob o compromisso mais antigo com uma matemática que se conecta rigorosamente ao mundo, não teria sido tão fácil para os neoclássicos usar a matemática para consolidar sua posição contra os institucionalistas, pois, como Mehrling sugeriu (durante discussões do ensaio de Weintraub), o institucionalismo americano compartilhava alguns dos preconceitos metodológicos da matemática do século XIX. A tradição mais recente da matemática do século XX se encaixa melhor com os preconceitos da teoria neoclássica, e assim a mudança no papel da matemática em relação à economia tendeu a apoiar a hegemonia neoclássica emergente. O declínio do institucionalismo e a ascensão da economia neoclássica foram moldados pela evolução da matemática em relação às ciências.

Usamos o termo "estilo" para descrever as diferenças implícitas na economia americana à medida que emergiu através do mundo da guerra fria das décadas de 1950 e 1960. Nosso termo estilo envolve, em primeiro lugar, uma linguagem, uma que restringiu o que poderia ser dito, uma linguagem não tão intimamente conectada ao mundo ou facilmente acessível, exceto para economistas profissionais e seus alunos. O termo poderia ser estendido para a ideia de uma "economia de laboratório" (sugestão de Bateman) na qual as ferramentas de modelagem matemática e econometria estatística, embora não mais totalmente integradas, poderiam ser reunidas em uma aliança mais tenuamente conectada ao mundo. (Aqui devemos notar que o termo "lab" também foi usado como um rótulo para o treinamento de "teoria aplicada" de nível profissional que a Universidade de Chicago desenvolveu nesse momento.) Essas mudanças são epitomizadas no desenvolvimento da teoria do consumidor, descrita neste volume por Philip Mirowski e D. Wade Hands. Aqui temos algo como a economia neoclássica americana arquetípica, preocupada com quebra-cabeças formais criados pelo uso de representações matemáticas, usando matemática para expressar preocupações e se aventurando, mas não muito longe, em dados estatísticos. Dentro disso, o desacordo técnico foi expresso em debate de nível profissional: o pluralismo da economia neoclássica. No entanto, nada depende diretamente do argumento - nenhum subsídio de renda será concedido ou abolido, embora talvez alguém consiga uma bolsa para fazer mais pesquisas!

Mas a noção de estilo envolve mais do que linguagem e formas, pois o ideal de laboratório também era prático com ferramentas práticas. De fato, a tradição institucionalista vive na economia aplicada americana, que tem uma reputação de cuidado e meticulosidade em suas investigações empíricas. As noções de Weintraub sobre os dois papéis da matemática também, ele sugere, podem ser vistas como subjacentes aos nossos rótulos modernos de economia teórica versus aplicada, categorias que se tornam coerentes na pesquisa de Backhouse apenas no período do pós-guerra. Assim, ao separar o estilo das descrições gerais de neoclassicismo versus pluralismo, faríamos bem em nos lembrar mais uma vez que isso não é uma história do pensamento, mas uma história de uma disciplina, com pessoas e instituições oferecendo serviços e outras pessoas e instituições exigindo esses serviços. Os patronos queriam que os economistas fossem capazes de resolver problemas reais (não quebra-cabeças matemáticos, teoria de alto nível ou temas históricos) em um estilo profissional (ou seja, especialista). Eles queriam ciência econômica utilizável, não algo esotérico, fosse chamado de neoclassicismo ou keynesianismo. Isso torna o resultado algo mais do que uma mudança de linguagem e a adoção de certas formas e ferramentas, algo mais semelhante a uma mudança de abordagem ou estilo.

A economia que emergiu foi uma em que os economistas aprenderam a dividir o problema em algo pequeno o suficiente para que pudesse ser resolvido, mas ainda realista o suficiente para que as pessoas pudessem se relacionar com ele. A mudança de estilo não envolveu nenhum grande novo compromisso metodológico ou teórico, mas foi uma parte importante da mudança de face da economia americana durante o período. E embora tal mudança de abordagem não necessariamente exigisse quaisquer mudanças iniciais nas crenças, ela teve implicações para as crenças a longo prazo. Como Mehrling descreve, uma continuidade inicial de ideias pode, através de uma mudança nos métodos de expressão, gradualmente provocar mudanças no conteúdo e nas crenças.

\subsubsection{\textbf{O Declínio do Institucionalismo}}

Nossa concentração na importância dos contextos da história econômica, política e científica tem se concentrado principalmente em por que a economia formal e neoclássica se fortaleceu com o curso dos eventos, mas muito do que foi dito também toca no declínio relativo do institucionalismo. Vale a pena reunir alguns dos aspectos dessa história, pois se aplica ao institucionalismo, particularmente porque vários dos ensaios nesta coleção fornecem insights sobre os detalhes dos processos envolvidos.

Como Goodwin mostra, o declínio do institucionalismo não foi rápido, e mesmo até 1948 a economia ainda era pluralista, oferecendo teoria abstrata, alta empiria e estudos institucionais. No entanto, em termos gerais, os mesmos fatores que gradualmente fortaleceram o neoclassicismo tiveram um efeito oposto no institucionalismo, de modo que a economia institucional, tão importante no período entre guerras, começou a diminuir à medida que a economia neoclássica se fortalecia. Isso, no entanto, é muito uma visão geral, e devemos nos lembrar da diversidade dentro do institucionalismo. Diferentes partes do movimento foram impactadas de maneiras diferentes e em momentos diferentes.

Os conceitos em mudança de ciência e de objetividade científica são casos em questão. No período entre guerras, o NBER mostrou uma boa combinação das duas noções de objetividade discutidas anteriormente. A objetividade como imparcialidade e como reconhecimento de diferenças de opinião pode ser vista na nomeação de diretores do NBER que representavam diferentes pontos de vista e diferentes constituintes. A objetividade como técnica está incorporada nos métodos quantitativos projetados para estabelecer fatos de maneira imparcial e na determinação de Mitchell de manter uma clara separação entre a busca de fatos científicos e o uso de fatos na defesa de políticas. O desejo de Mitchell de separar a ciência da defesa não era compartilhado por outros institucionalistas, e a diferença entre Mitchell e outros institucionalistas sobre essa questão pode explicar a capacidade do programa de Mitchell no NBER de manter sua posição relativamente alta por tanto tempo quanto fez. Em contraste, Hamilton sofreu ataques por suas propostas para a indústria do carvão, o que pode ter afetado sua posição na Brookings Graduate School (Ross 1991, 417), e ao longo de sua carreira, Commons estava no centro da controvérsia política. Tal postura profissional era exatamente o que Goodwin sugere que teria sido em desacordo com o novo estilo científico cada vez mais exigido por agências de financiamento e outros consumidores de análise econômica no período do pós-guerra.

Outros aspectos do destino da escola de institucionalismo de Wisconsin e do "velho" estilo de direito e economia são examinados neste volume por Jeff Biddle e por Medema. Biddle examina a hipótese de que os graduados da Universidade de Wisconsin entraram no governo em maior número do que os graduados de escolas mais ortodoxas, de modo que o institucionalismo ao estilo de Wisconsin falhou em se reproduzir dentro do mundo acadêmico. Isso pode ter sido um reflexo do viés ideológico e do treinamento específico fornecido por Commons e outros em Wisconsin. Biddle encontra apenas apoio limitado para sua hipótese, mas é certamente verdade que os alunos de Commons fizeram pouco para avançar seu esquema conceitual envolvendo o sistema jurídico e econômico e foram mais atraídos por seu trabalho em economia do trabalho e por alguns de seus esforços específicos de reforma. Os alunos de Commons replicaram seu tipo de trabalho mais concreto e orientado para o problema, um tipo de trabalho que gradualmente perdeu terreno para a economia de conjunto de ferramentas. Nesse aspecto, imagine como o tipo de institucionalismo de Commons em toda a sua complexidade era difícil de ensinar e aprender. Não parecia técnico (não era quantitativo), mas dependia de estudo detalhado, conhecimento de direito e economia, compreensão de personalidades e situações, experiência em mediação e integridade pessoal - um conjunto de habilidades que contrasta bastante com o que foi transmitido pelos workshops de teoria aplicada desenvolvidos na Universidade de Chicago e descritos por Emmett.

O tipo de direito e economia de Commons-Hamilton, retratado por Medema como multifacetado, pluralista em relação aos métodos, interdisciplinar e baseado em um conceito de direito e economia como mutuamente determinado e determinante, estava em desacordo com a mudança de temperamento, de modo que o velho tipo de direito e economia estava em sério declínio mesmo antes do desenvolvimento do novo tipo de direito e economia em Chicago. O novo tipo de direito e economia, com suas visões muito diferentes sobre competição, antitruste e outras questões de política, refletiu e ajudou a avançar a mudança ideológica da profissão para longe da agenda de reforma dos institucionalistas.

Outras seções do movimento institucionalista foram prejudicadas pelas experiências da Grande Depressão e do New Deal. Este é particularmente o caso entre institucionalistas como Tugwell e Means, que avançaram a visão "estruturalista" ou pró-planejamento durante a fase inicial do New Deal. Como Mayhew e Balisciano deixam claro, o resultado da experiência do New Deal foi não apenas uma mudança para longe do planejamento estrutural e para um estilo keynesiano de política macroeconômica, mas também uma mudança para longe de uma abordagem regulatória para a indústria e para uma postura mais pró-competitiva envolvendo a aplicação dos atos Sherman e Clayton.

Esses desenvolvimentos, envolvendo uma virada para longe do planejamento e regulação e em direção ao mercado e competição como instrumentos de controle, só poderiam ser reforçados pelo impacto ideológico da guerra fria.

\subsubsection{\textbf{Conclusão}}
Implícito nesta conta da transformação da economia americana está que o declínio do pluralismo na economia americana não foi nem um resultado simples nem óbvio do desenvolvimento da economia neoclássica e vice-versa. Nenhuma relação lógica diz que isso deve ter sido assim, nem as evidências apoiam tal história causal direta. Também tentamos evitar basear nossa conta da transformação em duas outras posições polares.

Uma é a conta de progresso Whiggish. Não está claro que as evidências possam apoiar uma história de que o neoclassicismo venceu porque ofereceu melhor teoria e melhores explicações. Argumentar que tal resultado era inevitável, que a economia neoclássica oferecia uma "ciência melhor", entra em conflito com nossa afirmação de que as noções em mudança de ciência e objetividade científica faziam parte do processo transformatório. Como argumentamos, a economia neoclássica cresceu em dominância à medida que a noção de ciência mudou e os dois desenvolvimentos estavam conectados. Sendo esse o caso, não havia critérios internos estáveis nos quais oferecer um julgamento histórico sobre "progresso".

A outra conta que evitamos é a teoria da conspiração, na qual economistas neoclássicos em posições de poder se uniram contra o heterodoxo. Como essa visão assume que a ciência heterodoxa poderia ter vencido a batalha, mas pelo poder social dos economistas neoclássicos, a conta implica de maneira semelhante que um grupo tem reivindicações historicamente mensuráveis para ser uma "ciência melhor". Isso é tão problemático quanto antes.

Além disso, ambas essas posições polares são baseadas em alguma dualidade reconhecível de grupos institucionalistas (ou heterodoxos) e neoclássicos que acreditamos não poder ser identificados no período entre guerras. Isso não é negar que se possa apontar economistas institucionalistas e neoclássicos individuais ou que eles tinham diferenças de opinião, mas sugerir que é difícil fazer a teoria do progresso ou da conspiração funcionar, porque o pluralismo do período entre guerras corta crenças individuais. A guerra foi um divisor de águas no qual o processo de transformação se resolveu repentinamente, de modo que depois desse tempo podemos começar a falar sensatamente em termos de tais grupos.

Também não queremos negar que houve batalhas individuais ou que o poder institucional importava para os resultados. Mas tais jogos de poder ocorreram dentro de estruturas envolvendo patronos e hierarquias operando dentro do contexto de uma sociedade política e econômica que apoiava pedidos de intervenção econômica no período entre guerras e por mercados livres no período do pós-guerra. Essas crenças econômicas e políticas mais amplas nunca são apenas cenários contra os quais indivíduos (e instituições) lutam guerras científicas; eles fornecem conteúdo para o debate e são elementos integrais em qualquer luta pelo poder.

Ao buscar fornecer uma conta geral da transformação na economia americana, nos concentramos em fatores explicativos nos quais as histórias individuais poderiam ser colocadas. Nosso objetivo principal tem sido fornecer uma conta da transformação consistente com o tempo e o caráter das mudanças sugeridas nos ensaios individuais neste volume: cada um tem sua própria história para contar, com circunstâncias contingentes separadas. Ao fazer uma conta na qual todos eles se encaixam com facilidade, nos concentramos nas contingências do mundo exterior: científico, político e econômico. Estes formaram a base de nossas explicações. Foi esse mundo que criou as circunstâncias às quais os economistas americanos se adaptaram e dentro das quais sua economia foi transformada.

A história da transformação está longe de ser encerrada. Muitos buracos devem ser preenchidos e muitas partes de nossa conta permanecem especulativas, exigindo uma pesquisa histórica substancial para transformá-las em história documentada. Qualquer força histórica que exista em nossa conta é extraída dos ensaios neste volume; as especulações e erros permanecem nossos.

\section{\textbf{Weintraub (2014)}}
\subsection{\textbf{Introdução: Contando a História da Economia do MIT no Período Pós-Guerra}}
Em 3 de outubro de 1940, o mentor de doutorado de Paul Samuelson, E. B. Wilson, escreveu para ele dizendo que tinha ouvido que Samuelson, recém-nomeado instrutor em Harvard, acabara de receber uma oferta para se juntar ao corpo docente do MIT como professor assistente de economia. Na carta, discutida em detalhes por Roger Backhouse neste volume, Wilson lembrou que ele mesmo, como professor assistente de primeiro ano em Yale, recebeu uma oferta para ir para o MIT como professor associado, aceitou a oferta e "nunca se arrependeu de ter feito isso". Ele continuou a dizer:

Pensei muito sobre a situação da economia no Tech. Quando [Francis Amasa] Walker era presidente... as perspectivas para a economia eram extremamente boas. [Desde a morte de Walker] a equipe [de economia] do Tech não foi notavelmente estatística ou matemática e de forma alguma capitalizou adequadamente em sua instrução [sobre] o histórico de seus alunos, que consiste em 2 anos de matemática obrigatória, 2 anos de física obrigatória, 1 ano de química obrigatória, e um ano de mecânica obrigatória... Parece que um curso muito mais poderoso sobre economia poderia ser dado se esse histórico fosse completamente utilizado... Acho que mostra uma inteligência extremamente aguçada por parte do Tech [e Freeman] tentar te conseguir... O que vai acontecer no Tech eu não sei, mas eles estão começando bem se te garantirem.

E em uma carta de acompanhamento datada de 14 de outubro, Wilson concluiu:

Espero que você avance rapidamente no Tech, talvez rápido o suficiente para que não pense em ir para outro lugar, como foi de fato a minha própria experiência. Se você ficar lá enquanto o Tech continua a nomear pessoas como você para sua equipe, então, quando você estiver no início da meia-idade, pode haver uma grande mudança no ensino no Tech e o Tech pode ter um departamento de pesquisa realmente distinto, como agora tem em matemática, física e química.

A previsão de Wilson, feita no final de 1940, foi perspicaz. Samuelson permaneceu no MIT, e o departamento, reforçado por várias nomeações de estudiosos tecnicamente fortes, tornou-se, no final dos anos 1950, um dos três ou quatro departamentos de pesquisa de economia mais distintos da América do Norte. Em outra década, se tornaria o departamento de economia mais conceituado do mundo. A história desse processo, e o importante artigo que forneceu o entendimento de base comum para os conferencistas, foi desenvolvido por Beatrice Cherrier (neste volume) a partir de várias fontes arquivísticas. Ao mesmo tempo, a ascensão do MIT à proeminência coincidiu com a notável transformação da economia americana no período pós-guerra. Cherrier é a primeira historiadora da economia a assumir o desafio narrativo de situar as circunstâncias particulares do MIT no contexto dessa transformação.

Ao longo dos últimos vinte e cinco anos, a faculdade de história da economia da Duke, juntamente com os bibliotecários de desenvolvimento de coleções (particularmente Robert Byrd e Will Hansen) na Biblioteca de Livros Raros e Manuscritos David M. Rubenstein, têm reunido os trabalhos de economistas notáveis (principalmente) do século XX no que agora é chamado de Projeto de Documentos de Economistas. Com o tempo, esse arquivo cresceu e se tornou central para a pesquisa histórica sobre economia no período pós-guerra. Os trabalhos de Edwin Burmeister, Evsey Domar, Franklin Fisher, Duncan Foley, Lawrence Klein, Franco Modigliani e Robert Solow, todos professores ou alunos do MIT, atraíram estudiosos de todo o mundo. Após a morte de Paul Samuelson em dezembro de 2009, seus trabalhos, por acordo prévio, foram para o Projeto de Documentos de Economistas e rapidamente se tornaram um ímã para historiadores da economia. Em resposta, no início de 2010, fui encorajado por meus colegas Bruce Caldwell, Neil De Marchi, Craufurd Goodwin e Kevin Hoover a planejar uma conferência na série de conferências anuais HOPE para examinar a história da economia do MIT. Após um ano de conversas e e-mails, convidei um grupo seleto de estudiosos para considerar o papel do MIT na transformação da economia americana no período pós-guerra. Essa conferência, realizada de 26 a 28 de abril de 2013, no Centro de Conferências R. David Thomas na Universidade Duke, foi patrocinada como de costume pela Duke University Press. No entanto, o generoso apoio financeiro da Fundação Alfred P. Sloan tornou possível a expansão da conferência HOPE "padrão" em uma que incluía um número maior de participantes e trabalhos. No final, os conferencistas aprenderam que contar a história do papel do MIT no período pós-guerra exigia atenção tanto às circunstâncias particulares que moldaram o MIT quanto às várias maneiras pelas quais a economia em si estava mudando.

\subsubsection{\textbf{O Desafio Historiográfico}}
Histórias na história da ciência envolvem tanto narrativas de continuidade quanto narrativas de mudança disruptiva. Historiadores da economia, que há muito empregam tal distinção, parecem concordar que a década de 1940 viu uma grande ruptura entre uma economia mais antiga e uma economia mais nova. Antes da Segunda Guerra Mundial, mesmo quando a economia abrangia abordagens, metodologias e teorias diversas, a maioria dos economistas ainda empregava os mesmos tipos de ferramentas, estudava os mesmos textos, examinava e reexaminava um cânone estável e compartilhava metas e práticas educacionais comuns. Retoricamente, eles eram indistinguíveis. Além das primeiras edições da Econometrica (e mais tarde da Review of Economic Studies), a maioria dos livros, revistas, artigos e resenhas publicados na década de 1930 exibia sua natureza literária, mesmo quando incorporavam argumentos geométricos e estatísticos descritivos ocasionais. Em contraste, no meio da década de 1950, a maioria das obras publicadas mais proeminentes tinha um caráter "científico" e apresentava argumentos matemáticos ou econométricos em apoio a conjecturas e hipóteses. Essas mudanças foram suficientemente bem reconhecidas na época para que a comunidade de economistas buscasse impor novos padrões para aqueles que buscavam credenciais profissionais como economistas.

Ao longo do próximo meio século, os historiadores da economia procuraram caracterizar essa ruptura e explicar suas fontes e consequências. Diferentes historiadores examinaram essa transformação de maneiras diferentes, nem todas consistentes entre si. A historiografia é adicionalmente complicada porque a década de 1940 é recente o suficiente para que várias das figuras centrais elas mesmas forneceram relatos da transformação aos historiadores. Mas, como é verdade para muitas histórias da ciência contemporânea, a tensão entre os interesses dos historiadores e dos observadores participantes complica o processo de construção de relatos convincentes do período pós-guerra. Contar a história da economia moderna deve começar interpretando a descontinuidade tanto no corpo do conhecimento econômico quanto na imagem do conhecimento econômico mantida pelos economistas no período pós-guerra. Mas existem muitas interpretações.

Se a tarefa específica aqui é contar a história do MIT enquanto emprega (ou rejeita) as histórias de descontinuidade, será útil primeiro olhar para longe do MIT para descrever como os estudiosos abordaram a história geral do período pós-guerra e as mudanças tanto no conhecimento econômico quanto na comunidade de economistas entre as décadas de 1930 e 1950. Depois de enquadrar essas questões, podemos ver melhor como os conferencistas abordaram as maneiras pelas quais a história do MIT é consistente e, no entanto, diferente daquelas histórias frequentemente contadas.

\subsubsection{\textbf{Narrativas Competitivas}}

Existem três abordagens narrativas bem estabelecidas para contar a história de como a economia mudou no período pós-guerra. Provavelmente a história mais antiga da transformação fala sobre o mundo antes do livro de John Maynard Keynes de 1936, A Teoria Geral do Emprego, Juros e Dinheiro, e o mundo depois que o livro de Keynes apareceu. Um exemplo antigo disso é o volume de 1967 de G. L. S. Shackle, Os Anos da Alta Teoria. Certamente A Revolução Keynesiana de Lawrence Klein ([1947] 1966), uma extensão de sua tese de doutorado escrita no MIT sob Paul Samuelson, contribuiu para essa maneira de pensar, assim como o próprio Keynes em seu livro de 1936, que começou caracterizando "economistas clássicos" como todo economista antes dele. O próprio Samuelson, que aprendeu economia keynesiana com Alvin Hansen em Harvard, era membro de uma geração que viu, na teoria de Keynes, uma saída para a Grande Depressão. Assim, quando A Estrutura das Revoluções Científicas de Thomas Kuhn apareceu em 1962, os economistas já se referiam à revolução keynesiana por duas décadas e foram capazes de apropriar os argumentos de Kuhn como adequados ao caso Keynes: a história se tornou uma em que a ciência neoclássica normal através dos anos 1920 enfrentou a crise da Depressão, e a teoria revolucionária de mudança de paradigma de Keynes criou um novo tipo de ciência econômica normal nos anos 1940. Essa história exigia que a economia se dividisse acentuadamente em antes e depois de Keynes. Mesmo argumentos sobre a mudança da natureza da econometria entre as décadas de 1930 e 1950, mudanças deploradas por Keynes e seus associados, foram moldados em termos desta revolução keynesiana (Louça 2007). Da mesma forma, o desenvolvimento das contas nacionais de renda por Simon Kuznets, James Meade e Richard Stone foi tornado necessário e facilitado pela revolução keynesiana, enquanto o planejamento de guerra empregava categorias keynesianas, e a coleta de dados estatísticos pós-Segunda Guerra Mundial refletia essas novas características da vida econômica. A revolução keynesiana nesta conta padrão produziu uma mudança na natureza da política econômica, à medida que os economistas se tornaram incorporados nos governos como técnicos e analistas. Nos Estados Unidos, a Lei do Emprego de 1946 reificou essa transformação com a criação do Conselho de Assessores Econômicos e o Relatório Econômico do Presidente.

Vários artigos neste volume da conferência envolvem essa narrativa mestra em maior ou menor grau. Perry Mehrling argumenta que as alianças keynesianas de Samuelson e Modigliani geraram um tipo particular de análise e política monetária. A teoria do crescimento que surgiu do volume de 1958 de Robert Dorfman, Samuelson e Solow sobre programação linear, e o artigo de Solow de 1957, era neoclássica no sentido da síntese neoclássica, mas essa síntese em si envolvia a teoria keynesiana, pois era isso que era "sintetizado" para a teoria neoclássica. Este trabalho no MIT é discutido nos artigos de Verena Halsmayer e de Mauro Boianovsky e Kevin D. Hoover.

A narrativa keynesiana enquadra a discussão da transformação da economia e permite novos tipos de investigações. Por exemplo, um trabalho importante em economia seria localizado naquelas instituições acadêmicas que adotaram mais rapidamente as ideias keynesianas. A assimilação das ideias keynesianas no nível do livro didático criou uma nova geração de livros didáticos elementares começando com os de Lorie Tarshis e Samuelson, publicados em 1947 e 1948, respectivamente. Este é o assunto do artigo de Yann Giraud neste volume. A "nova" economia pode ser rastreada até Keynes e aqueles que o cercavam em Cambridge, pelo menos de acordo com historiadores com alguma conexão com Cambridge. A relação complexa, mas confusa, entre essas duas redes keynesianas é o assunto do artigo de Roger Backhouse destruindo a controvérsia da teoria do capital de Cambridge.

Uma segunda narrativa de enquadramento surgiu do importante volume da conferência HOPE editado por Mary Morgan e Malcolm Rutherford intitulado "Do Pluralismo Inter-guerras ao Neoclassicismo Pós-guerra" (1998). Vários estudiosos haviam se tornado insatisfeitos com histórias caracterizando a transformação da economia do século XX nos Estados Unidos como uma luta na qual o institucionalismo perdeu adeptos em relação a um neoclassicismo ascendente. Estudiosos como Rutherford (2003) e Yuval Yonay (1998) começaram a reconstruir o período entre guerras como um de diversidade teórica. As lacunas nas histórias anteriormente contadas eram que nem o institucionalismo nem a economia neoclássica haviam sido monolíticos. A teoria da demanda e a teoria da produção e a teoria da firma haviam sido contestadas e contestadas nos anos entre guerras. No entanto, no meio da década de 1950, eles eram capítulos resolvidos em livros didáticos de microeconomia intermediária e de pós-graduação. Para os historiadores da economia, a questão emergente era: "Como a microeconomia do pós-guerra se estabilizou?" "Keynes" não poderia ser a resposta para esta pergunta. De uma economia entre guerras na qual um grande número de temas e tópicos na teoria econômica não eram necessariamente consistentes entre si, a década de 1950 apresentou uma visão coerente do que se tornou a microeconomia. A estabilidade da economia neoclássica tornou-se o ponto final a ser explicado. Contas desse processo de estabilização particular tornaram-se um conjunto mais ou menos convincente de argumentos de acordo com os quais os historiadores narravam a transformação da economia. O mais proeminente desses argumentos era que modelos matemáticos e o uso de técnicas econométricas para fornecer testes empíricos desses modelos desembaraçavam os vários enigmas teóricos, perplexidades e enigmas. Modelos geométricos simples e estatísticas descritivas não eram mais suficientes para estabelecer as reivindicações de um economista. A microeconomia do pós-guerra foi estabilizada pelo uso crescente de matemática e estatística nas análises econômicas (Weintraub 1991, 2002). O papel dos modelos tornou-se mais importante, e o papel dos julgamentos de valor e preocupações éticas tornou-se menos importante no ensino de graduação, currículos de pós-graduação e na socialização de novos ingressantes na profissão de economista. O período do pós-guerra testemunhou uma nova retórica da economia. Os artigos neste volume de (novamente) Halsmayer e Harro Maas abordam essas questões importantes.

Talvez a importância crescente de modelos mais formais, mais explícitos neste período estivesse associada à crescente consciência epistemológica, até mesmo autoconsciência, dos economistas neste momento de mudança metodológica. As referências de Samuelson ao "operacionalismo" de Percy Bridgman e ao "falsificacionismo" de Karl Popper, conforme apresentado aos economistas por Terence Hutchison, exigiam algo teórico, que na economia só poderia significar um modelo, para ser submetido a testes para reivindicar status científico (Morgan 2012). Apesar das diferenças residuais entre as comunidades de economia nacional (Fourcade 2009), a economia se tornou um discurso internacional cujos problemas e respostas foram estabilizados por um conjunto aceito de técnicas. Historiadores da economia trabalhando dentro deste espaço conceitual se concentrariam nas metodologias emergentes associadas à nova economia científica e na natureza técnica aumentada do assunto.

Como o departamento de economia do MIT foi identificado com essa nova economia técnica, os conferencistas passaram algum tempo detalhando os caminhos pelos quais essas análises mudaram os valores intelectuais e as estratégias retóricas da comunidade de economistas no período pós-guerra. Os artigos de Andrej Svorenčík e Pedro Garcia Duarte, em particular, examinam o movimento de pessoas e ideias dentro e fora do departamento de economia do MIT.

Um terceiro arcabouço historiográfico empregado para explicar a transformação da economia no período pós-guerra cresceu a partir de preocupações políticas e sociológicas. Histórias contadas a partir dessa consciência reconheceram que, antes da guerra, a economia estava conectada de maneira disciplinar a tradições nacionais específicas em países que produziam economia e economistas. Como resultado dos destroços da Segunda Guerra Mundial, os Estados Unidos se tornaram o principal produtor de análises econômicas não marxistas. A hegemonia da América estabilizou o discurso econômico ao substituir várias tradições nacionais pelas tradições emergentes da comunidade econômica americana. O artigo de Stephen Meardon sobre Charles Kindleberger aborda essas questões. Apenas os Estados Unidos tinham os recursos para educar, contratar e treinar economistas e para publicar pesquisas em economia em larga escala. Com o tempo, a economia se estabilizou no sentido de que tradições nacionais díspares foram submersas e marginalizadas em relação a uma economia mainstream americana. Muitas dessas tradições continuaram, é claro, como heterodoxia. Dois exemplos particulares foram o grupo pós-keynesiano do Reino Unido em torno de Joan Robinson em Cambridge (Backhouse, neste volume, "A Outra Cambridge"), e o grupo austríaco, neo-austríaco e de Chicago em torno de Friedrich Hayek que encontrou um lar entre o pequeno grupo de neoliberais conectados à Sociedade Mont Pelèrin. Além da economia de Chicago, as tradições heterodoxas não eram bem vistas e, de fato, produziram poucas ou nenhuma conexão séria com as estruturas de recompensa da economia mainstream, e não foram representadas no MIT.

Essas três narrativas de enquadramento da mudança na economia no período pós-guerra - a revolução keynesiana, a estabilização retórica da economia baseada em modelos matemáticos e econométricos, e a americanização da economia - é claro que não são mutuamente exclusivas nem exaustivas. Assim, houve espaço aberto na última década para a construção de narrativas alternativas. Alguns trabalhos de um pequeno número de historiadores da economia, bem como alguns membros da maior comunidade de estudos científicos, foram construídos a partir do tema da americanização e tiveram como uma de suas implicações necessárias que a ênfase da disciplina econômica do pós-guerra na construção e análise de modelos foi, nos termos de Andrew Pickering (1995), um movimento forçado e não um movimento livre. Alguns estudiosos fortemente críticos tanto da economia mainstream quanto dos economistas mainstream foram mais longe e caracterizaram a economia do pós-guerra como um desastre intelectual moldado e controlado pela ideologia da Guerra Fria. Philip Mirowski, particularmente em seu Machine Dreams (2002), e Sonja Amadae em seu Rationalizing Capitalist Democracy (2003) argumentaram que a contingência da Guerra Fria determinou a natureza e as instituições da economia do pós-guerra. É claro que ninguém nega que os economistas, trazidos para o serviço durante a guerra para analisar problemas de alocação de recursos - rotas de envio, fogo antiaéreo, movimentos de comboios, guerra anti-submarina, logística, estoques militares, controle de qualidade das linhas de produção militar, e assim por diante - ajudaram a criar um novo tipo de análise que logo seria chamado de pesquisa operacional. Mas Mirowski e Amadae ridicularizam RAND, Cowles e fundações como Ford e Rockefeller como contribuintes conscientes ou inconscientes para o projeto da Guerra Fria americana. A partir da perspectiva desta quarta narrativa, a teoria dos jogos, que eventualmente unificaria grandes áreas do que se tornou a economia neoclássica, teve origem na Guerra Fria militar e instanciou suposições sobre o comportamento humano que tornaram impossível aceitar visões mais generosas e díspares da ação humana. Esta historiografia enfatiza o papel dos fluxos de dinheiro do militar, subsídios, contratos e redes pessoais de guerreiros da Guerra Fria na economia emergente. Em um artigo neste volume rejeitando várias das afirmações de Mirowski, William Thomas examina como a pesquisa operacional se desenvolveu no MIT. Ao estabelecer que não se desenvolveu de forma alguma no departamento de economia do MIT, Thomas refuta a afirmação central de Mirowski.

Uma quinta linha narrativa, não frequentemente explorada por historiadores da economia, também se desenvolveu a partir da contingência da Segunda Guerra Mundial. O tranquilo remanso que era a universidade americana antes da guerra foi transformado não apenas pelo novo mundo de Vannevar Bush e pelo financiamento da ciência ligado ao interesse da Guerra Fria, mas também pela peça mais notável de legislação social no período imediatamente após a guerra. A Lei de Reajuste dos Militares de 1944, conhecida informalmente como GI Bill, reformulou o ensino superior americano. Faculdades e universidades estavam moribundas durante a Grande Depressão, pois reduziram o pessoal e a remuneração e adiaram os pensamentos de contratar novos professores e admitir mais alunos. No período pré-guerra, eles treinaram estudantes de pós-graduação em números tão pequenos que havia poucos instrutores disponíveis para ensinar a enxurrada de estudantes de graduação pagantes que se candidatavam à admissão. Foi nesse momento, por exemplo, que Lionel McKenzie, recentemente desmobilizado e enfrentando meio ano de desemprego antes de poder assumir sua bolsa Rhodes para Oxford, usou seu mestrado em economia de Princeton para garantir uma posição ensinando economia industrial para alunos de graduação do MIT. Novos economistas eram necessários como professores, analistas governamentais e economistas de negócios. Programas de pós-graduação em economia proliferaram. O grande número de estudantes de economia de graduação forçou mudanças nos estilos de instrução também: ferramentas e técnicas eram ensináveis de uma maneira que uma economia baseada em filosofia moral não era. O artigo de Pedro Teixeira lida precisamente com essas questões, assim como o de Giraud. Os GIs eram, em média, mais velhos e mais necessitados de credenciais de trabalho do que as elites estereotipadas que se pensava povoar os estabelecimentos da Ivy League ou os irmãos de fraternidade e irmãs de irmandade que habitavam muitas instituições estaduais americanas antes da guerra. Esses novos alunos também eram "melhores" do que as coortes anteriores: os melhores alunos de um grande grupo de candidatos serão pelo menos tão bons quanto os melhores alunos do grupo menor incluído. Esse influxo de estudantes não apenas convocou uma nova geração de professores universitários, mas também teve implicações profundas para o tipo de economia feita. As universidades se tornaram instrumentais para garantir o emprego para os veteranos que retornavam, temerosos de estar subqualificados para o emprego. Os veteranos que retornavam buscavam credenciais e conhecimento útil. Uma educação clássica baseada em humanidades era um luxo que esses alunos mais velhos não podiam pagar. A demanda por cursos de economia e de negócios explodiu (Augier e March 2011). De uma educação em artes liberais pré-guerra que desprezava a economia como um assunto prático adequado apenas para aqueles desqualificados por criação ou disponibilidade de lazer para estudar as humanidades, a economia do pós-guerra se tornou uma disciplina acadêmica importante.

Embora tenham havido poucas contextualizações sustentadas das mudanças na economia do pós-guerra que se baseiam extensivamente nessas questões educacionais e institucionais, e ainda menos por historiadores da economia, duas contribuições recentes ao longo dessas linhas chamaram a atenção de ambos os historiadores e economistas. A socióloga Marion Fourcade (2009) examinou a disciplina profissional de economia nos Estados Unidos, Grã-Bretanha e França de 1890 a 1990, e teceu as diferentes instituições sociais e educacionais de cada país em sua narrativa de mudança na economia. Mais recentemente, estudiosos se envolveram com uma excelente discussão dessas questões com relação à educação empresarial. A estudiosa da teoria organizacional Mie Augier e o teórico da educação James March, em seu livro "Raízes, Rituais e Retóricas da Mudança: Escolas de Negócios da América do Norte após a Segunda Guerra Mundial" (2011), contam uma nova história de uma economia em mudança no período pós-guerra através de uma história do crescimento das escolas de negócios nesse período. Muitos dos tópicos aludidos nas várias historiografias da economia estavam presentes também na educação empresarial, e a ruptura pré-guerra/pós-guerra foi ainda mais significativa nos negócios do que na economia. Mas, é claro, o destino da economia e da educação empresarial estavam entrelaçados.

O departamento de economia do MIT não estava isolado dos desafios mais gerais que enfrentavam tanto as universidades quanto a comunidade de economistas. Sua resposta ao afastamento da profissão da história é o assunto do artigo de Peter Temin sobre o papel da história econômica no programa do MIT. Com relação à resposta da disciplina à discriminação passada contra candidatos negros em muitas escolas de pós-graduação, e a ausência resultante de economistas negros nas faculdades de universidades de pesquisa, as tentativas do MIT de abordar isso foram, apesar das boas intenções, não mais bem-sucedidas do que as de outros departamentos de economia "top". O artigo de William Darity Jr. e Arden Kreeger fornece um relato detalhado dessa tentativa e suas consequências.

Qualquer uma ou várias ou todas as cinco narrativas - três mais antigas e duas mais recentes - podem "explicar" a crescente importância do departamento de economia do MIT no período pós-guerra. A revolução keynesiana? Paul Samuelson, confere. A natureza técnica emergente da disciplina? O treinamento técnico dos alunos do MIT, cientistas e engenheiros nascentes, e um corpo docente trazido para ensinar tais alunos, confere. Financiamento militar da Guerra Fria e o envolvimento de faculdades e instituições de economia com ambas as fontes de apoio e profissões de necessidade nacional? A história do MIT e do Rad Lab, e a explosão de gastos de defesa que passou pelo MIT neste período, confere. A influência internacional da economia e instituições americanas neste período? A conexão de uma instituição principalmente tecnológica com um mundo pós-guerra faminto pelo treinamento em engenharia e tecnologia que o MIT poderia fornecer, confere. Um influxo de estudantes na GI Bill transformando tanto a missão quanto a natureza do MIT como instituição? A criação de um programa real de economia no MIT exatamente neste período imediatamente após a guerra, e os primeiros movimentos que criariam a Sloan School, confere.

Paul Samuelson (2000), de fato, empregou duas dessas narrativas em sua própria contabilidade:

Dois fatores explicam nosso sucesso. Um, o renascimento do MIT após a Segunda Guerra Mundial como um recurso de pesquisa apoiado pelo governo federal. Dois, a revolução matemática na teoria macro e microeconômica e estatísticas: isso estava atrasado e era inevitável - o MIT era o lugar lógico para isso florescer. O que não era inevitável era que o Departamento de Economia do MIT mantivesse ao longo de seu desenvolvimento explosivo uma reputação merecida de amabilidade colegial.

A ideia de que o MIT surgiu do "nada" na década de 1930 para se tornar um dos três ou quatro locais mais importantes para a pesquisa econômica até meados da década de 1950 parece ser sobredeterminada em relação a essas várias narrativas. Não seria inadequado desenvolver a história da ascensão meteórica do MIT à proeminência ao longo de qualquer um desses cinco eixos narrativos. E, como a conferência deixou claro, há até uma sexta narrativa que apoia qualquer história de sucesso rápido do MIT, ou seja, que o período imediatamente após a guerra viu um colapso - em alguns lugares mais lento, em outros mais rápido - das barreiras para a contratação de professores judeus em faculdades e universidades americanas. Mais do que qualquer outra universidade privada ou pública de elite, particularmente as universidades da Ivy League, o MIT acolheu economistas judeus. Tanto a discussão de Backhouse sobre a mudança de Samuelson para o MIT, quanto a discussão do meu próprio artigo sobre o assunto, abriram este tópico para consideração da conferência.

\end{document}