\documentclass[a4paper,12pt]{article}[abntex2]
\bibliographystyle{abntex2-alf}
\usepackage{siunitx} % Fornece suporte para a tipografia de unidades do Sistema Internacional e formatação de números
\usepackage{booktabs} % Melhora a qualidade das tabelas
\usepackage{tabularx} % Permite tabelas com larguras de colunas ajustáveis
\usepackage{graphicx} % Suporte para inclusão de imagens
\usepackage{ragged2e} % Justificação de texto melhorada
\usepackage{setspace} % Controle do espaçamento entre linhas
\usepackage[a4paper, left=3.0cm, top=3.0cm, bottom=2.0cm, right=2.0cm]{geometry} % Personalização das margens do documento
\usepackage{lipsum} % Geração de texto dummy 'Lorem Ipsum'
\usepackage{fancyhdr} % Customização de cabeçalhos e rodapés
\usepackage{titlesec} % Personalização dos títulos de seções
\usepackage[portuguese]{babel} % Adaptação para o português (nomes e hifenização
\usepackage{hyperref} % Suporte a hiperlinks
\usepackage{indentfirst} % Indentação do primeiro parágrafo das seções
\sisetup{
  output-decimal-marker = {,},
  inter-unit-product = \ensuremath{{}\cdot{}},
  per-mode = symbol
}
\DeclareSIUnit{\real}{R\$}
\newcommand{\real}[1]{R\$#1}
\usepackage{float} % Melhor controle sobre o posicionamento de figuras e tabelas
\usepackage{footnotehyper} % Notas de rodapé clicáveis em combinação com hyperref
\hypersetup{
    colorlinks=true,
    linkcolor=black,
    filecolor=magenta,      
    urlcolor=cyan,
    citecolor=black,        
    pdfborder={0 0 0},
}
\usepackage[normalem]{ulem} % Permite o uso de diferentes tipos de sublinhados sem alterar o \emph{}
\usepackage{fontspec}
\setmainfont{Arial}
\makeatletter
\def\@pdfborder{0 0 0} % Remove a borda dos links
\def\@pdfborderstyle{/S/U/W 1} % Estilo da borda dos links
\makeatother
\onehalfspacing
\setlength{\headheight}{14.49998pt}
\begin{document}

\begin{titlepage}
    \centering
    \vspace*{1cm}
    \Large\textbf{INSPER – INSTITUTO DE ENSINO E PESQUISA}\\
    \Large ECONOMIA\\
    \vspace{1.5cm}
    \Large\textbf{Carta de Motivação}\\
    \textbf{PIBIC}\\
    \vspace{1.5cm}
    Prof. Paulo Cilas Marques Filho\\
    \vfill
    \normalsize
    Hicham Munir Tayfour, \href{mailto:hichamt@al.insper.edu.br}{hichamt@al.insper.edu.br}\\
    4º Período - Economia (B)\\
    \vfill
    São Paulo\\
    Mês/2024
\end{titlepage}

\newpage
Prezado Professor Paulo Cilas Marques Filho,

Meu nome é Hicham Munir Tayfour e estou no 4º semestre do curso de Economia. Tenho grande interesse em participar da iniciação científica no projeto nº 12, "Aprendizagem de máquina aplicada a problemas de causalidade".

Desde que iniciei minha jornada acadêmica e comecei a desenvolver trabalhos que envolviam pesquisas preliminares, estudos de caso e formulação de hipóteses, meu interesse em aprofundar meus estudos e realizar pesquisas mais fundamentadas aumentou. Procurei a orientação de professores das disciplinas que mais me interessavam para entender melhor o papel de um pesquisador, como se desenvolve um processo de pesquisa e quais passos devo seguir antes de iniciar um estudo específico.

Fui agraciado com feedbacks incríveis. Todos os professores com quem conversei foram extremamente abertos, receptivos e encorajadores em relação à minha trajetória acadêmica.

No 1º período, as disciplinas de Sistemas de Informação, com foco em programação em Python, e Cálculo 1 despertaram meu interesse. No 2º período, Cálculo 2 e Estatística 1, tanto a parte de probabilidade quanto a análise exploratória de dados usando R, reforçaram minha atração pela academia. As aplicações de cálculo em estatística, aliadas à habilidade de explorar dados usando R, foram determinantes nesse ponto. No 3º período, a disciplina de Estatística 2, onde aprendi a inferir a partir de uma amostra de observações da população e suas aplicações em R, me levou além do currículo da graduação. Agora, no 4º período, a disciplina de Econometria me fascina. A criação de modelos teóricos e computacionais usando R para entender como um conjunto de variáveis impacta outras, modelando regressões, respeitando suposições fundamentais e estimando os modelos para obter respostas, foi o que mais me motivou a querer realizar pesquisas.

Tive a oportunidade de entrar em contato com alguns princípios de Ciência de Dados através do livro "R for Data Science" de Hadley Wickham. Com este livro, aprofundei meus conhecimentos em funções do R, como otimizar processos, organizar dados, extrair informações relevantes, construir gráficos e usar bancos de dados na internet, através da linguagem SQL. Além disso, aprendi sobre markdown e LaTeX ao criar documentos no RStudio usando essas duas linguagens.

Além disso, tenho uma certa afinidade com a bibliografia básica do projeto, dado que é um tipo de material que aprecio consumir, e alguns deles foram citados em bibliografias presentes nas disciplinas da graduação, tanto regulares quanto eletivas.

Estou entusiasmado com a possibilidade de trabalhar sob sua orientação e progredir com o projeto "Aprendizagem de máquina aplicada a problemas de causalidade". Acredito que será uma excelente oportunidade para aprender ainda mais, não apenas sobre pesquisa, mas também adquirir um conhecimento vasto e altamente relevante para outros momentos da graduação. Estou disposto a me dedicar ao máximo a este projeto.

Vou deixar aqui os meus repositórios no GitHub de Python, R e Latex :

Repositório GitHub Python : \href{https://github.com/Hic-Tayfour/Python}{https://github.com/Hic-Tayfour/Python} 

Repositório GitHub R : \href{https://github.com/Hic-Tayfour/R}{https://github.com/Hic-Tayfour/R}

Repositório GitHub LaTex :\href{Repositório GitHub R : https://github.com/Hic-Tayfour/R Repositório GitHub LaTex : https://github.com/Hic-Tayfour/LaTex-OverLeaf  GitHub - Hic-Tayfour/LaTex-OverLeaf Contribute to Hic-Tayfour/LaTex-OverLeaf development by creating an account on GitHub. github.com  }{ https://github.com/Hic-Tayfour/LaTex-OverLeaf}

Agradeço antecipadamente a consideração pela minha candidatura.

Atenciosamente,

Hicham Munir Tayfour
\end{document}