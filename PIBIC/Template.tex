\documentclass[a4paper,12pt]{report}
\usepackage{graphicx}
\usepackage{indentfirst}
\usepackage{setspace}
\usepackage{titlesec}
\usepackage[top=3cm, bottom=2cm, left=3cm, right=2cm]{geometry}
\usepackage{tocloft}
\usepackage{amsmath}
\usepackage{hyperref}

% Configurações de formatação
\setlength{\parindent}{1.25cm}
\setlength{\parskip}{0cm}
\renewcommand{\cftsecfont}{\normalfont}
\renewcommand{\cftsecpagefont}{\normalfont}

% Definindo título dos capítulos e seções
\titleformat{\chapter}[hang]{\normalfont\bfseries\LARGE}{\thechapter}{2pc}{}
\titleformat{\section}[hang]{\normalfont\bfseries\Large}{\thesection}{1pc}{}
\titleformat{\subsection}[hang]{\normalfont\bfseries\normalsize}{\thesubsection}{1pc}{}

% Início do documento
\begin{document}

% Capa
\begin{titlepage}
    \centering
    \vspace*{5cm}
    {\bfseries\Large Nome da Instituição}\\[0.2cm]
    {\bfseries\Large Nome da Faculdade}\\[4cm]
    {\bfseries\Large Nome do Aluno}\\[6cm]
    {\bfseries\Large Título do Trabalho}\\[0.5cm]
    {\bfseries\large Subtítulo (se houver)}\\[6cm]
    {\bfseries\Large Cidade}\\[0.5cm]
    {\bfseries\Large Ano}
\end{titlepage}

% Folha de Rosto
\begin{titlepage}
    \centering
    \vspace*{5cm}
    {\bfseries\Large Nome do Aluno}\\[5cm]
    {\bfseries\Large Título do Trabalho}\\[0.5cm]
    {\bfseries\large Subtítulo (se houver)}\\[4cm]
    \hspace{7cm}
    \begin{minipage}{0.5\textwidth}
        \begin{flushright}
            Trabalho de conclusão de curso apresentado ao curso de Graduação em Nome do Curso da Nome da Instituição, como requisito parcial para a obtenção do título de Bacharel em Nome do Curso.\\[2cm]
            Orientador: Prof. Nome do Orientador
        \end{flushright}
    \end{minipage}\\[4cm]
    {\bfseries\Large Cidade}\\[0.5cm]
    {\bfseries\Large Ano}
\end{titlepage}

% Agradecimentos
\chapter*{Agradecimentos}
\begin{flushright}
    \vspace*{5cm}
    \textit{Lorem ipsum dolor sit amet, consectetur adipiscing elit. Integer posuere erat a ante.}
\end{flushright}

% Resumo
\chapter*{Resumo}
\addcontentsline{toc}{chapter}{Resumo}
Apresentação concisa das principais partes do trabalho fornecendo uma visão rápida e clara do conteúdo do trabalho: tema, objetivo, metodologia e conclusão. Em parágrafo único, texto com alinhamento justificado e sem tabulação. Recomenda-se extensão de 150 a 500 palavras.

Palavras-chave: termo1; termo2; termo3; termo4; termo5.

% Abstract
\chapter*{Abstract}
\addcontentsline{toc}{chapter}{Abstract}
Versão em inglês do resumo: apresentação concisa das principais partes do trabalho fornecendo uma visão rápida e clara do conteúdo do trabalho: tema, objetivo, metodologia e conclusão. Em parágrafo único, texto com alinhamento justificado e sem tabulação. Recomenda-se extensão de 150 a 500 palavras.

Keywords: keyword1; keyword2; keyword3; keyword4; keyword5.

% Sumário
\tableofcontents

% Introdução
\chapter{Introdução}

% Desenvolvimento
\chapter{Desenvolvimento}

% Conclusão
\chapter{Conclusão}

% Referências
\begin{thebibliography}{99}
    \addcontentsline{toc}{chapter}{Referências}
    \bibitem{referencia1} SOBRENOME, Nome. Título do livro. Edição. Local de publicação: Editora, Ano de publicação.
\end{thebibliography}

\end{document}
